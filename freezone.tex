\chapter{Freezone}
\chapterauthor{John Shirley}

John Shirley has often been first to tread frontiers that later became well-trampled cyberpunk turf. As a rock performer, he was heavily involved in the first virulent outbreak of punk on the West Coast. A prolific writer whose work includes such novels as City Come A-Walkin', The Brigade, and the horror extravaganza Cellars, Shirley is well known for his soaring, surreal imagery and bursts of extreme visionary intensity.

"Freezone" is an independent excerpt from Shirley's latest project, the Eclipse trilogy. Global in scope, Eclipse narrates a dizzying near-future where pop, politics, and paranoia collide in a high-tech struggle for survival. Always a pioneer, with a wide-ranging underground influence, Shirley's use of global issues may well portend a new upsurge of radical politics in SF.

John Shirley currently lives in Los Angeles and performs with his band.

\hrulefill

\firstletter{F}reezone floated in the Atlantic Ocean, a city afloat in the wash of international cultural confluence.

Freezone was anchored about a hundred miles north of Sidi Ifni, a drowsy city on the coast of Morocco in a warm, gentle current, and in a sector of the sea only rarely troubled by large storms. What storms arose here spent their fury on the maze of concrete wave-baffles Freezone Admin had spent years building up around the artificial island.

Originally, Freezone had been just another offshore drilling project. The massive oil deposit a quarter-mile below the artificial island was still less than a quarter tapped out. The drilling platform was owned in common by the Moroccan government and a Texas-based petroleum and electronics products company. TexMo. The company that bought Disneyland and Disneyworld and Disneyworld II—all three of which had closed in the wake of the CSD: the Computer Storage Depression. Also called the Dissolve Depression.

A group of Arab terrorists—at least, the US State Department claimed that's who did it—had arranged a well-placed electromagnetic pulse from a hydrogen bomb hidden aboard a routine orbital shuttle. The shuttle was vaporized in the blast, as well as two satellites, one of them manned; but when the CSD hit, no one took time to mourn the dead.

The orbital bomb had almost triggered Armageddon: three Cruise missiles had to be aborted, and fortunately two more were shot down by the Russians before the terrorist cell took credit for the upper atmospheric blast. Most of the bomb's blast had been directed upward; what came downward, though, was the side effect of its blast: the EMP. An electromagnetic pulse that—just as had been predicted since the 1970s—traveled through thousands of miles of wires and circuitry on the continent below the H-blast. The Defense Department was shielded; the banking system, for the most part, was not. The pulse wiped out ninety-three percent of the newly formed American Banking Credit Adjustment Bureau. ABCAB had handled seventy-six percent of the nation's buying and credit transferal. Most of what was bought, was bought through ABCAB or ABCAB related companies ... until the EMP wiped out ABCAB's memory storage, the pulse overburdening the circuits, melting them, and literally frying the data storage chips. And thereby kicking the crutches out from under the American economy. Millions of bank accounts were ``suspended'' until records could be restored—causing a run on remaining banks. The insurance companies and the Federal guarantee programs were overwhelmed. They just couldn't cover the loss.

The States had already been in trouble. The nation had lost its economic initiative in the early twenty-first century: its undereducated, badly trained workers, the outsourcing of jobs and manufacturing made US industry unable to compete with the Chinese and South American manufacturing booms. The EMP credit dissolve kicked the nation over the rim of recession and into the pit of depression—and made the rest of the world laugh. The Arab terrorist cell responsible—hard-core Islamic Fundamentalists—had been composed of seven men. Seven men who crippled a nation.

But America still had its enormous military spread, its electronics and medical innovators. And the war economy kept it humming, like a man with cancer taking amphetamines for a last burst of strength, while the endless malls and housing projects—built cheaply and in need of constant upkeep—got shabbier, uglier, trashier by the day. And more dangerous.

The States just weren't safe enough for the rich anymore. The resorts, the amusement parks, the exclusive affluent neighborhoods, places like Central Park West—all crumbled under the attrition of perennial strikes and persistent terrorist attacks. The swelling mass of the poor resented the recreations of the rich.

While the middle-class buffer was shrinking to insignificance there were still enclaves in the States where you could get lost in the media churn, hypnotized by the flashcards of desire into an iPad-trance fantasy of the American Dream as ten thousand companies vied for your attention, nagging you to buy and keep buying. Places that were walled city-states of middle-class illusions—like the place Hard-Eyes had come from.

But the affluent could feel the crumbling of their kingdom. They didn't feel safe in the States. They needed someplace outside, somewhere controlled. Europe was out now; Central and South America, too risky. The Pacific theater was another war zone.

So that's where Freezone came in.

A Texas entrepreneur—who hadn't had his money in ABCAB—saw the possibilities in the community that had grown up around the enormous complex of offshore drilling platforms. A paste-jewel necklace of brothels and arcades and cabarets had crystallized on derelict ships permanently anchored around the platforms. Hundreds of hookers and casino dealers worked the international melange of men who worked the oil rigs.

The entrepreneur made a deal with the Moroccan government, bought the rusting hulks and shanty nightclubs. And then he fired everyone.

The Texan owned a plastics company ... the company had developed light, super-tough plastic that the entrepreneur used in the rafts on which the new floating city was built. The community was now seventeen square miles of urban raft protected with one of the meanest security forces in the world. Freezone dealt in pleasant distractions for the rich in the exclusive section and—in the second-string places around the edge—for technickis from the drill rigs. And the second-string places sheltered a few thousand semi-illicit hangers-on, and a few hundred performers.

Like Rickenharp.

Rick Rickenharp stood against the south wall of the Semiconductor, letting the club's glare and blare wash over him, and mentally writing a song. The song went something like, ``Glaring blare, lightning stare/ Nostalgia for the electric chair.''

Then he thought, Fucking drivel.

All the while he was doing his best to look cool but vulnerable, hoping one of the girls flashing through the crowd would remember having seen him in the band the night before, would try to chat him up, play groupie. But they were mostly into wifi dancers.

And no fucking way Rickenharp was going to wire into minimono.

Rickenharp was a rock classicist; he was retro. He wore a black leather motorcycle jacket that was some seventy-some years old, said to have been worn by John Cale when he was still in the Velvet Underground. The seams were beginning to pop for the third time; three studs were missing from the chrome trimming. The elbows and collar edges were worn through the black dye to the brown animal the leather had come from. But the leather was second skin to Rickenharp. He wore nothing under it. His bony, hairless chest showed translucent-bluewhite between the broken zippers. He wore blue jeans that were only ten years old but looked older than the coat; he wore genuine Harley Davidson boots. Earrings clustered up and down his long, slightly too prominent ears, and his rusty brown hair looked like a cannon-shell explosion.

And he wore dark glasses.

And he did all this because it was gratingly unfashionable.

His band hassled him about it. They wanted their lead-git and frontman minimono.

``If we're gonna go minimono, we oughta just sell the fucking guitars and go wires,'' Rickenharp had told them.

And the drummer had been stupid and tactless enough to say, ``Well, fuck, man, maybe we should go to wires.''

Rickenharp had said, ``Maybe we should get a fucking drum machine, too, you fucking Neanderthal!'' and kicked the drum seat over, sending Murch into the cymbals with a fine crashing, so that Rickenharp added, ``you should get that good a sound outta those cymbals on stage. Now we know how to do it.''

Murch had started to throw his sticks at him, but then he'd remembered how you had to have them lathed up special because they didn't make them anymore, so he'd said, ``Suck my ass, big shot''and got up and walked out, not the first time. But that was the first time it meant anything, and only some heavy ambassadorial action on the part of Ponce had kept Murch from leaving the band.

The call from their agent had set the whole thing off. That's what it really was. Agency was streamlining its clientele. The band was out. The last two download albums hadn't sold, and in fact the engineers claimed that live drums didn't digitize well onto the miniaturized soundcaps that passed for CDs now. Rickenharp's holovid and the videos weren't getting much airplay.

Anyway, Vid-Co was probably going out of business. Another business sucked into the black hole of the depression. ``So it ain't our fault the stuffs not selling,'' Rickenharp said. ``We got fans but we can't get the distribution to reach 'em.''

Mose had said, ``Bullshit, we're out of the Grid, and you know it. All that was carrying us was the nostalgia wave anyway. You can't get more'n two bits out of a revival, man.''

Julio the bassist had said something in technicki which Rickenharp hadn't bothered to translate because it was probably stupid and when Rickenharp had ignored him he'd gotten pissed and it was his turn to walk out. Fucking touchy technickis anyway.

And now the band was in abeyance. Their train was stopped between the stations. They had one gig, just one: opening for a wifi act. And Rickenharp didn't want to do it. But they had a contract and there were a lot of rock nostalgia freaks on Freezone, so maybe that was their audience anyway and he owed it to them. Blow the goddamn wires off the stage.

He looked around the Semiconductor and wished the Retro-Club was still open. There'd been a strong retro presence at the RC, even some rockabillies, and some of the rockabillies actually knew what rockabilly sounded like. The Semiconductor was a minimono scene.

The minimono crowd wore their hair long, fanned out between the shoulders and narrowing to a point at the crown of the head, and straight, absolutely straight, stiff, so from the back each head had a black or gray or red or white teepee-shape. Those, in monochrome, were the only acceptable colors. Flat tones and no streaks. Their clothes were stylistic extensions of their hairstyles. Minimono was a reaction to Flare—and to the chaos of the war, and the war economy, and the amorphous shifting of the Grid. The Flare style was going, dying.

Rickenharp had always been contemptuous of the trendy Flares, but he preferred them to minimono. Flare had energy, anyway.

A flare was expected to wear his hair up, as far over the top of his head as possible, and that promontory was supposed to express. The more colors the better. In that scene, you weren't an individual unless you had an expressive flare. Screwshapes, hooks, aureola shapes, layered multicolor snarls. Fortunes were made in flare hair-shaping shops, and lost when it began to go out of fashion. But it had lasted longer than most fashions; it had endless variation and the appeal of its energy to sustain it. A lot of people copped out of the necessity of inventing individual expression by adopting a politically standard flare. Shape your hair like the insignia for your favorite downtrodden third world country (back when they were downtrodden, before the new marketing axis). Flares were so much trouble most people took to having flare wigs. And their drugs were styled to fit the fashion. Excitative neurotransmitters; drugs that made you seem to glow. The wealthier flares had nimbus belts, creating artificial auroras. The hipper flares considered this to be tastelessly narcissistic, which was a joke to nonflares, since all flares were floridly vain.

Rickenharp had never colored or shaped his hair, except to encourage its punk spikiness.

But Rickenharp wasn't a punkrocker. He identified with prepunk, late 1950s, mid-1960s, early 1970s. Rickenharp was a proud anachronism. He was simply a hard-core rocker, as out of place in the Semiconductor as bebop would have been in the 1980s dance clubs.

Rickenharp looked around at the flat-back, flat-gray, monochrome tunics and jumpsuits, the black wristfones, the cookie-cutter sameness of JAS's; at the uniform tans and ubiquitous FirStep Colony-shaped earrings (only one, always in the left ear). The high-tech-fetishist minimonos were said to aspire toward a place in the Colony the way Rastas had dreamed of a return to Ethiopia. Rickenharp thought it was funny that the Russians had blockaded the Colony. Funny to see the normally dronelike, antiflamboyant minimonos quietly simmering on ampheticool, standing in tense groups, hissing about the Russian blockade of FirStep, in why-doesn't-someone-do-something outrage.

The stultifying regularity of their canned music banged from the walls and pulsed from the floor. Lean against the wall and you felt a drill-bit vibration of it in your spine.

There were a few hardy, defiant flares here, and flares were Rickenharp's best hope for getting laid. They tended to respect old rock.

The music ceased; a voice boomed, ``Joel NewHope!'' and spots hit the stage. The first wifi act had come on. Rickenharp glanced at his watch. It was ten. He was due to open for the headline act at 11:30. Rickenharp pictured the club emptying as he hit the stage. He wasn't long for this club.

NewHope hit the stage. He was anorexic and surgically sexless: radical minimono. A fact advertised by his nudity: he wore only gray and black spray-on sheathing, his dick in a drag queen's tuck. How did the guy piss? Rickenharp, wondered. Maybe it was out of that faint crease at his crotch. A dancing mannequin. His sexuality was clipped to the back of his head: a single chrome electrode that activated the pleasure center of the brain during the weekly legally controlled catharsis. But he was so skinny—hey, who knows, maybe he went to a black-market cerebrostim to interface with the pulser. Though minimonos were supposed to be into stringent law and order.

The neural transmitters jacked into NewHope's arms and legs and torso transmitted to pickups on the stage floor. The long, funereal wails pealing from hidden speakers were triggered by the muscular contractions of his arms and legs and torso. He wasn't bad, for a minimono, Rickenharp thought. You can make out the melody, the tune shaped by his dancing, and it had a shade more complexity than the M'n'Ms usually had ... The M'n'M crowd moved into their geometrical dance configurations, somewhere between disco dancing and square dance, Busby Berkley kaleidoscopings worked out according to formulas you were simply expected to know, if you had the nerve to participate. Try to dance freestyle in their interlocking choreography, and sheer social rejection, on the wings of body language, would hit you like an arctic wind.

Sometimes Rickenharp did an acid dance in the midst of the minimono configuration, just for the hell of it, just to revel in their rejection. But his band had made him stop that. Don't alienate the audience at our only gig, man. Probably our last fucking gig ...

The wiredancer rippled out bagpipelike riffs over the digitalized rhythm section. The walls came alive.

A good rock club—in 1965 or 1975 or 1985 or 1995 or 2012 or 2039 should be narrow, dark, close, claustrophobic. The walls should be either starkly monochrome—all black or mirrored, say—or deliberately garish. Camp, layered with whatever was the contemporary avant-garde or gaudy graffiti.

The Semiconductor showed both sides. It started out butch, its walls glassy black; during the concert it went in gaudy drag as the sound-sensitive walls reacted to the music with color streaking, wavelengthing in oscilloscope patterns, shades of bluewhite for high end, red and purple for bass and percussion, reacting vividly, hypnotically to each note. The minimonos disliked reactive walls. They called it kitschy.

The dance spazzed the stage, and Rickenharp grudgingly watched, trying to be fair to it. Thinking, It's another kind of rock 'n' roll, is all. Like a Christian watching a Buddhist ceremony, telling himself, ``Oh, well, it's all manifestations of the One God in the end.'' Rickenharp thinking: But real rock is better. Real rock is coming back, he'd tell almost anyone who'd listen. Almost no one would.

A chaotichick came in, and he watched her, feeling less alone. Chaotics were much closer to real rockers. She was a skinhead, with the sides of her head painted. The Gridfriend insignia was tattooed on her right shoulder. She wore a skirt made of at least two hundred rags of synthetic material sewn to her leather belt—a sort of grass skirt of bright rags. The nipples of her bare breasts were pierced with thin screws. The minimonos looked at her in disgust; they were prudish, and calling attention to one's breasts was decidedly gauche with the M'n'Ms. She smiled sunnily back at them. Her handsome Semitic features were slashed randomly with paint. Her makeup looked like a spinpainting. Her teeth were filed.

Rickenharp swallowed hard, looking at her. Damn. She was his type.

Only ... she wore a blue-mesc sniffer. The sniffer's inverted question mark ran from its hook at her right ear to just under her right nostril. Now and then she tilted her head to it, and sniffed a little blue powder.

Rickenharp had to look away. Silently cursing.

He'd just written a song called ``Stay Clean.''

Blue mesc. Or syncoke. Or heroin. Or amphetamorphine. Or XTZ. But mostly he went for blue mesc. And blue mesc was addictive.

Blue mesc, also called boss blue. It offered some of the effects of mescaline and cocaine together, framed in the gelatinous sweetness of methaqualone. Only ... stop taking it after a period of steady use and the world drained of meaning for you. There was no actual withdrawal sickness. There was only a deeply resonant depression, a sense of worthlessness that seemed to settle like dust and maggot dung into each individual cell of the user's body.

Some people called blue mesc ``the suicide ticket.'' It could make you feel like a coal miner when the mineshaft caved in, only you were buried in yourself.

Rickenharp had squandered the money from his only major microdisc hit on boss blue and synthmorph. He'd just barely made it clean. And lately, at least before the band squabbles, he'd begun feeling like life was worth living again.

Watching the girl with the sniffer walk past, watching her use, Rickenharp felt stricken, lost, as if he'd seen something to remind him of a lost lover. An ex-user's syndrome. Pain from guilt of having jilted your drug.

And he could imagine the sweet burn of the stuff in his nostrils, the backward-sweet pharmaceutical taste of it in the back of his palate; the rush; the autoerotic feedback loop of blue mesc. Imagining it, he had a shadow of the sensation, a tantalizing ghost of the rush. In memory he could taste it, smell it, feel it ... Seeing her use brought back a hundred iridescent memories and with them came an almost irrepressible longing. (While some small voice in the back of his head tried to get his attention, tried to warn him, Hey, remember the shit makes you want to kill yourself when you run out; remember it makes you stupidly overconfident and boorish; remember it eats your internal organs ... a small, dwindling voice ... ) The girl was looking at him. There was a flicker of invitation in her eyes.

He wavered.

The small voice got louder.

Rickenharp, if you go to her, go with her, you'll end up using.

He turned away with an anguished internal wrenching. Stumbled through the wash of sounds and lights and monochrome people to the dressing room; to guitar and earphones and the safer sonic world.

``You gave him to me,'' Steinfeld said, leaning close to Purchase so he could be heard over the noise of the bar. ``And I give him back. And I think we'll both keep him.''

Purchase smiled and nodded. ``Stisky's a find. A piece of luck.''

Purchase was a big, sloppy-bodied man, his hair thin and his face wide. You could hear him breathe, even when he was at rest. But he laughed easily, and he didn't miss much. The two men liked one another, though they were NR for different reasons. Steinfeld had shaped the NR in the image of his own idealism. It was an extension of his convictions—some would say, his almost perverse obsession. Purchase worked for Witcher, Steinfeld's chief source of funding. But no, Steinfeld reflected, as they slid into a booth in the Freezone cocktail bar, Purchase worked for himself. That should have made him suspect. Only, it didn't. Steinfeld trusted him more than he trusted some of the NR's political zealots.

``Any problem with the blockades?'' Purchase asked, toying with his gold choker.

Steinfeld's brow furrowed. ``Yes and no. I got through—but it was close this time. No one actually fired on us. But they would have if they'd picked up on us sooner. Sometimes I feel like asking Witcher's pilots not to tell me if we're tracked. I'd rather not know if I'm about to be shot out of the air ... ''

``You bring anyone else through?''

``A few people. We can't get more than a handful out at any one time ... and it's a risk with just the handful. I won't be taking many more of these trips ... '' He grimaced and changed the subject. ``That's a silk suit, isn't it? It's a little hard to tell in these lights, but I think it's blue?''

``It is. Dark blue silk.'' Purchase signaled for a drink. When the puffy-eyed waitress arrived, yawning, rubbing her temples, he said, ``I want something big and glittery in an enormous glass. You choose. Something sweet. Sweet as whoever it was kept you up so late last night.''

She almost smiled. ``Something with a plastic mermaid? A little paper umbrella?''

``Both the umbrella and the mermaid are absolute necessities.''

``I'll have a scotch, please,'' Steinfeld said. ``On the rocks.'' They watched her walk away. She was wearing a gown that picked up wifi signals at random as the signals passed through the room and reproduced Web imagery down the svelte length of her. Collaged faces, mostly fashion models and breakfast-cereal-kids, rippled across her ass and the back of her thighs.

The bar was at the edge of a disco. Minimono droned and thudded on the dance floor. Lights whirled like UFOs landing in an old movie Steinfeld had seen as a boy.

They had to lean over the transparent plastic table to talk, but they'd picked the booth to discourage bugging.

The lights tinted Purchase's face, changing his color as if some expressionist painter were experimenting with his portrait. He was pinkish red dappled with blue when he asked, ``How'd Stisky take to training?''

``A fish to water. The more rigorous the better. Well, he was a priest, once, after all ... Does he have a name yet?''

``John Swenson. The cover had a good foundation: there was a John Swenson born the same year as Stisky. Died five years later. Looked a lot like Stisky did as a kid. His death went unregistered in his hometown—died in a boating accident with his parents on vacation, they all drowned. Death registered in Florida but never entered in computer records. We've put all the rest together. Worked up a set of false memories for mem-plantation ... I think we've got some likelies, to take the implants ... ''

The expression on Steinfeld's face made Purchase say, ``You've got qualms about mem-plants?''

``This business of toying with people's memories—I don't care which side it is doing it—no, I don't like it. It's—'' He shook his head.

``Too close to interfering with the soul?''

Steinfeld said, ``I am not sure I believe in the soul. But yes, it's too close—to interfering with the soul.''

``We're up against it. Outnumbered. We've got to use all the tools at hand. If it's any comfort, we don't implant our own people. We should, but we don't. Just the enemy.''

Steinfeld shrugged. ``So be it. How high can you place him?''

Purchase fidgeted, looking unsure of himself. The waitress came back with their drinks. Purchase's was some sort of phantasmagorical daiquiri. A cartoon character flew across the waitress's stomach (What was his name? Something the Gremlin) to be replaced instantly by a hydrogen-cell vehicle crashing head-on into another, both bursting into flames. ``Cars are crashing in your stomach,'' Purchase told her.

``That explains my heartburn,'' she said, snicking Purchase's Worldtalk expense account credit card through the credunit on her hip. She gave the card back and walked away, Marilyn Monroe waving at them from the small of her back. Monroe's breasts superimposed for one delectable instant on the waitress's buttocks.

``People are wearing the Grid now,'' Steinfeld said.

``Just pray to Gridfriend they don't make wallpaper like that. Come to think of it, they probably are making it ... ''

Steinfeld smiled; the smile was barely visible through his beard. He wore a cheap black-and-white flatsuit, a bit tacky here, but passable.

Purchase said, ``I think ... think, mind you ... I can place Stisky—or Swenson, now, if you like—I think I can place Swenson in the Second Circle itself, after a short, ah, probationary period. Within a few weeks.''

Steinfeld looked sharply at Purchase. ``It took us three years to get Devereaux into the Second Circle. And that was fast advancement. He was in the lower echelons, as you call it, two years and then—''

``I know all that. But ... '' Purchase leaned nearer. ``But I've gotten to know Crandall's sister. We modeled her transactional script patterns. She has an affair every two years—almost to the day! Usually something torrid. Then Rick gets rid of them or she loses interest. We believe that her next one will be somewhat more serious. And it's due in a week—and that's when I'll introduce her to Swenson. She has a growing need for long-term emotional security. We studied her preference profile: Swenson would be her archetype, which is why we picked him. She meets Swenson, Swenson romances her—and we both agree he's got the talent for that—and she will bring him with her. And of course he's done very well in their lower echelons.''

``You're very certain of that.''

``I'd swear to it: bet a cool million on it.''

Steinfeld nodded. ``A million. Well—you've just invoked the deity that means the most to you. I'm impressed. All right. If it gets that far ... Devereaux might ... ''

Purchase shook his head. ``You don't really think Devereaux is going to come through, do you? Do you know who's the new SA Security chief? Old Sackville-West. Devereaux's the nervous type. Old Sacks will smell that.''

``Then he may smell our Swenson.''

``I think not. Swenson has the talent. And he'll have Ellen Mae's support. Trust me.''

They took a moment to work on their drinks. Steinfeld looked down, through the table, and through the floor. The floor was transparent; the disco jutted from the side of a highmall rising two hundred stories over the main Freezone helicopter port; far below—and directly below—radio-controlled copters rose and landed, dragonfly bright in the ocean-burnished sunlight.

Steinfeld shivered with vertigo. He shifted his gaze to the expanse of cobalt sea. ``Funny how from up here, the waves look regular, perfect and orderly. Down close they're all chaotic.''

Purchase looked up from his drink. Without quite taking the straw from his mouth he said, ``That supposed to be a parable of some kind?''

``No. But I guess it could be: from up here we're taking too much for granted.'' The waitress walked by, her dress flashing with forty TV channels at once. It made Steinfeld's skin crawl. ``How much of that programming''—he nodded toward the televisioned dress—``is Worldtalk's doing?''

``Not a great deal in slices of time. But lots of it in small, regular pulses ... Worldtalk'll be active on the SA account this week. Naturally Crandall wants me to shepherd it. And I'll have to do a good job of it. You know that.''

``For a while. But try not to promote them brilliantly. Okay?''

Worldtalk. The globe-straddling agency for public relations and advertising. Purchase was a Chinese-boxes man, working from the inside box out: his own man and yet Witcher's man; Witcher's man and yet Steinfeld's; Steinfeld's and yet the SA's. The SA's and yet Worldtalk's. Steinfeld believed the sequence moved in that order of importance. He had to believe it, because he needed Purchase. There were too few like him.

There was Devereaux, of course. Who just might be a waste.

``You can pick up ... Swenson, in an hour, at ... '' He took a plastic-tagged hotel keycard from his pocket and gave it to Purchase, who pocketed the key casually but quickly. ``He'll be there. Report on placing him to Ben-Simon at the Israeli Embassy. He's still with me. And to Witcher. Let us know if he gets close to them ... to her.''

``You sound as if you doubt Devereaux's going to come through, yourself.'' (The light shifting, Purchase's face green, then blue.) ``Just thinking in contingencies.''

``If Devereaux doesn't come through, we'll have to roll up his backup team, fast.''

``They'll have to cope with it themselves. I'm leaving in a few hours. They'll do fine. They're ... basic. But good.''

He looked out over the sea, thinking, If Devereaux doesn't come through ...

There were eight people in the room, and, each in their own way, they were all killers. No: seven were killers. One was a man who had come here hoping to become a killer; to kill one of the other seven people within the hour, in this undersea conference room beneath Freezone.

Freezone's enormous octagonal raft was pocketed with air and layers of flotation synthetics. Most of the buildings in the exclusive Freezone Central complex—walled off from the rest of Freezone to guarantee safety for its inhabitants and visitors—extended like enormous undersea stalactites beneath the ``flotation support structure'' for greater stability and less vulnerability to winds.

In one of those buildings, the inverted wedge of the Fuji Hilton, Richard Crandall and Ellen Mae Crandall presided over the meeting ...

The room was dimmed for the briefing screen. Five men and the woman sat at the table. There were two Security men standing behind Crandall at the head of the table. All were bathed in the sickly electronic-blue light from the screen that filled the upper half of the wall to the right of the door.

On three sides, it was a standard convention meeting room, a forty-by-fifty-foot ``planning center.'' The walls were the usual imitation woodgrain, the table matching. The chairs were confoam swivels; soft track lighting overhead was muted now. A bank of remote controls for the screen and room service was inset at one end of the oblong table.

Behind Crandall, the fourth wall was a window of thick plate glass, looking out on the underside of the floating city. The dull blue vista was lit brokenly by flat white rectangles of light staggered along the down-juts of other buildings; the buildings looked like reflections in a pond, upside down. But look closer and you could see men in them: right-side-up men in upside-down buildings. Now and then some glossy, striped, gape-mouthed thing would swim up near the windows, attracted by the lights; jellyfish billowed up, pumping like disembodied heart valves.

Devereaux sat at the table, looking out into the undersea, carefully keeping up his mask of absentmindedness, carefully thinking only about jellyfish. Not permitting himself to think about the act he was about to perform. It was not yet time to think about it.

Best not to think about it at all. Best to let it happen the way an alarm clock goes off.

A voice droned through the air-conditioned room from the bluewhite screen, accompanying the charts and figures appearing and disappearing there. ``Alliance registration in Brussels,'' the voice said, ``rose forty-three percent in the last sixty days. Alliance coordinator for Brussels credits this abrupt rise to the anti-Russian/anti-American information campaign. Antipathy to the ‘foreign war-makers' has drawn increasing numbers of Belgians into the Alliance, their registration always hinging on the Alliance's guarantee of eventual ejection of all foreigners. Coordinator Casterman expects very little abreaction from Belgian inductees during the Final Phase, anticipating that thorough Camp indoctrination will obviate any significant resentment when, in the words of the resistance leader Chartres, ‘the country is taken over by foreigners who promised to protect us from foreigners.' ''

Crandall stabbed a button. The narrative froze. He dialed the lights up and turned to Sackville-West, head of Internal Security, asking sharply, ``Who wrote that report?'' Crandall's faint Southern accent was almost undetectable. He was slender, almost gaunt, his black eyes a little sunken. His wide, flexible mouth could flare into a smile like a dove flushed from cover; could just as easily clamp down into a frown firm enough to use as a metal-shop vise. His hair was receding, and he compensated with long sideburns. A strong nose, craggy cheekbones—a face almost like a beardless Lincoln. He wore a suit and tie, brown leather and cream-colored silk. His sister, to his left, looked unpleasantly like him, to Devereaux's eye. Even without sideburns. Her face was a little softer, her lips redder—but like him. Maybe it was the expression.

``That report ... '' Sackville-West muttered, clearing his throat several times as he looked through his pocket filer. ``Ah, that report was written by ... ah ... '' Sackville-West was a pinkfaced Britisher with three chins and a comma of hair on his forehead. He was always sweating, even in an air-conditioned room. ``Swenson wrote the report,'' Sackville-West said at last, looking up from his console.

Casually, Devereaux raised a hand to his cheek and pressed the stud under the skin, just below his right cheekbone. His Mossad-issue right eye increased its impressions-per-second ratio by five hundred percent. He rubbed at his left eye, as if it were tired, closing it so only the right took impressions. Details normally lost to the human eye showed up in his prosthetic perception: a flicker of fear in Sackville-West, the expression flashing through the face so rapidly Devereaux would have been unable to see it without the implant.

All it told Devereaux was that Sackville-West was physically intimidated by Crandall. Nothing new there.

Sackville-West was repeating, ``John Swenson. SA Number 34428, inducted February of—''

``I don't like him quoting Chartres,'' Crandall interrupted.

``I can see that, Rick.'' Sackville-West responded, nodding effusively. Calling him Rick but in a tone that meant Yes, sir. ``But—I think it was just his sense of humor. It says here his sense of irony is strong. I'd evaluate the remark as a kind of smug solidarity with us: mocking the resistance.''

``I'm not so sure of that,'' Crandall said. ``Have him observed.''

``Quite right, Rick—I've already punched out an order to that effect.'' He tapped in something more on the pocket filer; its tiny keys were almost too small for his pudgy fingers.

Most of Devereaux's attention went to watching Ellen Mae with his fast-action eye. And something flickered across her face—zip, and it was gone. But Devereaux had seen: anxiety. Concern. For— For Swenson. So they'd already managed to interest her in Swenson ...

She glanced at Devereaux, and away. He thought: Mustn't draw attention to myself, rubbing my eye so much.

He opened the left eye and, as if scratching his cheek, switched off the overdrive in the prosthetic one.

Devereaux turned to watch Crandall, who had gone on to comment on the Belgian brief. `` ... I do think, however, that the new disinfo campaign is working very well, and we should continue through the same means—making sure it's as anti-American as anti-Russian. Our friends at NATO''—he smiled, and that was a cue for a companionable chuckle around the table, which dutifully came—``would hardly approve of such indiscrimination, if it came out.''

Devereaux smiled and nodded as he was expected to. He glanced at the Security men behind Crandall, essentially mere bodyguards ... and wondered if they were bodyguards. The damn eye-blanking helmets made them maddeningly inscrutable. He thought he felt them watching him.

Don't get nervous, he told himself. Don't think about the job at all. You'll do it when the time comes.

But Crandall was eminently paranoid; his Security knew it, and knew they were expected to be suspicious of everyone. Even now they stood behind him, between him and the window, because Crandall had instantly mistrusted the window when he'd come into the room.

``I just don't like it, my friends,'' he'd say. ``Anybody could frogman right up to the window, fire a subaqueous rocket through it ... ''

But they were behind schedule, and he'd grudgingly agreed to the room with the glass wall.

Crandall switched the report back on. It droned out troop movement figures, confirming the Russian pullback, reporting the taking of crucial sectors. When it began on Paris, Devereaux felt the pressure rising in him again.

He wondered if the detection-shielding on the gun in his briefcase had been adequate. But they'd have arrested him by now if it hadn't been. He wondered, too, if he could shoot the Security men before Crandall, and still be assured of getting Crandall himself. No. He'd have to get Crandall first. Which meant Security would get him. And Devereaux would die.

He remembered a few lines from Rimbaud.

Monâme eternelle,

Observe ton voeu,

Malgré la nuit seule ...

My eternal soul,

Redeem your promise,

In spite of the night alone ...

It was a silent prayer of Devereaux's own, as Crandall stood to offer a prayer for the meeting.

The olive-drab simuleather briefcase was on the tabletop beside Devereaux's notetaker. Devereaux laid his hand on the table beside the briefcase.

It was almost time.

Crandall's prayers took about three minutes. Every head in the room was lowered for the prayer, even the guards. Even Devereaux's. But his finger closed on the latch at the corner of the briefcase. The latch that would release the gun into his hand.

Another thirty seconds, he told himself, as Crandall intoned in polished rhythms, ``We're asking your help, Lord, in this your battle, in this struggle to free the earth from the bonds of inbred social sin; from the sickness of miscegeny ... ''

Devereaux had twenty seconds. In those twenty seconds he found himself wondering, remembering— Stop thinking about it, he told himself. Steinfeld told you, again and again, When the moment comes, act, don't think. Act, don't think.

But he saw himself at the New Right meeting, in Nice, raising objections, eliciting odd looks from the other members, realizing that he didn't belong there anymore. Saw himself approached afterward by one of Steinfeld's men. Bashung. Bashung had heard him voice a few cautious misgivings about the proposal to demand that the government oust all recent immigrants from France. Bashung had watched Devereaux through a perceptual prosthetic, identical to the one he now carried in place of his right eye. Had seen the flash of confusion and worry and sorrow and anger the others had missed.

It hadn't taken long to recruit him. They revealed a great many things about Crandall that were usually kept suppressed. They took Devereaux along when they went to record statements by two widows whose husbands had been assassinated by the SA. Bashung and Steinfeld had played video of Crandall's early meetings with his coordinators, at which he made a series of insane statements in tones so calm and measured as to make the hair stand up on the back of Devereaux's neck. Devereaux had entered the New Right chiefly out of his hatred for the Russians. But when you grasped what Crandall's full intentions were ...

Crandall made the Russians look like playful imps.

Crandall was the one long awaited, long feared. The one they'd all known would come again.

And Devereaux had been recruited and trained to throw himself deeply into bogus support for the SA, to join the SA's ranks, to ascend to this very room, where he was adviser on the French SA takeover.

Devereaux seemed to hear the words of the boy poet again, the debaucher Rimbaud. Mon âme eternelle ... My eternal soul ...

``And we thank you, Lord,'' Crandall was saying, ``for advancing our struggle. We ask that you take charge now of our eternal souls ... ''

Redeem your promise ...

``In the name of Jesus the Redeemer ... ''

In spite of the night alone ...

`` ... we stand against the armies of darkness. Praise God, and Amen.''

What was the last line of the stanza? There was another line Devereaux had forgotten. It was ...

Ah, yes. He remembered, now, as he pressed the switch that ejected the cool grip of the gun into his palm, as the others intoned Amen.

He spoke the last two lines aloud as he stood and turned and raised the gun to fire at Crandall. ``Malgré la nuit seule, et le jour en Feu.''

In spite of the night alone, and the day on fire.

He fired the gun, the Teflon-coated slugs ripping through Crandall's bulletproof vest, but the Security men were already firing back. They had been watching him. He tried to track the gun muzzle to Ellen Mae, but the automatic pistols in the hands of the big men had punched holes in him, and he felt a terrible hollowness beneath his feet, as if someone had pulled out the center of the Earth, left it without a core, and it cracked open and he fell into the emptiness and died with a sickening pang of knowing that he hadn't hit Crandall squarely, the bastard would live, the bastard would live ...

\fancybreak{* * *}

Rickenharp was listening to a collector's item Velvet Underground tape, from 1968. It was capped into his Earmite. The song was ``White Light/White Heat.'' The guitarists were doing things that would make Baron Frankenstein say, ``There are some things man was not meant to know.'' He screwed the Earmite a little deeper so that the vibrations would shiver the bone around his ear, give him chills, chills that lapped through him in harmony with the guitar chords. He'd picked a visorclip to go with the music: a muted documentary on expressionist painters. Listening to the Velvets and looking at Edvard Munch. Man!

And then Julio dug a finger into his shoulder.

``Happiness is fleeting,'' Rickenharp muttered, as he flipped the visorclip back. Some visors came with camera eye and fieldstim. The fieldstim you wore snugged to the skin, as if it were a sheer corset. The camera picked up an image of the street you were walking down and routed it to the fieldstim, which tickled your back in the pattern of whatever the camera saw. Some part of your mind assembled a rough image of the street out of that. Developed for blind people in the 1980s. Now used by viddy addicts who walked or drove the streets wearing visors, watching TV, reflexively navigating by using the fieldstim, their eyes blocked off by the screen but never quite bumping into anyone. But Rickenharp didn't use a fieldstim.

So he had to look at Julio with his own eyes. ``What do YOU want?''

``N'ten,'' Julio said. Julio the technicki bassist. They went on in ten minutes.

Mose, Ponce, Julio, Murch. Rhythm guitar backup vocals. Keyboards. Bass. Drums.

Rickenharp nodded and reached up to flip the visor back in place, but Ponce flicked the switch on the visor's headset. The visor image shrank like a landscape vanishing down a tunnel behind a train, and Rickenharp felt like his stomach was shrinking inside him at the same rate. He knew what was coming down. ``Okay,'' he said, turning to look at them. ``What?''

They were in the dressing room. The walls were black with graffiti. All rock club dressing rooms will always be black with graffiti; flayed with it, scourged with it. Like the flat declaration THE PARASITES RULE, the cheerful petulance of symbiosis THE SCREAMIN' GEEZERS GOT FUCKING BORED HERE, the oblique existentialism of THE ALKOLOID BROTHERS LOVE YOU ALL BUT THINK YOU WOULD BE BETTER OFF DEAD, and the enigmatic ones like SYNC 66 CLICKS NOW. It looked like the patterning of badly wrinkled wallpaper. It was in layers; it was a palimpsest. Hallucinatory stylization as if tracing the electron firings of the visual cortex.

The walls, in the few places they were visible under the graffiti, were a gray-painted pressboard. There was just enough room for Rickenharp's band, sitting around on broken-backed kitchen chairs and one desk chair with three legs. Crowded between the chairs were instruments in their cases. The edges of the cases were false leather peeling away. Half the snaps broken.

Rickenharp looked at the band, looked clockwise one face to the next, taking a poll from their expressions: Mose on his left, a bruised look to his eyes; his hair a triple-Mohawk, the center spine red, the outer two white and blue; a smoky crystal ring on his left index finger that matched—he knew it matched—his smoky crystal amber eyes. Rickenharp and Mose had been close. Each looked at the other a little accusingly. There was a lover's sulkiness between them, though they'd never been lovers. Mose was hurt because Rickenharp didn't want to make the transition: Rickenharp was putting his own taste in music before the survival of the band. Rickenharp was hurt because Mose wanted to go minimono wifi act, a betrayal of the spiritual ethos of the band; and because Mose was willing to sacrifice Rickenharp. Replace him with a wire dancer. They both knew it, though it had never been said. Most of what passed between them was semiotically transmitted with the studied indirection of the terminally cool.

Tonight, Mose looked like serious bad news. His head was tilted as if his neck were broken, his eyes lusterless.

Ponce had gone minimono, at least in his look, and they'd had a ferocious fight over that. Ponce was slender—like everyone in the band—and fox-faced, and now he was sprayed battleship gray from head to toe, including hair and skin. In the smoky atmosphere of the clubs he sometimes vanished completely.

He wore silver contact lenses. Flat-out glum, he stared at a ten-slivered funhouse reflection in his mirrored fingernails.

Julio, yeah, he liked to give Rickenharp shit, and he wanted the change-up. Sure, he was loyal to Rickenharp, up to a point. But he was also a conformist. He'd argue for Rickenharp maybe, but he'd go with the consensus. Julio had lush curly black Puerto Rican hair piled prowlike over his head. He had a woman's profile and a woman's long-lashed eyes. He had a silver-stud earring, and wore classic retro-rock black leather like Rickenharp. He twisted the skull-ring on his thumb, returning a scowl for its grin, staring at it as if deeply worried that one of its ruby-red glass eyes was about to come out.

Murch was a thick slug of a guy with a glass crew cut. He was a mediocre drummer, but he was a drummer, with a trap set and everything, a species of musician almost extinct. ``Murch's rare as a dodo,'' Rickenharp said once, ``and that's not all he's got in common with a dodo.'' Murch wore horn-rimmed dark glasses, and he was holding a bottle of Jack Daniels on his knee. The Jack Daniels was a part of his outfit. It went with his cowboy boots, or so he thought.

Murch was looking at Rickenharp in open contempt. He didn't have the brains to dissemble.

``Fuck you, Murch,'' Rickenharp said.

``Whuh? I didn't say nothing.''

``You don't have to. I can smell your thoughts. Enough to gag a faggot maggot.'' Rickenharp stood and looked at the others. ``I know what's on your mind. Give me this: one last good gig. After that you can have it how you want.''

Tension lifted its wings and flew away.

Another bird settled over the room. Rickenharp saw it in his mind's eye: a thunderbird. Half made of an Indian teepee painting of a thunderbird, and half of chrome T-Bird car parts. When it spread its wings the pinfeathers glistened like polished bumpers. There were two headlights on its chest, and when the band picked up their instruments to go out to the stage, the headlights switched on.

Rickenharp carried his Stratocaster in its black case. The case was bandaged with duct tape and peeling with faded stickers. But the Strat was spotless. It was transparent. Its lines curved hot like a sports car.

They walked down a white plastibrick corridor toward the stage. The corridor narrowed after the first turn, so they had to walk sideways, holding the instruments out in front of them. Space was precious on Freezone.

The stagehand saw Murch go out first, and he signaled the DJ, who cut the canned music and announced the band through the PA. Old-fashioned, like Rickenharp requested: ``Please welcome ... Rickenharp.''

There was no answering roar from the crowd. There were a few catcalls and a smattering of applause.

Good, you bitch, fight me, Rickenharp thought, waiting for the band to take up their positions. He'd go on stage last, after they'd set up the spot for him. Always.

Rickenharp squinted from the wings to see past the glare of lights into the dark snakepit of the audience. Only about half minimono now. That was good, that gave him a chance to put this one over.

The band took its place, pressed their automatic tuners, fiddled with dials.

Rickenharp was pleasantly surprised to see that the stage was lit with soft red floods, which is what he'd requested. Maybe the lighting director was one of his fans. Maybe the band wouldn't fuck this one up. Maybe everything would fall into place. Maybe the lock on the cage door would tumble into the right combination, the cage door would open, the T-Bird would fly.

He could hear some of the audience whispering about Murch. Most of them had never seen a live drummer before, except for salsa. Rickenharp caught a scrap of technicki: ``Whuzziemackzut?'' What's he making with that, meaning: What are those things he's adjusting? The drums.

Rickenharp took the Strat out of its case and strapped it on. He adjusted the strap, pressed the tuner. When he walked onto the stage, the amp's reception field would trigger, transmit the Strat's signals to the stack of Marshalls behind the drummer. A shame, in a way, about miniaturization of electronics: the amps were small, though just as loud as twentieth century amps and speakers. But they looked less imposing. The audience was muttering about the Marshalls, too. Most of them hadn't seen old-fashioned amps. ``What's those for?'' Murch looked at Rickenharp. Rickenharp nodded.

Murch thudded 4/4, alone for a moment. Then the bass took it up, laid down a sonic strata that was kind of off-center strutting. And the keyboards laid down sheets of infinity.

Now he could walk on stage. It was like there'd been an abyss between Rickenharp and the stage, and the bass and drum and keyboards working together made a bridge to cross the abyss. He walked over the bridge and into the warmth of the floods. He could feel the heat of the lights on his skin. It was like stepping from an air-conditioned room into the tropics. The music suffered deliciously in a tropical lushness. The pure white spotlight caught and held him, focusing on his guitar, as per his directions, and he thought, Good, the lighting guy really is with me.

He felt as if he could feel what the guitar felt. The guitar ached to be touched.

Claire sat on the couch in her apartment, half the size of her father's, and waited, with quiet dread, for the InterColony news show to come on.

The main room of her apartment was now dialed to living room; the furniture changed shape for bedroom when she told it to. The walls around the screen were translucent, impregnated with a rain forest's greens and scarlets. The image shifted to a rain squall, and the enormous tropical leaves bounced in the rainwater, ran with crystal beads. A hidden aerator issued the scents of a jungle in the rain. She could almost feel the rainwater.

The all-media screen—a glaring anomaly in the projection of the jungle—showed a documentary about the European Congress of the New Right. The sound was turned off, but there were subtitles as the French Front National leader made a series of—she thought—wildly inflammatory statements with the calm of a TV chef explaining a recipe. The intense, pallid little man was saying `` ... the inevitability of conflict between cultures with fundamentally different roots can no longer be glossed over. The good intentions of those trying to reconcile Islamic Fundamentalists with Europeans only serve to prolong the pain of social redress. For, I assure you, social redress is necessary. Immigrants from cultures foreign to our own have muddied our cultural waters. It is foolish to assume we will ever occupy the same territory harmoniously. It is naive, unrealistic. This naivete costs us time, money, yes, human lives. The truth must be faced: some races will always be unable to reconcile! The answer is simple: expulsion. It is out of our hands as to whether we are forced to resort to violence in the execution of our solution to the immigrant problem. Cultural vitality and racial purity are synonymous—''

She turned away, sickened. She sensed some obscure connection between the European situation and the Colony.

She made herself a cocktail spiked with an antidepressant neurohormonal transmitter and sipped it, quickly feeling better—artificially—as she waited for the news.

There it was. She dialed up the sound.

`` ... Technicki radical leaders Molt and Bonham agreed today in principle to a meeting with Director Rimpler but said they could not schedule the meeting without a close look at security precautions for both sides.''

She shook her head sadly, muttering, ``They think we're going to arrest them at a meeting. The depth of mistrust ... '' She took another sip of the medicinal-tasting cocktail, thinking, Everything's worse than I thought it was ...

The news ran highlights from the last talk between Technicki Union leaders and Admin. There was the flatsuiter, Barkin, speaking in his nasal tone about `` ... a conflict of interest in the Colony's housing directors ... Admin is being puppeted by UNIC to run things according to UNIC's priorities, and its priority is profit, always. Admin maintains that the technicki housing project for the Open would be much costlier than was originally believed, and that's why it was put off—but they haven't put off developing Admin housing. We have completely lost sight of the fact that the UN's matching-funds program for the Colony was offered because Professor Rimpler promised a home to Earth's disadvantaged—the disadvantaged get here and find themselves in overcrowded, badly filtrated dorms—a drearier home than the one they left behind ... ''

Claire nodded, ever so slightly. There was something to it.

And since then the Russians had blockaded the Colony, cutting off shipments of food and other necessities from Earth. They weren't starving yet, but the warehoused supplies were running low. The technickis were reacting to the increased rationing. Admin was rationed, too, but the technickis were skeptical—and maybe they were right, Claire thought. Were Praeger and the UNIC people really eating less?

InterColony was showing a clip of the Colony riots now. One of the Radics, a guy named Molt, with a pipe wrench in his hand leading a charge down Corridor D. Forty technicki men and women followed behind him—including preteen boys carrying what looked like Molotov cocktails. The faces in the crowd looked almost delirious with release. The image was shot from above and to the side; she guessed it was one of the surveillance cameras. Molt was shouting something through bared teeth. He saw the camera, mounted near the ceiling, and turned toward it, ran at the viewer, threw the wrench. The wrench struck the camera lens— The image went black.

Without consciously knowing it, Rickenharp was moving to the music. Not too much. Not in the pushy, look-at-me way that some performers had. The way they had of trying to force enthusiasm from the audience, every move looking artificial.

No, Rickenharp was a natural. The music flowed through him physically, unimpeded by anxieties or ego knots. His ego was there: it was the fuel for his personal Olympics torch. But it was also as immaculate as a pontiff's robes.

The band sensed it: Rickenharp was in rare form tonight. Maybe it was because he was freed. The tensions were gone because he knew this was the end of the line: the band had received its death sentence: Now, Rickenharp was as unafraid as a true suicide. He had the courage of despair.

The band sensed it and let it happen. The chemistry was there, this time, when Ponce and Mose came into the verse section. Mose with a sinuous riffing picked low, almost on the chrome-plate that clamped the strings; Ponce with a magnificently redundant theme washed through the brass mode of the synthesizer. The whole band felt the chemistry like a pleasing electric shock, the pleasurable shock of individual egos becoming a group ego.

The audience was listening, but they were also resisting. They didn't want to like it. Still, the place was crowded—because of the club's rep, not because of Rickenharp—and all those packed-in bodies make a kind of sensitive atmospheric exo-skeleton, and he knew that made them vulnerable. He knew what to touch.

Feeling the Good Thing begin to happen, Rickenharp looked confident but not quite arrogant—he was too arrogant to show arrogance.

The audience looked at Rickenharp as a man will look at a smug adversary just before a hand-to-hand fight and wonder, ``Why's he so smug, what does he know?''

He knew about timing. And he knew there were feelings even the most aloof among them couldn't control, once those feelings were released: and he knew how to release them.

Rickenharp hit a chord. He let it shimmer through the room and he looked out at them. He made eye contact.

He liked seeing the defiant stares, because that was going to make his victory more complete.

Because he knew. He'd played five gigs with the band in the last two weeks, and for all five gigs the atmosphere had been strained, the electricity hadn't been there; like a Jacob's ladder where the two poles aren't properly lined up for the sparks to jump.

And like sexual energy, it had built up in them, dammed behind their private resentments; and now it was pouring through the dam, and the band shook with the release of it as Rickenharp thundered into his progression and began to sing ...

Strumming over the vocals, he sang,

You want easy overnight action

want it casually

A neat little chain reaction

and a little sympathy

You say it's just consolation

In the end it's a compensation

for insecurity

That way there's no surprises

That way no one gets hurt

No moral question tries us

No blood on satin shirts

But for me, yeah for me

PAIN IS EVERYTHING!

Pain is all there is

Babe take some of mine

or lick some of his

PAIN IS EVERYTHING!

Pain is all there is Babe take some of mine ...

From ``An Interview With Rickenharp: The Boy Methuselah,'' in Guitar Player Magazine, May 2037: GPM: You keep talking about group dynamics, but I have a feeling you don't mean dynamics in the usual musical sense.

RICKENHARP: The right way to create a band is for the members to simply find one another, the way lovers do. In bars or wherever. The members of the band are like five chemicals that come together with a specific chemical reaction. If the chemistry is right the audience becomes involved in this, this kind of—well, a social chemical reaction.

GPM: Could it be that all this is just your psychology? I mean, your emotional need for a really organically whole group?

RICKENHARP (after a long pause): It's true I need something like that. I need to belong. I mean—okay, I'm a ``nonconformist,'' but still, on some level I got a need to belong. Maybe rock bands are a surrogate family—the family unit is shot to hell, so ... the band is family for people. I'd do anything to keep it together. I need these guys. I'd be like a kid whose mom and dad and brothers and sisters were killed if I lost this band.

And he sang, PAIN IS EVERYTHING!

Pain is all there is

Babe take some of mine

Suck some of his

Yeah, said PAIN IS EVERYTHING—

Singing it insolently, half shouting, half warbling at the end of each note, with that fuck-you tone, performing that magic act: shouting a melody. He could see doors opening in their faces, even the minimonos, even the neutrals, all the flares, the rebs, the chaotics, the preps, the retros. Forgetting their subcultural classifications in the unification of the music. He was basted in sweat under the lights, he was squeezing sounds with his fingers and it was as if he could feel the sounds taking shape in his hands the way a sculptor feels clay under his fingers, and it was like there was no gap between his hearing the sound in his head and its coming out of the speakers. His brain, his body, his fingers had closed the gap, was one supercooled circuit breaker fused shut.

Some part of him was looking through the crowd for the chaotichick he'd spotted earlier. He was faintly disappointed when he didn't see her. He told himself, You ought to be happy, you had a narrow escape, she would've got you back into boss blue.

But when he saw her press to the front and nod at him ever so slightly in that smug insider's way, he was simply glad, and he wondered what his subconscious was planning for him ... All those thoughts were flickers. Most of the time his conscious mind was completely focused on the sound, and the business of acting out the sound for the audience. He was playing out of sorrow, the sorrow of loss. His family was going to die, and he played tunes that touched the chord of loss, in everyone ...

And the band was supernaturally tight. The gestalt was there, uniting them, and he thought: The band feels good, but it's not going to help when the gig's over.

It was like a divorced couple having a good time in bed but knowing that wouldn't make the marriage right again; the good time was a function of having given up.

But in the meantime there were fireworks.

By the last tune in the set the electricity was so thick in the club that—as Mose had said once, with a rocker's melodrama—``If you could cut it, it would bleed.'' The dope and smashweed and tobacco smoke moiling the air seemed to conspire with the stage lights to create an atmosphere of magical apartness. With each song-keyed shift in the light, red to blue to white to sulfurous yellow, a corresponding emotional wavelength rippled through the room. The energy built, and Rickenharp discharged it, his Strat the lightning rod.

And then the set ended.

Rickenharp bashed out the last five notes alone, nailing a climax onto the air. Then he walked offstage, hardly hearing the roar from the crowd. He found himself half running down the white, grimy plastibrick corridor, and then he was in the dressing room and didn't remember coming there. The graffiti seemed to writhe on the walls as if he'd taken a psychedelic. Everything felt more real than usual. His ears were ringing like Quasimodo's belfry.

He heard footsteps and turned, working up what he was going to say to the band. But it was the chaotichick and someone else, and then a third dude coming in after the someone else.

The someone else was a skinny guy with brown hair that was naturally messy, not messy as part of one of the cultural subcurrents. His mouth hung a little ajar, and one of his incisors was decayed black. His nose was windburned and the back of his bony hands were gnarled with veins. The third dude was Japanese; small, brown-eyed, nondescript, his expression was mild, just a shade more friendly than neutral. The skinny Caucasian guy wore an army jacket sans insignia, shiny jeans, and rotting tennis shoes. His hands were nervous, like there was something he was used to holding in them that wasn't there now. An instrument? Maybe.

The Japanese guy wore a Japanese Action Suit—surprise, surprise—sky-blue and neat as a pin. There was a lump on his hip—something he could reach by putting his right arm across his body and through the open zipper down the front of the suit—and Rickenharp was pretty sure it was a gun.

There was one thing all three of them had in common: they looked half-starved.

Rickenharp shivered—his gloss of sweat cooling on him, but he forced himself to say, ``Whusappnin'?'' It was wooden in his mouth. He was looking past them, waiting for the band.

``Band's in the wings,'' the chaotichick said. ``The bass player said to tell you ... well it was Telm zassouter.''

Rickenharp had to smile at her mock of Julio's technicki. Tell him, get his ass out here.

Then some of the druggy feeling washed away and he heard the shouts from the audience and he realized they wanted an encore.

``Jeez, an encore,'' he said without thinking. ``Been so fucking long.''

`` 'Ey mate,'' the skinny guy said, pronouncing mate like mite. Brit or Aussie. ``I saw you at Stone'enge five years ago when you 'ad yer second 'it.''

Rickenharp winced a little when the guy said your second hit, inadvertently underlining the fact that Rickenharp had had only two, and everyone knew he wasn't likely to have any more.

``I'm Carmen,'' the chaotichick said. ``This is Willow and Yukio.'' Yukio was standing sideways from the others, and something about the way he did it told Rickenharp he was watching down the corridor without seeming to.

Carmen saw Rickenharp looking at Yukio and said, ``Cops are coming down.''

``Why?'' Rickenharp asked. ``The club's licensed.''

``Not for you or the club. For us.''

He looked at her and said, ``Hey, I don't need to get busted ... '' He picked up his guitar and went into the hall. ``I got to do my encore before they lose interest.''

She followed along, into the hall and the echo of the encore stomps, and asked, ``Can we hang out in the dressing room for a while?''

``Yeah, but it ain't sacrosanct. You come back here, the cops can, too.'' They were in the wings now. Rickenharp signaled to Murch and the band started playing.

Standing beside him, she said, ``These aren't exactly cops. They probably don't know these kind of places, they'd look for us in the crowd, not the dressing room.''

``You're an optimist. I'll tell the bouncer to stand here, and if he sees anyone else start to come back, he'll tell 'em it's empty back here 'cause he just checked. Might work, might not.''

``Thanks.'' She went back to the dressing room. He spoke to the bouncer and went on stage. Feeling drained, the guitar heavy on him. But he picked up on the energy level in the room and it carried him through two encores. He left them wanting more—and, sticky with sweat, walked back to his dressing room.

They were still there. Carmen, Yukio, Willow.

``Is there a stage door?'' Yukio asked. ``Into alley?''

Rickenharp nodded. ``Wait in the hall; I'll come out and show you in a minute.''

Yukio nodded, and they went into the hall. The band came in, filed past Carmen and Yukio and the Brit without much noticing them, assuming they were backstage hangout flotsam, except Murch stared at Carmen's tits and swaggered a bit, twirling his drumsticks.

The band sat around laughing in the dressing room, slapping palms, lighting several kinds of smokes. They didn't offer Rickenharp any; they knew he didn't use it.

Rickenharp was packing his guitar away, when Mose said, ``You blew good.''

``You mean he gave you a good head?'' Murch said, and Julio snickered.

``Yeah,'' Ponce said, ``the guy gives a good head, good collarbone, good kidneys—''

``Good kidneys? Rick sucks on your kidneys? I think I'm gonna puke.''

And the usual puerile band banter because they were still high from a good set and putting off what they knew had to come, till Rickenharp said, ``What you want to talk about, Mose?''

Mose looked at him, and the others shut up.

``I know there's something on your mind,'' Rickenharp said softly. ``Something you haven't come out with yet.''

Mose said, ``Well, it's like—there's an agent Ponce knows, and this guy could take us on. He's a technicki agent and we'd be taking on a technicki circuit, but we'd work our way back from there, that's a good base. But this guy says we have to get a wire act in.''

``You guys been busy,'' Rickenharp said, shutting the guitar case.

Mose shrugged, ``Hey, we ain't been doing it behind your back; we didn't hear from the guy till yesterday night. We didn't have a real chance to talk to you till now, so, uh, we have the same personnel but we change costumes, change the band's name, write new tunes.''

``We'd lose it,'' Rickenharp said. Feeling caved-in. ``We'd lose the thing we got, doing that shit, because it's all superimposed.''

``Rickenharp—rock 'n' roll is not a fucking religion,'' Mose said.

``No, it's not a religion, it's a way of life. Now, here's my proposal: we write new songs in the same style as always. We did good tonight. It could be the beginning of a turnaround for us. We stay here, build on the base audience we established tonight.''

It was like throwing coins into the Grand Canyon. You couldn't even hear them hit bottom.

The band just looked back at him.

``Okay,'' Rickenharp said. ``Okay. We've been through this ten fucking times. Okay. That's all.'' He'd had an exit speech worked out for this moment, but it caught in his throat. He turned to Murch and said, ``You think they're going to keep you on, they tell you that? Bullshit! They'll be doing it without a drummer, man. You better learn to program computers, fast.'' Then he looked at Mose. ``Fuck you, Mose.'' He said it quietly.

He turned to Julio, who was looking at the far wall as if to decipher some particularly cryptic piece of graffiti. ``Julio, you can have my amp, I'll be traveling light.''

He turned and, carrying his guitar, walked out, leaving silence behind him.

He nodded at Yukio and his friends and they followed him to the stage door. At the door, Carmen said, ``Any chance you could help us find a little cover?''

Rickenharp needed company, bad. He nodded and said, ``Yeah, if you'll gimme a hit of that blue boss.''

She said, ``Sure.'' And they went into the alley.

FirStep. The Colony.

They all had to sit on the floor in Bitchie's because there weren't enough chairs. And Molt didn't want some of them on chairs and some on the floor. He wanted them all on the same level, where he could make easy eye contact.

There were twenty-two of them, eighteen men and four women, sitting in a circle on the mattress-covered floor. They were technickis just off the shift, or waiting to go on shift. The little room was fuggy with their smell; the air cycler in this part of the space colony was overtaxed. Had been wafting them green air lately anyway.

Wilson was going on, and on, his technickese slurring the sentences together so the monologue sounded like one big sentence, even translated: `` ... so the thing to do is to push for confrontation and then stop short of the actual confrontation and hold the confrontation out as a threat to make them deal with us because if we really push it and get down to fighting with them we're going to lose but even though they know they're going to win they still don't want the confrontation because it's going to endanger the Colony's air-tigh integ and it'll cost them a few men and it'll be expensive to repair everything that gets busted so I think we oughta ... ''

On and on. Molt had had enough. Wilson was short, thick-bodied, his blond hair had been flared into a corona-shape, was losing its shape like a dying dandelion; he had small, squinty blue eyes and a bulbous nose, little red mouth always going, probably got an in with the pharmacy techs for uppers. Wearing his greasy air-systems mechanic's overalls. Wilson wanted badly to be a radic leader. Molt and Bonham and Barkin were the acknowledged leaders, and Wilson muttered about it because Molt and Bonham weren't real technickis, spoke it with the accent of someone raised in Standard English. Molt's opinion was: Wilson was a grasping scheming runt. Molt had tried to keep Wilson out of this meeting, but the little prick had wormed in, nudged his friends about till they got Barkin to invite him. Little prick'd sell out his granny, Molt thought.

Molt broke in on him with, ``You had it a minute ago, Wilson, when you said they're scared of confrontation even though they'd win it. But you didn't carry it far enough.'' Though of course Molt said it in technicki. ``They're scared enough of it, what we got to do is, hit 'em harder, go for it—we got more power than they want to admit because they're scared we'll—''

Bonham broke in, ``There's something else we could do.''

Molt glared at Bonham. He didn't like being interrupted. And he was beginning to mistrust Bonham, too. All their chumminess had evaporated when they got to be rivals in the new party. Bonham, sitting beside Wilson, shifted, wincing, to keep his long legs from going to sleep, ran his fingers through his hair with that goddamn Che Guevara took on his face, going on, ``We could put up a barricade. Several. Take control of the heart of technickitown. That's a confrontation, but then again it isn't. I mean, a big barricade, maybe block off Corridor D completely.''

``They'd be on us before we got it half done,'' Barkin said. No flatsuit today, Molt noticed. The fucking hypocrite wearing mech's overalls, like he ever ventured into the repair hangars. He was squatting in a way that kept him from coming into direct contact with the greasy mattress, except for the bottom of his feet. Doesn't want to get dried cum and sweat from Bitchie's customers on his legs. Nice clean mech overalls, probably bought 'em at a costume shop. Christ.

Bonham shook his head. ``We stage a diversion, smoke bombs, whatever, up at the main crossover. We have forklifts, everything else we need, get most of it in place in no time.''

``You defend a barricade in Corridor D,'' Molt said, ``you got to use guns and you got to shoot to kill. Because they won't stand for a barricade, they'll rush it right off.''

Wilson shook his head like a terrier with earmites. ``No, hey, I think Bonham's right we'd just fire warning shots, capture some territory, they wouldn't want to move in right away because it'd force confrontation and they don't want that, leastways not yet ... ''

``I'm not so sure they don't want that,'' Barkin said, but no one was listening, except Molt. They were all jabbering at once now, hot on the idea of a barricade. And then a whistle came through the intercom, signal that the admin bulls were coming down the corridor, were going to raid Bitchie's, so the radics started moving out according to the drill, going out into the little service hallway and through the kitchen; the bulls would find the place empty ...

But how did they know about the meeting?

\fancybreak{* * *}

It hurt with every breath. But after a while that particular pain was part of the rhythm of being alive, was almost reassuring, and Smoke ceased to take much notice of it. The monotony, and the bedlam noises and smells of the place—that's what was hard to take. He tried to keep himself amused by guessing where he was, what was going on. But the body cast (and damn its itching!) kept him from looking around much. And there was no one who spoke English near him, at first. After a couple of days, he worked out that he was in Belgium, southeast of Brussels, in some kind of military hospital.

After they'd put on the body cast he'd spoken to the doctor only once. ``You are lucky,'' the doctor said, in a heavy Belge accent. ``We find no brain damage. Zare ess some internal bleeding and we stop it. You have fracture breastbone, fracture arm, fracture collarbone. Slight concussion. Burns—second degree, not zo bad. You are lucky alzo zince we have ... '' And then he said something in Belgian.

``What's that?'' Smoke asked.

``A machine puts a current in zuh broken places of the bone, helps to heal faster. Good-bye.'' There was finality in that good-bye, and Smoke never saw him again except out of the corner of his eye as he ghosted around the beds of other patients in the big hospital dorm room.

``He is a bastard of a Belge,'' said the man next to him. A Frenchman. That was all Smoke could tell about him, because his casts made it impossible to turn and look that far to the side. ``The Belge are imbeciles, all Belge,'' the Frenchman said. ``And this electricity cure, this will kill you bientôt.''

It hurt Smoke too much to talk, at that point, so he didn't reply, and that was their entire conversation. Two days later the Frenchman died.

Sometimes Smoke played with the pain. It came in waves, and when the waves were in a peak, the pain was something palpable. He had always had what he thought of as his inner hand. It was the area low in his chest where he felt the center of his sensations. The place that glows for gratifications and aches for emotional hurt. Sometimes he felt he could shape the locus of sensation there into a kind of ectoplasmic hand—he knew it wasn't ectoplasm, but he pictured it that way—and he could imagine reaching with that hand into other parts of his body, to test them. Reach into the left leg and it will tingle with sensitivity. If it was in pain, he could reach in and touch the pain. Now when the waves of pain came strongest, he reached out his inside hand and caught the waves of pain in the hand and parted them, split them up, or squeezed them like something gelatinous between fingers of internal self-sensation; and this ``contact'' produced, in his mind's eye, a kind of rainbow-on-oil shimmer he watched with childlike fascination. In this way the pain became objectified into visual terms, and was rendered neutral, defused. The pain became almost painless.

But sometimes the misery of the ward overwhelmed him. The sick were in cots and their cots were everywhere; there were, lately, men laid out on the floor. The place stank, of course, and sometimes the smell was given the extra pungency of humiliation when much of the stink came from himself—the overworked nurses were slow about his bedpans. And the noise of the place diminished at night, but it never ceased. There was moaning and, always, bitching in four or five languages. There were men babbling obscenities, an unceasing bubbling over of mental ugliness, and that was perhaps the worst. He was perversely grateful for the occasional CRUMP and quaking of the shellings—or were they bombs?—in the countryside around the hospital. They made it possible to visualize a world outside the infinitely monotonous grind of life in the hospital.

For a time some of the patients were refugees, adding the sirening of wailing children to the dissonant symphony of complaints bouncing from the ceiling. But there was a rule about the hospital being used only for NATO soldiers—Smoke heard a British Red Cross nurse complain about it—and the refugees were moved out to a camp where, it was said, death was certain for the very ill. There wasn't enough food to go around in the refugee camps. In keeping with triage, critically ill refugees simply were not fed.

Smoke had seen the Dutch refugee camps. Had heard the stories ... Stories of a hundred thousand, two hundred thousand—an ever-swelling multitude of the displaced and homeless tramping the roads outside the European cities. At first they'd fled the war in cars—but the highways had become impassable with rubble and craters, and anyway fuel was hard to get. Now they walked, or pulled carts—often whole families pulling a cart made of a small stripped-down fiberglass car, propane or electric engines removed. Legions of people yoked to automotive shapes, as if enslaved to serve cars ... Part of a dust cloud in summer, slogging through icy mud in the winter; learning about trenchfoot and scurvy, cholera and hepatitis, gangrene and lice. Some formed tribes for self-protection. The tribes were usually ethnocentric, which festered racial awareness. People who, before the war, had been indifferent to their neighbor's race, were reviling the ``scheming Jews hoarding food'' or the ``thieving Arabs, steal your last crust if you're not watching with a gun in your hand!'' By some unspoken consensus, the roads were usually a neutral place, where the tribes merged into one mass of tramping, weeping, cursing, death-eyed misery. Thousands more took to sea in improvised boats and those who didn't founder and drown sometimes found refuge in the Middle East, in Israel and Egypt; a few thousand were admitted to Scotland; thousands more to Canada and the USA. But the anti-immigrant feeling was strong in North America, now, with the global warming crisis and the propaganda, and the quota was quickly filled. The flow of refugees to America became a trickle and then stopped with the near-cessation of civilian air and sea traffic over the Atlantic.

Most of the refugees were trapped in Europe. And most had been cosmopolitan urbanites, whose major baseline concerns before the war had been the acquisition of new technology, or car repair, or money for the August holidays. And now their worries were food, water, weapons, shelter, warmth, medicine. The refugee camps provided enough food to prolong the suffering, but not enough to generate the energy to find a way out of the suffering. The camps were called ``the shitpits'' by the English speakers. Camp shelters were made from waterproofed cardboard, which turned out to be waterproof for only three or four rainfalls. At first the refugee camps were clean, and run like military bases, dreary but livable. But as the war dragged on the volunteers fell sick, or lost heart; the military could no longer spare men to help out; the Russians blockaded emergency-civilian supplies, believing they might also be supply ships for NATO. The Second Alliance was involved in shipping relief supplies, and Steinfeld claimed they diverted much of it for their own use. The camps swelled and rotted, teeming with people the way cysts teem with bacteria. Riots against the camp administration flared—and quickly died out. They accomplished nothing. But inter-tribal melees followed by guerrilla warfare became a fact of life, as one refugee racial group attacked another for food and medical supplies. And here and there were the advance agents of the Second Alliance, quietly distributing small amounts of food, and great bags of promises. Recruiting those the SA saw as having ``special potential.'' These would disappear from the refugee camps, would turn up later in the Second Alliance, unswervingly loyal to the organization that had brought them out of starvation and squalor and hopelessness, shown them purpose and order and a reinforcement of their most cherished prejudices ...

Smoke wondered for a while if he would be taken to one of the refugee camps, since he was no NATO soldier. But an orderly wheeling him for the bone-healing treatment, referred to him as ``the American soldier.'' Perhaps he would be taken from the hospital when they discovered the mistake. Or perhaps Steinfeld had arranged this ``mistake.'' Why? It must have cost him several favors. Why had Steinfeld done so much for him? Steinfeld was not an altruist by reflex. Steinfeld was a man obsessed.

Working on the fringes of the New Resistance operation, Smoke had picked up pieces of Steinfeld's history, had fitted them together. Smoke was sometimes privy to intelligence about the NR which didn't reach its rank and file. He had learned that Steinfeld had once been a field operative for the Mossad: Israeli intelligence.

Steinfeld had operated a listening post and then had been promoted to field officer, running agents. As Mossad field officer Steinfeld had run-ins with agents of the Second Alliance as they went about their recruiting. He became interested in them and gathered evidence that their ranks were riddled with active anti-Semites, including men who, decades before, had sheltered the doddering, wheelchair-bound Nazi war criminals from war crimes investigators. Steinfeld became a bit shrill in trumpeting the dangers of the Second Alliance to the Mossad. He was believed to have lost his objectivity. This, combined with his known sympathy for the Palestinians, cost him his post. He was pressured into resigning. He set up his own network, ``going indie,'' at first cadging funds here and there from sympathizers—some said even from Palestinians. Now, an American businessman named Quincy Witcher paid Steinfeld's bills. And no one was quite sure why.

Steinfeld had his sympathizers in the Mossad; occasionally one of these gave him intelligence, or a little extra credit-grease, or food, or weapons. The Mossad brass pretended not to know about this, because Steinfeld was still useful to them. But he was also on their yellow list: the list of those who would be assassinated, should the correct juxtaposition of circumstances arise; should Steinfeld be viewed as dangerous. There were those who would have relegated Steinfeld to the red list: assassinate ASAP. Suppose he was captured? they argued. He has seen us on the inside; there is much he knows. Still, over tea in commissaries and wine in the better restaurants in Tel Aviv, it was decided that Steinfeld would not be shot or blown up or poisoned, at least not right away. Not by the Mossad. After all, he was doing work that was useful to the Mossad, but which they could truly disavow.

Lying rigid in his plaster carapace like a paralyzed lobster, staring at the same grime spots on the yellowing ceiling week after week, Smoke thought about Steinfeld a great deal. So it was somehow not a great surprise when Steinfeld came to see him. It was as if Smoke had conjured him.

Steinfeld was wearing a blue nylon windbreaker. It rode up a little on his big belly. The New Resistance was based in Paris now, which was relatively comfortable compared to Amsterdam.

``Looks like there's more to eat in Paris,'' Smoke rasped when Steinfeld sat carefully on an unsoiled corner of Smoke's bed.

Steinfeld smiled and nodded. He looked at the IV stand, then at the lesions on Smoke's forearm. ``You don't look so bad,'' he said. ``Except for this arm. What's this?''

``It became infected,'' Smoke said. ``The IV needle. They put it in the wrong place a few times, missed the vein. What's worse is when they forget to change the bottle. The damn thing empties and turns vampire, sucks blood out of me. The blood runs up the tube. Hurts like the devil.''

Steinfeld said, ``They have too much to do.''

``I know. I don't complain—anyway, they ignore complaints.''

``But once,'' Steinfeld said, looking at him, ``you tried to tell them you are not a soldier, that you should not be here. So I heard.''

``They don't listen no matter what you say.''

``If they had, you'd probably be dead by now. Do you still have a death wish, Smoke?'' Steinfeld asked.

Smoke said nothing.

``I think you do. That's the only problem with it.''

``With what?''

Steinfeld said, ``With the fact that you owe me now, Smoke.''

Smoke said, with a faint smile, ``I see.''

Steinfeld nodded.

``You have plans for me,'' Smoke said.

Now it was Steinfeld's turn to say nothing.

``It itches in this cast,'' Smoke said. It was good to have someone to complain to.

``Yes. And the food here is ... ?''

``Execrable,'' Smoke said.

``Go on,'' Steinfeld said.

``They rarely change the sheets,'' Smoke said with alacrity, ``and they rarely turn me. I get bedsores, which they sometimes allow to become infected. Then they give me a general antibiotic, and the sores ease, and then they forget to turn me and the sores come back. And so forth. The crying of the others is an assault on sanity.''

``I would say that it is better to be in such a place than dead in a shell of a building in Amsterdam—given that you won't be here forever. But we come again to the problem of your death wish.''

``Are the others alive? Hard-Eyes and the others?''

``So far as I know. I've been away from Paris for a while.''

There was something more that Smoke wanted to ask, but he felt foolish. And in this place there was little dignity; what one could scrape up, one hoarded.

He didn't have to ask it, as it happened: Steinfeld guessed what was in Smoke's mind. ``The crow lived, and came along to the boat. I have it in my flat, in Paris. Someone's taking care of it.''

Smoke felt an absurdly profound relief.

Steinfeld stood up. He took a chocolate bar and a vitaminpak from his pocket and put them in Smoke's usable hand.

``They're giving me a treatment with electric currents to heal the bones,'' Smoke said to keep Steinfeld there just a little longer. ``A Frenchman told me it would hurt me, but I think it's helping. The pain is much less, It's just a few weeks since they started doing it.''

Steinfeld nodded. ``It works. We'll come to get you when they decide the casts can come off.''

He turned to go. Smoke said quickly, desperately, ``Tell me something. Anything. I need something to think about. You have plans for me. Tell me about it. Something.''

``There isn't much I can say here.''

``Then only what you can say.''

Steinfeld nodded at the IV bottle. ``I'll see to it they refill that thing.''

``Tell them to take it away. I don't need it. Tell me something, Steinfeld.''

Steinfeld took a deep breath, tugged at his beard, blew the breath out again. He looked at Smoke. ``I know who you are. I found out the day before the jumpjet hit us. For a while I too thought Smoke was a nickname.''

``Wait—'' Smoke felt he was going to choke.

But Steinfeld bulled grimly on. ``You don't want me to talk about it. You've become expert in not thinking about it, and you don't want me to undermine that expertise. Tough. You wanted something to think about. So think about this: you're Jack Brendan Smoke. You're American. You were in Amsterdam when the war broke out, to see a psychiatrist at the Leydon clinic. Before that, you won the United Nations Literary Committee prize for your Search for a Contemporary Reality. You were the spokesman for all the people who felt lost in the accelerated rate of change. You wrote a second series of essays in which you said, generally, that there were people manipulating the Grid for political ends, and you named Worldtalk. You predicted a return of fascism and you quoted something you'd heard about the Second Circle, the secret inner circle of the Second Alliance. The ones who make the SA's long-term goals ... That essay was never published. Evidently someone at your publishing company was SA. Some men came to the clinic in ski masks. You were taken in the night and they—''

``Please ... '' A great weight on his chest made it hard to breathe. ``Steinfeld ... ''

``They tortured you. They gave you a drug that made you feel that you were choking ... ''

He stopped, seeing Smoke was gagging. He waited. After a minute the spasm passed.

Smoke lay staring at the ceiling, breathing shallowly.

``I'm going to go on, Smoke,'' Steinfeld said.

Smoke just lay there.

Steinfeld said, ``They wanted to know who you got the information from. About the Second Circle. You didn't tell them. They tortured you in many ways. In many imaginative ways. And then the choking drug, again. They tried to move you to another place, where they had access to extraction. You escaped, en route, and went back to the hospital, where you broke down completely. You were sent in secret to another clinic. The men would have found the clinic anyway, eventually, would have come for you again—if not for the war. That was the day the Russian tanks crossed into Germany. And a little while later the Russians were moving in on Amsterdam, and they shelled the city. Your clinic was shelled. Almost everyone killed ... ''

``I was in my safe, locked room,'' Smoke said, taking it up in a small voice. ``But then the wall was blown in. I went to get someone to put the wall back. They were all dead. Except Dr. Van Henk. I saw him—his face was bloody. The sight of him bloody like that frightened me. I don't know why it affected me so strongly—I ran from him, we lost sight of each other in all the burning. Was it Van Henk who—?''

``Yes. I had this from Van Henk. He's still alive. So far as I know.''

``I was the only patient not killed. Wherever I went there were only the dead. I wandered out of there. Sometimes the choking, from the drug—it would start again. It seemed to come back, maliciously, after me. The choking and the dead everywhere ... For a long time I couldn't remember who I was. When I could remember—I wanted to forget again. Wanted to be someone else ... '' His voice was cracked glass.

Steinfeld said, ``Sometimes your face looked familiar to me. But under all that grime ... and the way a man gets wasted ... '' He shrugged. ``So you wanted me to tell you something. There's this: you were a great writer. A great speaker, great humanist. The torture didn't break you—but then again it did. Even so, Smoke: you could help us. In the States the only ones who believe that the fascists are coming again are the ones trying to help them. As for the others—'' He shook his head sadly. ``Worldtalk pushes their buttons. But if people keep speaking up in the underGrid ... You could help us! People remember you.''

Smoke said, ``I can't.''

``Sometimes I see a man who's broken, or bent by torture, and I know he'll never change. Never heal. When I saw you with that crow,'' he smiled ruefully, ``I knew you would heal. That meant the other possibilities I saw in you could become real. Now you have something to think about.''

Steinfeld nodded, once, to say good-bye and went away. Leaving Smoke to just lie there, staring at the ceiling.

Hard-Eyes and Jenkins were walking through the Parc Monceau. It was the flaccid end of late afternoon; the trees stood leafless and stark as nude crones. The forest looked dead, and misty wet, all grays and blue shadows and browns leached of life. But the smell of moldering leaves was good. Hard-Eyes inhaled it in hungry breaths and the cold air bit at his sore nostrils; pins and needles danced in his cheeks.

The HK-21 assault rifle in his hands was cold as a stone crucifix and felt nearly as heavy. He was tired; he was hungry. One meal a day was not enough.

``Steinfeld promised us more to eat,'' Jenkins complained.

Hard-Eyes had been thinking the same thing, but he said, ``We're lucky to get what they give us. You get a look at that refugee camp outside town?''

Jenkins grunted. ``You got a point.'' His hands were red from cold on the blue steel and plastic stock of his assault rifle. They'd traded in the Weatherby and the .22 for more practical weapons.

Hard-Eyes glanced over his shoulder, wondering why the instructors were hanging back so far. And then he stopped.

He couldn't see them at all now.

``They're fucking with us, Jenkins,'' Hard-Eyes said.

Jenkins stopped and they watched the trail behind them, expecting to see their guerrilla-warfare instructors strolling around the bend, through the attenuated bristle of bluish underbrush. Nothing.

There were no bird sounds. There had been, a few minutes before.

Hard-Eyes swallowed.

``Those guys don't like us,'' Jenkins said, his voice hushed. ``What'd they say about ‘robber packs'?''

``Said we hadda be careful because the noise of the gunfire might attract robber packs. Guys that live on the other side of the park, in the shack slum over there.''

``Shit!—you think they set us up?''

``Steinfeld wouldn't risk men like that.''

``But I'm telling you these guys don't like us. They've decided that Americans suck. They think we're CIA or some shit. And Steinfeld ain't around.''

``They don't like us but they wouldn't—'' He broke off, staring through the mist. There were men coming out of the woods.

The trees to the right of the trail were not thickly dispersed. The well-spaced trunks came together in the compression of distance, becoming a corrugated wall of gray about fifty yards away. From here it looked liked a solid wall of trees. So when the men came out of that solid-seeming wall it looked as if they were squeezed out, like man-shaped drops of liquid. They looked as gray as the tree trunks, except there were smears of orange-pink for faces and pencil-thin strokes of blue-black and brown in their hands. Rifles.

Hard-Eyes counted eight and after that stopped counting. And looked for a place to run to. To his left was a broad, cracked parking lot. The French government scarcely existed now, and the skeleton of it that remained had no resources for park maintenance, so the parking lot was choked with blown branches and leaves; here and there were the rusty humps of stripped and abandoned cars. But the cars were too far away to be used for cover. He and Jenkins would be shot in the back if they ran across the parking lot.

The trail up ahead looked safer, but the instructors had told them, ``When pressed, ask yourself if this is a situation when we must disperse—or regroup? The answer depends on the nature of the enemy and their position with relation to your own unit's command.''

If Hard-Eyes ran ahead, the enemy would become a wedge between Hard-Eyes' unit and the unit's command, the instructors. The command would be threatened, hemmed in by the enemy and when possible he was to regroup to protect command.

So Hard-Eyes said, ``Back down the trail.''

Jenkins said, ``Shit, man—''

``Come on!''

The pack was close enough so that Hard-Eyes could make out the features in the orange-pink smudges. He turned and ran.

Hard-Eyes thought, This isn't the enemy, this is a bunch of half-starved Parisians hoping we'll be carrying something valuable or edible.

And he thought, The instructors have set us up so the robber pack becomes ``the enemy.'' Like this is a war game where the other side doesn't know it's playing a game.

And then he thought, The fucking instructors are hoping we'll get our asses blown away. Or maybe we'll burn down a few of these problem thugs for them.

... As he heard the first popping sounds, and the echoes like a sheet of aluminum shaken to simulate thunder. A piece of turf threw itself in the air near his feet. Irrationally, he leapt back, as if the bit of jumping ground itself were the threat.

Hard-Eyes ran on, Jenkins a little behind and falling farther behind. The trees danced crazily past; the sky made a jerky windshield wiper movement. He ran past a tree and it spat bark at him; a piece of yellow wood-flesh showed where the bullet had scored away bark.

He heard Jenkins returning fire behind him, a thudding rattle, probably no hope of hitting anyone, trying to suppress them.

Thirty feet ahead the trail sank into thickets of blue bushes. There was not much cover at the opening of the trail between the bushes. If the pack got directly behind them, they would simply stop and shoot down the line of the trail, and cut them down.

If they reached that first bend, cutting left into the bushes, they might make it.

But just then Jenkins stumbled and gave a strangely high-pitched cry as he went down, skidding over the iron-hard, iron-cold earth, his rifle clattering. Hard-Eyes wanted to run on, part of his mind already making up excuses.

But he stopped. Huffing and cursing, he turned, going upstream against his own impulse to get the fuck out, fighting the current in himself. He heard a scornful humming, and he knew that a bullet had missed his head by an inch or two. Jenkins was getting to his knees, puffing. How could they miss him? He was such a big target! Hard-Eyes bent and tried to help him up, but Jenkins shook his arm off—both of them annoyed—and said, ``Just cover me,'' as he reached for his rifle.

Hard-Eyes turned and opened up without aiming, the automatic rifle jumping in his hand; he felt like a fool when he saw that he'd shot six holes in the bole of a tree between him and the pack. Then he saw one of them coming at him from the right. The man paused about forty feet away and raised the rifle to his shoulder, aiming, like a man shooting at rabbits. He had a big nose, weak chin, gaunt cheeks. He wore a tagged brown cap. He fired. The bullet cut the air overhead. The man struggled to reload his rifle ...

Hard-Eyes swiveled the HK-21 and fired another burst from the hip. He had ludicrous mental images of himself as a boy taking turns with his big brother cutting the lawn because when you started the lawn-mower it made a noise like the assault rifle. He saw himself spraying a water hose at his brother—shooting an automatic weapon sometimes felt like shooting a high-pressure water hose at someone; when you were close enough to the enemy with no time to aim, you pointed the hose, raking back and forth, and hoped for the best. The man in the brown cap spun half around and staggered, dropped his rifle, but didn't fall. He looked confused, then he turned and ran, holding his side. Wounded. Others were coming on through the trees, spread out. Hard-Eyes emptied the magazine at them, firing in little bursts. They dodged behind trees for cover—and then Hard-Eyes realized that Jenkins was up and running for the brush.

Hard-Eyes ran after him. Someone on his left shot at him. He felt a tightening sensation at the left side of his head: entirely psychosomatic; that was the place he imagined the bullets would hit him. Anticipating the sickening crack of a bullet impacting. Jenkins was about ten feet ahead, running with a wallowing motion, with poor coordination, looking as if he'd like to throw the encumbering rifle away.

And then the brush was sweeping past and Hard-Eyes felt a surge of relief as he turned the bend in the trail. For the moment he was out of their line of sight. Up ahead the trail stretched straight for a ways. That would be a good place to get shot in the back.

``Jenkins!'' he hissed. ``Hey—go find the instructors, I'm gonna be here in the brush on the left side, left side going this way, don't shoot in it when you come back even if you hear gunfire in it 'cause that'll be me!'' He couldn't be sure Jenkins heard, but Hard-Eyes thought he saw him bob his head in response.

Hard-Eyes angled left, then pressed close to the brush, turned to move back up, parallel to the trail. The brush here hooked in a question-mark shape, roughly, and he was moving up the stem of the question mark toward the inside of its hook. The pack was on the other side of the hook. He was breathing hard as much from fear as exertion, his breath smoking out white in front of him, and he thought, What if they see my breath steam above the brush; they'll know my position ...

He heard a babble of voices in French. He pressed into the wall of brush at the hook of the question mark, biting his lip to keep from yelling when a twig stung his right eye, other tiny jags raking his cheek and neck and hands.

He turned sideways to elbow deeper into the brush, thinking, Maybe this is stupid, maybe the brush will just hold me in place and I won't be able to run, and they'll see me in here and shoot into it till they get me.

He scrunched down, so that the thicker part of the brush was over him, and he felt better about it, because he could move here, the branches making arches over him. He heard voices and footsteps. He began to worm between the thick, horny stems of the bushes toward the bottleneck in the trail, dragging the rifle in his right hand, trying to keep dirt out of it. Pulling himself along on his elbows. The cold ground sent an ache up through his elbow bone. His cheek itched fiercely where the twigs had lashed him. His eye burned where it'd been scratched. It hurt when he blinked.

He could see the trail through the screen of brush now. He brought the rifle up and wedged it into firing position against his shoulder, about thirty degrees out of alignment with his body, his elbows planted, the breech propped in his hands, and sighted at the trail. And then he heard the French voices again and knew they were arguing. Some wanted to go down the trail into the brush. Others thought it might be too dangerous. Then three of them trotted down the trail, in a formation neat as bowling pins. He angled the rifle up a little more and then thought, Shit, I didn't put another clip in it! Idiot!

He laid the gun down, quietly, carefully, as they drew abreast of him. The front man was just fourteen feet away, ten feet beyond the screen of brush. Hard-Eyes reached behind him, fished in his pack. The angle was awkward. He ground his teeth in frustration. The man was walking past. Still fishing in the pack, Hard-Eyes felt something metallic cold under his fingers. He drew it out and looked at it. A full clip. He ejected the other clip and slapped the full one in—and heard a shot. Someone was bending to look in the brush. A rifle barked and a piece of twig lopped neatly in two, fell delicately across his rifle barrel.

Hard-Eyes sighted on the guy crouching in the trail. He took a deep breath, let it out, and when it had gone out of him and his body was still before the next breath, he squeezed the trigger—and at the same time the other man fired. Something sizzled past Hard-Eyes' right cheek. The man who'd shot at him did a little dance of frustration, dancing backward—no, that's not what he was doing, he was staggering back as Hard-Eyes' assault rifle stitched three rounds into his chest. Hard-Eyes expected to see bloodied holes but the places the bullets struck looked like black dots. The guy fell. Hard-Eyes kept firing, raking, centering the sights on the silhouettes of two other running men ...

The rifle kicking his shoulder, acrid blue smoke clinging to the arching brush just overhead. A twig smoldering from muzzle flash. His ears aching with the detonations, the vibrations.

The men had stopped running. Were all, like him, on the ground; but they were on their backs. One of them making a mewling sound and a pedaling motion with his feet. Another turning to vomit blood. Hand clawing the ground. Twitching. Then not moving at all.

Hard-Eyes waited, but no one else came down the trail. After a while, when his hands were going stiff with cold, his elbows aching, his cheek throbbing, he heard Jenkins shout something. And then French voices behind him, and he knew one as the petulant voice of one of his instructors.

There was another sound: wham wham wham wham wham wham. After a moment he realized it was the sound of his heart pounding. He was amazed that he could hear it so clearly.

He wormed up, thrust head and shoulders out of the brush just enough to look down the trail both ways. He saw no one either way, except the dead. The three men he'd shot were all dead now. They weren't silhouettes anymore. But he couldn't help noticing that one of them was just a boy. Maybe fifteen. A boy with a rifle gripped in his white hands.

He stood up and brushed thorns and dried leaves off himself, feeling dizzy but energized.

Thinking, with more wonder than regret: They were just hungry. That's really all it was.

His instructors came around the bend in the trail, their rifles raised.

``Hold your fire!'' Hard-Eyes yelled. Or tried to. The words came out mush because his mouth was numb from cold. His right ear felt cold too. Funny: just the right one.

They slowed, looking at him. They were frowning. He knew they'd have some complaint about how he'd done it. Jenkins was right: They didn't like Americans. But Hard-Eyes knew he'd mostly done it right.

Jenkins came lumbering along. He stared at Hard-Eyes open-mouthed.

``Your ear,'' he said. ``You lost an ear.''

Molt was walking down the corridor, thinking he'd got off at the wrong level. It felt like Level 02. He felt heavier here than he should.

The corridor was deserted, which he thought was strange, too. It should be work time in this section. Wilson had said they'd meet on Level 00. He was sure of that. He was sure he'd pressed 00. But he saw a coordination indicator, Level 03, Corridor C13—no indicator for function.

He was in a part of the Colony he'd never been in before. The walls were the same kind of utilitarian studded gray metal you found down in Recycling or around a power station.

I pressed 00, he thought. I'm sure I did.

He turned to go back to the elevator. A section of the ceiling four inches wide slid down to become a wall, in front of him, ceiling down to floor. Zi-ip: that fast. It was a transparent wall, plastic but thick, and he knew he'd never be able to break it. He stared, feeling a panic of the sort he hadn't felt since having his first really bad childhood nightmares. He touched the wall to be sure it was real.

He looked around, gut clenching with a growing suspicion.

He'd pressed 00, but the elevator had taken him to 03. The bastards could control the elevators independently. Of course. They had brought him here.

He looked down the hall. There were other sections of the ceiling that looked as if they could slide down. He was sure he hadn't seen a ceiling like that at launch level or at the dorms. But he had seen them somewhere. Around Admin—when you went to Admin to get your pay chit stamped you saw ceilings with those sections in them, and you wondered what they were ...

He backed away from the transparent wall, turned, and ran. He got forty feet. Ten feet ahead of him another section of wall slid down. He was boxed in.

He slowed, the run becoming a trot and then a walk. He walked up to the wall and pressed his forehead against it, looking down the stretch of empty corridor on the other side. He was breathing hard, clouding the transparent plastic. He slammed his fist against the wall—three times, almost fracturing the bones of his hand. He knew he couldn't break the wall. He hit it to let them know how he felt. Because he knew they were watching him.

He looked at the metal corridor walls between the transparent barriers, wondering ... what next? Poison gas maybe? Or maybe he'd be ejected into space. No. The liberal wimps in Admin—at least on the Rimpler side of the board—didn't have the honesty to do it that way.

There was a door in one of the walls. It was opening.

Slowly, sliding back into the wall. It whirred faintly.

Molt thought, I'm expected to walk through there. Fuck that.

He moved to where he could look through the door without standing too near it. Through the doorway he saw an almost bare room. There was a rectangular panel in the wall that would be the cot, when it was pulled out. There was a toilet, a sink, a shower stall. Air-conditioning vents—not big enough to crawl through. That was all. A detention cell.

He sat down with his back to the wall across from the door. He wasn't going to give them the satisfaction of seeing him walk in here. Not right away.

He wondered, idly, why they'd done it this way. If they'd tracked him, why hadn't they sent the bulls in to arrest him?

Because the sneaky bastards knew if they'd sent the bulls in, it would have been a political act. It would have martyred him. They had to do it so no one would see. This way they could spread rumors he'd deserted to the other side or gone into hiding. Make him look like a coward.

Wilson. That skeevy runt must have sold him out.

He looked around, wondering where the cameras were. He looked at the ceiling and nodded to himself. Somewhere in the ceiling, one of those panels is two-way.

He stood up and dropped his pants ...

In the Admin conference room, they sat around a table shaped like a backward S and watched Molt on the screen. Molt dropped his pants, took hold of his dick, waved it at them, pointed at it with his other hand, and mouthed, clearly, Suck this, you motherfuckers.

Claire winced and looked away.

The curves of the S-table were softly contoured. Praeger sat inside the curve across and to the left from Claire. Her father sat in the form-fitting chair across from her. Ganzio, the Brazilian UNIC rep, sat to her immediate left, scowling. He'd been here for an inspection visit—and had been stranded when the Russians had blockaded the Colony. He wanted to go home.

Judith Van Kips, the Afrikaner rep, sat to Gaazio's left. To Van Kips' left sat Messer-Krellman, officially the union rep appointed by UNIC—puppeted by UNIC. Across from Messer Krellman was Scanlon, the Colony Security chief.

The room was lit with soft, shadowless indirection. On Claire's far right, at one end of the cornerless, roughly rectangular room was the screen, and on the screen was Molt. On Claire's left, opposite the screen, was a bronze sculpture of a flock of birds taking flight from a pond.

Claire glanced at Molt, saw he was doing something even more obscene now, and fastened her eyes on the sculpture—with almost equal distaste. The sculpture seemed as false, as abstract and convenient, as UNIC's protestations of classless fairness. Everyone will have a chance here, Admin had been saying over InterColony channel. Everyone will have an opportunity to move into the Open when the time comes. When the blockade is lifted, we'll discuss pay raises and greater recreational credits. But in the meantime ...

In the meantime they discussed security measures.

``Isolating this man isn't going to isolate the rebellion,'' Claire said. ``The rebellion is widely supported. And it'll continue to be supported in the Colony—as long as we're hypocritical. We complain of not having money to improve their housing, but we sink four million newbux into expanding the security system—well before the rebellion began. And two million more into Admin housing improvements—''

Scanlon said, ``Looks like we improved security not a moment too soon. The riots ... ''

``The riots don't have to be,'' Claire said wearily. ``There would be no riots if the technickis were given what they were promised in the Articles. The technickis are convinced we've betrayed their trust.''

``Are they really convinced?'' Praeger asked. ``I think not.'' Praeger was half-bald, and his pinkish head always made Claire think of a pencil eraser rounded by use. His eyes were weak, and he had some kind of phobia of implantation eye operations, so he wore thick, rimless glasses. His lips were bloodless, the same color as the skin of his face. He was thick-bodied, an athletic man—something you wouldn't think he'd be, looking at his head—and muscular under the gray three-piece suit. ``They're reacting to stimuli, according to their social programming. They could just as easily react another way—with other stimuli. And if we're wise, we'll provide that.''

``And let them know only what we want them to know,'' Rimpler said suddenly, startling them with his humorous tone. ``And if they find out about the rest—tell them it's a communications problem.'' The ``communications problem'' was a reference to Praeger's failing to inform Rimpler of the emergency while he was on vacation. Praeger had claimed he'd given the order to a subordinate, who'd failed to implement it by simple oversight. In due course Praeger had produced a subordinate who claimed to be responsible for the error. The man had been put on pay suspension, and probably been well paid off. ``Just a little communica-shuns prob-lemmmm,'' Rimpler said, dreamily singsong. Making Claire think of the dormouse at the mad tea party. And making her think, What's happening to him?

Van Kips sighed. ``I really think there's no point in dragging that one over the coals again, Doctor.'' Pursing her lips—the severest expression she allowed herself. Or, perhaps, that Praeger allowed her. Supposedly, she worshiped Praeger. She was an implausibly beautiful woman. Shaped to some artist's conception. Metal-flake blue eyes; a model's narrow, doe-elegant face. Her long, perfectly straight flaxen hair was parted in the middle, to fall over her shoulders with impossible artfulness. She wore a dove-gray suit and white silk blouse; the suit clung to her tall, willowy body when she moved. But now she sat rigidly upright, her hands folded in her lap. Moving only her eyes when she looked at someone.

``At this point,'' Praeger said, ``it's meaningless to try to pin down the cause of the riots. First, we must quell the riots, the vandalism, the strikes. If we come out now and say, Yes, you're right, we've been remiss—well, that would encourage them in the idea that violence is the way to get through to us. The violence must cease before we concede anything.''

``I sure have to agree with that, Bill,'' Scanlon said, in his faint Southern accent. He was a big, boyish-looking man, with tired eyes and a lot of seams in his wide, friendly face. Friendly face, and he'll have a jolly twinkle in his eye, Claire thought, when he gets around to ordering my arrest. ``If we give in now we'll have to give in every time they threaten us. Things'll just get worse—for them and for us, too.'' He shifted in his seat and waited for a response, smiling like an angel. Claire remembered having heard he was some kind of born-again Christian.

``For them and for us, too?'' Claire said. ``That ‘them and us' mentality is one of our problems. I move we release the prisoners Security took during the riots, on their own recognizance. Just to ease the tension a bit. Then we try to set up another meeting with the Radics—and we allow them to send a technicki representative to the meetings. Those aren't such great concessions.''

``Jack here,'' Praeger said, nodding toward Messer-Krellman, ``represents them. He's the union rep, is he not?'' Messer Krellman was a ferret-faced man with a bored expression and a habit of sighing after each statement.

``Yes, I seem to recall that's my function,'' he said sarcastically and sighed, looking with mild reproach at Claire.

Claire shook her head. ``It should be a technicki rep! Born and bred a technicki! Someone who speaks technicki because he was raised in it. Jack has simply lost their confidence. It wouldn't be a concession to—''

``It would,'' Praeger said. ``Because it's on their list of demands. Along with the release of so-called political prisoners. His demands.'' Nodding now at the screen. At Molt.

``Look at him,'' Judith Van Kips muttered, shaking her head. ``This is one of the technicki leaders. You'd want someone like this at our meetings? Here?''

``He's not a technicki, actually,'' Claire said. ``Not precisely ... We'd pick someone more, um—''

``Look at him,'' Van Kips repeated, hissing it.

On the screen, Molt was pivoting in a circle, wagging his dick at each point of the compass.

Judith Van Kips made a noise of revulsion. ``The man is evidently on drugs.''

Rimpler shook his head. ``I think not.'' He chuckled. ``Molt knows we're watching, but he doesn't know where we are, so he's saying fuck you in every direction, just to make sure we get the message.''

``You seem to approve, Doctor,'' Ganzio, commented. He was a slim, dark man with a mustache so neat-edged it looked stenciled, and small, forever-shifting black eyes. He wore a gold-colored suit, which everyone privately thought vulgar.

``Oh, no, no,'' Rimpler said airily. ``But one has to admire his nerve.''

Molt was making an even ruder gesture now, and Praeger stabbed a finger at the tabletop's terminal. The image on the screen reticulated, folded into itself, was replaced with a view of the Strip. There was a crowd around the café, listening to someone standing on a table speak. Praeger punched for a close-up on the speaker. The image zoomed in. It was Bonham. They didn't have the audio on, but the crowd looked mesmerized by the speech. ``Now, there's a fellow with talent,'' Praeger said. ``Suppose he was speaking for our benefit. And suppose we controlled the technicki TV channel. If we provided the right stimuli, the technickis would drop their inane, self-indulgent rebellion of their own initiative. Willingly.''

Claire felt a chill. She looked to her father, wishing he'd take some active part in supporting their side of things. He was looking wistfully at the refreshment panels in the wall across from him, probably wanting to dial up a cocktail.

Maybe it had been a mistake to insist he come to the meeting at all, Claire thought. He had changed, in the last few years. In the beginning, her father had considered the Colony an extension of himself, and, if anything, he'd been a micromanager, too fervently responsible for its development and maintenance. And then Mother had left him, refusing to make the move to the Colony. He'd considered it a personal betrayal. Claire had been almost relieved by the divorce, really—she'd never felt close to her mother. The woman was cold, self-involved ... As if to compensate for his wife's betrayal of his dream—she had called the Colony ``a vanity unprecedented in the history of mankind'' and ``a monument to the misbegotten''—Rimpler was more control-compulsive than ever.

But with Terry's death, he began to change. At first he became, by turns, defensive, sullen, inward. That stage had also been marked by feverish overwork.

And then he'd collapsed, in Admin Central Command, after spending twenty straight hours overseeing the installation of the new computer system—and dealing with all the problems that arose while the old system was down. Then came another stage, a sort of manic-depressive period. Claire suspected he was using his pass to the pharmaceuticals storerooms too liberally. He'd begun using intermediaries to hire girls out of Bitchie's and the other technicki Afters. And he became increasingly abstracted at work—as if he was thinking only of getting home, to another sexual psychodrama ...

Still—he'd done what was expected of him, as an administrator—until the riots, and the news that he'd debauched right through a Colony life-support emergency. He reacted as if the Colony itself had rejected him. And he buckled under the psychological disorientation brought on by the sudden loss of control. Became childlike, prone to tantrums. Now, too often, she found herself forced into the role of chiding mother. He seemed to enjoy seeing her in that role—and at the same time he was afraid of her. More than once she'd found herself sick inside with self-disgust when she'd realized he'd drawn her into some almost incestuous dominatrix-style role-playing. She'd refused to play along—and he withdrew even more into drugs, drink, the search for oblivion—and when the real world intruded on his quest for oblivion, he responded by jeering at the thing he'd devoted most of his life to building ...

What was it Praeger had said?

... provided the right stimuli the technickis would drop their inane, self-indulgent rebellion of their own initiative. Willingly.

Claire took a deep breath and turned to Praeger. ``You feel they can be swayed with a broader media campaign. It won't work. Not with the blockade building up the pressure, making everyone a little more afraid every day ... ''

Praeger said, ``Media campaign?'' He seemed abstracted. He smiled faintly. ``Not precisely. Nothing so transparent ... I think we've lost sight of the problem at this meeting. The problem is sabotage! The problem is a life-support risk! This is a life-threatening emergency, Claire! For their sakes as well as—well, for everyone's sakes, we have to take the reins in our hands. All of the reins.''

Claire looked at the screen. ``They're not so stupid as to damage the life supports. They don't want to eat vacuum any more than we do.''

``When people get excited,'' Praeger said calmly, ``they tend to forget common-sense considerations. The thing could get out of control—farther out of control than any one of them would like. An individual technicki is logical—a mob of technickis is not.''

``And you propose to defuse them by taking control of their media? That'll only infuriate them!''

``You misunderstand me. I mean—we'll control it indirectly. They won't know we're doing it, if we do it right.''

``But that's ... '' She was at a loss for words. She looked again to her father. But he was standing up.

``Well, it's been delightful,'' Rimpler said. Smiling vacantly. He walked to the door, without saying anything more; without even looking around. Leaving her alone with them.

Claire grated, ``Dad! Dammit—take some responsibility!''

He paused at the door, turned to her the look of a bad little boy caught doing what he shouldn't.

She looked away. Scornfully: ``Oh, forget it. Go on.''

He shrugged, turned, and opened the door. She thought, Maybe he manipulated me. Knew I couldn't handle that little-kid shit. Knew that'd force me to let him go ...

Scanlon was looking thoughtfully after Rimpler. Something icy-cold about the expression on Scanlon's face frightened Claire.

Rimpler closed the door behind him—effectively closing the door on his leadership in Colony Admin.

Praeger was gazing at the screen. ``This man Bonham could be very useful to us,'' he said.

Messer-Krellman said, ``I believe Claire made a motion a little while ago. Does anyone second it? Should we vote on it?'' He liked the formalities. And he knew how the vote would turn out.

``Don't bother,'' Claire said. ``I suggest we table any further action till 0900 tomorrow. We all need to think about this. Just keep in mind: the situation is explosive, with the blockade of the Colony. They know the Colony is blockaded, they know resources will run low. They're going to be more insistent than ever—you won't be able to manipulate them.'' She got up and followed her Father out the door.

She paused for a moment before going out, and looked over her shoulder.

Van Kips and Praeger were looking at the screen. Praeger said something to Van Kips. She nodded.

Feeling helpless, Claire left the room.

\fancybreak{* * *}

Rickenharp put on his dark glasses, because of the way the Walk tugged at him.

The Walk wound through the interlinked Freezone outfloats for a half-mile, looping up and back, a hairpin canyon of arcades crusted with neon and glowflake, holos and screens. It was involuted, intensified by layering and a blaze of colored light.

Stoned, very stoned: Rickenharp and Carmen walked together through the sticky-warm night, almost in step. Yukio walked behind, Willow ahead, and Rickenharp felt like part of a jungle patrol formation. And he had another feeling: that they were being followed, or watched. Maybe it was suggestion, from seeing Yukio and Willow glance over their shoulders now and then ...

Rickenharp felt a ripple of kinetic force under his feet, an arc of wallow moving in languid whiplash through the flexible streetstuff, telling him that the breakers were up today, the baffles around the artificial island feeling the strain.

The arcades ran three levels above the narrow street; each level had its own sidewalk balcony; people stood at the railing to look down at the segmented snake of street traffic. The stack of arcades funneled a rich wash of scents to Rickenharp: the french-fry toastiness of the fast food; the sweet harshness of smashweed smoke, gyno-smoke, tobacco smoke—the cloy of perfumes; the mixed odors of fish-ka-bob stands, urine, rancid beer, popcorn, sea air; and the faint ozone smell of the small, eerily quiet electric cars jockeying on the street. His first time here, Rickenharp had thought the place smelled wrong for a red light cluster. ``It's wimpy,'' he'd said. Then he'd realized he was missing the bass-bottom of carbon monoxides. There were no combustion cars on Freezone. Some parts of America still permitted pollutive, resource-greedy gasoline cars, and Rickenharp, being a retro, had preferred those places.

The sounds splashed over Rickenharp in a warm wave of cultural fecundity; pop tunes from thudders and wrist-boxes swelled in volume as they passed, the guys exuding the music insignificant in comparison to the noise they carried, the skanky tripping of protosalsa or the calculatedly redundant pulse of minimono.

Rickenharp and Carmen walked beneath a fiberglass arch—so covered with graffiti its original commemorative meaning was lost—and ambled down the milky walkway under the second-story arcade boardwalk. The multinational crowd thickened as they approached the heart of the Walk. The soft lights glowing upward from beneath the polystyrene walkway gave the crowd a 1940s-horror-movie look; even through the dark glasses the place tugged at Rickenharp with a thousand subliminal come-hithers.

Rickenharp was still riding the blue mesc surf, but the wave was beginning to break; he could feel it crumbling under him. He looked at Carmen. She glanced back at him, and they understood one another. She looked around, then nodded toward the darkened doorway of a defunct movie theater, a trash-cluttered recess twenty feet off the street. They went into the doorway; Yukio and Willow stood with their backs to the door, blocking the view from the street, so that Rickenharp and Carmen could each do a double hit of blue mesc. There was a kind of little-kid pleasure in stepping into seclusion to do drugs, a rush of outlaw in-crowd romance to it. On the second sniff the graffiti on the pad-locked, fiberglass doors seemed to writhe with significance. ``I'm running low,'' Carmen said, checking her mesc bottle.

``Running low on drugs? Whoever heard of that happening?'' Rickenharp said and they both burst in peals of laughter. His mind was racing now, and he felt himself click into the boss blue verbal mode. ``You see that graffiti? You're gonna die young because the ITE took the second half of your life. You know what that is? I didn't know what ITE was till yesterday, I used to see those things and wonder and then somebody said—''

``Immortality something or other,'' she said, licking blue mesc off her sniffer.

``Immortality Treatment Elite. Supposedly some people keeping an immortality treatment to themselves because the government doesn't want the public to live too long and overpopulate the place. Another bullshit conspiracy theory.''

``You don't believe in conspiracies?''

``I don't know—some. Nothing that far-fetched. But—I think people are being manipulated all the time. Even here ... this place tugs at you, you know. Like—''

Willow said, ``Right, we'll 'ave our sociology class later children, you gotter? Where's this place with the bloke can get us off the fooking island?''

``Yeah, okay, come on,'' Rickenharp said, leading them back into the flow of the crowd—but seamlessly picking up his blue mesc rap. ``I mean, this place is a Times Square, right? You ever read the old novels about that place? That was the archetype. Or some places in Bangkok. I mean, these places are carefully arranged. Maybe subconsciously. But arranged as carefully as Japanese florals, only with the inverse esthetic. Sure, every whining, self-righteous tightassed evangelist who ever preached the diabolic seductiveness of places like this was right—in a way—was fully justified 'cause, yeah, the places titillate and they seduce and they vampirize people. Yeah, they're Venus's-flytraps. Architectural Svengalis. Yes to all the clichés about the bad part of town. All the reverend preachers—Reverend who, Reverend—what's his name?—Rick Crandall ... ''

She looked sharply at him. He wondered why but the mesc swept him on.

``All the preachers are right, but the reason they're right is why they're wrong, too. Everything here is trying to sell you something. Lots of lights and whirligig suction to seduce you into throwing your energy into it—in the form of money. People mostly come here to buy or to be titillated up to the verge of buying. The tension between wanting to buy and the resistance to buying can give you a charge. That's what I get into: I let it tickle my glands, but I hold back from paying into it. You know? Just constant titillation but no orgasm, because you waste your money or you get a social disease or mugged or sold bad drugs or something ... I mean, anything sold here is pointless bullshit. But it's harder for me to resist tonight ... '' Because I'm stoned. ``Makes you susceptible. Receptive to subliminals worked into the design of the signs, that gaudy kinetics, those fucking on/off bulbs—makes you flash on the old computer-thinking models, binomial thinking, on-off, on-off, blink blink—all those neon tubes, pulling you like the hypnotist's spiral pendant in the old movies ... And the kinds of colors they use, the energy of the signs, the rate of pulse, the rate of on/offing in the bulbs, all of it's engineered according to principles of psychology the people who make them don't even know they're using, colors that hint about, you know, glandular discharges and tingly chemical flows to the pleasure center ... like obscenities you pay for in the painted mouth of a whore ... like video games ... I mean—''

``I know what you mean,'' she said, in desperation buying a waxpaper cup of beer. ``You must be thirsty after that monologue. Here.'' She shoved the foaming cup under his nose.

``Talking too much. Sorry.'' He drank off half the beer in three gulps, took a breath, finished it, and it was paradise for a moment. A wave of quietude soothed him—and then evaporated mesc burned through again. Yeah, he was wired.

``I don't mind listening to you talk,'' she said, ``except you might say too much, and I'm not sure if we're being scanned.''

Rickenharp nodded sheepishly, and they walked on. He crushed the cup in his hand, began methodically to shred it as they went.

Rickenharp luxuriated in the colors of the place, colors that mixed and washed over the crowd, making the stream of hats and heads into a living swatch of iridescent gingham; shining the cars into multicolored lumps of mobile ice.

You take the word lurid, Rickenharp thought, and you put it raw in a vat filled with the juice of the word appeal. You leave it and let the acids of appeal leach the colors out of lurid, so that you get a kind of gasoline rainbow on the surface of the vat. You extract the petro-rainbow on the surface of the vat with cheesecloth and strain it into a glass tube, dilute heavily with oil of cartoon innocence and extract of pure subjectivity. Now run a current through the glass tube and all the other tubes of the neon signs interlacing Freezone's Walk.

The Walk, stretching ahead of them, was itself almost a tube of colored lights, converging in a kaleidoscope; the concave fronts of the buildings to either side were flashing with a dozen varieties of signs. The sensual flow of neon data in primary colors was broken at cunningly irregular intervals by stark trademark signs: SYNTHLIFE SYSTEMS and MICROSOFT-APPLE and NIKE and COCA-COLA and WARNER AMEX and NASA CHEMCO and BRAZILIAN EXPORTS INTL and EXXON ELECTRICS and NESSIO. In all of that only one hint of the war: two unlit signs, FABRIZZIO and ALLINNE—an Italian and a French company, killed by the Russian blockades. The signs were unlit, dead.

They passed a TV-shirt shop; tourists walked out with their shirts flashing video imagery, fiberoptics woven into the shirtfront playing the moving sequence of your choice.

Sidewalk hawkers of every race sold beta candy spiked with endorphins; sold shellfish from Freezone's own beds, tempura'd and skewered; sold holocube pornography key rings; sold instapix of you and your wife, oh that's your boyfriend ... Despite the nearness of Africa, black Africans were few here: Freezone Admin considered them a security risk and few on the contiguous coast could afford the trip. The tourists were mostly Japanese, Canadian, Brazilians—riding the crest of the Brazilian boom—South Koreans, Chinese, Arabs, Israelis, and a smattering of Americans; damned few Americans anymore, with the depression. Screens scanned them, one of them caught Rickenharp with a facial recognition program and on it a sexy animated Asian woman cooed, ``Rick Rickenharp—try Wilcox Subsensors and walk in a glow of excitement ... ''

As they got deeper into the Walk the atmosphere became even more hot-house. It was a multicolored steam bath. The air was sultry, the various smokes of the place warping the neon glow, filtering and smearing the colors of signs and TV shirts and DayGlo jewelry. High up, between the not-quite-fitted jigsaw parts of signs and lights, were blue-black slices of night sky. At street level the jumble was given shape and borders by the doors opening on either side: by people using the doors to check out malls and stimsmoke parlors and memento shops and cubey theaters and, especially, tingler galleries. Dealers drifted up like reef fish, nibbling and moving on, pausing to offer, ``DH, gotcher good Dee Ech'': Direct Hookup, illegal cerebral pleasure center stimulation. And drugs: synth-cocaine and smokeable herbs; stims, and downs. About half of the dealers were burn artists, selling baking soda or pseudostims. The dealers tended to hang on to Rickenharp and Carmen because they looked like users, and Carmen was wearing a sniffer. Blue mesc and sniffers were illegal, but so were lots of things the Freezone cops ignored. You could wear a sniffer, carry the stuff, but the understanding was, you don't use it openly, you step into someplace discreet.

And whores of both sexes cruised the street, flagrantly soliciting. Freezone Admin was supposed to regulate all prostitution, but black-market pros were tolerated as long as somebody paid off the beat security and as long as they didn't get too numerous.

The crowd streaming past was a perpetually unfolding revelation of human variety. It unfolded again and a specialty pimp appeared, pushing a man and woman ahead of him; they had to hobble because they were straitjacket-packaged in black-rubber bondage gear. Their faces were ciphers in blank black-rubber masks; aluminum racks held their mouths wide-open, intended to be inviting, but to Rickenharp whispered to Carmen, ``Victims of a mad orthodontist!'' and she laughed.

Studded down the streets were Freezone security guards in bulletproofed uniforms that made Rickenharp think of baseball umpires, faces caged in helmets. Their guns were locked by combination into their holsters; they were trained to open the four-digit combination in one second.

Mostly they stood around, gossiped on their helmet radios. Now two of them hassled a sidewalk three-card-monte artist—a withered little black guy who couldn't afford the baksheesh—pushing him back and forth between them, bantering one another through helmet amplifiers, their voices booming over the discothud from the speakers on the download shops: ``WHAT THE FUCK YOU DOING ON MY BEAT SCUMBAG. HEY BILL YOU KNOW WHAT THIS GUY'S DOING ON MY BEAT.''

``FUCK NO I DUNNO WHAT'S HE DOING ON YOUR BEAT.''

``HE'S MAKlNG ME SICK WITH THIS RIP-OFF MONTE BULLSHIT IS WHAT HE'S DOING.''

One of them hit the guy too hard with the waldo-enhanced arm of his riot suit and the monte dealer spun to the ground like a top running out of momentum, out cold.

``LOITERING ON THE ZONE'S WALKS, YOU SEE THAT BILL.''

``I SEE AND IT MAKES ME SICK JIM.''

The bulls dragged the little guy by the ankle to a lozenge-shaped kiosk in the street and pushed him into a man-capsule. They sealed the capsule, scribbled out a report, pasted it onto the capsule's hard plastic hull. Then they shoved the man-capsule into the kiosk's chute. The capsule was sucked by mail-tube principle to Freezone Lockup.

``Looks like they're using some kind of garbage disposal to get rid of people here,'' Carmen said when they were past the cops.

Rickenharp looked at her. ``You weren't nervous walking by the cops. So it's not them we're avoiding, huh?''

``Nope.''

``You wanna tell me who it is we're supposed to be avoiding?''

``Uh-uh, I do not.''

``How do you know these out-of-town cops you're worried about haven't gone to the locals and recruited some help?''

``Yukio says they won't, they don't want anybody to scan what they're doing here because the Freezone admin don't like 'em.''

Rickenharp guessed: the who they were avoiding was the Second Alliance. Freezone's chairman was Jewish. The Second Alliance could meet in Freezone—the idea was, the place was open to anyone for meetings, or recreation; anyone, even people the Freezone boss would like to see gassed—but the SA couldn't operate here, except covertly.

The fucking SA bulls! Shit! ... The blue mesc worked with his paranoia. Adrenaline spurted, making his heart bang. He began to feel claustrophobic in the crowd; began to see patterns in the movement around him, patterns charged with meaning superimposed by his own fear-galvanized mind. Patterns that taunted him with, The SA's close behind. He felt a stomach-churning combination of horror and elation.

All night he'd worked hard at suppressing thoughts of the band. And of his failure to make the band work. He'd lost the band. And it was almost impossible to make anyone understand why that was, to him, like a man losing his wife and children. And there was the career. All those years of pushing for that band, struggling to program a place for it in the Grid. Shot to hell now, his identity along with it. He knew, somehow, that it would be futile to try to put together another band. The Grid just didn't want him; and he didn't want the fucking Grid. And the elation was this: that ugly pit of displacement inside him closed up, was just gone, when he thought about the SA bulls. The bulls threatened his life, and the threat caught him up in something that made it possible to forget about the band. He'd found a way out.

But the horror was there, too. If he got caught up in this ... if the SA bulls got hold of him ...

Fuck it. What else did he have?

He grinned at Carmen, and she looked blankly back at him, wondering what the grin meant.

So now what? he asked himself. Get to the OmeGaity. Find Frankie. Frankie was the doorway.

But it was taking so long to get there. Thinking. The drug's fucking with your sense of duration. Heightened perception makes it seem to take longer.

The crowd seemed to get thicker, the air hotter, the music louder, the lights brighter. It was getting to Rickenharp. He began to lose the ability to make the distinction between things in his mind and things around him. He began to see himself as an enzyme molecule floating in some macrocosmic bloodstream—the sort of things that always OD'd him when he did an energizing drug in a sensory-overflow environment.

What am I?

Sizzling orange-neon arrows on the marquee overhead seemed to crawl off the marquee, slither down the wall, down into the sidewalk, snaking to twine around his ankles, to try to tug him into a tingler emporium. He stopped and stared. The emporium's display holos writhed with fleshy intertwinings; breasts and buttocks jutted out at him, and he responded against his will, like all the clichés, getting hard in his pants: visual stimuli; monkey see, monkey respond. He thought: Bell rings and dog salivates.

He looked over his shoulder. Who was that guy with the sunglasses back there? Why was he wearing sunglasses at night? Maybe he's SA— Noooo, man. I'm wearing sunglasses at night. Means nothing.

He tried to shrug off the paranoia, but somehow it was twined into the undercurrent of sexual excitement. Every time he saw a whore or a pornographic video sign, the paranoia hooked into him as a kind of scorpion stinger on the tail of his adolescent surge of arousal. And he could feel his nerve ends begin to extrude from his skin. After having been clean so long, his-blue mesc tolerance was low.

Who am I? Am I the crowd?

He saw Carmen look at something in the street, then whisper urgently to Yukio.

``What's the matter?'' Rickenharp asked.

She whispered, ``You see that silver thing? Kind of a silvery fluttering? There—over the cab ... Just look, I don't wanna point.''

He looked into the street. A cab was pulling up at the curb. Its electric motor whined as it nosed through a heap of refuse. Its windows were dialed to mercuric opacity. Above and a little behind it a chrome bird hovered, its wings a hummingbird blur. It was about thrush-sized, and it had a camera-lens instead of a head. ``I see it. Hard to say whose it is.''

``I think it's run from inside that cab. That's like them. They'll send it after us from there. Come on.'' She ducked into a tingler gallery; Willow and Yukio and Rickenharp followed her. They had to buy a swipe card to get in. A bald, jowly old dude it the counter took the cards, swiped them without looking, his eyes locked on a wrist-TV screen. On his wrist a miniature newscaster was saying in a small tinny voice, `` ... attempted assassination of SA director Crandall today ... '' Something mumbled, distorted. `` ... Crandall is in serious condition and heavily guarded at Freezone Medicenter ... ''

The turnstile spun for them and they went into the gallery. Rickenharp heard Willow mutter to Yukio, ``The bastard's still alive.''

Rickenharp put two and two together.

The tingler gallery was predominantly fleshtone, every available vertical surface taken up by emulsified nude humanity. As you passed from one photo or holo to the next, you saw the people in them were inverted or splayed or toyed with, turned in a thousand variations on coupling, as if a child had been playing with unclothed dolls and left them scattered. A sodden red light hummed in each booth: the light snagged you, a wavelength calculated to produce sexual curiosity. In each ``privacy booth'' was a screen and a tingler. An oxygen mask that dropped from a ceiling trap pumped out a combination of amyl nitrite and pheromones. The tingler looked like a twentieth-century vacuum cleaner hose with an oversized salt-shaker top on one end: You watched the pictures, listened to the sounds, and ran the tingler over your erogenous zones; the tingler stimulated the appropriate nerve ends with a subcutaneously penetrative electric field, very precisely attenuated. You could pick out the guys in the health-club showers who'd used a tingler too long: use it more than the ``recommended thirty-five-minute limit'' and it made your skin look sunburned. One time Rickenharp's drummer had asked him if he had any lotion: ``I got ‘tingler dick,' man.''

``To phrase it in the classic manner,'' Yukio said abruptly, ``is there another way out of here?''

Rickenharp nodded. ``Yeah ... Uh—somewhere.''

Willow was staring at a teaser blurb under a still-image of two men, a woman and a goat. He took a step closer, squinting at the goat.

``You looking for a family resemblance, Willow?'' Rickenharp said.

``Shut your 'ole, ya retro greaser.''

The booth sensed his nearness: the images on the sample placard began to move, bending, licking, penetrating, reshaping themselves with a weirdly formalized awkwardness; the booth's light increased its red glow, puffed out a tease of pheromone and amyl nitrite, trying to seduce him.

``Well, where is the other door?'' Carmen hissed.

``Huh?'' Rickenharp looked at her. ``Oh! I'm sorry, I'm so—uh I'm not sure.'' He glanced over his shoulder, lowered his voice. ``The bird didn't follow us in.''

Yukio murmured, ``The electric fields on the tinglers confuse the bird's guidance system. But we must keep a step ahead.''

Rickenharp looked around—but he was still stoned: the maze of black booths and fleshtones seemed to twist back on itself, to turn ponderously, as if going down some cubistic drain ...

``I will find the other door,'' Yukio said. Rickenharp followed him gratefully. He wanted out.

They hurried through the narrow hall between tingler booths. The customers moved pensively—or strolled with excessive nonchalance—from one booth to another, reading the blurbs, scanning the imagery, sorting through fetishistic indexings for their personal libido codes, not looking at one another except peripherally, carefully avoiding the margins of personal-space.

Chuffing, sighing music played from somewhere; the red lights were like the glow of blood in a hand held over a bright light. But the place was rigorously Calvinistic in its obstacle course of tacit regulations. And here and there, at the turns in the hot, narrow passageways between rows of booths, bored security guards rocked on their heels and told the browsers, No loitering please, you can purchase more time at the front desk.

Rickenharp flashed that the place wanted to drain his sexuality, as if the vacuum-cleaner hoses in the booths were going to vacuum his orgone energy, leave him chilled as a gelding.

Get the fuck out of here.

Then he saw EXIT, and they rushed for it, through it.

They were in an alley. They looked up, around, half expecting to see the metal bird. No bird. Only the gray intersection of styroconcrete planes, stunningly monochrome after the hungry chromatics of the tingler gallery.

They walked out to the end of the alley, stood for a moment watching the crowd. It was like standing on the bank of a torrent. Then they stepped into it, Rickenharp, blue mesc'd, fantasizing that he was getting wet with the liquefied flesh of the rush of humanity as he steered by sheer instinct to his original objective: the OmeGaity.

They pushed through the peeling black chessboard doors into the dark mustiness of the OmeGaity's entrance hall, and Rickenharp gave Carmen his coat to hide her bare breasts. ``Men only, in here,'' he said, ``but if you don't shove your femaleness into their line of sight, they might let us slide.''

Carmen pulled the jacket on, zipped it up—very carefully—and Rickenharp gave her his dark glasses.

Rickenharp banged on the window of the screening kiosk beside the locked door that led into the cruising rooms. Beyond the glass, someone looked up from a fat-screen TV. ``Hey, Carter,'' Rickenharp said.

``Hey.'' Carter grinned at him. Carter was, by his own admission, ``a trendy faggot.'' He was flexicoated battleship gray with white trim, a minimono style. But the real M'n'Ms would have spurned him for wearing a luminous earring—it blinked through a series of words in tiny green letters—Fuck ... you ... if ... you ... don't ... like ... it ... Fuck ... you ... if—and they'd have considered that unforgivably ``Griddy.'' And anyway Carter's wide, froggish face didn't fit the svelte minimono look. He looked at Carmen. ``No girls, Harpie.''

``Drag queen,'' Rickenharp said. He slipped a folded twenty newbux note through the slot in the window. ``Okay?''

``Okay, but she takes her chances in there,'' Carter said, shrugging. He tucked the twenty in his charcoal bikini briefs.

``Sure.''

``You hear about Geary?''

``Nope.''

``Snuffed hisself with China White 'cause he got green pissed.''

``Oh, shit.'' Rickenharp's skin crawled. His paranoia flared up again, and to soothe it he said, ``Well, I'm not gonna be licking anybody's anything. I'm looking for Frankie.''

``That asshole. He's there, holding court or something. But you still got to pay admission, honey.''

``Sure,'' Rickenharp said.

He took another twenty newbux out of his pocket, but Carmen put a hand on his arm and said, ``We'll cover this one.'' She slapped a twenty down.

Carter took it, chuckling. ``Man, that queen got some real nice larynx work.'' Knowing damn well she was a girl. ``Hey, Rick, you still playing at the—''

``I blew the gig off,'' Rickenharp cut in, trying to head off the pain. The boss blue had peaked and left him feeling like he was made out of cardboard inside, like any pressure might make him buckle. His muscles twitched now and then, fretful as restive children scuffing feet. He was crashing. He needed another hit. When you were up, he thought, things showed you their frontsides, their upsides; when you peaked, things showed you their hideous insides. When you were down, things showed you their backsides, their downsides. File it away for lyrics.

Carter pressed the buzzer that unlocked the door. It razzed them as they walked through.

Inside it was dim, hot, humid.

``I think your blue was cut with coke or meth or something,'' Rickenharp told Carmen as they walked past the dented lockers. ``Cause I'm crashing harder than I should be.''

``Yeah, probably ... What'd he mean ‘he got green pissed'?''

``Positive test for AIDS-three. The HIV that kills you in three weeks. You drop this testing pill in your urine and if the urine turns green you got AIDS. There's no cure for the new HIV yet, won't be in three weeks, so the guy ... '' He shrugged.

``What the 'ell is this place?'' Willow asked.

In a low voice Rickenharp told him, ``It's a kind of bathless gay baths, man. Cruising places for 'mos. But about a lotta the people are straights who ran out of bux at the casinos, use it for a cheap place to sleep, you know?''

``Yeah? And 'ow come you know all about it, 'ey?''

Rickenharp smirked. ``You saying I'm gay? The horror, the horror.''

Someone in a darkened alcove to one side laughed at that.

Willow was arguing with Yukio in an undertone. ``Oi don't like it, that's all, fucking faggots got a million fucking diseases. Some side o' beef with a tan going to wank on me leg.''

``We just walk through, we don't touch,'' Yukio said. ``Rickenharp knows what to do.''

Rickenharp thought, Hope so.

Maybe Frankie could get them safely off Freezone, maybe not.

The walls were black pressboard. It was a maze like a tingler gallery but in the negative. There was a more ordinary red light; there was the peculiar scent that lots of skin on skin generates and the accretion of various smokes, aftershaves, cheap soap, and an ingrained stink of sweat and semen gone rancid. The walls stopped at ten feet up and the shadows gathered the ceiling into themselves, far overhead. It was a converted warehouse space, with a strange vibe of stratification: claustrophobia layered under agoraphobia. They passed mossy dark cruising warrens. Faces blurred by anonymity turned to monitor them as they passed, expressions cool as video cameras.

They strolled through the game room with its stained pool tables and stammering holo-games, its prized-open vending machines. Peeling from the walls between the machines were posters of men—caricatures with oversized genitals and muscles that seemed themselves a kind of sexual organ, faces like California surfers. Carmen bit her finger to keep from laughing at them, marveling at the idiosyncratic narcissism of the place.

They passed through a cruising room designed to look like a barn. Two men ministered to one another on a wooden bench inside a ``horse stall'' with wet fleshy noises. Willow and Yukio looked away. Carmen stared at the gay sex in fascination. Rickenharp walked past without reacting, led the way through other midnight nests of pawing men; past men sleeping on benches and couches, sleepily slapping unwanted hands away.

And found Frankie in the TV lounge.

The TV lounge was bright, well-lit, the walls cheerful yellow. The OmeGaity was cheap—there were no holo cubes. There were motel-standard living-room lamps on end tables; a couch; a regular color screen showing a rock video channel; and a bank of monitors on the wall. It was like emerging from the underworld. Frankie was sitting on the couch, waiting for customers.

Frankie dealt on a porta-terminal he'd plugged into a Grid-socket. The buyer gave him an account number or credit card; Frankie checked the account, transferred the funds into his own (registered as consultancy fees), and handed over the packets.

The walls of the lounge were inset with video monitors; one showed the orgy room, another a porn vid, another ran a Grid network satellite channel. On that one a newscaster was yammering about the attempted assassination, this time in technicki, and Rickenharp hoped Frankie wouldn't notice it and make the connection. Frankie the Mirror was into taking profit from whatever came along, and the SA paid for information.

Frankie sat on the torn blue vinyl couch, hunched over the pocket-sized terminal on the coffee table. Frankie's customer was a disco 'mo with a blue sharkfin flare, steroid muscles, and a white karate robe; the guy was standing to one side, staring at the little black canvas bag of blue packets on the coffee table as Frankie completed the transaction.

Frankie was black. His bald scalp had been painted with reflective chrome; his head was a mirror, reflecting the TV screens in fish-eye miniature. He wore a pinstriped three-piece gray suit. A real one, but rumpled and stained like he'd slept in it, maybe fucked in it. He was smoking a Nat Sherman cigarette, down to the gold filter. His synthcoke eyes were demonically red. He flashed a yellow grin at Rickenharp. He looked at Willow, Yukio, and Carmen, made a mocking scowl. ``Fucking narcs—get more fancy with their setups every day. Now they got four agents in here, one of 'em looks like my man Rickenharp, other three took like refugees and a computer designer. But that Jap hasn't got a camera. Gives him away.''

``What's this 'ere about—'' Willow began.

Rickenharp made a dismissive gesture that said, He isn't serious, dumbshit. ``I got two purchases to make,'' he announced and looked at Frankie's buyer. The buyer took his packet and melted back into the warrens.

``First off,'' Rickenharp said, taking his card from his wallet, ``I need some blue blow, three grams.''

``You got it, homeboy.'' Frankie ran a lightpen over the card, then punched a request for data on that account. The terminal asked for the private code number. Frankie handed the terminal to Rickenharp, who punched in his code, then erased it from visual. Then he punched to transfer funds to Frankie's account. Frankie took the terminal and double-checked the transfer. The terminal showed Rickenharp's adjusted balance and Frankie's gain.

``That's gonna eat up half your account, Harpie,'' Frankie said.

``I got some prospects.''

``I heard you and Mose parted company.''

``How'd you get that so fast?''

``Ponce was here buying.''

``Yeah, well—now I've dumped the dead weight, my prospects are even better.'' But as he said it he felt dead weight in his gut.

`` 'S your bux, man.'' Frankie reached into the canvas carry-on, took out three pre-weighed bags of blue powder. He looked faintly amused. Rickenharp didn't like the look. It seemed to say, I knew you'd come back, you sorry little wimp.

``Fuck off, Frankie,'' Rickenharp said, taking the packets.

``What's this sudden squall of discontent, my child?''

``None of your business, you smug bastard.''

Frankie's smugness tripled. He glanced speculatively at Carmen and Yukio and Willow. ``There's something more, right?''

``Yeah. We got a problem. My friends here—they're getting off the raft. They need to slip out the back way so Tom and Huck don't see 'em.''

``Mmm. What kind of net's out for them?''

``It's a private outfit. They'll be watching the copter port, everything legit ... ''

``We had another way off,'' Carmen said suddenly. ``But it was blown—''

Yukio silenced her with a look. She shrugged.

``Verr-rry mysterious,'' Frankie said. ``But there are safety limits to curiosity. Okay. Three grand gets you three berths on my next boat out. My boss's sending a team to pick up a shipment. I can probably get 'em on there. That's going east, though. You know? Not west or south or north. One direction and one only.''

``That's what we need,'' Yukio said, nodding, smiling. Like he was talking to a travel agent. ``East. Someplace Mediterranean.''

``Malta,'' Frankie said. ``Island of Malta. Best I can do.'' Yukio nodded. Willow shrugged. Carmen assented by her silence.

Rickenharp was sampling the goods. In the nose, to the brain, and right to work. Frankie watched him placidly. Frankie was a connoisseur of the changes drugs made in people. He watched the change of expression on Rickenharp's face. He watched Rickenharp's visible shift into ego drive.

``We're gonna need four berths, Frankie,'' Rickenharp said.

Frankie raised an eyebrow. ``You better decide after that shit wears off.''

``I decided before I took it,'' Rickenharp said, not sure if it was true.

Carmen was staring at him. He took her by the arm and said, ``Talk to you a minute?'' He led her out of the lounge, into the dark hallway. The skin of her arm was electrically sweet under his fingers. He wanted more. But he dropped his hand from her and said, ``Can you get the bux?''

She nodded. ``I got a fake card, dips into—well, it'll get it for us. I mean, for me and Yukio and Willow. I'd have to get authorization to bring you. And I can't do that.''

``Know what? I won't help you get out otherwise.''

``You don't know—''

``Yeah, I do. I'm ready to go. I just go back and get my guitar.''

``The guitar'll be a burden where we're going. We're going into occupied territory, to get where we want to be. You'd have to leave the guitar.''

He almost wavered at that. ``I'll check it into a locker. Pick it up someday. Thing is—if they watched us with that bird, they saw me with you. They'll assume I'm part of it. Look, I know what you're doing. The SA's looking for you. Right? So that means you're—''

``Okay, hold it, shit; keep your voice down. Look—I can see where maybe they marked you, so you got to get off the raft, too. Okay, you go with us to Malta. But then you—''

``I got to stay with you. The SA's everywhere. They marked me.''

She took a deep breath and let it out in a soft whistle through her teeth. She stared at the floor. ``You can't do it.'' She looked at him. ``You're not the type. You're a fucking artist.''

He laughed. ``You say that like it's the lowest insult you can come up with. Look—I can do it. I'm going to do it. The band is dead. I need to ... '' He shrugged helplessly. Then he reached up and took her sunglasses off, looked at her shadowed eyes. ``And when I get you alone I'm going to batter your cervix into jelly.''

She punched him hard in the shoulder. It hurt. But she was smiling. ``You think that kind of talk turns me on? Well, it does. But it's not going to get you into my pants. And as for going with us—What you think this is? You've seen too many movies.''

``The SA's marked me, remember? What else can I do?''

``That's not a good enough reason to ... to become part of this thing. You got to really believe in it, because it's hard. This is not a celebrity game show.''

``Jesus. Give me a break. I know what I'm doing.''

That was bullshit. He was trashed. He was blown. My computer's experiencing a power surge. Motherboard fried. Hell, then burn out the rest.

He was living a fantasy. But he wasn't going to admit it. He repeated, ``I know what I'm doing.''

She snorted. She stared at him. ``Okay,'' she said.

And after that everything was different.
