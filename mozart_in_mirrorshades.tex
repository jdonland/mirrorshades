\chapter{Mozart in Mirrorshades}
\chapterauthor{Bruce Sterling and Lewis Shiner}

This footloose time-travel fantasy emerged in a happy spirit of Movement camaraderie. Its headlong energy and aggressive political satire are sure signs of writers who feel they have points to make: points about America, about the Third World, about "development" and "exploitation." And a point about science fiction: that energy and fun are its natural birthrights.

The figure of Wolfgang Amadeus Mozart seems to have a special resonance for this decade, appearing in films, Broadway plays, and rock videos, as well as in SF. It's an interesting case of cultural synchronicity. Something is loose in the 1 980s. And we are all in it together.

\hrulefill

\firstletter{F}rom the hill north of the city, Rice saw eighteenth-century Salzburg spread out below him like a half-eaten lunch.

Huge cracking towers and swollen, bulbous storage tanks dwarfed the ruins of the St. Rupert Cathedral. Thick white smoke billowed from the refinery’s stacks. Rice could taste the familiar petrochemical tang from where he sat, under the leaves of a wilting oak.

The sheer spectacle of it delighted him. You didn’t sign up for a time-travel project, he thought, unless you had a taste for incongruity. Like the phallic pumping station lurking in the central square of the convent, or the ruler-straight elevated pipelines ripping through Salzburg’s maze of cobbled streets. A bit tough on the city, maybe, but that was hardly Rice’s fault. The temporal beam had focused randomly in the bedrock below Salzburg, forming an expandable bubble connecting this world to Rice’s own time.

This was the first time he’d seen the complex from outside its high chain-link fences. For two years, he’d been up to his neck getting the refinery operational. He’d directed teams all over the planet, as they caulked up Nantucket whalers to serve as tankers, or trained local pipefitters to lay down line as far away as the Sinai and the Gulf of Mexico.

Now, finally, he was outside. Sutherland, the company’s political liaison, had warned him against going into the city. But Rice had no patience with her attitude. The smallest thing seemed to set Sutherland off. She lost sleep over the most trivial local complaints. She spent hours haranguing the “gate people,” the locals who waited day and night outside the square-mile complex, begging for radios, nylons, a jab of penicillin.

To hell with her, Rice thought. The plant was up and breaking design records, and Rice was due for a little R and R. The way he saw it, anyone who couldn’t find some action in the Year of Our Lord 1775 had to be dead between the ears. He stood up, dusting windblown soot from his hands with a cambric handkerchief.

A moped sputtered up the hill toward him, wobbling crazily. The rider couldn’t seem to keep his high-heeled, buckled pumps on the pedals while carrying a huge portable stereo in the crook of his right arm. The moped lurched to a stop at a respectful distance, and Rice recognized the music from the tape player: Symphony No. 40 in G Minor.

The boy turned the volume down as Rice walked toward him. “Good evening, Mr. Plant Manager, sir. I am not interrupting?”

“No, that’s okay.” Rice glanced at the bristling hedgehog cut that had replaced the boy’s outmoded wig. He’d seen the kid around the gates; he was one of the regulars. But the music had made something else fall into place. “You’re Mozart, aren’t you?”

“Wolfgang Amadeus Mozart, your servant.”

“I’ll be goddamned. Do you know what that tape is?”

“It has my name on it.”

“Yeah. You wrote it. Or would have, I guess I should say. About fifteen years from now.”

Mozart nodded. “It is so beautiful. I have not the English to say how it is to hear it.”

By this time most of the other gate people would have been well into some kind of pitch. Rice was impressed by the boy’s tact, not to mention his command of English. The standard native vocabulary didn’t go much beyond radio, drugs, and fuck. “Are you headed back toward town?” Rice asked.

“Yes, Mr. Plant Manager, sir.”

Something about the kid appealed to Rice. The enthusiasm, the gleam in the eyes. And, of course, he did happen to be one of the greatest composers of all time.

“Forget the titles,” Rice said. “Where does a guy go for some fun around here?”
squares

At first Sutherland hadn’t wanted Rice at the meeting with Jefferson. But Rice knew a little temporal physics, and Jefferson had been pestering the American personnel with questions about time holes and parallel worlds.

Rice, for his part, was thrilled at the chance to meet Thomas Jefferson, the first President of the United States. He’d never liked George Washington, was glad the man’s Masonic connections had made him refuse to join the company’s “godless” American government.

Rice squirmed in his Dacron double knits as he and Sutherland waited in the newly air-conditioned boardroom of the Hohensalzburg Castle. “I forgot how greasy these suits feel,” he said.

“At least,” Sutherland said, “you didn’t wear that goddamned hat today.” The VTOL jet from America was late, and she kept looking at her watch.

“My tricorne?” Rice said. “You don’t like it?”

“It’s a Masonista hat, for Christ’s sake. It’s a symbol of anti-modern reaction.” The Freemason Liberation Front was another of Sutherland’s nightmares, a local politico-religious group that had made a few pathetic attacks on the pipeline.

“Oh, loosen up, will you, Sutherland? Some groupie of Mozart’s gave me the hat. Theresa Maria Angela something-or-other, some broken-down aristocrat. They all hang out together in this music dive downtown. I just liked the way it looked.”

“Mozart? You’ve been fraternizing with him? Don’t you think we should just let him be? After everything we’ve done to him?”

“Bullshit,” Rice said. “I’m entitled. I spent two years on startup while you were playing touch football with Robespierre and Thomas Paine. I make a few night spots with Wolfgang and you’re all over me. What about Parker? I don’t hear you bitching about him playing rock and roll on his late show every night. You can hear it blasting out of every cheap transistor in town.”

“He’s propaganda officer. Believe me, if I could stop him I would, but Parker’s a special case. He’s got connections all over the place back in Realtime.” She rubbed her cheek. “Let’s drop it, okay? Just try to be polite to President Jefferson. He’s had a hard time of it lately.”

Sutherland’s secretary, a former Hapsburg lady-in-waiting, stepped in to announce the plane’s arrival. Jefferson pushed angrily past her. He was tall for a local, with a mane of blazing red hair and the shiftiest eyes Rice had ever seen. “Sit down, Mr. President.” Sutherland waved at the far side of the table. “Would you like some coffee or tea?”

Jefferson scowled. “Perhaps some Madeira,” he said. “If you have it.”

Sutherland nodded to her secretary, who stared for a moment in incomprehension, then hurried off. “How was the flight?” Sutherland asked.

“Your engines are most impressive,” Jefferson said, “as you well know.” Rice saw the subtle trembling of the man’s hands; he hadn’t taken well to jet flight. “I only wish your political sensitivities were as advanced.”

“You know I can’t speak for my employers,” Sutherland said. “For myself, I deeply regret the darker aspects of our operations. Florida will be missed.”

Irritated, Rice leaned forward. “You’re not really here to discuss sensibilities, are you?”

“Freedom, sir,” Jefferson said. “Freedom is the issue.” The secretary returned with a dust-caked bottle of sherry and a stack of clear plastic cups. Jefferson, his hands visibly shaking now, poured a glass and tossed it back. Color returned to his face. He said, “You made certain promises when we joined forces. You guaranteed us liberty and equality and the freedom to pursue our own happiness. Instead we find your machinery on all sides, your cheap manufactured goods seducing the people of our great country, our minerals and works of art disappearing into your fortresses, never to reappear!” The last line brought Jefferson to his feet.

Sutherland shrank back into her chair. “The common good requires a certain period of, uh, adjustment—”

“Oh, come on, Tom,” Rice broke in. “We didn’t ‘join forces,’ that’s a lot of crap. We kicked the Brits out and you in, and you had damn-all to do with it. Second, if we drill for oil and carry off a few paintings, it doesn’t have a goddamned thing to do with your liberty. We don’t care. Do whatever you like, just stay out of our way. Right? If we wanted a lot of backtalk we could have left the damn British in power.”

Jefferson sat down. Sutherland meekly poured him another glass, which he drank off at once. “I cannot understand you,” he said. “You claim you come from the future, yet you seem bent on destroying your own past.”

“But we’re not,” Rice said. “It’s this way. History is like a tree, okay? When you go back and mess with the past, another branch of history splits off from the main trunk. Well, this world is just one of those branches.”

“So,” Jefferson said. “This world—my world—does not lead to your future.”

“Right,” Rice said.

“Leaving you free to rape and pillage here at will! While your own world is untouched and secure!” Jefferson was on his feet again. “I find the idea monstrous beyond belief, intolerable! Howcan you be party to such despotism? Have you no human feelings?”

“Oh, for God’s sake,” Rice said. “Of course we do. What about the radios and the magazines and the medicine we hand out? Personally I think you’ve got a lot of nerve, coming in here with your smallpox scars and your unwashed shirt and all those slaves of yours back home, lecturing us on humanity.”

“Rice?” Sutherland said.

Rice locked eyes with Jefferson. Slowly, Jefferson sat down. “Look,” Rice said, relenting. “We don’t mean to be unreasonable. Maybe things aren’t working out just the way you pictured them, but hey, that’s life, you know? What do you want, really? Cars? Movies? Telephones? Birth control? Just say the word and they’re yours.”

Jefferson pressed his thumbs into the corners of his eyes. “Your words mean nothing to me, sir. I only want ... I want only to return to my home. To Monticello. And as soon as possible.”

“Is it one of your migraines, Mr. President?” Sutherland asked. “I had these made up for you.” She pushed a vial of pills across the table toward him.

“What are these?”

Sutherland shrugged. “You’ll feel better.”

After Jefferson left, Rice half expected a reprimand. Instead, Sutherland said, “You seem to have a tremendous faith in the project.”

“Oh, cheer up,” Rice said. “You’ve been spending too much time with these politicals. Believe me, this is a simple time, with simple people. Sure, Jefferson was a little ticked off, but he’ll come around. Relax!”
squares

Rice found Mozart clearing tables in the main dining hall of the Hohensalzburg Castle. In his faded jeans, camo jacket, and mirrored sunglasses, he might almost have passed for a teenager from Rice’s time.

“Wolfgang!” Rice called to him. “How’s the new job?”

Mozart set a stack of dishes aside and ran his hands over his short-cropped hair. “Wolf,” he said. “Call me Wolf, okay? Sounds more ... modern, you know? But yes, I really want to thank you for everything you have done for me. The tapes, the history books, this job—it is so wonderful just to be around here.”

His English, Rice noticed, had improved remarkably in the last three weeks. “You still living in the city?”

“Yes, but I have my own place now. You are coming to the gig tonight?”

“Sure,” Rice said. “Why don’t you finish up around here, I’ll go change, and then we can go out for some sachertorte, okay? We’ll make a night of it.”

Rice dressed carefully, wearing mesh body armor under his velvet coat and knee britches. He crammed his pockets with giveaway consumer goods, then met Mozart by a rear door.

Security had been stepped up around the castle, and floodlights swept the sky. Rice sensed a new tension in the festive abandon of the crowds downtown.

Like everyone else from his time, he towered over the locals; even incognito he felt dangerously conspicuous.

Within the club Rice faded into the darkness and relaxed. The place had been converted from the lower half of some young aristo’s town house; protruding bricks still marked the lines of the old walls. The patrons were locals, mostly, dressed in any Realtime garments they could scavenge. Rice even saw one kid wearing a pair of beige silk panties on his head.

Mozart took the stage. Minuet-like guitar arpeggios screamed over sequenced choral motifs. Stacks of amps blasted synthesizer riffs lifted from a tape of K-Tel pop hits. The howling audience showered Mozart with confetti stripped from the club’s hand-painted wallpaper.

Afterward Mozart smoked a joint of Turkish hash and asked Rice about the future.

“Mine, you mean?” Rice said. “You wouldn’t believe it. Six billion people, and nobody has to work if they don’t want to. Five-hundred-channel TV in every house. Cars, helicopters, clothes that would knock your eyes out. Plenty of easy sex. You want music? You could have your own recording studio. It’d make your gear on stage look like a goddamned clavichord.”

“Really? I would give anything to see that. I can’t understand why you would leave.”

Rice shrugged. “So I’m giving up maybe fifteen years. When I get back, it’s the best of everything. Anything I want.”

“Fifteen years?”

“Yeah. You got to understand how the portal works. Right now it’s as big around as you are tall, just big enough for a phone cable and a pipeline full of oil, maybe the odd bag of mail, heading for Realtime. To make it any bigger, like to move people or equipment through, is expensive as hell. So expensive they only do it twice, at the beginning and the end of the project. So, yeah, I guess we’re stuck here.”

Rice coughed harshly and drank off his glass. That Ottoman Empire hash had untied his mental shoelaces. Here he was opening up to Mozart, making the kid want to emigrate, and there was no way in hell Rice could get him a Green Card. Not with all the millions that wanted a free ride into the future—billions, if you counted the other projects, like the Roman Empire or New Kingdom Egypt.

“But I’m really glad to be here,” Rice said. “It’s like ... like shuffling the deck of history. You never know what’ll come up next.” Rice passed the joint to one of Mozart’s groupies, Antonia something-or-other. “This is a great time to be alive. Look at you. You’re doing okay, aren’t you?” He leaned across the table, in the grip of a sudden sincerity. “I mean, it’s okay, right? It’s not like you hate all of us for fucking up your world or anything?”

“Are you making a joke? You are looking at the hero of Salzburg. In fact, your Mr. Parker is supposed to make a tape of my last set tonight. Soon all of Europe will know of me!” Someone shouted at Mozart, in German, from across the club. Mozart glanced up and gestured cryptically. “Be cool, man.” He turned back to Rice. “You can see that I am doing fine.”

“Sutherland, she worries about stuff like all those symphonies you’re never going to write.”

“Bullshit! I don’t want to write symphonies. I can listen to them any time I want! Who is this Sutherland? Is she your girlfriend?”

“No. She goes for the locals. Danton, Robespierre, like that. How about you? You got anybody?”

“Nobody special. Not since I was a kid.”

“Oh, yeah?”

“Well, when I was about six I was at Maria Theresa’s court. I used to play with her daughter—Maria Antonia. Marie Antoinette she calls herself now. The most beautiful girl of the age. We used to play duets. We made a joke that we would be married, but she went off to France with that swine, Louis.”

“Goddamn,” Rice said. “This is really amazing. You know, she’s practically a legend where I come from. They cut her head off in the French Revolution for throwing too many parties.”

“No they didn’t.... “

“That was our French Revolution,” Rice said. “Yours was a lot less messy.”

“You should go see her, if you’re that interested. Surely she owes you a favor for saving her life. “

Before Rice could answer, Parker arrived at their table, surrounded by ex-ladies-in-waiting in spandex capris and sequined tube tops. “Hey, Rice,” Parker shouted, serenely anachronistic in a glitter T-shirt and black leather jeans. “Where did you get those unhip threads? Come on, let’s party!”

Rice watched as the girls crowded around the table and gnawed the corks out of a crate of champagne. As short, fat, and repulsive as Parker might be, they would gladly knife one another for a chance to sleep in his clean sheets and raid his medicine cabinet.

“No, thanks,” Rice said, untangling himself from the miles of wire connected to Parker’s recording gear.

The image of Marie Antoinette had seized him and would not let go.
squares

Rice sat naked on the edge of the canopied bed, shivering a little in the air conditioning. Past the jutting window unit, through clouded panes of eighteenth-century glass, he saw a lush, green landscape sprinkled with tiny waterfalls.

At ground level, a garden crew of former aristos in blue denim overalls trimmed weeds under the bored supervision of a peasant guard. The guard, clothed head to foot in camouflage except for a tricolor cockade on his fatigue cap, chewed gum and toyed with the strap of his cheap plastic machine gun. The gardens of Petit Trianon, like Versailles itself, were treasures deserving the best of care. They belonged to the Nation, since they were too large to be crammed through a time portal.

Marie Antoinette sprawled across the bed’s expanse of pink satin, wearing a scrap of black-lace underwear and leafing through an issue of Vogue. The bedroom’s walls were crowded with Boucher canvases: acres of pert silky rumps, pink haunches, knowingly pursed lips. Rice looked dazedly from the portrait of Louise O’Morphy, kittenishly sprawled on a divan, to the sleek, creamy expanse of Toinette’s back and thighs. He took a deep, exhausted breath. “Man,” he said, “that guy could really paint.”

Toinette cracked off a square of Hershey’s chocolate and pointed to the magazine. “I want the leather bikini,” she said. “Always, when I am a girl, my goddamn mother, she keep me in the goddamn corsets. She think my what-you-call, my shoulder blade sticks out too much. “

Rice leaned back across her solid thighs and patted her bottom reassuringly. He felt wonderfully stupid; a week and a half of obsessive carnality had reduced him to a euphoric animal “Forget your mother, baby. You’re with me now. You want ze goddamn leather bikini, I get it for you.”

Toinette licked chocolate from her fingertips. “Tomorrow we go out to the cottage, okay, man? We dress up like the peasants and make love in the hedges like noble savages.”

Rice hesitated. His weekend furlough to Paris had stretched into a week and a half; by now security would be looking for him. To hell with them, he thought. “Great,” he said. “I’ll phone us up a picnic lunch. Foie gras and truffles, maybe some terrapin—”

Toinette pouted. “I want the modem food. The pizza and burritos and the chicken fried.” When Rice shrugged, she threw her arms around his neck. “You love me, Rice?”

“Love you? Baby, I love the very idea of you.” He was drunk on history out of control, careening under him like some great black motorcycle of the imagination. When he thought of Paris, take-out quiche-to-go stores springing up where guillotines might have been, a six-year-old Napoleon munching Dubble Bubble in Corsica, he felt like the archangel Michael on speed.

Megalomania, he knew, was an occupational hazard. But he’d get back to work soon enough, in just a few more days....

The phone rang. Rice burrowed into a plush house robe formerly owned by Louis XVI. Louis wouldn’t mind; he was now a happily divorced locksmith in Nice.

Mozart’s face appeared on the phone’s tiny screen. “Hey, man, where are you?”

“France,” Rice said vaguely. “What’s up?”

“Trouble, man. Sutherland flipped out, and they’ve got her sedated. At least six key people have gone over the hill, counting you.” Mozart’s voice had only the faintest trace of accent left.

“Hey, I’m not over the hill. I’ll be back in just a couple days. We’ve got, what, thirty other people in Northern Europe? If you’re worried about the quotas—”

“Fuck the quotas. This is serious. There’s uprisings. Comanches raising hell on the rigs in Texas. Labor strikes in London and Vienna. Realtime is pissed. They’re talking about pulling us out.”

“What?” Now he was alarmed.

“Yeah. Word came down the line today. They say you guys let this whole operation get sloppy. Too much contamination, too much fraternization. Sutherland made a lot of trouble with the locals before she got found out. She was organizing the Masonistas for some kind of passive resistance and God knows what else.”

“Shit.” The fucking politicals had screwed it up again. It wasn’t enough that he’d busted ass getting the plant up and on line; now he had to clean up after Sutherland. He glared at Mozart. “Speaking of fraternization, what’s all this we stuff? What the hell are you doing calling me?”

Mozart paled. “Just trying to help. I got a job in communications now.”

“That takes a Green Card. Where the hell did you get that?”

“Uh, listen, man, I got to go. Get back here, will you? We need you.” Mozart’s eyes flickered, looking past Rice’s shoulder. “You can bring your little time-bunny along if you want. But hurry.”

“I ... oh, shit, okay,” Rice said.
squares

Rice’s hovercar huffed along at a steady 80 kph, blasting clouds of dust from the deeply rutted highway. They were near the Bavarian border. Ragged Alps jutted into the sky over radiant green meadows, tiny picturesque farmhouses, and clear, vivid streams of melted snow.

They’d just had their first argument. Toinette had asked for a Green Card, and Rice had told her he couldn’t do it. He offered her a Gray Card instead, that would get her from one branch of time to another without letting her visit Realtime. He knew he’d be reassigned if the project pulled out, and he wanted to take her with him. He wanted to do the decent thing, not leave her behind in a world without Hersheys and Vogues.

But she wasn’t having any of it. After a few kilometers of weighty silence she started to squirm. “I have to pee,” she said finally. “Pull over by the goddamn trees.”

“Okay,” Rice said. “Okay.”

He cut the fans and whirred to a stop. A herd of brindled cattle spooked off with a clank of cowbells. The road was deserted.

Rice got out and stretched, watching Toinette climb a wooden stile and walk toward a stand of trees.

“What’s the deal?” Rice yelled. “There’s nobody around. Get on with it!”

A dozen men burst up from the cover of a ditch and rushed him. In an instant they’d surrounded him, leveling flintlock pistols. They wore tricornes and wigs and lace-cuffed highwayman’s coats; black domino masks hid their faces. “What the fuck is this?” Rice asked, amazed. “Mardi Gras?”

The leader ripped off his mask and bowed ironically. His handsome Teutonic features were powdered, his lips rouged. “I am Count Axel Ferson. Servant, sir.”

Rice knew the name; Ferson had been Toinette’s lover before the Revolution. “Look, Count, maybe you’re a little upset about Toinette, but I’m sure we can make a deal. Wouldn’t you really rather have a color TV?”

“Spare us your satanic blandishments, sir!” Ferson roared. “I would not soil my hands on the collaborationist cow. We are the Freemason Liberation Front!”

“Christ,” Rice said. “You can’t possibly be serious. Are you taking on the project with these popguns?”

“We are aware of your advantage in armaments, sir. This is why we have made you our hostage.” He spoke to the others in German. They tied Rice’s hands and hustled him into the back of a horse-drawn wagon that had clopped out of the woods.

“Can’t we at least take the car?” Rice asked. Glancing back, he saw Toinette sitting dejectedly in the road by the hovercraft.

“We reject your machines,” Ferson said. “They are one more facet of your godlessness. Soon we will drive you back to hell, from whence you came!”

“With what? Broomsticks?” Rice sat up in the back of the wagon, ignoring the stink of manure and rotting hay. “Don’t mistake our kindness for weakness. If they send the Gray Card Army through that portal, there won’t be enough left of you to fill an ashtray.”

“We are prepared to sacrifice! Each day thousands flock to our worldwide movement, under the banner of the All-Seeing Eye! We shall reclaim our destiny! The destiny you have stolen from us!”

“Your destiny?” Rice was aghast. “Listen, Count, you ever hear of guillotines?”

“I wish to hear no more of your machines.” Ferson gestured to a subordinate. “Gag him.”
squares

They hauled Rice to a farmhouse outside Salzburg. During fifteen bone-jarring hours in the wagon he thought of nothing but Toinette’s betrayal. If he’d promised her the Green Card, would she still have led him into the ambush? That card was the only thing she wanted, but how could the Masonistas get her one?

Rice’s guards paced restlessly in front of the windows, their boots squeaking on the loosely pegged floorboards. From their constant references to Salzburg he gathered that some kind of siege was in progress.

Nobody had shown up to negotiate Rice’s release, and the Masonistas were getting nervous. If he could just gnaw through his gag, Rice was sure he’d be able to talk some sense into them.

He heard a distant drone, building slowly to a roar. Four of the men ran outside, leaving a single guard at the open door. Rice squirmed in his bonds and tried to sit up.

Suddenly the clapboards above his head were blasted to splinters by heavy machine-gun fire. Grenades whumped in front of the house, and the windows exploded in a gush of black smoke. A choking Masonista lifted his flintlock at Rice. Before he could pull the trigger a burst of gunfire threw the terrorist against the wall.

A short, heavyset man in flak jacket and leather pants stalked into the room. He stripped goggles from his smoke-blackened face, revealing Oriental eyes. A pair of greased braids hung down his back. He cradled an assault rifle in the crook of one arm and wore two bandoliers of grenades. “Good,” he grunted. “The last of them.” He tore the gag from Rice’s mouth. He smelled of sweat and smoke and badly cured leather. “You are Rice?”

Rice could only nod and gasp for breath.

His rescuer hauled him to his feet and cut his ropes with a bayonet. “I am Jebe Noyon. Trans-Temporal Army.” He forced a leather flask of rancid mare’s milk into Rice’s hands. The smell made Rice want to vomit. “Drink!” Jebe insisted. “Is koumiss, is good for you! Drink, Jebe Noyon tells you!”

Rice took a sip, which curdled his tongue and brought bile to his throat. “You’re the Gray Cards, right?” he said weakly.

“Gray Card Army, yes,” Jebe said. “Baddest-ass warriors of all times and places! Only five guards here, I kill them all! I, Jebe Noyon, was chief general to Genghis Khan, terror of the earth, okay, man?” He stared at Rice with great, sad eyes. “You have not heard of me. “

“Sorry, Jebe, no.”

“The earth turned black in the footprints of my horse.”

“I’m sure it did, man.”

“You will mount up behind me,” he said, dragging Rice toward the door. “You will watch the earth turn black in the tireprints of my Harley, man, okay?”
squares

From the hills above Salzburg they looked down on anachronism gone wild.

Local soldiers in waistcoats and gaiters lay in bloody heaps by the gates of the refinery. Another battalion marched forward in formation, muskets at the ready. A handful of Huns and Mongols, deployed at the gates, cut them up with orange tracer fire and watched the survivors scatter.

Jebe Noyon laughed hugely. “Is like siege of Cambaluc! Only no stacking up heads or even taking ears any more, man, now we are civilized, okay? Later maybe we call in, like, grunts, choppers from ‘Nam, napalm the son-of-a-bitches, far out, man.”

“You can’t do that, Jebe,” Rice said sternly. “The poor bastards don’t have a chance. No point in exterminating them.”

Jebe shrugged. “I forget sometimes, okay? Always thinking to conquer the world.” He revved the cycle and scowled. Rice grabbed the Mongol’s stinking flak jacket as they roared downhill. Jebe took his disappointment out on the enemy, tearing through the streets in high gear, deliberately running down a group of Brunswick grenadiers. Only panic strength saved Rice from falling off as legs and torsos thumped and crunched beneath their tires.

Jebe skidded to a stop inside the gates of the complex. A jabbering horde of Mongols in ammo belts and combat fatigues surrounded them at once. Rice pushed through them, his kidneys aching.

Ionizing radiation smeared the evening sky around the Hohensalzburg Castle. They were kicking the portal up to the high-energy maximum, running cars full of Gray Cards in and sending the same cars back loaded to the ceiling with art and jewelry.

Over the rattling of gunfire Rice could hear the whine of VTOL jets bringing in the evacuees from the US and Africa. Roman centurions, wrapped in mesh body armor and carrying shoulder-launched rockets, herded Realtime personnel into the tunnels that led to the portal.

Mozart was in the crowd, waving enthusiastically to Rice. “We’re pulling out, man! Fantastic, huh? Back to Realtime!”

Rice looked at the clustered towers of pumps, coolers, and catalytic cracking units. “It’s a goddamned shame,” he said. “All that work, shot to hell.”

“We were losing too many people, man. Forget it. There’s plenty of eighteenth centuries.”

The guards, sniping at the crowds outside, suddenly leaped aside as Rice’s hovercar burst through the gates. Half a dozen Masonic fanatics still clung to the doors and pounded on the windscreen. Jebe’s Mongols yanked the invaders free and axed them while a Roman flamethrower unit gushed fire across the gates.

Marie Antoinette leaped out of the hovercar. Jebe grabbed for her, but her sleeve came off in his hand. She spotted Mozart and ran for him, Jebe only a few steps behind.

“Wolf, you bastard!” she shouted. “You leave me behind! What about your promises, you merde, you pig-dog!”

Mozart whipped off his mirrorshades. He turned to Rice. “Who is this woman?”

“The Green Card, Wolf! You say I sell Rice to the Masonistas, you get me the card!” She stopped for breath and Jebe caught her by one arm. When she whirled on him, he cracked her across the jaw, and she dropped to the tarmac.

The Mongol focused his smoldering eyes on Mozart. “Was you, eh? You, the traitor?” With the speed of a striking cobra he pulled his machine pistol and jammed the muzzle against Mozart’s nose. “I put my gun on rock and roll, there nothing left of you but ears, man.”

A single shot echoed across the courtyard. Jebe’s head rocked back, and he fell in a heap.

Rice spun to his right. Parker, the DJ, stood in the doorway of an equipment shed. He held a Walther PPK. “Take it easy, Rice,” Parker said, walking toward him. “He’s just a grunt, expendable.”

“You killed him!”

“So what?” Parker said, throwing one arm around Mozart’s frail shoulders. “This here’s my boy! I transmitted a couple of his new tunes up the line a month ago. You know what? The kid’s number five on the Billboard charts! Number five!” Parker shoved the gun into his belt. “With a bullet!”

“You gave him the Green Card, Parker?”

“No,” Mozart said. “It was Sutherland.”

“What did you do to her?”

“Nothing! I swear to you, man! Well, maybe I kind of lived up to what she wanted to see. A broken man, you know, his music stolen from him, his very soul?” Mozart rolled his eyes upward. “She gave me the Green Card, but that still wasn’t enough. She couldn’t handle the guilt. You know the rest.”

“And when she got caught, you were afraid we wouldn’t pull out. So you decided to drag me into it! You got Toinette to turn me over to the Masons. That was your doing!”

As if hearing her name, Toinette moaned softly from the tarmac. Rice didn’t care about the bruises, the dirt, the rips in her leopard-skin jeans. She was still the most gorgeous creature he’d ever seen.

Mozart shrugged. “I was a Freemason once. Look, man, they’re very uncool. I mean, all I did was drop a few hints, and look what happened.” He waved casually at the carnage all around them. “I knew you’d get away from them somehow.”

“You can’t just use people like that!”

“Bullshit, Rice! You do it all the time! I needed this seige so Realtime would haul us out! For Christ’s sake, I can’t wait fifteen years to go up the line. History says I’m going to be dead in fifteen years! I don’t want to die in this dump! I want that car and that recording studio!”

“Forget it, pal,” Rice said. “When they hear back in Realtime how you screwed things up here—”

Parker laughed. “Shove off, Rice. We’re talking Top of the Pops, here. Not some penny-ante refinery.” He took Mozart’s arm protectively. “Listen, Wolf, baby, let’s get into those tunnels. I got some papers for you to sign as soon as we hit the future.”

The sun had set, but muzzle-loading cannon lit the night, pumping shells into the city. For a moment Rice stood stunned as cannonballs clanged harmlessly off the storage tanks. Then, finally, he shook his head. Salzburg’s time had run out.

Hoisting Toinette over one shoulder, he ran toward the safety of the tunnels.