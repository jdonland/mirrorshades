\chapter{Petra}
\chapterauthor{Greg Bear}

Greg Bear sold his first short story in 1966—at the age of fifteen. He hit his stride in the late Seventies and early Eighties, when a flurry of short stories and novels established him as a writer to watch.

Bear's work is strongly rooted in the best of SF's intellectual tradition. A prolific yet disciplined writer, he prizes speculative rigor and respect for scientific fact. This attitude linked him with traditional hard SF—despite much well-praised work in fantasy.

As his career developed, Bear's great imaginative gifts came strongly to the fore, given even greater impact by the disciplined craftsmanship he learned early on. The combination has produced a genuinely radical hard SF with extreme visionary power, demonstrated in such widely lauded novels as Blood Music and Eon.

The following story, published early in 1982, marked Bear's quantum leap past traditional limits into a mind-boggling new realm. With its thorough and detailed development of a truly fantastic concept, it shows Bear's technique at its best.

\textit{```God is dead, God is dead'... Perdition! When God dies, you'll know it.''}

—Confessions of St. Argentine

\hrulefill

\firstletter{I}'m an ugly son of stone and flesh, there's no denying it. I don't remember my mother. It's possible she abandoned me shortly after my birth. More than likely she is dead. My father—ugly beaked, half-winged thing, if he resembles his son—I have never seen.

Why should such an unfortunate aspire to be a historian? I think I can trace the moment my choice was made. It's among my earliest memories, and it must have happened about thirty years ago, though I'm sure I lived many years before that—years now lost to me. I was squatting behind thick, dusty curtains in a vestibule, listening to a priest instructing other novitiates, all of pure flesh, about Mortdieu. His words are still vivid.

``As near as I can discover,'' he said, ``Mortdieu occurred about seventy-seven years ago. Learned ones deny that magic was set loose on the world, but few deny that God, as such, had died.''

Indeed. That's putting it mildly. All the hinges of our once-great universe fell apart, the axis tilted, cosmic doors swung shut, and the rules of existence lost their foundations. The priest continued in measured, awed tones to describe that time.

``I have heard wise men speak of the slow decline. Where human thought was strong, reality's sudden quaking was reduced to a tremor. Where thought was weak, reality disappeared completely, swallowed by chaos. Every delusion became as real as solid matter.'' His voice trembled with emotion. ``Blinding pain, blood catching fire in our veins, bones snapping and flesh powdering. Steel flowing like liquid. Amber raining from the sky. Crowds gathering in streets that no longer followed any maps, if the maps themselves had not altered. They knew not what to do. Their weak minds could not grab hold...''

Most humans, I take it, were entirely too irrational to begin with. Whole nations vanished or were turned into incomprehensible whirlpools of misery and depravity. It is said that certain universities, libraries, and museums survived, but to this day we have little contact with them.

I think often of those poor victims of the early days of Mortdieu. They had known a world of some stability; we have adapted since. They were shocked by cities turning into forests, by their nightmares taking shape before their eyes. Prodigal crows perched atop trees that had once been buildings, pigs ran through the streets on their hind legs... and so on. (The priest did not encourage contemplation of the oddities. ``Excitement,'' he said, ``breeds even more monsters.'') Our Cathedral survived. Rationality in this neighborhood, however, had weakened some centuries before Mortdieu, replaced only by a kind of rote. The Cathedral suffered. Survivors—clergy and staff, worshipers seeking sanctuary—had wretched visions, dreamed wretched dreams. They saw the stone ornaments of the Cathedral come alive. With someone to see and believe, in a universe lacking any other foundation, my ancestors shook off stone and became flesh. Centuries of stone celibacy weighed upon them. Forty-nine nuns who had sought shelter in the Cathedral were discovered and were not entirely loath, so the coarser versions of the tale go. Mortdieu had had a surprising aphrodisiacal effect on the faithful and conjugation took place.

No definite gestation period has been established, for at that time the great stone wheel had not been set twisting back and forth to count the hours. Nor had anyone been given the chair of Kronos to watch over the wheel and provide a baseline for everyday activities.

But flesh did not reject stone, and there came into being the sons and daughters of flesh and stone, including me. Those who had fornicated with the inhuman figures were cast out to raise or reject their monstrous young in the highest hidden recesses. Those who had accepted the embraces of the stone saints and other human figures were less abused but still banished to the upper reaches. A wooden scaffolding was erected, dividing the great nave into two levels. A canvas drop cloth was fastened over the scaffold to prevent offal raining down, and on the second level of the Cathedral the more human offspring of stone and flesh set about creating a new life.

I have long tried to find out how some semblance of order came to the world. Legend has it that it was the archexistentialist Jansard crucifier of the beloved St. Argentine—who, realizing and repenting his error, discovered that mind and thought could calm the foaming sea of reality.

The priest finished his all-too-sketchy lecture by touching on this point briefly: ``With the passing of God's watchful gaze, humanity had to reach out and grab hold the unraveling fabric of the world. Those left alive—those who had the wits to keep their bodies from falling apart—became the only cohesive force in the chaos.''

I had picked up enough language to understand what he said; my memory was good—still is—and I was curious enough to want to know more.

Creeping along stone walls behind the curtains, I listened to other priests and nuns intoning scripture to gaggles of flesh children. That was on the ground floor, and I was in great danger; the people of pure flesh looking on my kind as abominations. But it was worth it.

I was able to steal a Psalter and learned to read. I stole other books; they defined my world by allowing me to compare it with others. At first I couldn't believe the others had ever existed; only the Cathedral was real. I still have my doubts. I can look out a tiny round window on one side of my room and see the great forest and river that surround the Cathedral, but I can see nothing else. So my experience with other worlds is far from direct.

No matter. I read a great deal, but I'm no scholar. What concerns me is recent history—the final focus of that germinal hour listening to the priest. From the metaphysical to the acutely personal.

I am small—barely three English feet in height—but I can run quickly through most of the hidden passageways. This lets me observe without attracting attention. I may be the only historian in this whole structure. Others who claim the role disregard what's before their eyes, in search of ultimate truths, or at least Big Pictures. So if you prefer history where the historian is not involved, look to the others. Objective as I try to be, I do have my favorite subjects.

\fancybreak{* * *}

In the time when my history begins, the children of stone and flesh were still searching for the Stone Christ. Those of us born of the union of the stone saints and gargoyles with the bereaved nuns thought our salvation lay in the great stone celibate, who came to life as all the other statues had.

Of smaller import were the secret assignations between the bishop's daughter and a young man of stone and flesh. Such assignations were forbidden even between those of pure flesh; and as these two lovers were unmarried, their compound sin intrigued me.

Her name was Constantia, and she was fourteen, slender of limb, brown of hair, mature of bosom. Her eyes carried the stupid sort of divine life common in girls that age. His name was Corvus, and he was fifteen. I don't recall his precise features, but he was handsome enough and dexterous: he could climb through the scaffolding almost as quickly as I. I first spied them talking when I made one of my frequent raids on the repository to steal another book. They were in shadow, but my eyes are keen. They spoke softly, hesitantly. My heart ached to see them and to think of their tragedy, for I knew right away that Corvus was not pure flesh and that Constantia was the daughter of the bishop himself. I envisioned the old tyrant meting out the usual punishment to Corvus for such breaches of level and morality—castration. But in their talk was a sweetness that almost masked the closed-in stench of the lower nave.

``Have you ever kissed a man before?''

``Yes.''

``Who?''

``My brother.'' She laughed.

``And?'' His voice was sharper; he might kill her brother, he seemed to say.

``A friend named Jules.''

``Where is he?''

``Oh, he vanished on a wood-gathering expedition.''

``Oh.'' And he kissed her again.

I'm a historian, not a voyeur, so I discreetly hide the flowering of their passion. If Corvus had had any sense, he would have reveled in his conquest and never returned. But he was snared and continued to see her despite the risk. This was loyalty, love, faithfulness, and it was rare. It fascinated me.

\fancybreak{* * *}

I have just been taking in sun, a nice day, and looking out over the buttresses.

The Cathedral is like a low-bellied lizard, the nave its belly, the buttresses its legs. There are little houses at the base of each buttress, where rainspouters with dragon faces used to lean out over the trees (or city or whatever was down below once). Now people live there. It wasn't always that way—the sun was once forbidden. Corvus and Constantia from childhood were denied its light, and so even in their youthful prime they were pale and dirty with the smoke of candles and tallow lamps. The most sun anyone received in those days was obtained on wood-gathering expeditions.

After spying on one of the clandestine meetings of the young lovers, I mused in a dark corner for an hour, then went to see the copper giant Apostle Thomas. He was the only human form to live so high in the Cathedral. He carried a ruler on which was engraved his real name—he had been modeled after the Cathedral's restorer in times past, the architect Viollet-le-Duc. He knew the Cathedral better than anyone, and I admired him greatly. Most of the monsters left him alone—out of fear, if nothing else. He was huge, black as night, but flaked with pale green, his face creased in eternal thought. He was sitting in his usual wooden compartment near the base of the spire, not twenty feet from where I write now, thinking about times none of the rest of us ever knew: of joy and past love, some say; others say of the burden that rested on him now that the Cathedral was the center of this chaotic world.

It was the giant who selected me from the ugly hordes when he saw me with a Psalter. He encouraged me in my efforts to read. ``Your eyes are bright,'' he told me. ``You move as if your brain were quick, and you keep yourself dry and clean. You aren't hollow like the rainspouters—you have substance. For all our sakes, put it to use and learn the ways of the Cathedral.''

And so I did.

He looked up as I came in. I sat on a box near his feet and said, ``A daughter of flesh is seeing a son of stone and flesh.''

He shrugged his massive shoulders. ``So it shall be, in time.''

``Is it not a sin?''

``It is something so monstrous it is past sin and become necessity,'' he said. ``It will happen more as time passes.''

``They're in love, I think, or will be.''

He nodded. ``I—and One Other—were the only ones to abstain from fornication on the night of Mortdieu,'' he said. ``I am—except for the Other—alone fit to judge.''

I waited for him to judge, but he sighed and patted me on the shoulder. ``And I never judge, do I, ugly friend?''

``Never,'' I said.

``So leave me alone to be sad.'' He winked. ``And more power to them.''

The bishop of the Cathedral was an old, old man. It was said he hadn't been bishop before the Mortdieu, but a wanderer who came in during the chaos, before the forest had replaced the city. He had set himself up as titular head of this section of God's former domain by saying it had been willed to him.

He was short, stout, with huge hairy arms like the clamps of a vise. He had once killed a spouter with a single squeeze of his fist, and spouters are tough things, since they have no guts like you (I suppose) and I. The hair surrounding his bald pate was white, thick, and unruly, and his eyebrows leaned over his nose with marvelous flexibility. He rutted like a pig, ate hugely, and shat liquidly (I know all). A man for this time, if ever there was one.

It was his decree that all those not pure of flesh be banned and that those not of human form be killed on sight.

When I returned from the giant's chamber, I saw that the lower nave was in an uproar. They had seen someone clambering about in the scaffold, and troops had been sent to shoot him down. Of course it was Corvus. I was a quicker climber than he and knew the beams better, so when he found himself trapped in an apparent cul-de-sac, it was I who gestured from the shadows and pointed to a hole large enough for him to escape through. He took it without a breath of thanks, but etiquette has never been important to me. I entered the stone wall through a nook a spare hand's width across and wormed my way to the bottom to see what else was happening. Excitement was rare.

A rumor was passing that the figure had been seen with a young girl, but the crowds didn't know who the girl was. The men and women who mingled in the smoky light, between the rows of open-roofed hovels, chattered gaily. Castrations and executions were among the few joys for us then; I relished them too, but I had a stake in the potential victims now and I worried.

My worry and my interest got the better of me. I slid through an unrepaired gap and fell to one side of the alley between the outer wall and the hovels. A group of dirty adolescents spotted me. ``There he is!'' they screeched. ``He didn't get away!''

The bishop's masked troops can travel freely on all levels. I was almost cornered by them, and when I tried one escape route, they waited at a crucial spot in the stairs—which I had to cross to complete the next leg—and I was forced back. I prided myself in knowing the Cathedral top to bottom, but as I scrambled madly, I came upon a tunnel I had never noticed before. It led deep into a broad stone foundation wall. I was safe for the moment but afraid that they might find my caches of food and poison my casks of rainwater. Still, there was nothing I could do until they had gone, so I decided to spend the anxious hours exploring the tunnel.

The Cathedral is a constant surprise; I realize now I didn't know half of what it offered. There are always new ways to get from here to there (some, I suspect, created while no one is looking), and sometimes even new theres to be discovered. While troops snuffled about the hole above, near the stairs—where only a child of two or three could have entered—I followed a flight of crude steps deep into the stone. Water and slime made the passage slippery and difficult. For a moment I was in darkness deeper than any I had experienced before—a gloom more profound than mere lack of light could explain. Then below I saw a faint yellow gleam. More cautious, I slowed and progressed silently. Behind a rusting, scabrous metal gate, I set foot into a lighted room. There was the smell of crumbling stone, a tang of mineral water, slime—and the stench of a dead spouter. The beast lay on the floor of the narrow chamber, several months gone but still fragrant.

I have mentioned that spouters are very hard to kill—and this one had been murdered. Three candles stood freshly placed in nooks around the chamber, flickering in a faint draft from above. Despite my fears, I walked across the stone floor, took a candle, and peered into the next section of tunnel.

It sloped down for several dozen feet, ending at another metal gate. It was here that I detected an odor I had never before encountered—the smell of the purest of stones, as of rare jade or virgin marble. Such a feeling of lightheadedness passed over me that I almost laughed, but I was too cautious for that. I pushed aside the gate and was greeted by a rush of the coldest, sweetest air, like a draft from the tomb of a saint whose body does not corrupt but rather, draws corruption away and expels it miraculously into the nether pits. My beak dropped open. The candlelight fell across the darkness onto a figure I at first thought to be an infant. But I quickly disagreed with myself. The figure was several ages at once. As I blinked, it became a man of about thirty, well formed, with a high forehead and elegant hands, pale as ice. His eyes stared at the wall behind me. I bowed down on scaled knee and touched my forehead as best I could to the cold stone, shivering to my vestigial wing-tips. ``Forgive me, Joy of Man's Desiring,'' I said. ``Forgive me.'' I had stumbled upon the hiding place of the Stone Christ.

``You are forgiven,'' He said wearily. ``You had to come sooner or later. Better now than later, when...'' His voice trailed away and He shook His head. He was very thin, wrapped in a gray robe that still bore the scars of centuries of weathering. ``Why did you come?''

``To escape the bishop's troops,'' I said.

He nodded. ``Yes. The bishop. How long have I been here?''

``Since before I was born, Lord. Sixty or seventy years.'' He was thin, almost ethereal, this figure I had imagined as a husky carpenter. I lowered my voice and beseeched, ``What may I do for you, Lord?''

``Go away,'' He said.

``I could not live with such a secret,'' I said. ``You are salvation. You can overthrow the bishop and bring all the levels together.''

``I am not a general or a soldier. Please go away and tell no—''

I felt a breath behind me, then the whisper of a weapon. I leaped aside, and my hackles rose as a stone sword came down and shattered on the floor beside me. The Christ raised His hand. Still in shock, I stared at a beast much like myself. It stared back, face black with rage, stayed by the power of His hand. I should have been more wary—something had to have killed the spouter and kept the candles fresh.

``But, Lord,'' the beast rumbled, ``he will tell all.''

``No,'' the Christ said. ``He'll tell nobody.'' He looked half at me, half through me, and said, ``Go, go.''

Up the tunnels, into the orange dark of the Cathedral, crying, I crawled and slithered. I could not even go to the giant. I had been silenced as effectively as if my throat had been cut.

The next morning I watched from a shadowy corner of the scaffold as a crowd gathered around a lone man in a dirty sackcloth robe. I had seen him before—his name was Psalo, and he was left alone as an example of the bishop's largess. It was a token gesture; most of the people regarded him as barely half-sane.

Yet this time I listened and, in my confusion, found his words striking responsive chords in me. He was exhorting the bishop and his forces to allow light into the Cathedral again by dropping the canvas tarps that covered the windows. He had talked about this before, and the bishop had responded with his usual statement—that with the light would come more chaos, for the human mind was now a pesthole of delusions. Any stimulus would drive away whatever security the inhabitants of the Cathedral had.

\fancybreak{* * *}

At this time it gave me no pleasure to watch the love of Constantia and Corvus grow. They were becoming more careless. Their talk grew bolder: ``We shall announce a marriage,'' Corvus said.

``They will never allow it. They'll... cut you.''

``I'm nimble. They'll never catch me. The church needs leaders, brave revolutionaries. If no one breaks with tradition, everyone will suffer.''

``I fear for your life—and mine. My father would push me from the flock like a diseased lamb.''

``Your father is no shepherd.''

``He is my father,'' Constantia said, eyes wide, mouth drawn tight.

I sat with beak in paws, eyes half-lidded, able to mimic each statement before it was uttered. Undying love... hope for a bleak future... shite and onions! I had read it all before, in a cache of romance novels in the trash of a dead nun. As soon as I made the connection and realized the timeless banality—and the futility—of what I was seeing, and when I compared their prattle with the infinite sadness of the Stone Christ, I went from innocent to cynic. The transition dizzied me, leaving little backwaters of noble emotion, but the future seemed clear. Corvus would be caught and executed; if it hadn't been for me, he would already have been gelded, if not killed. Constantia would weep, poison herself; the singers would sing of it (those selfsame warble-throats who cheered the death of her lover); perhaps I would write of it (I was planning this chronicle even then), and afterward, perhaps, I would follow them both, having succumbed to the sin of boredom.

With night, things become less certain. It is easy to stare at a dark wall and let dreams become manifest. At one time, I've deduced from books, dreams could not take shape beyond sleep or brief fantasy. All too often I've had to fight things generated in my dreams, flowing from the walls, suddenly independent and hungry. People often die in the night, devoured by their own nightmares.

That evening, falling to sleep with visions of the Stone Christ in my head, I dreamed of holy men, angels, and saints. I came awake abruptly, by training, and sat that one had stayed behind. The others I saw flitting outside the round window, where they whispered and made plans for flying off to heaven. The wraith that had lingered made a dark, vague shape in one corner. His breath came new to him, raw and harsh. ``I am Peter,'' he said, ``also called Simon. I am the Rock of the Church, and popes are told that they are heir to my task.''

``I'm rock, too,'' I said. ``At least, part of me is.''

``So be it, then. You are heir to my task. Go forth and be Pope. Do not revere the Stone Christ, for a Christ is only as good as He does, and if He does nothing, there is no salvation in Him.''

The shadow reached out to pat my head. I saw his eyes grow wide as he made out my form. He muttered some formula for banishing devils and oozed out the window to join his fellows.

I imagined that if such a thing were actually brought before the council, it would be decided under the law that the command of a dream saint is not binding. I did not care. The wraith had given me better orders than any I'd had since the giant told me to read and learn.

But to be Pope, one must have a hierarchy of servants to carry out one's plans. The biggest of rocks does not move by itself. So it was that, swollen with power, I decided to appear in the upper nave and announce myself to the people.

\fancybreak{* * *}

It took a great deal of courage to show myself in broad daylight, without my cloak, and to walk across the scaffold's surface, on the second level, through crowds of vendors setting up the market for the day. Some saw me and reacted with typical bigotry. They kicked and cursed at me. My beak was swift and discouraged them.

I clambered to the top of a prominent stall and stood in the murky glow of a small lamp, rising to my full height and clearing my throat, making ready to give my commands. Under a hail of rotten pomegranates and limp vegetables, I told the throng who I was. I boldly told them about my vision. I tried to make myself speak clearly, starting over from the beginning several times, but the deluge of opprobrium was too thick. Jeweled with beads of offal, I jumped down and fled to a tunnel entrance too small for most men. Some boys followed, ready to do me real harm, and one lost his finger while trying to slice me with a fragment of colored glass.

I recognized, almost too late, that the tactic of open revelation was worthless. There are levels of fear and bigotry, and I was at the very bottom.

My next strategy was to find some way to disrupt the Cathedral from top to bottom. Even bigots, when reduced to a mob, could be swayed by the presence of one obviously ordained and capable. I spent two days skulking through the walls. There had to be a basic flaw in so fragile a structure as the church, and, while I wasn't contemplating total destruction, I wanted something spectacular, unavoidable.

While I cogitated, hanging from the bottom of the second scaffold, above the community of pure flesh, the bishop's deep gravelly voice roared over the noise of the crowd. I opened my eyes and looked down. The masked troops were holding a bowed figure, and the bishop was intoning over its head, ``Know all who hear me now, this young bastard of flesh and stone—''

Corvus, I told myself. Finally caught. I shut one eye, but the other refused to close out the scene.

``—has violated all we hold sacred and shall atone for his crimes on this spot, tomorrow at this time. Kronos! Mark the wheel's progress.'' The elected Kronos, a spindly old man with dirty gray hair down to his buttocks, took a piece of charcoal and marked an X on the huge bulkhead chart, behind which the wheel groaned and sighed in its circuit.

The crowd was enthusiastic. I saw Psalo pushing through the people.

``What crime?'' he called out. ``Name the crime!''

``Violation of the lower level!'' the head of the masked troops declared.

``That merits a whipping and an escort upstairs,'' Psalo said. ``I detect a more sinister crime here. What is it?''

The bishop looked Psalo down coldly. ``He tried to rape my daughter, Constantia.''

Psalo could say nothing to that. The penalty was castration and death. All the pure humans accepted such laws. There was no other recourse.

I mused, watching Corvus being led to the dungeons. The future that I desired at that moment startled me with its clarity. I wanted that part of my heritage that had been denied to me—to be at peace with myself, to be surrounded by those who accepted me, by those no better than I. In time that would happen, as the giant had said. But would I ever see it? What Corvus, in his own lusty way, was trying to do was equalize the levels, to bring stone into flesh until no one could define the divisions.

Well, my plans beyond that point were very hazy. They were less plans than glowing feelings, imaginings of happiness and children playing in the forest and fields beyond the island as the world knit itself under the gaze of God's heir. My children, playing in the forest. A touch of truth came to me at this moment. I had wished to be Corvus when he tupped Constantia.

So I had two tasks, then, that could be merged if I was clever. I had to distract the bishop and his troops, and I had to rescue Corvus, fellow revolutionary.

I spent that night in feverish misery in my room. At dawn I went to the giant and asked his advice. He looked me over coldly and said, ``We waste our time if we try to knock sense into their heads. But we have no better calling than to waste our time, do we?''

``What shall I do?''

``Enlighten them.''

I stomped my claw on the floor. ``They are bricks! Try enlightening bricks!''

He smiled his sad, narrow smile. ``Enlighten them,'' he said.

I left the giant's chamber in a rage. I did not have access to the great wheel's board of time, so I couldn't know exactly when the execution would take place. But I guessed—from memories of a grumbling stomach—that it would be in the early afternoon. I traveled from one end of the nave to the other and, likewise, the transept. I nearly exhausted myself.

Then, traversing an empty aisle, I picked up a piece of colored glass and examined it, puzzled. Many of the boys on all levels carried these shards with them, and the girls used them as jewelry—against the wishes of their elders, who held that bright objects bred more beasts in the mind. Where did they get them?

In one of the books I had perused years before, I had seen brightly colored pictures of the Cathedral windows. ``Enlighten them,'' the giant had said.

Psalo's request to let light into the Cathedral came to mind.

Along the peak of the nave, in a tunnel running its length, I found the ties that held the pulleys of the canvases over the windows. The best windows, I decided, would be the huge ones of the north and south transepts. I made a diagram in the dust, trying to decide what season it was and from which direction the sunlight would come—pure theory to me, but at this moment I was in a fever of brilliance. All the windows had to be clear. I could not decide which was best.

I was ready by early afternoon, just after sext prayers in the upper nave. I had cut the major ropes and weakened the clamps by prying them from the walls with a pick stolen from the bishop's armory. I walked along a high ledge, took an almost vertical shaft through the wall to the lower floor, and waited.

Constantia watched from a wooden balcony, the bishop's special box for executions. She had a terrified, fascinated look on her face. Corvus was on the dais across the nave, right in the center of the cross of the transept. Torches illumined him and his executioners, three men and an old woman.

I knew the procedure. The old woman would castrate him first, then the men would remove his head. He was dressed in the condemned red robe to hide any blood. Blood excitement among the impressionable was the last thing the bishop wanted. Troops waited around the dais to purify the area with scented water.

I didn't have much time. It would take minutes for the system of ropes and pulleys to clear and the canvases to fall. I went to my station and severed the remaining ties. Then, as the Cathedral filled with a hollow creaking sound, I followed the shaft back to my viewing post.

In three minutes the canvases were drooping. I saw Corvus look up, his eyes glazed. The bishop was with his daughter in the box. He pulled her back into the shadows. In another two minutes the canvases fell onto the upper scaffold with a hideous crash. Their weight was too great for the ends of the structure, and it collapsed, allowing the canvas to cascade to the floor many yards below. At first the illumination was dim and bluish, filtered perhaps by a passing cloud. Then, from one end of the Cathedral to the other, a burst of light threw my smoky world into clarity. The glory of thousands of pieces of colored glass, hidden for decades and hardly touched by childish vandals, fell upon upper and lower levels at once. A cry from the crowds nearly wrenched me from my post. I slid quickly to the lower level and hid, afraid of what I had done. This was more than simple sunlight. Like the blossoming of two flowers, one brighter than the other, the transept windows astounded all who beheld them.

Eyes accustomed to orangey dark, to smoke and haze and shadow, cannot stare into such glory without drastic effect. I shielded my own face and tried to find a convenient exit.

But the population was increasing. As the light brightened and more faces rose to be locked, phototropic, the splendor unhinged some people. From their minds poured contents too wondrous to be accurately cataloged. The monsters thus released were not violent, however, and most of the visions were not monstrous.

The upper and lower nave shimmered with reflected glories, with dream figures and children clothed in baubles of light. Saints and prodigies dominated. A thousand newly created youngsters squatted on the bright floor and began to tell of marvels, of cities in the East, and times as they had once been. Clowns dressed in fire entertained from the tops of the market stalls. Animals unknown to the Cathedral cavorted between the dwellings, giving friendly advice. Abstract things, glowing balls in nets of gold and ribbons of silk, sang and floated around the upper reaches. The Cathedral became a great vessel of all the bright dreams known to its citizens.

Slowly, from the lower nave, people of pure flesh climbed to the scaffold and walked the upper nave to see what they couldn't from below. From my hideaway I watched the masked troops of the bishop carrying his litter up narrow stairs. Constantia walked behind, stumbling, her eyes shut in the new brightness.

All tried to cover their eyes, but none for long succeeded.

I wept. Almost blind with tears, I made my way still higher and looked down on the roiling crowds. I saw Corvus, his hands still wrapped in restraining ropes, being led by the old woman.

Constantia saw him, too, and they regarded each other like strangers, then joined hands as best they could. She borrowed a knife from one of her father's soldiers and cut his ropes away. Around them the brightest dreams of all began to swirl, pure white and blood-red and sea-green, coalescing into visions of all the children they would innocently have.

I gave them a few hours to regain their senses—and to regain my own. Then I stood on the bishop's abandoned podium and shouted over the heads of those on the lowest level.

``The time has come!'' I cried. ``We must all unite now; we must unite—''

At first they ignored me. I was quite eloquent, but their excitement was still too great. So I waited some more, began to speak again, and was shouted down. Bits of fruit and vegetables arced up. ``Freak!'' they screamed, and drove me away.

I crept along the stone stairs, found the narrow crack, and hid in it, burying my beak in my paws, wondering what had gone wrong. It took a surprisingly long time for me to realize that, in my case, it was less the stigma of stone than the ugliness of my shape that doomed my quest for leadership.

I had, however, paved the way for the Stone Christ. He will surely be able to take His place now, I told myself. So I maneuvered along the crevice until I came to the hidden chamber and the yellow glow. All was quiet within. I met first the stone monster, who looked me over suspiciously with glazed gray eyes. ``You're back,'' he said. Overcome by his wit, I leered, nodded, and asked that I be presented to the Christ.

``He's sleeping.''

``Important tidings,'' I said.

``What?''

``I bring glad tidings.''

``Then let me hear them.''

``His ears only.''

Out of the gloomy corner came the Christ, looking much older now. ``What is it?'' He asked.

``I have prepared the way for You,'' I said. ``Simon called Peter told me I was the heir to his legacy, that I should go before You—''

The Stone Christ shook His head. ``You believe I am the fount from which all blessings flow?''

I nodded, uncertain.

``What have you done out there?''

``Let in the light,'' I said.

He shook His head. ``You seem a wise enough creature. You know about Mortdieu.''

``Yes.''

``Then you should know that I barely have enough power to keep myself together, to heal myself, much less to minister to those out there.'' He gestured beyond the walls. ``My own source has gone away,'' He said mournfully. ``I'm operating on reserves, and those none too vast.''

``He wants you to go away and stop bothering us,'' the monster explained.

``They have their light out there,'' the Christ said. ``They'll play with that for a while, get tired of it, go back to what they had before. Is there any place for you in that?''

I thought for a moment, then shook my head. ``No place,'' I said. ``I'm too ugly.''

``You are too ugly, and I am too famous,'' He said. ``I'd have to come from their midst, anonymous, and that is clearly impossible. No, leave them alone for a while. They'll make me over again, perhaps, or better still, forget about me. About us. We don't have any place there.''

I was stunned. I sat down hard on the stone floor, and the Christ patted me on my head as He walked by. ``Go back to your hiding place; live as well as you can,'' He said. ``Our time is over.''

I turned to go. When I reached the crevice, I heard His voice behind, saying, ``Do you play bridge? If you do, find another. We need four to a table.''

I clambered up the crack, through the walls, and along the arches over the revelry. Not only was I not going to be Pope—after an appointment by Saint Peter himself!—but I couldn't convince someone much more qualified than I to assume the leadership.

It is the sign of the eternal student, I suppose, that when his wits fail him, he returns to the teacher.

I returned to the copper giant. He was lost in meditation. About his feet were scattered scraps of paper with detailed drawings of parts of the Cathedral. I waited patiently until he saw me. He turned, chin in hand, and looked me over.

``Why so sad?''

I shook my head. Only he could read my features and recognize my moods.

``Did you take my advice below? I heard a commotion.''

``Mea maxima culpa,'' I said.

``And... ?''

I hesitantly made my report, concluding with the refusal of the Stone Christ. The giant listened closely without interrupting. When I was done, he stood, towering over me, and pointed with his ruler through an open portal.

``Do you see that out there?'' he asked. The ruler swept over the forests beyond the island, to the far green horizon. I replied that I did and waited for him to continue. He seemed to be lost in thought again.

``Once there was a city where trees now grow,'' he said. ``Artists came by the thousands, and whores, and philosophers, and academics. And when God died, all the academics and whores and artists couldn't hold the fabric of the world together. How do you expect us to succeed now?''

Us? ``Expectations should not determine whether one acts or not,'' I said. ``Should they?''

The giant laughed and tapped my head with the ruler. ``Maybe we've been given a sign, and we just have to learn how to interpret it correctly.''

I leered to show I was puzzled.

``Maybe Mortdieu is really a sign that we have been weaned. We must forage for ourselves, remake the world without help. What do you think of that?''

I was too tired to judge the merits of what he was saying, but I had never known the giant to be wrong before. ``Okay. I grant that. So?''

``The Stone Christ tells us His charge is running down. If God weans us from the old ways, we can't expect His Son to replace the nipple, can we?''

``No...''

He hunkered next to me, his face bright. ``I wondered who would really stand forth. It's obvious He won't. So, little one, who's the next choice?''

``Me?'' I asked, meekly. The giant looked me over almost pityingly.

``No,'' he said after a time. ``I am the next. We're weaned!'' He did a little dance, startling my beak up out of my paws. I blinked. He grabbed my vestigial wing-tips and pulled me upright. ``Stand straight. Tell me more.''

``About what?''

``Tell me all that's going on below, and whatever else you know.''

``I'm trying to figure out what you're saying,'' I protested, trembling a bit.

``Dense as stone!'' Grinning, he bent over me. Then the grin went away, and he tried to look stern. ``It's a grave responsibility. We must remake the world ourselves now. We must coordinate our thoughts, our dreams. Chaos won't do. What an opportunity, to be the architect of an entire universe!'' He waved the ruler at the ceiling. ``To build the very skies! The last world was a training ground, full of harsh rules and strictures. Now we've been told we're ready to leave that behind, move on to something more mature. Did I teach you any of the rules of architecture? I mean, the aesthetics. The need for harmony, interaction, utility, beauty?''

``Some,'' I said.

``Good. I don't think making the universe anew will require any better rules. No doubt we'll need to experiment, and perhaps one or more of our great spires will topple. But now we work for ourselves, to our own glory, and to the greater glory of the God who made us! No, ugly friend?''

\fancybreak{* * *}

Like many histories, mine must begin with the small, the tightly focused, and expand into the large. But unlike most historians, I don't have the luxury of time. Indeed, my story isn't even concluded yet.

Soon the legions of Viollet-le-Duc will begin their campaigns. Most have been schooled pretty thoroughly. Kidnapped from below, brought up in the heights, taught as I was. We'll begin returning them, one by one.

I teach off and on, write off and on, observe all the time.

The next step will be the biggest. I haven't any idea how we're going to do it.

But, as the giant puts it, ``Long ago the roof fell in. Now we must push it up again, strengthen it, repair the beams.'' At this point he smiles to the pupils. ``Not just repair them. Replace them! Now we are the beams. Flesh and stone become something much stronger.''

Ah, but then some dolt will raise a hand and inquire, ``What if our arms get tired holding up the sky?''

Our task, I think, will never end.