\chapter{Red Star, Winter Orbit}
\chapterauthor{Bruce Sterling and William Gibson}

Collaborative stories are a tradition in science fiction. And collaborative work has flourished in cyberpunk, as writers, already working closely together in concept and criticism, take the next logical step---to joint creation. In a sense, collaboration, by combining voices, allows the Movement to speak with a voice of its own.

Mirrorshades concludes with two collaborations. The present story, from 1983, is the only joint work to date by William Gibson and Bruce Sterling---both widely seen as central figures in cyberpunk. ``Red Star, Winter Orbit'' demonstrates cyberpunk's global point of view as well as its love of closely researched, fully realized detail.

William Gibson wrote ``The Gernsback Continuum,'' which led this collection.

Bruce Sterling's first novel was published in 1977. He has written three novels and a score of short stories. His work ranges widely through the SF field, including comic satires and historical fantasies. He is perhaps best known for his ``Shaper series,'' including the novel Schismatrix, and for his sense of irony, which sometimes leads him to speak of himself in the third person.

He lives in Austin, Texas.

\hrulefill

\firstletter{C}olonel Korolev twisted slowly in his harness, dreaming of winter and gravity. Young again, a cadet, he whipped his horse across the late November steppes of Kazakhstan into dry red vistas of Martian sunset.

That's wrong, he thought---

And woke---in the Museum of the Soviet Triumph in Space---to the sounds of Romanenko and the KGB man's wife. They were going at it again behind the screen at the aft end of the Salyut, restraining straps and padded hull creaking and thudding rhythmically. Hooves in the snow.

Freeing himself from the harness, Korolev executed a practiced kick that propelled him into the toilet stall. Shrugging out of his threadbare coverall, he clamped the commode around his loins and wiped condensed steam from the steel mirror. His arthritic hand had swollen again during sleep; the wrist was bird-bone thin from calcium loss. Twenty years had passed since he'd last known gravity; he'd grown old in orbit.

He shaved with a suction razor. A patchwork of broken veins blotched his left cheek and temple, another legacy from the blowout that had crippled him.

When he emerged, he found that the adulterers had finished. Romanenko was adjusting his clothing. The political officer's wife, Valentina, had ripped the sleeves from her brown coverall; her white arms were sheened with the sweat of their exertion. Her ash-blond hair rippled in the breeze from a ventilator. Her eyes were purest cornflower blue, set a little too closely together, and they held a look half-apologetic, half-conspiratorial. `See what we've brought you, Colonel---'

She handed him a tiny airline bottle of cognac.

Stunned, Korolev blinked at the Air France logo embossed on the plastic cap.

`It came in the last Soyuz. In a cucumber, my husband said.' She giggled. `He gave it to me.'

`We decided you should have it, Colonel,' Romanenko said, grinning broadly. `After all, we can be furlough at any time.' Korolev ignored the sidelong, embarrassed glance at his shriveled legs and pale, dangling feet.

He opened the bottle, and the rich aroma brought a sudden tingling rush of blood to his cheeks. He raised it carefully and sucked out a few milliliters of brandy. It burned like acid. `Christ,' he gasped, `it's been years. I'll get plastered!' he said, laughing, tears blurring his vision.

`My father tells me you drank like a hero, Colonel, in the old days.'

`Yes,' Korolev said, and sipped again, `I did.' The cognac spread through him like liquid gold. He disliked Romanenko. He'd never liked the boy's father, either---an easygoing Party man, long since settled into lecture tours, a dacha on the Black Sea, American liquor, French suits, Italian shoes...The boy had the father's looks, the same clear gray eyes utterly untroubled by doubt.

The alcohol surged through Korolev's thin blood. `You are too generous,' he said. He kicked once, gently, and arrived at his console. `You must take some samisdata, American cable broadcasts, freshly intercepted. Racy stuff! Wasted on an old man like me.' He slotted a blank cassette and punched for the material.

`I'll give it to the gun crew,' Romanenko said, grinning. `They can run it on the tracking consoles in the gun room.' The particle-beam station had always been known as the gun room. The soldiers who manned it were particularly hungry for this sort of tape. Korolev ran off a second copy for Valentina.

`It's dirty?' She looked alarmed and intrigued. `May we come again, Colonel? Thursday at 2400?'

Korolev smiled at her. She had been a factory worker before she'd been singled out for space. Her beauty made her useful as a propaganda tool, a role model for the proletariat. He pitied her now, with the cognac coursing through his veins, and found it impossible to deny her a little happiness. `A midnght rendezvous in the museum, Valentina? Romantic!'

She kissed his cheek, wobbling in free fall. `Thank you, my Colonel.'

`You're a prince, Colonel,' Romanenko said, slapping Korolev's matchstick shoulder as gently as he could. After countless hours on an exerciser, the boy's arms bulged like a blacksmith's.

Korolev watched the lovers carefully make their way out into the central docking sphere, the junction of three aging Salyuts and two corridors. Romanenko took the `north' corridor to the gun room; Valentina went in the opposite direction to the next junction sphere and the Salyut where her husband slept.

There were five docking spheres in Kosmograd, each with its three linked Salyuts. At opposite ends of the complex were the military installation and the satellite launchers. Popping, humming and wheezing, the station had the feel of a subway and the dank metallic reek of a tramp steamer.

Korolev had another pull at the bottle. Now it was half-empty. He hid it in one of the museum's exhibits, a NASA Hasselblad recovered from the site of the Apollo landing. He hadn't had a drink since his last furlough, before the blowout. His head swam in a pleasant, painful current of drunken nostalgia.

Drifting back to his console, he accessed a section of memory where the collected speeches of Alexei Kosygin had been covertly erased and replaced with his personal collection of samisdata, digitized pop music, his boyhood favorites from the Eighties. He had British groups taped from West German radio, Warsaw Pact heavy metal, American imports from the black market. Putting on his headphones, he punched for the Czestochowa reggae of Brygada Cryzis.

After all the years, he no longer really heard the music, but images came rushing back with an aching poignancy. In the Eighties he'd been a long-haired child of the Soviet elite, his father's position placing him effectively beyond the reach of the Moscow police. He remembered feedback howling through the speakers in the hot darkness of a cellar club, the crowd a shadowy checkerboard of denim and bleached hair. He'd smoked Marlboros laced with powdered Afghani hash. He remembered the mouth of an American diplomat's daughter in the back seat of her father's black Lincoln. Names and faces came flooding in on a warm haze of cognac. Nina, the East German who'd shown him her mimeographed translations of dissident Polish newssheets--- Until the night she didn't turn up at the coffee bar. Whispers of parasitism, of anti-Soviet activity, of the waiting chemical horrors of the psikuska--- Korolev started to tremble. He wiped his face and found it bathed in sweat. He took off the headphones.

It had been fifty years, yet he was suddenly and very intensely afraid. He couldn't remember ever having been this frightened, not even during the blowout that had crushed his hip. He shook violently. The lights. The lights in the Salyut were too bright, but he didn't want to go to the switches. A simple action, one he performed regularly, yet...The switches and their insulated cables were somehow threatening. He stared, confused. The little clockwork model of a Lunokhod moon rover, its Velcro wheels gripping the curved wall, seemed to crouch there like something sentient, poised, waiting. The eyes of the Soviet space pioneers in the official portraits were fixed on him with contempt.

The cognac. His years in free fall had warped his metabolism. He wasn't the man he'd once been. But he would remain calm and try to ride it out. If he threw up, everyone would laugh.

Someone knocked at the entrance to the museum, and Nikita the Plumber, Kosmograd's premier handyman, executed a perfect slow-motion dive through the open hatch. The young civilian engineer looked angry. Korolev felt cowed. `You're up early, Plumber,' he said, anxious for some facade of normality.

`Pinhead leakage in Delta Three.' He frowned. `Do you understand Japanese?' The Plumber tugged a cassette from one of the dozen pockets that bulged on his stained work vest and waved it in Korolev's face. He wore carefully laundered Levi's and dilapidated Adidas running shoes. `We accessed this last night.'

Korolev cowered as though the cassette were a weapon. `No, no Japanese.' The meekness of his own voice startled him. `Only English and Polish.' He felt himself blush. The Plumber was his friend; he knew and trusted the Plumber, but---`Are you well, Colonel?' The Plumber loaded the tape and punched up a lexicon program with deft, callused fingers.' You look as though you just ate a bug. I want you to hear this.'

Korolev watched uneasily as the tape flickered into an ad for baseball gloves. The lexicon's Cyrillic subtitles raced across the monitor as a Japanese voice-over rattled maniacally.

`The newscast's coming up,' said the Plumber, gnawing at a cuticle.

Korolev squinted anxiously as the translation slid across the face of the Japanese announcer: AMERICAN DISARMAMENT GROUP CLAIMS...PREPARATIONS AT BAIKONUR COSMODROME...PROVE RUSSIANS AT LAST READY...TO SCRAP ARMED SPACE STATION COMIC CITY

`Cosmic,' the Plumber muttered. `Glitch in the lexicon.'

BUILT AT TURN OF CENTURY AS BRIDGEHEAD TO SPACE...AMBITIOUS PROJECT CRIPPLED BY FAILURE OF LUNAR MINING...EXPENSIVE STATION OUTPERFORMED BY OUR UNMANNED ORBITAL FACTORIES...CRYSTALS, SEMICONDUCTORS AND PURE DRUGS...

`Smug bastards.' The Plumber snorted. `I tell you, it's that goddamned KGB man Yefremov. He's had a hand in this!'

STAGGERING SOVIET TRADE DEFICITS...POPULAR DISCONTENT WITH SPACE EFFORT...RECENT DECISIONS BY POLITBURO AND CENTRAL COMMITTEE SECRETARIAT...

`They're shutting us down!' The Plumber's face contorted with rage.

Korolev twisted away from the screen, shaking uncontrollably. Sudden tears peeled from his lashes in free-fall droplets. `Leave me alone! I can do nothing!'

`What's wrong, Colonel?' The Plumber grabbed his shoulders. `Look me in the face. Someone's dosed you with the Fear!'

`Go away,' Korolev begged.

`That little spook bastard! What has he given you? Pills? An injection?'

Korolev shuddered. `I had a drink---'

`He gave you the Fear! You, a sick old man! I'll break his face!' The Plumber jerked his knees up, somersaulted backward, kicked off from a handhold overhead, and catapulted out of the room.

`Wait! Plumber!' But the Plumber had zipped through the docking sphere like a squirrel, vanishing down the corridor, and now Korolev felt that he couldn't bear to be alone. In the distance he could hear metallic echoes of distorted, angry shouts.

Trembling, he closed his eyes and waited for someone to help him.
squares

He'd asked Psychiatric Officer Bychkov to help him dress in his old uniform, the one with the Star of the Tsiolkovsky Order sewn above the left breast pocket. The black dress boots of heavy quilted nylon, with their Velcro soles, would no longer fit his twisted feet; so his feet remained bare.

Bychkov's injection had straightened him out within an hour, leaving him alternately depressed and furiously angry. Now he waited in the museum for Yefremov to answer his summons.

They called his home the Museum of the Soviet Triumph in Space, and as his rage subsided, to be replaced with an ancient bleakness, he felt very much as if he were simply another one of the exhibits. He stared gloomily at the gold-framed portraits of the great visionaries of space, at the faces of Tsiolkovsky, Rynin, Tupolev. Below these, in slightly smaller frames, were portraits of Verne, Goddard, and O'Neill.

In moments of extreme depression he had sometimes imagined that he could detect a common strangeness in their eyes, particularly in the eyes of the two Americans. Was it simply craziness, as he sometimes thought in his most cynical moods? Or was he able to glimpse a subtle manifestation of some weird, unbalanced force that he had often suspected of being human evolution in action?

Once, and only once, Korolev had seen that look in his own eyes---on the day he'd stepped on to the soil of the Coprates Basin. The Martian sunlight, glinting within his helmet visor, had shown him the reflection of two steady, alien eyes---fearless, yet driven---and the quiet secret shock of it, he now realized, had been his life's most memorable, most transcendental moment.

Above the portraits, oily and inert, was a painting that depicted the landing in colors that reminded him of borscht and gravy, the Martian landscape reduced to the idealistic kitsch of Soviet Socialist realism. The artist had posed the suited figure beside the lander with all of the official style's deeply sincere vulgarity.

Feeling tainted, he awaited the arrival of Yefremov, the KGB man, Kosmograd's political officer.

When Yefremov finally entered the Salyut, Korolev noted the split lip and the fresh bruises on the man's throat. He wore a blue Kansai jump suit of Japanese silk and stylish Italian deck shoes. He coughed politely. `Good morning, Comrade Colonel.'

Korolev stared. He allowed the silence to lengthen. `Yefremov,' he said heavily, `I am not happy with you.'

Yefremov reddened, but he held his gaze. `Let us speak frankly to each other, Colonel, as Russian to Russian. It was not, of course, intended for you.'

`The Fear, Yefremov?'

`The beta-carboline, yes. If you hadn't pandered to their antisocial actions, if you hadn't accepted their bribe, it would not have happened.'

`So I am a pimp, Yefremov? A pimp and a drunkard? You are a cuckold, a smuggler, and an informer. I say this,' he added, `as one Russian to another.'

Now the KGB man's files assumed the official mask of bland and untroubled righteousness.

`But tell me, Yefremov, what it is that you are really about. What have you been doing since you came to Kosmograd? We know that the complex will be stripped. What is in store for the civilian crew when they return to Baikonur? Corruption hearings?'

`There will be interrogation, certainly. In certain cases there may be hospitalization. Would you care to suggest, Colonel Korolev, that the Soviet Union is somehow at fault for Kosmograd's failures?'

Korolev was silent.

`Kosmograd was a dream, Colonel. A dream that failed. Like space. We have no need to be here. We have an entire world to put in order. Moscow is the greatest power in history. We must not allow ourselves to lose the global perspective.'

`Do you think we can be brushed aside that easily? We are an elite, a highly trained technical elite.'

`A minority, Colonel, an obsolete minority. What do you contribute, aside from reams of poisonous American trash? The crew here were intended to be workers, not bloated black marketeers trafficking in jazz and pornography.' Yefremov's face was smooth and calm. `The crew will return to Baikonur. The weapons are capable of being directed from the ground. You, of course, will remain, and there will be guest cosmonauts: Africans, South Americans. Space still retains a degree of its former prestige for these people.'

Korolev gritted his teeth. `What have you done with the boy?'

`Your Plumber?' The political officer frowned. `He has assaulted an officer of the Committee for State Security. He will remain under guard until he can be taken to Baikonur.'

Korolev attempted an unpleasant laugh. `Let him go. You'll be in too much trouble yourself to press charges. I'll speak with Marshal Gubarev personally. My rank may be entirely honorary, Yefremov, but I do retain a certain influence.'

The KGB man shrugged. `The gun crew are under orders from Baikonur to keep the communications module under lock and key. Their careers depend on it.'

`Martial law, then?'

`This isn't Kabul, Colonel. These are difficult times. You have the moral authority here; you should try to set an example.'

`We shall see,' Korolev said.
squares

Kosmograd swung out of Earth's shadow into raw sunlight. The walls of Korolev's Salyut popped and creaked like a nest of glass bottles. A Salyut's viewports, Korolev thought absently, fingering the broken veins at his temple, were always the first things to go.

Young Grishkin seemed to have the same thought. He drew a tube of caulk from an ankle pocket and began to inspect the seal around the viewport. He was the Plumber's assistant and closest friend.

`We must now vote,' Korolev said wearily. Eleven of Kosmograd's twenty-four civilian crew members had agreed to attend the meeting, twelve if he counted himself. That left thirteen who were either unwilling to risk involvement or else actively hostile to the idea of a strike. Yefremov and the six-man gun crew brought the total number of those not present to twenty. `We've discussed our demands. All those in favor of the list as it stands---' He raised his good hand. Three others raised theirs. Grishkin, busy at the viewport, stuck out his foot.

Korolev sighed. `There are few enough as it is. We'd best have unanimity. Let us hear your objections.'

`The term military custody,' said a biological technician named Korovkin, `might be construed as implying that the military, and not the criminal Yefremov, is responsible for the situation.' The man looked acutely uncomfortable. `We are in sympathy otherwise but will not sign. We are Party members.' He seemed about to add something but fell silent. `My mother,' his wife said quietly, `was Jewish.'

Korolev nodded, but he said nothing.

`This is all criminal foolishness,' said Glushko, the botanist. Neither he nor his wife had voted. `Madness. Kosmograd is finished, we all know it, and the sooner home the better. What has this place ever been but a prison?' Free fall disagreed with the man's metabolism; in the absence of gravity, blood tended to congest in his face and neck, making him resemble one of his experimental pumpkins.

`You are a botanist, Vasili,' his wife said stiffly, `while I you will recall, am a Soyuz pilot. Your career is not at stake.'

`I will not support this idiocy!' Glushko gave the bulkhead a savage kick that propelled him from the room. His wife followed, complaining bitterly in the grating undertone crew members learned to employ for private arguments.

`Five are willing to sign,' Korolev said, `out of a civilian crew of twenty-four.'

`Six,' said Tatjana, the other Soyuz pilot, her dark hair drawn back and held with a braided band of green nylon webbing. `You forget the Plumber...'

`The sun balloons!' cried Grishkin, pointing toward the earth. `Look!'

Kosmograd was above the coast of California now, clean shorelines, intensely green fields, vast decaying cities whose names rang with a strange magic. High above a fleece of stratocumulus floated five solar balloons, mirrored geodesic spheres tethered by power lines; they had been a cheaper substitute for a grandiose American plan to build solar-powered satellites. The things worked, Korolev supposed, because for the last decade he'd watched them multiply.

`And they say that people live in those things?' Systems Officer Stoiko had joined Grishkin at the viewport.

Korolev remembered the pathetic flurry of strange American energy schemes in the wake of the Treaty of Vienna. With the Soviet Union firmly in control of the world's oil flow, the Americans had seemed willing to try anything. Then the Kansas meltdown had permanently soured them on reactors. For more than three decades they'd been gradually sliding into isolationism and industral decline. Space, he thought ruefully, they should have gone into space. He'd never understood the strange paralysis of will that had seemed to grip their brilliant early efforts. Or perhaps it was simply a failure of imagination, of vision. You see, Americans, he said silently, you really should have tried to join us here in our glorious future, here in Kosmograd.

`Who would want to live in something like that?' Stoiko asked, punching Grishkin's shoulder and laughing with the quiet energy of desperation.
squares

`You're joking,' said Yefremov. `Surely we're all in enough trouble as it is.'

`We're not joking, Political Officer Yefremov, and these are our demands.' The five dissidents had crowded into the Salyut the man shared with Valentina, backing him against the aft screen. The screen was decorated with a meticulously airbrushed photograph of the premier, who was waving from the back of a tractor. Valentina, Korolev knew, would be in the museum now with Romanenko, making the straps creak. The colonel wondered how Romanenko so regularly managed to avoid his duty shifts in the gun room.

Yefremov shrugged. He glanced down the list of demands. `The Plumber must remain in custody. I have direct orders. As for the rest of this document---'

`You are guilty of unauthorized use of psychiatric drugs!' Grishkin shouted.

`That was entirely a private matter,' said Yefremov calmly.

`A criminal act,' said Tatjana.

`Pilot Tatjana, we both know that Grishkin here is the station's most active samisdata pirate! We are all criminals, don't you see? That's the beauty of our system, isn't it?' His sudden, twisted smile was shockingly cynical. `Kosmograd is not the Potemkin, and you are not revolutionaries. And you demand to communicate with Marshal Gubarev? He is in custody at Baikonur. And you demand to communicate with the minister of technology? The minister is leading the purge.' With a decisive gesture he ripped the printout to pieces, scraps of yellow flimsy scattering in free fall like slow-motion butterflies.
squares

On the ninth day of the strike, Korolev met with Grishkin and Stoiko in the Salyut that Grishkin would ordinarily have shared with the Plumber.

For forty years the inhabitants of Kosmograd had fought an antiseptic war against mold and mildew. Dust, grease, and vapor wouldn't settle in free fall, and spore lurked everywhere---padding, in clothing, in the ventilation ducts. In the warm, moist petri-dish atmosphere, they spread like oil slicks. Now there was a reek of dry rot in the air, overlaid with ominous whiffs of burning insulation.

Korolev's sleep had been broken by the hollow thud of a departing Soyuz lander, Glushko and his wife, he supposed. During the past forty-eight hours, Yefremov had supervised the evacuation of the crew members who had refused to join the strike. The gun crew kept to the gun room and their barracks ring, where they still held Nikita the Plumber.

Grishkin's Salyut had become strike headquarters. None of the male strikers had shaved, and Stoiko had contracted a staph infection that spread across his forearms in angry welts. Surrounded by lurid pinups from American television, they ressembled some degenerate trio of pornographers. The lights were dim; Kosmograd ran on half-power. `With the others gone,' Stoiko said, `our hand is strengthened.'

Grishkin groaned. His nostrils were festooned with white streamers of surgical cotton. He was convinced that Yefremov would try to break the strike with beta-carboline aerosols. The cotton plugs were just one symptom of the general level of strain and paranoia. Before the evacuation order had come from Baikonur, one of the technicians had taken to playing Tchaikovsky's 1812 Overture at shattering volume for hours on end. And Glushko had chased his wife, naked, bruised, and screaming, up and down the length of Kosmograd. Stoiko had accessed the KGB man's files and Bychkov's psychiatric records; meters of yellow printout curled through the corridors in flabby spirals, rippling in the current from the ventilators.

`Think what their testimony will be doing to us ground-side,' muttered Grishkin. `We won't even get a trial. Straight to the psikuska.' The sinister nickname for the political hospitals seemed to galvanize the boy with dread. Korolev picked apathetically at a viscous pudding of chlorella.

Stoiko snatched a drifting scroll of printout and read aloud. `Paranoia with a tendency to overesteem ideas! Revisionist fantasies hostile to the social system!' He crumpled the paper. `If we could seize the communications module, we could tie into an American comsat and dump the whole thing in their laps. Perhaps that would show Moscow something about our hostility!'

Korolev dug a stranded fruit fly from his algae pudding. Its two pairs of wings and bifurcated thorax were mute testimony to Kosmograd's high radiation levels. The insects had escaped from some forgotten experiment; generations of them had infested the station for decades. `The Americans have no interest in us,' Korolev said. `Moscow can no longer be embarrassed by such revelations.'

`Except when the grain shipments are due,' Grishkin said.

`America needs to sell as badly as we need to buy.' Korolev grimly spooned more chlorella into his mouth, chewed mechanically, and swallowed. `The Americans couldn't reach us even if they desired to. Canaveral is in ruins.'

`We're low on fuel,' Stoiko said.

`We can take it from the remaining landers,' Korolev said.

`Then how in hell would we get back down?' Grishkin's fists trembled. `Even in Siberia, there are trees, trees; the sky! To hell with it! Let it fall to pieces! Let it fall and burn!'

Korolev's pudding spattered across the bulkhead.

`Oh, Christ,' Grishkin said, `I'm sorry, Colonel. I know you can't go back.'
squares

When he entered the museum, he found Pilot Tatjana suspended before that hateful painting of the Mars Landing, her cheeks slick with tears.

`Do you know, Colonel, they have a bust of you at Baikonur? In bronze. I used to pass it on my way to lectures.' Her eyes were red-rimmed with sleeplessness.

`There are always busts. Academies need them.' He smiled and took her hand.

`What was it like that day?' She still stared at the painting.

`I hardly remember. I've seen the tapes so often, now I remember them instead. My memories of Mars are any schoolchild's.' He smiled for her again. `But it was not like this bad painting. In spite of everything, I'm still certain of that.'

`Why has it all gone this way, Colonel? Why is it ending now? When I was small I saw all this on television. Our future in space was forever---'

`Perhaps the Americans were right. The Japanese sent machines instead, robots to build their orbital factories. Lunar mining failed for us, but we thought there would at least be a permanent research facility of some kind. It all had to do with purse strings, I suppose. With men who sit at desks and make decisions.'

`Here is their final decision with regard to Kosmograd.' She passed him a folded scrap of flimsy. `I found this in the printout of Yefremov's orders from Moscow. They'll allow the station's orbit to decay over the next three months.'

He found that now he too was staring fixedly at the painting he loathed. `It hardly matters anymore,' he heard himself say.
squares

And then she was weeping bitterly, her face pressed hard against Korolev's crippled shoulder. `But I have a plan, Tatjana,' he said, stroking her hair. `You must listen.'

He glanced at his old Rolex. They were over eastern Siberia. He remembered how the Swiss ambassador had presented him with the watch in an enormous vaulted room in the Grand Kremlin Palace.

It was time to begin.

He drifted out of his Salyut into the docking sphere, batting at a length of printout that tried to coil around his head.

He could still work quickly and efficiently with his good hand. He was smiling as he freed a large oxygen bottle from its webbing straps. Bracing himself against a handhold, he flung the bottle across the sphere with all his strength. It rebounded harmlessly with a harsh clang. He went after it, caught it, and hurled it again.

Then he hit the decompression alalrm.

Dust spurted from speakers as a Klaxon began to wail. Triggered by the alarm, the docking bays slammed shut with a wheeze of hydraulics. Korolev's ears popped. He sneezed, then went after the bottle again.

The lights flared to maximum brilliance, then flickered out. He smiled in the darkness, groping for the steel bottle. Stoiko had provoked a general systems crash. It hadn't been difficult. The memory banks were already riddled to the point of collapse with bootlegged television broadcasts. `The real bareknuckle stuff,' he muttered, banging the bottle against the wall. The lights flickered on weakly as emergency cells came on line.

His shoulder began to ache. Stoically he continued pounding, remembering the din a real blowout caused. It had to be good. It had to fool Yefremov and the gun crew.

With a squeal, the manual wheel of one of the hatches began to rotate. It thumped open, finally, and Tatjana looked in, grinning shyly.

`Is the Plumber free?' he asked, releasing the bottle.

`Stoiko and Umansky are reasoning with the guard.' She drove a fist into her open palm. `Grishkin is preparing the landers.'

He followed her up to the next docking sphere. Stoiko was helping the Plumber through the hatch that led from the barracks ring. The Plumber was barefoot, his face greenish under a scraggly growth of beard. Meteorologist Umansky followed them, dragging the limp body of a soldier.

`How are you, Plumber?' Korolev asked.

`Shaky. They've kept me on the Fear. Not big doses, but---and I thought that that was a real blowout!'

Grishkin slid out of the Soyuz lander nearest Korolev, trailing a bundle of tools and meters on a nylon lanyard. `They all check out. The crash left them under their own automatics. I've been at their remotes with a screwdriver so they can't be overriden by ground control. How are you doing, my Nikita!' he asked the Plumber. `You'll be going in steep to central China.'

The Plumber winced, shook himself, and shivered, `I don't speak Chinese.'

Stoiko handed him a printout. `This is in phonetic Mandarin. I WISH TO DEFECT, TAKE ME TO THE NEAREST JAPANESE EMBASSY.'

The Plumber grinned and ran his fingers through his thatch of sweat-stiffened hair. `What about the rest of you?' he asked.

`You think we're doing this for your benefit alone?' Tatjana made a face at him. `Make sure the Chinese news services get the rest of that scroll, Plumber. Each of us has a copy. We'll see that the world knows what the Soviet Union intends to do to Colonel Yuri Vasilevich Korolev, first man on Mars!' She blew the Plumber a kiss.

`How about Filipchenko here?' Umansky asked. A few dark spheres of congealing blood swung crookedly past the unconscious soldier's cheek.

`Why don't you take the poor bastard with you,' Korolev said.

`Come along then, shithead,' the Plumber said, grabbing Filipchenko's belt and towing him toward the Soyuz hatch. `I, Nikita the Plumber, will do you the favor of your miserable lifetime.'

Korolev watched as Stoiko and Grishkin sealed the hatch behind them.

`Where are Romanenko and Valentina?' Korolev asked, checking his watch again.

`Here, my colonel,' Valentina said, her blond hair floating around her face in the hatch of another Soyuz. `We have been checking this one out.' She giggled.

`Time enough for that in Tokyo,' Korolev snapped. `They'll be scrambling jets in Vladivostok and Hanoi within minutes.'

Romanenko's bare, brawny arm emerged and yanked her back into the lander. Stoiko and Grishkin sealed the hatch.

`Peasants in space.' Tatjana made a spitting noise.

Kosmograd boomed hollowly as the Plumber, with the unconscious Filipchenko, cast off. Another boom and the lovers were off as well.

`Come along, friend Umansky,' said Stoiko. `And farewell, Colonel!' The two men headed down the corridor.

`I'll go with you,' Grishkin said to Tatjana. He grinned. `After all, you're a pilot.'

`No,' she said. `Alone. We'll split the odds. You'll be fine with the automatics. Just don't touch anything on the board.'

Korolev watched her help him into the sphere's last Soyuz.

`I'll take you dancing, Tatjana,' Grishkin said, `in Tokyo.' She sealed the hatch. Another boom, and Stoiko and Umansky had cast off from the next docking sphere.

`Go now, Tatjana,' Korolev said. `Hurry. I don't want them shooting you down over international waters.'

`That leaves you here alone, Colonel, alone with our enemies.'

`When you've gone, they'll go as well,' he said. `And I depend on your publicity to embarrass the Kremlin into keeping me alive here.'

`And what shall I tell them in Tokyo, Colonel? have you a message for the world?'

`Tell them...' and every cliché came rushing to him with absolute lightness that made him want to laugh hysterically: One small step...We came in peace...Workers of the world...' You must tell them that I need it,' he said, pinching his shrunken wrist, `in my very bones.'

She embraced him and slipped away.
squares

He waited alone in the docking sphere. The silence scratched away at his nerves; the systems crash had deactivated the ventilation system, whose hum he'd lived with for twenty years. At last he heard Tatjana's Soyuz disengage.

Someone was coming down the corridor. It was Yefremov, moving clumsily in a vacuum suit. Korolev smiled.

Yefremov wore his bland, official mask behind the Lexan faceplate, but he avoided meeting Korolev's eyes as he passed. He was heading for the gun room.

`No!' Korolev shouted.

The Klaxon blared the station's call to full battle alert.

The gun-room hatch was open when he reached it. Inside, the soldiers were moving jerkily in the galvanized reflex of constant drill, yanking the broad straps of their console seats across the chests of their bulky suits.

`Don't do it!' He clawed at the stiff accordion fabric of Yefremov's suit. One of the accelerators powered up with a staccato whine. On a tracking screen, green cross hairs closed in on a red dot.

Yefremov removed his helmet. Calmly, with no change in his expression, he backhanded Korolev with the helmet.

`Make them stop!' Korolev sobbed. The walls shook as a beam cut loose with the sound of a cracking whip. `Your wife, Yefremov! She's out there!'

`Outside, Colonel.' Yefremov grabbed Korolev's arthritic hand and squeezed. Korolev screamed. `Outside.' A gloved fist struck him in the chest.

Korlev pounded helplessly on the vacuum suit as he was shoved out into the corridor. `Even I, Colonel, dare not come between the Red Army and its orders.' Yefremov looked sick now; the mask had crumbled. `Fine sport,' he said. `Wait here until it's over.'

Then Tatjana's Soyuz struck the beam installation and the barracks ring. In a split-second daguerreotype of raw sunlight, Korolev saw the gun room wrinkle and collapse like a beer can crushed under a boot; he saw the decapitated torso of a soldier spinning away from a console; he saw Yefremov try to speak, his hair streaming upright as vacuum tore the air in his suit out through his open helmet ring. Fine twin streams of blood arced from Korolev's nostrils, the roar of escaping air replaced by a deeper roaring in his head.

The last thing Korolev remembered hearing was the hatch door slamming shut.

When he woke, he woke to darkness, pulsing agony behind his eyes, remembering old lectures. This was as great a danger as the blowout itself, nitrogen bubbling through the blood to strike with white-hot, crippling pain...

But it was all so remote, so academic, really. He turned the wheels of the hatches out of some strange sense of noblesse oblige, nothing more. The labor was quite onerous, and he wished very much to return to the museum and sleep.
squares

He could repair the leaks with caulk, but the systems crash was beyond him. He had Glushko's garden. With the vegetables and algae, he wouldn't starve or smother. The communications module had gone with the gun room and the barracks ring, sheared from the station by the impact of Tatjana's suicidal Soyuz. He assumed that the collision had perturbed Kosmograd's orbit, but he had no way of predicting the hour of the station's final incandescent meeting with the upper atmosphere. He was often ill now, and he often thought that he might die before burnout, which disturbed him.

He spent uncounted hours screening the museum's library of tapes. A fitting pursuit for the Last Man in Space who had once been the First Man on Mars.

He became obsessed with the icon of Gagarin, endlessly rerunning the grainy television images of the Sixties, the newsreels that led so unalterably to the cosmonaut's death. The stale air of Kosmograd swam with the spirits of martyrs. Gagarin, the first Salyut crew, the Americans roasted alive in their squat Apollo...

Often he dreamed of Tatjana, the look in her eyes like the look he imagined in the eyes of the museum's portraits. And once he woke, or dreamed he woke, in the Salyut where she had slept, to find himself in his old uniform, with a battery-powered work light strapped across his forehead. From a great distance, as though he watched a newsreel on the museum's monitor, he saw himself rip the Star of the Tsiolkovsky Order from his pocket and staple it to her pilot's certificate.

When the knocking came, he knew that it must be a dream as well.

The hatch wheeled open.

In the bluish, flickering light from the old film, he saw that the woman was black. Long corkscrews of matted hair rose like cobras around her head. She wore goggles, a silk aviator's scarf twisting behind her in free fall. `Andy,' she said in English, `you better come see this!'

A small, muscular man, nearly bald, and wearing only a jockstrap and a jangling toolbelt, floated up behind her and peered in. `Is he alive?'

`Of course I am alive,' said Korolev in slightly accented English.

The man called Andy sailed in over her head. `You okay, Jack?' His right bicep was tattooed with a geodesic balloon above crossed lightning bolts and bore the legend SUNSPARK 15, UTAH. `We weren't expecting anybody.'

`Neither was I,' said Korolev, blinking.

`We've come to live here,' said the woman, drifting closer.

`We're from the balloons. Squatters, I guess you could say. Heard the place was empty. You know the orbit's decaying on this thing?' The man executed a clumsy midair somersault, the tools clattering on his belt. `This free fall's outrageous.'

`God,' said the woman, `I just can't get used to it! It's wonderful. It's like skydiving, but there's no wind.'

Korolev stared at the man, who had the blundering, careless look of someone drunk on freedom since birth. `But you don't even have a launchpad,' he said.

`Launchpad?' the man said, laughing. `What we do, we haul these surplus booster engines up the cables to the balloons, drop 'em, and fire 'em in midair.'

`That's insane,' Korolev said.

`Got us here, didn't it?'

Korolev nodded. If this was all a dream, it was a very peculiar one. `I am Colonel Yuri Vasilevich Korolev.'

`Mars!' The woman clapped her hands. `Wait'll the kids hear that.' She plucked the little Lunokhod moon-rover model from the bulkhead and began to wind it.

`Hey,' the man said, `I gotta work. We got a bunch of boosters outside. We gotta lift this thing before it starts burning.'

Something clanged against the hull. Kosmograd rang with the impact. `That'll be Tulsa,' Andy said, consulting a wrist watch. `Right on time.'

`But why?' Korolev shook his head, deeply confused. `Why have you come?'

`We told you. To live here. We can enlarge this thing, maybe build more. They said we'd never make it living in the balloons, but we were the only ones who could make them work. It was our one chance to get out here on our own. Who'd want to live out here for the sake of some government, some army brass, a bunch of pen pushers? You have to want a frontier---want it in your bones, right?'

Korolev smiled. Andy grinned back. `We grabbed those power cables and just pulled ourselves straight up. And when you get to the top, well, man, you either make that big jump or else you rot there.' His voice rose. `And you don't look back, no sir! We've made that jump, and we're here to stay!'

The woman placed the model's Velcro wheels against the curved wall and released it. It went scooting along above their heads, whirring merrily. `Isn't that cute? The kids are just going to love it.'

Korolev stared into Andy's eyes. Kosmograd rang again, jarring the little Lunokhod model on to a new course.

`East Los Angeles,' the woman said. `That's the one with the kids in it.' She took off her goggles, and Korolev saw her eyes brimming over with a wonderful lunacy.

`Well,' said Andy, rattling his toolbelt, `you feel like showing us around?