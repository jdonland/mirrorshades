\chapter{Tales of Houdini}
\chapterauthor{Rudy Rucker}

Rudy Rucker, an associate professor of computer science at San Jose State University, is perhaps the most wildly visionary science fiction writer working today. He goes against the grain of many scientist writers in SF, in that his work reflects not nuts and-bolts hard tech but radical visions drawn from the esoteric reaches of mathematics. Such widely praised Rucker novels as White Light and Software draw their imaginative power from Rucker's study of information theory, multidimensional to pology, and infinite sets.

But Rucker's work is marked not by philosophical aridity but by a raucous gutter-level humanity. And his narrative skill and fertile imagination extend beyond metaphysical idea pieces. The following story is a brief but perfectly constructed fantasy. Taken from his short-story collection The 57th Franz Kafka, it shows Rucker's boldly inventive originality at its most exhilarating.

His latest book is Mind Tools, his fourth nonfiction work of popular science, dealing with the conceptual roots of mathematics and information theory.

\hrulefill

\firstletter{H}oudini is broke. The vaudeville circuit is dead, ditto big-city stage. Mel Rabstein from Pathé News phones him up looking for a feature.

“Two G advance plus three points gross after turnaround.”

“You’re on.”

The idea is to get a priest, a rabbi and a judge to be on camera with Houdini in all the big scenes. It’ll be feature-length and play in the Loews chain. All Houdini knows for sure is that there’ll be escapes, bad ones, with no warnings.

It starts at four in the morning, July 8, 1948. They bust into Houdini’s home in Levittown. He lives there with his crippled mom. Opening shot of priest and rabbi kicking the door down. Close on their thick-soled black shoes. Available light. The footage is grainy, jerky, can’t-help-it-cinema-verité. It’s all true.

The judge has a little bucket of melted wax, and they seal up Houdini’s eyes and ears and nose-holes. The dark mysterioso face is covered over and over before he fully awakes, relaxing into the events, leaving dreams of pursuit. Houdini is ready. They wrap him up in Ace bandages and surgical tape, a mummy, a White Owl cigar.

Eddie Machotka, the Pathé cameraman, time-lapses the drive out to the airstrip. He shoots a frame every ten seconds so the half-hour drive only takes two minutes on screen. Dark, the wrong angles, but still convincing. There’s no cuts. In the back of the Packard, on the laps of the priest, judge and rabbi, lies Houdini, a white loaf crusty with tape, twitching in condensed time.

The car pulls right onto the airstrip, next to a B-15 bomber. Eddie hops out and films the three holy witnesses unloading Houdini. Pan over the plane. “The Dirty Lady,” is lettered up near the nose.

The Dirty Lady! And it’s not crop dusters or reservists flying it, daddyo, it’s Johnny Gallio and his Flying A-Holes! Forget it! Johnny G., the most decorated World War II Pacific combat ace, flying, with Slick Tires Jones navigating, and no less a man than Moanin’ Max Moscowitz in back.

Johnny G. jumps down out of the cockpit, not too fast, not too slow, just cool, flight-jacket Johnny. Moanin’ Max and Slick Tires lean out the bomb-bay hatch, grinning and ready to roll.

The judge pulls out a turnip pocket-watch. The camera zooms in and out, four-fifty A. M., the sky is getting light.

Houdini? He doesn’t know they’re handing him into the bomb-bay of The Dirty Lady. He can’t even hear or see or smell. But he’s at peace, glad to have all this out in the open, glad to have it really happen.

Everyone gets in the plane. Bad camera-motion as Eddie climbs in. Then a shot of Houdini, long and white, worming around like an insect larva. He’s snuggled right down in the bomb cradle with Moanin’ Max leaning over him like some wild worker ant.

The engines fire up with a hoarse roar. The priest and the rabbi sit and talk. Black clothes, white faces, grey teeth.

“Do you have any food?” the priest asks. He’s powerfully built, young, with thin blonde hair. One hell of a Notre Dame linebacker under those robes.

The rabbi is a little fellow with a fedora and a black beard. He’s got a Franz Kafka mouth, all ticks and teeth. “It’s my understanding that we’ll breakfast in the terminal after the release.”

The priest is getting two hundred for this, the rabbi three. He has a bigger name. If the rushes work out they’ll be witnessing the other escapes as well.

It’s not a big plane, really, and no matter which way Eddie points the camera, there’s always a white piece of Houdini in the frame. Up front you can see Johnny G. in profile, handsome Johnny not looking too good. There’s sweat-beads on his long upper lip, booze sweat. Peace is coming hard to Johnny.

“Just spiral her on up,” Slick Tires says softly. “Like a bed-spring, Johnny.”

Out the portholes you can see the angled horizon sweep by, until they hit the high mattress of clouds. Max watches the altimeter, grinning and showing his teeth. They punch out of the clouds, into high slanting sunlight. Johnny holds to the helix…he’d go up forever if no one said stop…but now it’s high enough.

“Bombs away!” Slick Tires calls back. The priest crosses himself and Moanin’ Max pulls the release handle. Shot of white-wrapped Houdini in the coffin-like bomb-cradle. The bottom falls out, and the long form falls slowly, weightlessly at first. Then the slipstream catches one end, and he begins to tumble, dark white against the bright white of the clouds below.

Eddie holds the shot as long as he can. There’s a big egg-shaped cloud down there, with Houdini falling towards it. Houdini begins to unwrap himself. You can see the bandages trailing him, whipping back and forth like a long flagellum, then thip he’s spermed his way into that rounded white cloud.

On the way back to the airstrip, Eddie and the sound-man go around the plane asking everyone if they think Houdini’ll make it.

“I certainly hope so,” the rabbi.

“I have no idea,” the priest, hungry for his breakfast.

“There’s just no way,” Moanin’ Max. “He’ll impact at two hundred miles per.”

“Everyone dies,” Johnny G.

“In his position I expect I’d try to drogue-chute the bandages,” Slick Tires.

“It’s a conundrum,” the judge.

The clouds drizzle and the plane throws up great sheets of water when it lands. Eddie films them getting out and filing into the small terminal, deserted except …

Across the room, with his back to them, a man in pajamas is playing pinball. Cigar-smoke. Someone calls to him, and he turns—Houdini.

Houdini brings his mother to see the rushes. Everyone except for her loves it. She’s very upset, though, and tears at her hair. Lots of it comes out, lots of white old hair on the floor next to her wheelchair.

Back at home Houdini gets down on his knees and begs and begs until she gives him permission to finish the movie. Rabstein at Pathé figures two more stunts will do it.

“No more magic after that,” Houdini promises. “I’ll use the money to open us a little music shop.”

“Dear boy.”

For the second stunt they fly Houdini and his mom out to Seattle. Rabstein wants to use the old lady for reaction shots. Pathé sets the two of them up in a boarding house, leaving the time and nature of the escape indeterminate.

Eddie Machotka sticks pretty close, filming bits of their long strolls down by the docks. Houdini eating a Dungeness crab. His mom buying taffy. Houdini getting her a wig.

Four figures in black slickers slip down from a fishing boat. Perhaps Houdini hears their footfalls, but he doesn’t deign to turn. Then they’re upon him: the priest, the judge, the rabbi, and this time a doctor as well—could be Rex Morgan.

While the old lady screams and screams, the doctor knocks Houdini out with a big injection of sodium pentathol. The great escape artist doesn’t resist, just watches and smiles till he fades. The old lady bashes the doctor with her purse before the priest and rabbi get her and Houdini bundled onto the fishing boat.

On the boat, it’s Johnny G. and the A-Holes again. Johnny can fly anything, even a boat. His eyes are bloodshot and all over the place, but Slick Tires guides him out of the harbor and down the Puget Sound to a logging river. Takes a couple of hours, but Eddie time-lapses it all…Houdini lying in half of a hollowed log and the doc shooting him up every so often.

Finally they get to a sort of mill-pond with a few logs in it. Moanin’ Max and the judge have a tub of plaster mixed up, and they pour it in around Houdini. They tape over his head-holes, except for the mouth, which gets a breathing tube. What they do is to seal him up inside a big log, with the breathing tube sticking out disguised as a branch-stub. Houdini is unconscious and locked inside the log by a plaster-of-Paris filling…sort of like a worm dead inside a Twinkie. The priest and the rabbi and the judge and the doctor heave the log overboard.

It splashes, rolls, and mingles with the other logs waiting to get sawed up. There’s ten logs now, and you can’t tell for sure which is the one with Houdini in it. The saw is running and the conveyor belt snags the first log.

Shot of the logs bumping around. In the foreground, Houdini’s mom is pulling the hair out of her wig. Big SKAAAAAZZT sound of the first log getting cut up. You can see the saw up there in the background, a giant rip-saw cutting the log right down the middle.

SKAAAZZZZT! SKAAAAZZZZZT! SKAAAZZZZT! The splinters fly. One by one the logs are hooked and dragged up to the saw. You want to look away, but you can’t…just waiting to see blood and used food come flying out. SKAAAZZZZT!

Johnny G. drinks something from a silver hip-flask. His lips move silently. Curses? Prayers? SKAAAZZZZT! Moanin’ Max’s nervous horse-face sweats and grins. Houdini’s Mom has the wig plucked right down to the hair-net. SKAAAZZZZZT! Slick Tire’s eyes are big and white as hard-boiled eggs. He helps himself to Johnny’s flask. SKAAAZZT! The priest mops his forehead and the rabbi…SKNAKCHUNKFWEEEEE!

Plaster dust flies from the ninth log. It falls in two, revealing only a negative of Houdini’s body. An empty mold! They all scramble onto the mill dock, camera pointing around, looking for the great man. Where is he?

Over the shouts and cheers you can hear the jukebox in the mill-hands’ cafeteria. The Andrews Sisters. And inside there’s…Houdini, tapping his foot and eating a cheeseburger.

“Only one more escape,” Houdini promises, “And then we’ll get that music shop.”

“I’m so frightened, Harry,” his bald mom says. “If only they’d give you some warning.”

“They have, this time. Piece of cake. We’re flying out to Nevada.”

“I just hope you stay away from those show-girls.”

The priest and the rabbi and the judge and the doctor are all there, and this time a scientist, too. A low-ceilinged concrete room with slits for windows. Houdini is dressed in a black rubber wet suit, doing card-tricks.

The scientist, who’s a dead ringer for Albert Einstein, speaks briefly over the telephone and nods to the doctor. The doctor smiles handsomely into the camera, then handcuffs Houdini and helps him into a cylindrical tank of water. Refrigeration coils cool it down, and before long they’ve got Houdini frozen solid inside a huge cake of ice.

The priest and the rabbi knock down the sides of the tank, and there’s Houdini like a big firecracker with his head sticking out for a fuse. Outside is a truck with a hydraulic lift. Johnny G. and the A-Holes are there, and they load Houdini in back. The ice gets covered with pads to keep it from melting in the hot desert sun.

Two miles off, you can see a spindly test-tower with a little shed on top. This is an atom-bomb test range, out in some godforsaken desert in the middle of Nevada. Eddie Machotka rides the truck with Houdini and the A-Holes.

Shot of the slender tower looming overhead, the obscene bomb-bulge at the top. God only knows what strings Rabstein had to pull to get Pathé in on this.

There’s a cylindrical hole in the ground right under the tower, right at ground zero, and they slip the frozen Houdini in there. His head, flush with the ground, grins at them like a peyote cactus. They drive back to the bunker, fast.

Eddie films it all in real time, no cuts. Houdini’s mom is in the bunker, of course, plucking a lapful of wigs. The scientist hands her some dice.

“Just to give him fighting chance, we won’t detonate until you are rolling a two. Is called snake-eyes, yes?”

Close on her face, frantic with worry. As slowly as possible, she rattles the dice and spills them onto the floor.

Snake-eyes!

Before anyone else can react, the scientist has pushed the button, a merry twinkle in his faraway eyes. The sudden light filters into the bunker, shading all the blacks up to grey. The shock wave hits next, and the judge collapses, possibly from heart attack. The roar goes on and on. The crowded faces turn this way and that.

Then it’s over, and the noise is gone, gone except for…an insistent honking, right outside the bunker. The scientist undogs the door and they all look out, Eddie shooting over their shoulders.

It’s Houdini! Yes! In a white convertible with a breast-heavy show-girl!

“Give me my money!” he shouts. “And color me gone!”