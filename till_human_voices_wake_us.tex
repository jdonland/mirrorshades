\chapter*{Till Human Voices Wake Us}
\chapterauthor{Lewis Shiner}

Since his first publication in 1977, Lewis Shiner has written a widely ranging spectrum of short stories: mysteries, fantasies, and horror as well as SF. But the 1984 appearance of his first novel, Frontera, demonstrated his important role in Movement fiction. Frontera combined classic hard-SF structure with a harrowing portrait of postindustrial society in the early twenty first century. The book's gritty realism and deflating treatment of SF icons aroused much comment.

Shiner's work is marked by thorough research and coolly meticulous construction. His lean, vigorous prose shows his allegiance to hard-boiled mystery fiction as well as to such quasi-mainstream authors as Elmore Leonard and Robert Stone.

The son of an anthropologist, Shiner has a fondness for odd belief structures, such as Zen, quantum physics, and mythic archetypes. Though he is capable of bizarre flights of fancy, his work of late has tended toward direct, unsentimental realism and an increasing interest in global politics. The following Shiner story, from 1984, combines mythic images and technosocial politics in a classic cyberpunk mix.

\hrulefill

They were at forty feet, in darkness. Inside the narrow circle of his dive light, Campbell could see coral polyps feeding, their ragged edges transformed into predatory flowers.

If anything could have saved us, he thought, this week should have been it.

Beth's lantern wobbled as she flailed herself away from the white-petaled spines of a sea urchin. She wore nothing but a white T-shirt over her bikini, despite Campbell's warnings, and he could see gooseflesh on her thighs. Which is as much of her body, he thought, as I've seen in ... how long? Five weeks? Six? He couldn't remember the last time they'd made love.

As he moved his light away he thought he saw a shape in the darkness. Shark, he thought, and felt his throat tighten. He swung the lamp back again.

That was when he saw her.

She was frozen by the glare, like any wild animal. Her long straight hair floated up from her shoulders and blended into the darkness. The nipples of her bare breasts were elliptical and purple in the night water.

Her legs merged into a green, scaly tail.

Campbell listened to his breath rasp into the regulator. He could see the width of her cheekbones, the paleness of her eyes, the frightened tremor of the gills around her neck.

Then reflex took over and he brought up his Nikonos and fired. The flare of the strobe shocked her to life. She shuddered, flicked her crescent tail toward him, and disappeared.

A sudden, inexplicable longing overwhelmed him. He dropped the camera and swam after her, legs pumping, pulling with both arms. As he reached the edge of a hundred-foot drop-off, he swept the light in an arc that picked up a final glimpse of her, heading down and to the west. Then she was gone.
squares

He found Beth on the surface, shivering and enraged. "What the hell was the idea of leaving me alone like that? I was scared to death. You heard what that guy said about sharks--"

"I saw something," Campbell said.

"Fan-fucking-tastic." She rode low in the water, and Campbell watched her catch a wave in her open mouth. She spat it out and said, "Were you taking a look or just running away?"

"Blow up your vest," Campbell said, feeling numb, desolate, "before you drown yourself." He turned his back to her and swam for the boat.
squares

Showered, sitting outside his cabin in the moonlight, Campbell began to doubt himself.

Beth was already cocooned in a flannel nightgown near her edge of the bed. She would lie there, Campbell knew, sometimes not even bothering to close her eyes, until he was asleep.

His recurring, obsessive daydreams were what had brought him here to the island. How could he be sure he hadn't hallucinated that creature out on the reef?

He'd told Beth that they'd been lucky to be picked for the vacation, that he'd applied for it months before. In fact, his fantasies had so utterly destroyed his concentration at work that the company had ordered him to come to the island or submit to a complete course of psych testing.

He'd been more frightened than he was willing to admit. The fantasies had progressed from the mild violence of smashing his CRT screen to a bizarre, sinister image of himself floating outside his shattered office windows, not falling the forty stories to the street, just drifting there in the whitish smog.

High above him, Campbell could see the company bar, glittering like a chrome-and-steel monster just hatched from its larval stage. He shook his head. Obviously he needed sleep. Just one good night's rest and things would get back to normal.
squares

In the morning Campbell went out on the dive boat while Beth slept in. He was distracted, clumsy, bothered by shadows in his peripheral vision.

The dive master wandered over while they were changing tanks and asked him, "You nervous about something!"

"No," Campbell said. "I'm fine."

"There's no sharks on this part of the reef, you know."

"It's not that," Campbell said. "There's no problem. Really."

He read the look in the dive master's eyes: another case of shell shock. The company must turn them out by the dozens, Campbell thought. The stressed-out executives and the boardroom victims, all with the same glazed expressions.

That afternoon they dove a small wreck at the east end of the island. Beth paired off with another woman, so Campbell stayed with his partner from the morning, a balding pilot from the Cincinnati office.

The wreck was no more than a husk, an empty shell, and Campbell floated to one side as the others crawled over the rotting wood. His sense of purpose had disappeared, left him wanting only the weightlessness and lack of color of the deep water.
squares

After dinner he followed Beth out onto the patio. He'd lost track of how long he'd been watching the clouds over the dark water when she said, "I don't like this place."

Campbell looked back at her. She was sleek and pristine in her white linen jacket, the sleeves pushed up to her elbows, her still-damp hair twisted into a chignon and spiked with an orchid. She'd been sulking into her brandy since they'd finished dinner, and once again she'd astonished him with her ability to exist in a completely separate mental universe from his own.

"Why not?"

"It's fake. Unreal. This whole island." She swirled the brandy but didn't drink any of it. "What business does an American company have owning an entire island? What happened to the people who used to live here?"

"In the first place," Campbell said, "it's a multinational company, not just American. And the people are still living here, only now they've got jobs instead of starving to death." As usual, Beth had him on the defensive. In fact he wasn't all that thrilled with the Americanization of the island. He'd imagined natives with guitars and congas, not portable stereos that blasted electronic reggae and neo-funk. The hut where he and Beth slept was some kind of geodesic dome, air-conditioned and comfortable, but he missed the sound of the ocean.

"I just don't like it," Beth said. "I don't like top secret projects that have to be kept behind electric fences. I don't like the company flying people out here for vacations the way they'd throw a bone to a dog."

Or a straw to a drowning man, Campbell thought. He was as curious as anybody about the installations at the west end of the island, but of course that wasn't the point. He and Beth were walking through the steps of a dance that Campbell now saw would inevitably end in divorce. Their friends had all been divorced at least once, and an eighteen-year marriage probably seemed as anachronistic to them as a 1957 Chevy.

"Why don't you just admit it?" Campbell said. "The only thing you really don't like about the island is the fact that you're stuck here with me."

She stood up, and Campbell felt, with numbing jealousy, the stares of men all around them focus on her. "I'll see you later," she said, and heads turned to follow the clatter of her sandals.

Campbell ordered another Salva Vida and watched her walk downhill. The stairs were lit with Japanese lanterns and surrounded by wild purple and orange flowers. By the time she reached the sandbar and the line of cabins, she was no more than a shadow, and Campbell had finished most of the beer.

Now that she was gone, he felt drained and a little dizzy. He looked at his hands, still puckered from the long hours in the water, at the cuts and bruises of three days of physical activity. Soft hands, the hands of a company man, a desk man. Hands that would push a pencil or type on a CRT for another twenty years, then retire to the remote control of a big-screen TV.

The thick, caramel-tasting beer was starting to catch up to him. He shook his head and got up to find the bathroom. His reflection shimmered and melted in the warped mirror over the bathroom sink. He realized he was stalling, staying away from the chill, sterile air of the cabin as long as he could.

And then there were the dreams. They'd gotten worse since he'd come to the island, more vivid and disturbing every night. He couldn't remember details, only slow, erotic sensations along his skin, a sense of floating in thin, crystalline water, of rolling in frictionless sheets. He'd awaken from them gasping for air like a drowning fish, his penis swollen and throbbing.

He brought another beer back to his table, not really wanting it, just needing to hold it in his hands. His attention kept wandering to a table on a lower level, where a rather plain young woman sat talking with two men in glasses and dress shirts. He couldn't understand what was so familiar about her until she tilted her head in a puzzled gesture and he recognized her. The broad cheekbones, the pale eyes.

He could hear the sound of his own heart. Was it just some kind of prank, then? A woman in a costume? But what about the gill lines he'd seen on her neck? How in God's name had she moved so quickly?

She stood up, made apologetic gestures to her friends. Campbell's table was near the stairs, and he saw she would have to pass him on her way out. Before he could stop to think about it, he stood up, blocking her exit, and said, "Excuse me?"

"Yes?" She was not that physically attractive, he thought, but he was drawn to her anyway, in spite of the heaviness of her waist, her solid, shortish legs. Her face was older, more tired than the one he'd seen out on the reef. But similar, too close for coincidence. "I wanted to ... could I buy you a drink?" Maybe, he thought, I'm just losing my mind.

She smiled, and her eyes crinkled warmly. "I'm sorry. It's really very late, and I have to be at work in the morning."

"Please," Campbell said. "Just for a minute or two." He could see her suspicion, and behind that a faint glow of flattered ego. She wasn't used to being approached by men, he realized. "I just want to talk with you."

"You're not a reporter, are you?"

"No, nothing like that." He searched for something reassuring. "I'm with the company. The Houston office."

The magic words, Campbell thought. She sat down in Beth's chair and said, "I don't know if I should have any more. I'm about half looped as it is."

Campbell nodded, said, "You work here, then."

"That's right."

"Secretary?"

"Biologist," she said, a little sharply. "I'm Dr. Kimberly." When he didn't react to the name, she softened it by adding, "Joan Kimberly."

"I'm sorry," Campbell said. "I always thought biologists were supposed to be homely." The flirtation came easily. She had the same beauty as the creature on the reef, a sort of fierce shyness and alien sensuality, but in the woman they were more deeply buried.

My God, Campbell thought, I'm actually doing this, actually trying to seduce this woman. He glanced at the swelling of her breasts, knowing what they would look like without the blue oxford shirt she wore, and the knowledge became a warmth in his groin.

"Maybe I'd better have that drink," she said. Campbell signaled the waiter.

"I can't imagine what it would be like to live here," he said. "To see this every day."

"You get used to it," she said. "I mean, it's still unbearably beautiful sometimes, but you have your work, and your life goes on. You know?"

"Yes," Campbell said. "I know exactly what you mean."
squares

She let Campbell walk her home. Her loneliness and vulnerability were like a heavy perfume, so strong it repelled him at the same time that it pulled him irresistibly toward her.

She stopped at the doorway of her cabin, another geodesic. This one sat high on the hill, buried in a grove of palms and bougainvilleas. The sexual tension was so strong that Campbell could feel his shirtfront trembling.

"Thank you," she said, her voice rough. "You're very easy to talk to."

He could have turned away then, but he couldn't seem to unravel himself. He put his arms around her, and her mouth bumped against his, awkwardly. Then her lips began to move and her tongue flicked out eagerly. She got the door open without moving away from him, and they nearly fell into the house.
squares

He pushed himself up on extended arms and watched her moving beneath him. The moonlight through the trees was green and watery, falling in slow waves across the bed. Her breasts swayed heavily as she arched and twisted her back, the breath bubbling in her throat. Her eyes were clenched tight, and her legs wrapped around his and held them, like a long forked tail.
squares

Before dawn he slid out from under her limp right arm and got into his clothes. She was still asleep as he let himself out.

He'd meant to go back to his cabin, but instead he found himself climbing to the top of the island's rocky spine to wait for the sun to come up.

He hadn't even showered. Kimberly's perfume and musk clung to his hands and crotch like sexual stigmata. It was Campbell's first infidelity in eighteen years of marriage, a final, irreversible act.

He knew most of the jargon. Mid-life crisis and all that. He'd probably seen Kimberly at the bar some other night and not consciously remembered her, projected her face onto a fantasy with obvious Freudian water/rebirth connotations.

In the dim, fractionated light of the sunrise, the lagoon was gray, the line of the barrier reef a darker smudge broken by whitecaps that curved like scales on the skin of the ocean. Dry palm fronds rustled in the breeze, and the island birds began to chirp and stutter themselves awake. A shadow broke from one of the huts on the beach below and climbed toward the road, weighted down with a large suitcase and a flight bag. Above her, in the asphalt lot at the top of the stairs, a taxi coasted silently to a stop and doused its lights.

If he had run, he could have reached her and maybe could even have stopped her, but the hazy impulse never became strong enough to reach his legs. Instead, he sat until the sun was hot on his neck and his eyes were dazzled into blindness by the white sand and water.
squares

On the north side of the island, facing the mainland, the village of Espejo sprawled in the mud for the use of the resort and the company. A dirt track ran down the middle of it, oily water standing in the ruts. The cinder-block houses on concrete piers and the Fords rusting in the yards reminded Campbell of an American suburb in the fifties, warped by nightmare.

The locals who worked in the company's kitchens and swept the company's floors lived here, and their kids scuffled in alleys that smelled of rotting fish or lay in the shade and threw rocks at three-legged dogs. An old woman sold Saint Francis flour-sack shirts from ropes tied between pilings of her house. Under an awning of corrugated green plastic, bananas lay in heaps and flies swarmed over haunches of beef. At the end of the main street was a farmacia with a faded yellow Kodak sign that promised One Day Service.

Campbell blinked and found his way to the back, where a ten-year-old boy was reading La Novela Policiaca. The boy set the comic on the counter and said, "Yes, sir?"

"How soon can you develop these?" Campbell shoved the cartridge toward him.

"¿Mande?"

Campbell gripped the edge of the counter. "Ready today?" he asked slowly.

"Tomorrow. This time."

Campbell took a twenty out of his wallet and held it face down on the scarred wood. "This afternoon?"

"Momentito." The boy tapped something out on a computer terminal at his right hand. The dry clatter of the keys filled Campbell with distaste. "Tonight, okay?" the boy said. "A las seis." He touched the dial of his watch and said, "Six."

"All right," Campbell said. For another five dollars he bought a pint of Canadian Club, and then he went back onto the street. He felt like a sheet of weakly colored glass, as if the sun shone clear through him. He was a fool, of course, to be taking this kind of chance with the film, but he needed that picture. He had to know.
squares

He anchored the boat as close as possible to where it had been the night before. He had two fresh tanks and about half the bottle of whiskey left. It was barely noon, the sun a white ball of fire in the sky.

Diving drunk and alone was against every rule anyone had ever tried to teach him, but the idea of a simple, clean death by drowning seemed ludicrous to Campbell, not even worth consideration. Fate obviously had something more convoluted in mind for him.

His diving jeans and sweatshirt, still damp and salty from the night before, were suffocating him. He got into his tank as quickly as he could and rolled over the side.

The cool water revived him, washed him clean. He purged the air from his vest and dropped straight to the bottom. Dulled by whiskey and lack of sleep, he floundered for a moment in the sand before he could get his buoyancy neutral.

At the edge of the drop-off he hesitated, then swam to his right, following the edge of the cliff. His physical condition made him burn air faster than he wanted to; going deeper would only make it worse.

The bright red of a Coke can winked at him from a coral head. He crushed it and stuck it in his belt, suddenly furious with the company and its casual rape of the island, with himself for letting them manipulate him, with Beth for leaving him, with the entire world and the human race. He kicked hard, driving himself through swarms of jack and blue tang, hardly noticing the twisted, brilliantly colored landscape that moved beneath him.

Some of the drunkenness burned off in his first burst of energy, and he gradually slowed, wondering what he possibly could hope to accomplish. It was useless, he thought. He was chasing a phantom. But he didn't turn back.

He was still swimming when he hit the net.

It was nearly invisible, a web of monofilament in one-foot squares, strong enough to hold a shark or a school of porpoises. He tested it with the serrated edge of his diver's knife, with no luck. He was close to the west end of the island, where the company kept their research facility. The net followed the line of the reef as far down as he could see and extended out into the open water.

She was real, he thought. They built this to keep her in. But how did she get past it?

When he'd last seen her she'd been heading down. Campbell checked his seaview gauge, saw that he had less than five hundred pounds of air left. Enough to take him down to a hundred feet or so and right back up. The sensible thing to do was to return to the boat and bring a fresh tank back with him.

He went down anyway.

He could see the fine wires glinting as he swam past them. They seemed bonded to the coral itself, by some process he could not even imagine. He kept his eyes moving between the depth gauge and the edge of the net. Much deeper than a hundred feet and he would have to worry about decompression as well as an empty tank.

At 100 feet he tripped his reserve lever. Three hundred pounds and counting. All the reds had disappeared from the coral, leaving only blues and purples. The water was noticeably darker, colder, and each breath seemed to roar into his lungs like a geyser. Ten more feet, he told himself, and at 125 he saw the rip in the net.

He snagged his backpack on the monofilament and had to back off and try again, fighting panic. He could already feel the constriction in his lungs again, as if he were trying to breathe with a sheet of plastic over his mouth. He'd seen tanks that had been sucked so dry that the sides caved in. They found them on divers trapped in rockslides and tangled in fishing line.

His tank slipped free and he was through, following his bubbles upward. The tiny knot of air in his lungs expanded as the pressure around him let up, but not enough to kill his need to breathe. He pulled the last of the air out of the tank and forced himself to keep exhaling, forcing the nitrogen out of his tissues.

At fifty feet he slowed and angled toward a wall of coral, turned the comer, and swam into a sheltered lagoon.

For a few endless seconds he forgot that he had no air.

The entire floor of the lagoon was laid out in squares of greenery: kelp, mosses, and something that looked like giant cabbage. A school of red snappers circled past him, herded by a metal box with a blinking light on the end of one long antenna. Submarines with spindly mechanical arms worked on the ocean floor, thinning the vegetation and darkening the water with chemicals. Two or three dolphins were swimming side by side with human divers, and they seemed to be talking to each other.

His lungs straining, Campbell turned his back on them and kicked for the surface, trying to stay as close to the rocks as he could. He wanted to stop for a minute at ten feet, to give at least a nod to decompression, but it wasn't possible. His air was gone.

He broke the surface less than a hundred feet from a concrete dock. Behind him was a row of marker buoys that traced the line of the net all the way out to sea and around the far side of the lagoon.

The dock lay deserted and steaming in the sun. Without a fresh tank, Campbell had no chance of getting out the way he'd come in. If he swam out on the surface, he'd be as conspicuous as a drowning man. He had to find another tank or another way out.

Hiding his gear under a sheet of plastic, he crossed the hot concrete slab to the building behind it, a wide, low warehouse full of wooden crates. A rack of diving gear was built into the left-hand wall. Campbell was starting for it when he heard a voice behind him.

"Hey, you! Hold it!"

Campbell ducked behind a wall of crates, saw a tiled hallway opening into the back of the building, and ran for it. He didn't get more than three or four steps before a uniformed guard stepped out and pointed a .38 at his chest.
squares

"You can leave him with me."

"Are you sure, Dr. Kimberly?"

"I'll be all right," she said. "I'll call you if there's any trouble."

Campbell collapsed in a plastic chair across from her desk. The office was strictly functional--waterproof and mildew-resistant. A long window behind Kimberly's head showed the lagoon and the row of marker buoys.

"How much did you see?" she asked.

"I don't know. I saw what looked like farms. Some machinery."

She slid a photograph across the desk to him. It showed a creature with a woman's breasts and the tail of a fish. The face was close enough to Kimberly's to be her sister.

Or her clone's.

Campbell suddenly realized the amount of trouble he was in.

"The boy at the farmacia works for us," Kimberly said.

Campbell nodded. Of course he did. Where else would he get a computer? "You can have the picture," Campbell said, blinking the sweat out of his eyes. "And the negative."

"Let's be realistic," she said, tapping the keys of her CRT and studying the screen. "Even if we let you keep your job, I don't see how we could hold your marriage together. And then you have two kids to put through college..." She shook her head. "Your brain is full of hot information. There are too many people who would pay to have it, and there are just too many ways you can be manipulated. You're not much of a risk, Mister Campbell." She radiated hurt and betrayal, and he wanted to slink away from her in shame.

She got up and looked out the window. "We're building the future here," she said. "A future we couldn't even imagine fifteen years ago. And that's just too valuable to let one person screw up. Plentiful food, cheap energy, access to a computer net for the price of a TV set, a whole new form of government--"

"I've seen your future," Campbell said. "Your boats have killed the reef for over a mile around the hotel. Your Coke cans are lying all over the coral bed. Your marriages don't last and your kids are on drugs and your TV is garbage. I'll pass."

"Did you see that boy in the drugstore? He's learning calculus on that computer, and his parents can't even read and write. We're testing a vaccine on human subjects that will probably prevent leukemia. We've got laser surgery and transplant techniques that are revolutionary. Literally."

"Is that where she came from?" Campbell asked, pointing to the photograph.

Kimberly's voice dropped. "It's synergistic, don't you see? To do the transplants we had to be able to clone cells from the donor. To clone cells we had to have laser manipulation of the genes..."

"They cloned your cells? Just for practice?"

She nodded slowly. "Something happened. She grew, but she stopped developing, kept her embryonic form from the waist down. There was nothing we could do except ... make the best of it."

Campbell took a longer look at the picture. No, not the romantic myth he had first imagined. The tail was waxy looking in the harsh light of the strobe, the fins more clearly undeveloped legs. He stared at the photo in queasy fascination. "You could have let her die."

"No. She was mine. I don't have much, and I wouldn't give her up." Kimberly's fists clenched at her sides. "She's not unhappy, she knows who I am. In her own way I suppose she cares for me." She paused, looking at the floor. "I'm a lonely woman, Campbell. But of course you know that."

Campbell's throat was dry. "What about me?" he rasped, and managed to swallow. "Am I going to die?"

"No," she said. "Not you either...."
squares

Campbell swam for the fence. His memories were cloudy and he had trouble focusing his thoughts, but he could visualize the gap in the net and the open ocean beyond it. He kicked down easily to 120 feet, the water cool and comforting on his naked skin. Then he was through, drifting gently away from the noise and stink of the island, toward some primal vision of peace and timelessness.

His gills rippled smoothly as he swam.