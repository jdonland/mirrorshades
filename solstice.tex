\chapter{Solstice}
\chapterauthor{James Patrick Kelly}

James Patrick Kelly's first publication came in 1975. His career accelerated through the early 1980s; he has written almost two dozen short stories and two novels. His second book, Freedom Beach, was written with John Kessel and attracted praise for its lively invention and playful literary erudition.

Like Kessel, Kelly has been associated with a loose group of Eighties SF writers, generally identified as SF's new 'literary wing," as (theoretically) opposed to the harder tech interests of the cyberpunks.

In 1985, Kelly gleefully complicated matters by publishing the following story, a high-tech extravaganza of headlong visionary daring. He followed it with two more stories, equally inventive and original, in a self-proclaimed "cyberpunk trilogy." By his example, Kelly has demonstrated a truism in SF—that where critics divide and analyze, writers will unite and synthesize.

\hrulefill

\firstletter{O}nce a year they open it to the public. Some spend lifetimes planning for the day. Others arrive by chance, fortunate sightseers swarming out of the tour hovers. They record it all but only rarely understand what they are witnessing. Years later a few of the disks will come out to pep up drooping parties. Most will be forgotten.

It happens on the summer solstice. One of two points on the ecliptic at which its distance from the celestial equator is greatest. The longest day of the year. A turning point.
squares

They arrived late in the afternoon when the crowd was starting to thin. A tall man in his early forties. A teenaged girl. They had the same gray eyes. Her straw-colored hair had begun to darken, just as his had darkened when he was seventeen. There was an inescapable similarity in the way they whispered jokes to one another and laughed at the people around them. Neither carried a camera.

They had come to wander among the sarsen stones of what Tony Cage considered the most extraordinary antiquity in the world. Yes, the pyramids were older, bigger, but they had long since yielded their mystery. The Parthenon had once been more beautiful but the acids of history had etched it beyond recognition. But Stonehenge … Stonehenge was unique. Essential. It was a mirror in which each age could observe the quality of its imagination, in which every man could measure his size.

They joined the queue waiting to enter the dome. Occasional screams of synthesized music pierced the buzz of the crowd; the free festival being held in a nearby field was hitting a frenzied peak. Perhaps later they would explore its delights, but now they had reached the entrance to the exterior shell of the dome. The girl laughed as she popped through the bubble membrane.

“It’s like being kissed by a giant,” she said.

They were in the space between the exterior and interior shells of the dome. On any other day this would have been as close as they could have come to the stone circles. The dome was made of hardened optical plastic with a low refractive index. Walkways spiraled upward in the space between the shells; tourists who climbed to the top had a bird’s-eye view of Stonehenge.

They entered the inner shell. There was a reporter with a microcam standing near the Heel Stone; he spotted them and started waving. “Pardon, sir, pardon!” Cage pulled the girl out of the flow of the crowd and waited; he did not want the fool calling his name in front of all these people.

“You’re the drug artist.” The reporter drew them aside. A daisy smile bloomed on his obsidian face. “Case, Cane…” he tapped the skull plug behind his ear as if to dislodge the memory from his wetware.

“Cage.”

“And this?” The smile became a smirk. “Your lovely daughter?”

Cage thought about punching the man. He thought about walking away. The girl laughed.

“I’m Wynne.” She shook the reporter’s hand.

“Name’s Zomboy. Wiltshire stringer for SONIC. Have you seen the old stones before? I could show you around.” Cage kept expecting the microcam’s red light to come on but the reporter seemed strangely hesitant. “I say, you wouldn’t by any chance be holding any free samples? For one of your major fans?”

Wynne bit her lip to stifle a giggle and reached into her pocket. “I doubt you could tell Tony much about Stonehenge. Sometimes I think he lives for this place.” She produced a plastic bottle, shook some green capsules into her palm and offered them to the reporter.

He took one and inspected it carefully. “No label on the casing.” He fixed his suspicion on Cage. “You’re sure it’s safe?”

“Hell no,” said Wynne. She popped two of the capsules into her mouth. “Very experimental. Turn your brains to blood pudding.” She offered one to Cage and he took it. He wished Wynne would stop playing these twisted games. “We’ve been eating them all day,” said Wynne. “Can’t you tell?”

Gingerly, the reporter put one in his mouth. Then the red light came on. “So you’re a devotee of Stonehenge, Mr. Cage?”

“Oh yes.” Wynne was babbling. “He comes here all the time. Gives lectures to whoever will listen. Says there’s a kind of magic to the place.”

“Magic?” The lens stared at Cage, had never left him.

“Not the kind of magic you’re thinking of, I’m afraid.” Cage hated looking into cameras when he was twisted. “No wizards or human sacrifices or bolts of lightning. A subtle kind of magic, the only kind still possible in this overly explained world.” The words rolled out unbidden—perhaps because he had spoken them before. “It has to do with the way a mystery captures the imagination and becomes an obsession. A magic that works exclusively in the mind.”

“And who better to contemplate mind magic than the celebrated drug artist, Mr. Tony Cage.” The reporter spoke not to them but to an unseen audience.

Cage smiled into the camera.
squares

In 1130 Henry of Huntingdon, an archdeacon at Lincoln, was commissioned by his bishop to write a history of England. His was the first written account of a place called “Stanenges, where stones of wonderful size have been erected after the manner of doorways, so that doorway appears to have been raised upon doorway; and no one can conceive of how such great stones have been raised aloft, or why they were built there.” The name derives from the Old English, “stan”: stone, and “hengen”: gallows. Medieval gallows consisted of two posts and a crosspiece. There is no record of executions at Stonehenge, although Geoffrey of Monmouth, writing six years after Henry, describes the massacre of four hundred and sixty British lords by treacherous Saxons. Geoffrey claims that, as a memorial to the dead, Uther Pendragon and Merlin stole the sacred megaliths known as the Giants’ Dance from the Irish by magic and force of arms and reerected them on the Wiltshire Plain. The “Merlin theory” of Stonehenge’s construction, while certainly true to the spirit of Anglo-Irish relations, was a piece with the rest of Geoffrey’s Arthurian tapestry: a jingoistic fairy tale.
squares

“Wake up.”

Cage had been dreaming of sleep. A vast treeless pasture, green waves rolling to the horizon. The animals shied away from him as he wandered among them. He was lost.

“Tony.”

The cryogenicists claimed that stiffs did not dream. Strictly speaking they were right, but as the tank thawed him out, his synapses begin to fire again. Then dreams came.

“Wake up, Tony.”

His eyelids flickered. “Go ‘way.” He felt like a pincushion. He opened his eyes and stared at her. For a moment he thought he was still dreaming. Wynne had shaved her hair off except for a spiky multicolored fan that ran from ear to ear. From the looks of it she had just had a new body tint done. In blue.

“I’m leaving, Tony. I only stayed around to make sure you thawed all right. I’m all packed.”

He mumbled something sarcastic. It did not make much sense, even to him, but the tone of voice was right. He knew she was not as strong as she thought she was. Otherwise she wouldn’t have tried to spring this on him while he was still groggy. He sat up in the tank.

“Leave, then,” he said. “Help me out of here.”

He huddled on the couch in the drawing room and tried not to feel cold as he stared at the mist which hung over Galway Bay. There was no horizon; both the sky and the water were the color of old thatch. Exactly the same kind of day it had been when he had climbed into the tank. He had never much liked Ireland. But when the Republic had extended its tax benefits to drug artists, his accountants had forced citizenship on him.

Wynne had a fire going; the room had filled with the bitter smell of burning peat. She brought him a cup of coffee. There was a red and green pill on the saucer. “What’s this?” He held it up.

“New. Serentol. Helps you relax.”

“I’ve been stiff for six months, Wynne. I’m plenty relaxed.”

She shrugged, took the pill from him and popped it into her mouth. “No sense wasting it.”

“Where will you go?” he said.

She seemed surprised that he would ask, as if she had expected an argument first. “England for a while,” she said. “After that I don’t know.”

“All right.” He nodded. “No sense staying here any longer than you have to. But you will come back when it’s tank time again?”

She shook her head; the peacock hair fluttered. He decided that he could get used to it.

“How much will it cost to change your mind?”

She smiled. “You haven’t got enough.”

He matched the smile. “Come give us a kiss then.” He pulled her down onto his lap. She was twenty-two years old and very beautiful. He knew it was immodest of him to think this because when he looked at her he saw himself. The best thing about these revivals was watching her catch up in age as he hibernated the winters away to establish residency for tax purposes. In another thirty-odd years they would both be in their fifties. “I love you,” he said.

“Sure.” Her voice was slurred. “Daddy loves his little girl.”

Cage was shocked. He had never heard her talk that way before. Something had happened while he was in the tank. But then she giggled and put her hand on his thigh. “You can come with us, if you want.”

“Us?” He brushed his fingertips across the smooth scalp and wondered how many serentols she had taken that day.
squares

James I was fascinated by Stonehenge, so much so that he commissioned the celebrated architect Inigo Jones to draw a plan of the stones and determine their purpose. The results of Jones’ studies were published posthumously in 1655 by his son-in-law. Jones rejected the notion that such a structure could have been raised by any indigenous people since “the ancient Britains (were) utterly ignorant, as a Nation wholly addicted to Wars, never applying themselves to the Study of Arts, or troubling their thoughts with any Excellency.” Instead Jones, who had learned his art in Renaissance Italy and was a student of classical architecture, declared that Stonehenge must be a Roman temple, a blending of the Tuscan and Corinthian styles, possibly built during the reign of the Flavian emperors.

In 1663 Dr. Walter Charlton, a physician to Charles II, disputed Jones’ theory, maintaining that Stonehenge was built by the Danes “to be a Court Royal, or a place for the Election and Inauguration of their Kings.” The poet Dryden applauded Charlton in verse, “Stoneheng, once thought a Temple, you have found A Throne, where Kings, our Earthly Gods, were crown’d”

In fact, many pointed to the crown-like shape of Stonehenge as proof of this theory. Of couse these speculations, coming so soon after Charles had been restored to the throne following a long exile, were politically convenient. The most astute courtiers spared no effort to discredit Cromwell’s republic and to curry royal favor by reasserting the antiquity of divine right of kings.
squares

Wynne had been Cage’s greatest extravagance. He had never really sought the money; the entertainment multinationals kept forcing it on him. Once he had acquired a Raphael and a Constable and a Klee, vacationed in the Mindanao Trench on the Habitat Three and at the Disney on the moon, he found precious little else worth the trouble of buying.

People envied him: the rich, the famous drug artist. But when Cage first hit it at Western Amusement, he had almost suffocated in his new wealth. The problem was that the money would not just sit there and keep quiet. It screamed for attention. It had to be collected, managed and disbursed by an endless procession of people with tight smiles and firm handshakes who insisted on giving him advice no matter how much he paid them to leave him alone. To them he was Tony Cage, Incorporated.

It was while he was developing Focus that Cage decided he needed someone to help him spend the money. He felt no particular urge to contract a marriage. None of the women he was sleeping with at the time mattered to him. He knew that they had been drawn by that irresistible pheromone: the smell of success. He wanted to share his life with someone who would be bound to him by ties no lawyer could break. Someone who would be uniquely his. Forever. Or so he imagined. Perhaps there was nothing romantic about it at all. Maybe the sociobiologists were right and what was at work was an instinct that had been wired into the brains of vertebrates back in the Devonian: reproduce, reproduce.

Wynne was carried in an artificial womb. It was cleaner that way, medically and legally. All it took was a tissue culture from a few of Cage’s intestinal epithelial cells and some gene sculpturing to change the “Y” chromosome to an “X,” as well as a few other miscellaneous improvements. Just this and a little matter of one-point-two-million new dollars and Wynne was his.

He told himself that he must reject all the labels that they tried to put on Wynne. He refused to think of her as his daughter. Nor was she exactly his clone. She was like a twin, except that they were carried to term in different wombs and her birth came some twenty-six years after his and the abusive environment that twisted him never touched her. Which was to say she was nothing like a twin. She was something new, something infinitely precious. There were no rules for her behavior, no boundaries for her abilities. He liked to brag that he had got exactly what he had ordered. “She’s prettier than me, smarter, a better tennis player,” he would joke, “worth every cent.”

Cage did not have much time for Wynne when she was a toddler. Back in those days he was still testing the product on himself and often as not would stagger home quite twisted. He found her an English nanny—the best kind. He did not pay Mrs. Detling to love the little girl; Wynne earned that on her own. The fierce old woman spent truckloads of Cage’s money on Wynne; their philosophy was to treat the girl as if she were a blank disk on which must be recorded only the most important information. For Wynne’s sake they traveled whenever Cage could get away from the lab. Detling helped her develop an Old World command of languages; Wynne spoke English, Russian, Spanish, a smattering of Japanese, and she could read her Virgil in Latin. When she entered third form she tested in the ninety-ninth percentile for her age group on the Geneva Culture-Free Intelligence Profile.

It was not until she was seven that Cage began to take real pleasure in her company. Her charm was an incongruous mix of maturity and childishness.

He came home from the lab one day to find Wynne networking a game on the telelink.

“I thought you were going to see your friend. What’s her name?” he said.

“Haidee? I decided not to when Nanny told me you were coming home early.”

“I just came home to change.” At the time he was working on Laughers and still had a buzz from a morning dose. He did not want to start giggling like a fool in front of the child so he opened the bar and poked a pressure syringe filled with neuroleptic to straighten himself out. “I have a date. Have to go out at six.”

She signed off from the game. “With that new one? Jocelyn?”

“Jocelyn, yes.” He held out his hand for the telelink controller. “Mind if I check the mail?”

She gave it to him. “I miss you when you’re at work, Tony.”

He had heard this before. “I miss you too, Wynne.” He brought up the mail menu on the screen and began the sort.

She snuggled next to him and watched in silence. “Tony,” she said at last, “do grownups ever cry?”

“Hmmm.” Western was bitching about the delays with Laughers, threatened to hold up the bonus from Soar. “Sometimes, I guess.”

“They do?” She sounded shocked. “If they fall down and scrape their knees?”

“Usually it’s because something sad happens.”

“Like what?”

“Something sad.” Long silence. “You know.” He wanted her to change the subject.

“I saw Jocelyn crying.”

She had his attention.

“The other night,” Wynne said. “She came and sat on the couch, waiting for you. I was playing house behind the chair. She didn’t know I was here. She’s ugly, you know, when she cries. The stuff under her eyes makes her tears black. Then she got up and she was going toward the bathroom and she saw me and looked at me like it was my fault she was crying. But she kept going and didn’t say anything. When she came out, she was happy again. At least she wasn’t crying. Did you make her sad?”

“I don’t know, Wynne.” He felt as thought he should be angry but he did not know at whom. “Maybe I did.”

“Well, I don’t think that was a very grownup thing to do. I don’t think I like her much.” Wynne looked at him then to see if she had gone too far.

“Well, what does she have to be sad about? She sees you more than I do and I don’t cry.”

He hugged her. “You’re a good girl, Wynne.” He decided then not to see Jocelyn that night. “I love you.”

Many people try to make a division between personal life and life at work. Before Wynne, Cage had always been lonely, no matter whom he was with. He hated facing the void at the center of his personal life; throwaway women like Jocelyn only fed the emptiness. He went to work to escape himself; this was the secret of his success. But as Wynne grew older he had to change, gradually making room for her in his life until she filled it.
squares

William Stukeley belonged to the grand tradition of English eccentrics. From 1719 to 1724 this impressionable young antiquarian spent his summers exploring Stonehenge. His meticulous fieldwork was not to be equaled for a century and a half. Stukeley made precise measurements of the relationships between the stones. He explored the surrounding countryside and discovered that the circle was but a part of a much larger neolithic complex. He was the first to point out the orientation of Stonehenge’s axis toward the summer solstice. He did not, however, publish these findings until ten years later. In the interim he took holy orders, married, moved from London to Lincolnshire and decided he was a Druid.

From his quirky reading of the Bible, Pliny, and Tacitus, Stukeley had deduced that the Druids must be direct descendants of the Biblical Abraham, who had hitched a ride to England on Phoenician ships. Although his book contained an account of the superb fieldwork at Stonehenge, Stukeley’s polemical intent was best summed up in the frontispiece, a portrait of the author as Chyndonax, a prince of the Druids. It was “a chronological history of the origin and progress of true religion, and of idolatry.” Stukeley painted a vision of noble sages practicing a pure natural religion, the modern equivalent of which, he was at pains to point out, was none other than his own beloved Church of England! The Druids had built Stonehenge as a temple to their serpent god. Although Stukeley believed that the rites practiced there may have included human sacrifice, he was inclined to forgive his spiritual forebears their excesses. Perhaps they had got Abraham’s example wrong.

A hundred years later Stukeley’s Druidical fantasy had wormed its way both into the Encyclopaedia Britannica and the popular imagination. In 1857 a direct rail link between London and Salisbury was established and the Victorians descended in droves. To some Stonehenge was splendid confirmation of the ancient and present greatness of Britannia, to others it was a dark dream of disemboweled maidens and pagan license. It was about this time that the summer solstice became a spectacle. The pubs in nearby Amesbury stayed open all night, although by license only tourists were to be served. If the skies were clear those who staggered on to Stonehenge might number in the thousands. It was not a respectful crowd. They would break bottles against the bluestones and climb the sarsens, dancing in the midsummer dawn. The dreaming stillness of the Wiltshire plain would be shattered by rowdy laughter and the clatter of vehicles.
squares

Cage never liked Tod Schluermann. He told himself the fact that Tod had become Wynne’s lover while Cage was in the tank had nothing to do with it. Nor did it matter that Tod had convinced her to go with him to England. Tod had bounced around the world in his twenty-four years; his father had been an Air Force doctor. Born in the Philippines, he had grown up on bases in Germany, Florida, and Colorado. He had flunked out of the Air Force Academy and had attended several other colleges without acquiring anything more substantial than distaste for getting up early.

Tod was a skinny kid who looked good in the gaudy skintights that had come into fashion. He was handsome in a streamlined way. Beneath his face was the delicate bone structure of a Renaissance madonna. In order to get into the Academy he had needed cochlear implants to correct a slight hearing problem; he had ordered the surgeons to clip his ears. He had no hair on him at all except for a black brush on his head. Like Wynne he had a pale blue skin tint; in some lights he looked like a corpse.

He and Wynne had met at a drug club; she was doing Soar at a light table when he sat down next to her. Cage never understood exactly what Tod was doing at the club. Tod did not often use psychoactive drugs and, although he tried to hide it, seemed to disapprove of regular users. A good candidate for the Drug Temperance League. There was a streak of the puritan in him that distanced him from his licentious generation. In his years in and out of college, Tod had read widely but not well. Like many self-taught men, hesuspected expertise. He had native intelligence, that was clear, but arrogance often made him seem stupid.

“And where are you two going to get the money to live?” Cage asked him over dinner the night before they left Ireland.

Tod swirled a premier cru Chablis in a Waterford crystal wineglass and smiled. “Money is only a problem if you think too much about it, man.”

“Tony, would you stop worrying and pass the veal?” Wynne said. “We’ll be fine.” No one spoke as Tod helped himself to seconds and passed her the serving dish. “After all,” she continued, “we’ll have my allowance.”

There was a spot of Madeira sauce on Tod’s chin. “I don’t want your money, Wynne.”

Cage knew that was for his benefit. Wynne’s allowance was generous enough to support a barrister in Mayfair; he didn’t want her wasting it on Tod. “What makes you think you can learn to program a video synthesizer? People go to school for that, you know.”

“School, yes.” He and Wynne exchanged glances. “Well, you know, the problem is that by the time the teachers get done with you, they’ve mashed your creativity flat. Talk to the good little ‘A’ students who catch on with the big companies and you find that they’ve forgotten why they became artists in the first place. All they know how to do is recycle the stale old crap they learned at school. Anyone can see it. Just call up some videos on the telelink. Yesterday’s news, man.”

“Tod’s been studying very hard. And he’s had some experience already,” said Wynne. “Besides, it isn’t as hard to learn to program as it used to be. They’ve really been working to make the interface a lot more accessible.”

“They? You mean the stale old corporate grinds?”

“Tony.” Wynne pushed away from the table.

“No,” said Tod. “He’s right.” She settled down again. Cage hated the way she always gave in to Tod. “Look, man, I’m not saying that everything you learn in school is corrupt. Look at you. I mean, you could never have developed Soar or anything if you hadn’t done your time. I give you a lot of credit for coming out of that whole. Your work is brilliant. I know artists who can’t even think about a project until they poke a few ml’s of your Focus. But that’s what it’s about, man. What’s important is the art, not the technology.”

“We’re talking about computer-driven video synthesizers, Tod.” Cage laid his fork across the plate. The conversation had killed his appetite. “I happen to know a little something about them. I’ve had programmers working for me, remember. They’re complicated machines. And expensive to use. How are you going to afford the access time you’ll need?”

Tod was the only one still eating. “There are ways,” he said between bites. “The small shops are open to hackers after business hours. Go in at three in the morning and work until five. Cheap.”

“Even if you come up with anything worthwhile, you have to get it distributed. The multinationals like Western Amusement won’t even touch freelance.”

Tod shrugged. “So? I’ll start at the bottom. That’s why we’re going to England. British telelink has plenty of open slots on community access stations. Once people see what I’ve got, it’ll be easy. I know it.”

Wynne poured a volatile stimulant called Bliss into a brandy snifter, breathed deeply of the fumes, and passed it. Tod’s sniff was quick and disapproving; he offered the glass to Cage. Colleen came in with the dessert and Cage realized that there was nothing he could say. It was obvious that Tod did not have the resiliency to fight through the inevitable setbacks. In six months it would be another scheme. Tod would blame Wynne or Cage—someone else!—for his failure and continue his aimless life without them, secure in the delusion that he was a genius trapped in a world full of fools. It was obvious.

But there was Wynne, his beautiful Wynne, beaming at Tod as if he were the second coming of Leonardo. The son of a bitch was going to take her away.
squares

Sir Edmund Antrobus, the baronet who owned Stonehenge, died without an heir in 1915. For years he had squabbled with the Church of the Universal Bond, a modern reincarnation of Druidism based on equal parts of wish fulfillment and bad scholarship, over access to the site. The Chief Druid announced that it had been a Druid curse which had struck Sir Edmund down. Several months later his estate came up for sale. Mr. Cecil Chubb bought Stonehenge at auction for 6600 pounds. He claimed it was an impulse purchase. Three years later Chubb offered Stonehenge to the nation and was knighted by Lloyd George for his generosity.

To the cautious bureaucrats in the Office of Works, Stonehenge was a disaster waiting to happen. Several leaning stones threatened to collapse; wobbly lintels needed readjustment. The government sought help from the Society of Antiquaries for this work. The antiquarians seized the opportunity to expand the repairs into a grandiose, and disastrous, excavation of the entire monument. The government, however, withdrew funding soon after the stones were straightened and for years the Society struggled to finance the dig itself. More often than not Colonel William Hawley had to work alone, living in a drafty hut on the site. In 1926 the project was mercifully suspended, having accomplished little more than to disturb evidence and embarrass the Society. As the bewildered Hawley told the Times: “The more we dig, the more the mystery appears to deepen.”
squares

Like many people, Cage did not chose his career; he became a drug artist by accident. When he started at Cornell he intended to study genetic engineering. At that time Boggs was developing viruses that could alter chromosomes in existing cells. Kwabena had published her pioneering work reconstructing algae for human consumption. It seemed as if every month a different geneticist stepped forward to promise a miracle that would change the world. Cage wanted to make miracles, too. At the time, idealism did not seem so foolish.

Unfortunately, genetic engineering excited every other bright kid in the country. The competition at Cornell was fierce. Cage started doing drugs in his sophomore year just to keep up with the course work. Soon he was the king of the all-nighters. He started with small doses of metrazine; it was only supposed to be psychologically addicting. Cage knew he was tougher than any drug. He did not much care for the recreationalstuff back then. No time. He tried THC on occasion: both pot and the new aerosols from Sweden. Once over a spring break a woman he had been seeing gave him some mescal buttons. She said it would give him new insight. It did—he realized he was wasting his time with her.

Three semesters later it all went wrong. By then he was poking megamphetamines in massive doses, sometimes over eighty milligrams. The initial rush felt like a whole-body orgasm; he did not feel like studying much afterward. His advisor told him to switch out of the program after he took a “C” in genetic chemistry. He was burning up brain cells and losing weight; he had already lost his direction. He knew he had to get clean and start over again.

He had signed up for a course in psychopharmacology on a paranoid whim. If he had to study something, why not the chemistry of what he was doing to himself with his habit? Bobby Belotti was a good teacher; he soon became a friend. He helped Cage get off the megs, helped him salvage a plain vanilla degree in biology and encouraged him to apply to graduate school. Much of Cage’s idealism had been seared away during those twisted semesters of amphetamine psychosis. Maybe that was why it was so easy to convince himself that developing new drugs was just as noble as curing hemophilia.

Cage wrote his master’s thesis on the effects of indole hallucinogens on serontonergic and dopaminergic receptors. The eary indole hallucinogens like LSD and DMT had long since been thought to inhibit production of the neuroregulator serotonin, not surprising since their chemical structures were remarkably similar. His work showed that hallucinogens of this family also effect the dopamine-producing system and that many of the reported effects of these drugs resulted from interactions between these neuroregulators. It was not, he had to admit, particularly innovative or brilliant work; the foundations had been laid long ago. But by then he had grown tremendously bored with being a student. The work reflected it.

He took his degree in the middle of the brief, inglorious rule of the America First Party, a pack of libertarian fanatics bent on dismantling the government of the United States. Sunsetting the Food and Drug Administration sparked the revolution in recreational drug use. Cage was still debating whether to slog on for his doctorate when Bobby Belotti called to say that he was leaving Cornell. Western Amusement was recruiting people to do R\&D for its new psychoactive drug division. Belotti was going. Did Cage want in? Of course.

Belotti’s team was supposed to be looking for a businessman’s flash. Something fast and dirty: fat-soluble so that it could pass quickly into the brain and reach its site of action within minutes after ingestion. It had to beeasily metabolized so that the psychoactive effect would fade within an hour or two. No needles, keep the tolerance effect low. They did not want the users to see God or experience the ultimate orgasm, just a little psychic distortion, some pretty visuals, and leave them with a smile.

Since Cage had already worked with the indole hallucinogens, Belotti gave him pretty much of a free hand. After a couple of frustrating months, he began to look seriously at DMD. It seemed to fit the specifications, except that animal tests did not show significant psychoactive effect. He worried that it might be too subtle. No matter how safe it was, the stuff was no good if it left the user straight as a Baptist accountant. Still Cage was able to convince Belotti to authorize microiontophoretic tests on rats.

Bobby Belotti was a thoroughly disheveled man. His curly black hair resisted combing. He was forever tucking in his shirt; his paunch tugged it out again. There were rings of dried coffee on the upper strata of memos and reports piled on his desk; dust gathered undisturbed in the nooks of his terminal. For all his ability, he was the kind of employee that management preferred to hide from the outside world.

“Look at this.” Cage burst into Belotti’s office and dropped a ten centimeter stack of fanfold paper on his desk. “The DMD results. The stuff inhibits the hell out of the serotonergic system.”

Belotti pulled off his glasses and rubbed his eye with the back of his hand. “Great. Have you got an effect to show me?”

“No, but these numbers say there has to be one. Must be some kind of trigger.”

Belotti sighed and began to shuffle through the papers on his desk. “The front office is screaming for something to sell, Tony. I don’t see that DMD is the answer. Do you?”

“A couple of weeks, Bobby. I’m almost there—I can taste it.”

Belotti found a memo, handed it to Cage. “Give it a rest, Tony. Let’s get a couple of products under our belt, then maybe you can try again.” The memo reassigned Cage to work directly under Belotti’s supervision.

They argued. Cage had never learned to argue and he had a hair-trigger temper. Belotti was too calm, too damn understanding. Although it was never mentioned, the debt that Cage owed Belotti only fueled his outrage. He felt as if he were the wayward student being corrected once again by his kindly professor.

Fuming, Cage brought the odious memo back to his cubicle, shut down his terminal and glared at the empty screen. He was in a mood to lash out, do something crazy. The idea came to him in anger, a stunt straight right out of a mad scientist video. He requisitioned ten milligrams of DMD and went home to try it on himself.

Half an hour after eating the drug, he was lying on the bed in a darkened room, waiting for something,anything to happen. He felt jittery, as if he had just poked some mild speed. His pulse rate was up, he was sweating. He knew from the tests that the drug must have already found its way to his brain. He felt nothing—he was not even angry anymore. At last he got out of bed, turned up the lights and went into the kitchen to make himself a snack. He settled in front of the telelink with a ham and cheese sandwich and turned the monitor on. News. Change channel. Click, click.

No signal. Just visual static. Exactly what it took to trigger DMD’s psychoactive effect. He never ate that sandwich.

Instead he spent the next hour gazing intently at the screen of red, green, and blue phosphors flashing at random. Except that to Cage they were not at all random. He saw patterns, wonderful patterns: wheels of fire, amber waves of grain, angels dancing on the head of a pin, demon faces. He felt as if he himself were a pattern. He was liberated from his body, soaring into the screen to play amidst the beautiful lights.

And then it was over, a very clean finish. It had been an hour and a half since he had eaten the pill; the peak had lasted about forty-five minutes. It was perfect. With a sophisticated light show to trigger DMD’s effect, it might be the most popular drug since alcohol. And it was his, he realized. All his.

After all, Belotti had cut himself out of the action with his memo. It was Cage who had taken the risk, put his body and sanity on the line. Friendship was friendship but Cage knew that if he played this right he could change his life. So he made sure that management heard about DMD from him, making the case that Belotti had tried to stifle important research. If his co-workers resented him for stepping on a friend’s face on his scramble up the ladder, Cage learned not to care. The front office was secretly relieved; Cage was much more presentable than Belotti. Soon he was in charge of the team, then the whole lab.

Cage expected Bobby Belotti to leave, go back to Cornell, but he never did. Perhaps Belotti intended it as a subtle kind of revenge: showing up for work every day, drinking coffee with the man who had betrayed him. Cage refused to be ashamed. He found ways to avoid Belotti, eventually burying him on a minor project that had little chance of success. The two never spoke much after that.

They called the drug Soar and proceeded to market the hell out of it. Western Amusement’s PR flacks made Cage famous before he understood quite what they were doing to him. The interviewers on telelink could not get enough of him. A sanitized bio appeared on most of the major information utilities: the brilliant young researcher, the daring breakthrough, the first step of an incredible psychic journey—at first Cage was amused by it all.

When he could get to the lab he spent much of his time brainstorming mechanisms to trigger Soar’s psychoactive effect. The light table, which can read EEG patterns and transform them into hi-res computer pyrotechnics, was the most successful but there were others. In fact, the hardware aftermarket made Western Amusement almost as much as the drug itself. Cage’s lab turned into a money machine. To keep thecorporate headhunters from stealing him away, Western Amusement gave him participation in the profits. He was soon one of the richest young men in the world.

There were three parts to the recreational drug experience: the chemical itself, the mental state of the user and the environment in which the drug was consumed. What Cage liked to call the surround. As the years passed he became much less involved in developing chemicals. The kids coming out of grad school were better researchers than he had ever been. He was more interested in conceptual design and especially liked dreaming up new surrounds: the sensory deprivation helmet, the alpha strobe. The flacks made the best of his evolving interests; he was no longer a psychopharmacological researcher. He was anointed as the first drugartist.

However, the real reason Cage was forced to cut back on his involvement with drug development had nothing to do with artistic yearning. He had the classic addictive personality; he really loved to get twisted. Over the years he had let some vicious psychoactive chemicals sink claws into his synapses. Although he always managed to pull free, management was nervous. They had made Tony Cage a corporate symbol; they could not afford a meltdown.

Cage should not have been surprised to see his taste for drugs mirrored in Wynne. She began using them when she was nine. By the time she was eleven he was letting her poke some of the major psychoactives. It could hardly have been otherwise if Wynne was to share his life. One of Cage’s perks was a personal bar that put most drug clubs to shame. And his own lab was developing a cannabinol chewing gum aimed at the preteen market. Despite what the Temperance League preached, Cage had not made the drug culture: it had made him. Kids all over the world were getting twisted, reaching for the brightest flash. Still, Wynne’s zest for drugs disturbed him.

Cage tried to ensure that Wynne was never addicted to any one chemical. He saw to it that her habit was various. If she started to build up a cross-tolerance to hallucinogens, for example, he would make her give the whole family a vacation and switch to opiates. Nor was she constantly twisted. She would go on sprees that would last anywhere from a few hours to a few days. Then nothing for a week or two. Still, Cage worried about her. She took some astounding doses.

The summer before she met Tod they flew from the States into da Vinci airport and checked into the Hilton. Even though they had taken the suborbital they were having a hard time getting their biological clocks reset. Since Cage had business in Rome the next day he could not afford to stay jet-lagged. Wynne called room service and had them bring up a couple of strawberry placidex shakes. Cage settled back on his bed; the stuff made him feel as if he were melting into the mattress. Wynne sat in a thermal chair and listlessly switched channels on the telelink. Finally she shut it off and asked him if he thought he took too many drugs.

Cage had been about to doze off; suddenly he was alert as anyone with placidex seeping into his brain can be. “Sure, I think about it all the time. Right now I think I’m okay. There have been times, though, when I thought that I might be in trouble.”

She nodded. “How do you know when you’re in trouble?”

“One sign is when you stop worrying about it.”

She folded her arms as if she were chilled. “That’s a hell of a thing to say. You’re only safe if you’re worried?”

“Or if you’re clean.”

“Oh, come on. What’s the longest you’ve been clean? Recently.”

“Six months. When I was in the tank.” They both laughed. “Since you brought it up,” he said, “let me ask you. Think you do too much?”

She considered, as if the question had surprised her. “Nah,” she said at last. “I’m young. I can take it.”

He told her then about how he had been hooked on amphetamines at Cornell. The story did not seem to impress her.

“But you beat it, obviously,” she said. “So it couldn’t have been that bad.”

“Maybe you’re right,” he agreed. “But it seems to me that I was lucky. A couple more months and I might never have been able to get clean again.”

“I like getting twisted,” she said. “But there are other things I like just as much.”

“For instance?”

“Sex, as if you didn’t know.” She stretched. “Space, weightlessness. Losing myself in a book or a play or a video. Spending your money.” She yawned. The words were coming slower and slower. “Falling asleep.”

“Come to bed, then,” he said. “You’re keeping both of us up.”

She touched the shoulder clasp and her wrapper uncoiled, crinkling, into a pile on the floor. She climbed in next to him. Her skin was cool to the touch. “Who invented placidex anyway?” she said as she snuggled next to him. He could feel the smoothness of her belly against his back. “Man knew what he was doing.”

“The man did not know what he was doing.” It was the placidex that laughed; Cage would rather have made the point. Still, it was funny in a macabre way. “Took a big dose one day, fell asleep in a thermal chair. He had overridden the timer. Baked to death.”

“Died happy, anyway.” She patted his hip and rolled over. “Pleasant dreams.”
squares

In 1965 the astronomer Gerald Hawkins published a book with an immodestly bold title: Stonehenge Decoded. Earlier explainers had always looked beyond Stonehenge for evidence to back up their theories. Some ages found authority in the Bible and church tradition, others in the ruins of Rome or the great historians of antiquity. Like his predecessors, Hawkins invoked the authorities of his time to support his ingenious theory. Using the Harvard-Smithsonian IBM 7090 computer to analyze patterns of solar and lunar alignments at Stonehenge, Hawkins reached a conclusion that electrified the world. Stonehenge had been built as an observatory for ancient astronomers. In fact, he claimed that parts of it formed a “Neolithic computer” which had been used by its builders to predict eclipses of the moon.

Hawkins’ theory caught the popular imagination, in large part due to uncomprehending coverage by the old printed newspapers. Reporters dithered over this marvel: Stone Age scientists had built a computer of sarsen and bluestone that only a modern electronic brain could “decode.” There was even a television special on some of the old pre-telelink networks. Much was made of Hawkins’ use of the computer despite the fact that the numbers it had crunched could easily have been done by hand. And what Hawkins had, in fact, proved was entirely different from what he claimed to have proved. The computer studies showed that the Aubrey holes, a ring of fifty-six regularly spaced pits, could be used to predict eclipses. They did not show that the builders of Stonehenge had had any such purpose in mind. Others soon offered conflicting interpretations and closely-reasoned Stonehenge astronomies proliferated. The problem was soon recognized: Stonehenge had too much astronomical significance. It was a mirror in which any theoretician could see his ideas reflected.
squares

Cage did not immediately follow Tod and Wynne to England. Instead he flew back to the States to check with Western Amusement after his cryogenic vacation. Cage was no longer an actual employee of the company. An independent contractor, he was himself a corporation. Still, there were no doors shut to him at the lab he had made famous, no secrets that he could not learn. The hot news was that in the six months Cage had been in the tank, Bobby Belotti had made a breakthrough on the Share project.

Cage had started the Share project years before when he was still working at the lab full time. He had been thinking about the way social reinforcement seemed to energize recreational drug use. Most users preferred to get twisted with other users, at drug clubs and private parties or before making love or eating a fine meal or free-fall dancing in space. If socialization enhanced pleasure, why not try to find a way for users to share an identical experience? Not just by creating identical surrounds but by synchronizing the effect on a synaptic level. Direct stimulation of the sensory cortex. A kind of artificial telepathy.

Corporate headquarters was skeptical. The mere mention of telepathy gave the whole project the smell of pseudoscience. And it seemed expensive. At the time Cage had thought that the effect would have to be created electro-chemically; the interaction of psychoactive drugs with electronic brain stimulation. Some kind of wetware would probably be necessary. But marketing research showed that many people were afraid of skull plugs. The zombie factor, they called it.

Cage kept after them. If nothing else, he thought Share might be a powerful aphrodisiac. It could redefine intimacy. What did it matter how expensive it was, if it turned out to be the ultimate erotic experience? He pointed out that no one had ever gone broke selling love potions, and they let him do a feasibility study.

He had to doctor the study; there were a lot of holes that only basic research could fill. But the research was being done, if not at Western Amusement then elsewhere. What he was finally able to sell them was a small ongoing effort. The perfect place to bury Bobby Belotti. A side bet on a long shot.

And now, years later, Belotti had a something which looked very promising. He had borrowed a drug, 7, 2-DAPA, which had been developed by neuropathologists studying language disorders. It could induce an euphoric anomia, disrupting the process of associating certain visual inputs with words. Users had trouble naming what they saw. Nouns, especially abstract nouns and proper names, were especially difficult. The severity of the anomia was related not only to dosage but to the complexity of the visual environment. For example, a user shown a single long-stemmed rose might be unable to speak the words “flower” or “rose” even though he could otherwise carry on intelligent conversation about gardening; show him into a greenhouse and he might well be speechless. However, if he picked the rose up, or smelled it or heard the word “rose” he would make the connection. And in that moment of recognition enkephalin neurons would start pumping like crazy; the brain would be awash in the pleasure of discovery.

“The problem is,” Belotti explained to Cage, “there’s no way yet to predict exactly which words will be lost. Too much individual variation. For instance, maybe I can’t say “rose” but you can. In that case I can get a flash from you; you get nothing. It’s only if both of us lose the same word and then get an appropriate cue that we share the effect.”

“Doesn’t sound as if it’s going to replace sex.” Cage laughed; Belotti winced. The man had not changed.What was left of his hair still needed combing. There were webs of broken veins beneath his wrinkled skin. He seemed very old, very empty. Cage found it hard to remember the time when they had been friends.

“Well, Shared sex might be interesting.” Belotti sounded as if he were repeating excuses he had made before. “But you wouldn’t get much effect by telling someone he’s having an orgasm. Too tactile, very little to do with visual input. Still, since the enkephalin supresses pain impulses, pleasure would be correspondingly enhanced. But remember, this is fairly mild at the dosages we’re looking at. Take too much and there’s a tendency to withdraw. You get into hallucinations. It’s unpredictable—dangerous.”

“Can the effect be blocked?”

“So far the neuroleptics are the only true antagonists we’ve found. And they’re pretty slow-acting.” Belotti shrugged. “Testing isn’t finished yet. Actually I haven’t paid that much attention. They took me off it, you know. I spent ten years chasing the specs you wrote and now I’m running computer simulations—make work.”

Cage had not thought about Bobby Belotti in a long time; suddenly he was sorry for the old man. “What would you use for it, Bobby?”

“As I said, not my decision. Marketing will find someone to peddle it to, I’m sure. I guess they’re a little disappointed that it didn’t turn out to be the aphrodisiac you promised them.”

“It’s fine work, Bobby. You don’t have to apologize to anyone. But I can’t believe that you’ve worked as hard and as long as you have without thinking of commercial applications.”

“Well if you could control which words were lost, then you could use guides to supply the necessary cues.” Belotti scratched the back of his neck. “Maybe you could blend in an hypnotic to give the guides more psychological authority. It might help, say, in art appreciation classes. Or maybe museums could sell it along with those tape recorded tours.”

Wonderful. A flash for museums. Cage could imagine the ads. The topless vidqueen says to her silver boyfriend, Hey, bucko, let’s shank down to the National Gallery and get twisted! No wonder they had taken it away from him. “Why bother? Sounds like all you need are two people sitting at a kitchen table shooting words at each other.”

“But words—it’s not that simple. We’re not talking fancy lights here: we’re talking about internalized symbols which can trigger complex mental states. Emotions, memories…”

“Sure, Bobby. Look, I’ll talk to the front office. See if we can get you a new project, you own team.”

“Don’t bother.” His expression was stony. “They’ve offered me early retirement and I’m going to take it. I’m sixty-one years old, Tony. How old are you these days?”

“I’m sorry, Bobby. I think you’ve done wonders bringing Share this far.” He gave Belotti his deal-closing smile. “Where can I get some samples?”

Belotti nodded, as if he had been expecting Cage to ask. “Still can’t keep your hands off the product? They’re keeping a pretty tight lid on the stuff, you know. Until they decide what they’ve got.”

“I’m a special case, Bobby. You ought to know that by now. Some rules just don’t apply to me.”

Belotti hesitated. He looked as if he were trying to balance some incredibly complex equation.

“Come on, Bobby. For an old friend?”

With a poisonous grin, Belotti thumbed a printreader to unlock his desk, took a green bottle from the top drawer and tossed it to Cage. “One at a time, understand? And you didn’t get it from me.”

Cage popped the top. Six pills: yellow powder in clear casings. For a moment he was suspicious; Belotti seemed awfully eager to break company rules. But Cage had long since made up his mind about the man. He could not bring himself to worry about someone for whom he had so little respect. He tried to imagine what it would have been like to be ordinary like poor Belotti: old, at the end of a failed career, bitter, and tired. What kept a man like that alive? He shivered and pushed the fantasy away as he pocketed the green bottle. “What time is it, anyway?” he said. “I told Shaw I’d meet him for lunch.”

Belotti touched the temple of his eyeglasses and the lenses opaqued. “You know, I really used to hate you. Then I realized it: you didn’t know what the hell you were doing. Might as well blame a cat for batting around a bloody mouse. You don’t see anyone, Tony. I’ll bet you don’t even see yourself.” His hands shook. “That’s all right, I’ll shut up now.” He powered down his terminal. “I’m going home. Only reason I came in was because they said you wanted a meeting.”

Taking no chances, Cage had one of Belloti’s samples analyzed: it was pure. Then, rather than risk any more confrontation, Cage moved on. There were lawyers in Washington and accountants in New York. He spoke at the American Psychopharmacological Association’s annual meeting at Hilton Head in South Carolina and gave half a dozen telelink interviews. He met a Japanese woman and they made reservations to spend a weekend in orbit at Habitat Three. Afterwards they went to Osaka where he found out she was a corporate spy for Unico. It had been almost two months. Time, he thought, for Tod to have screwed up, for Wynne to have recognized that he was born to fail, and for their impossible affair to have collapsed under its own weight. Cage caught the suborbital to Heathrow. He was so sure.

It was a nasty surprise: Tod Schluerman had been lucky.

The video Burn London was only five minutes long. It started with a shot of silos. Countdown. Launch. London was under attack. No missiles—enormous naked Wynnes left rainbows across the sky as they hurtled down on the city. They exploded not in flame but in foliage, smothering entire city blocks with trees and brush. Soon the city disappeared beneath a forest. The camera zoomed to a clearing where a band called Flog was playing. They had been providing the dreamy sound track. The tempo picked up, the group played faster and faster until their instruments caught fire, consuming them and the forest. The final shot was a pan over ash and charred stumps. Cage thought it was dumb.

No one could have predicted that sixteen-year-olds across the UK would choose that moment to take Flog into their callow hearts. When they made Burn London with Tod, Flog was unknown. In the span of a month they went from a basement in Leeds to a floor of Claridges in London. Although Tod did not make much money from Burn London, he had earned a name. The kid who had once compared himself to Nam June Paik was instead making videos for pubescent music fans.

He and Wynne were living at a tube rack in Battersea. She could have afforded better; he insisted that they live within his means. There were about two hundred plastic sleep tubes stacked in what had once been a warehouse. Each was three meters long; the singles were a meter and a half in diameter, the doubles two. Each was furnished with a locker beneath a gel mattress, a telelink terminal, and a water bubbler passing for a sink. There was always a line for the showers. The toilets smelled.

It was all right for Tod; he spent most of his time haunting the video labs or dealing with band managers. He even had a desk at VidStar and a regularly scheduled session on its synthesizer: Four to five A.M., Tuesdays, Thursdays, and Saturdays. But Wynne was only in the way at VidStar. And although they went out almost every night to clubs around London to hear bands play and show Tod’s videos, there seemed to be very little for Wynne to do. Cage could not understand why she seemed so happy.

“Because I’m in love,” she said. “For the first time in my life.”

“I’m glad for you, Wynne. Believe me.” They were sitting over lagers in a pub, waiting for Tod to finish work and join them for dinner. It was dark. It was easier to lie in the dark. “But how long can it last unless you find something to do? Something for yourself.”

“So I can be famous. Like you?” She chuckled as she rubbed her finger along the rim of her glass. “Why should you care about that now, Tony? You were the one who said I should take some time off after I finished sixth form.”

“I’ve thought a lot since you’ve been with Tod. You could get into any school you wanted.”

“You know how Tod feels about school. Still, I have considered taking some business courses. I thought I might be Tod’s manager. That would give him more time to do the important work. He’s really good and he’s still learning, that’s what’s so incredible. Did you get a chance to see Burn London yet?”

Cage nodded.

“Did you recognize the women?”

“Of course.”

She smiled. She was proud of being in Tod’s video. Cage realized his plan of inaction had gone very, very wrong. He would have to intervene in their affair or he might never get Wynne back.

“Good news,” said Tod as he slipped onto the bench beside Wynne. They kissed. “I sold them on the idea. I’ve got a commission to shoot a thirty minute video at the free festival.”

Wynne hugged him. “That’s great, Tod. I knew you could do it.”

“Free festival?” said Cage. “What are you talking about?”

“You know, man.” Tod finished the rest of Wynne’s lager. “You’re always lecturing us about it; that’s when I got the idea. I’m going to do a video of the solstice celebration. At Stonehenge.”
squares

History does not record the first use of drugs at Stonehenge. However, there is little doubt that most of the major hallucinogens available in 1974 were ingested during the first Stonehenge free festival. An offshore pirate music station, Radio Caroline, had urged its listeners to come to Stonehenge for a festival of “love and awareness.” On solstice day that year a horde of scruffy music fans in their late teens and twenties set up camp in the field next to the car park. The music back then was called rock; apparently no pun was intended. The empty landscape around the stones was filled with tents and teepees, cars and caravans. Electric guitars screamed and there was a whiff of marijuana on the summer breeze. There are tapes of those early festivals. A vast psychedelia of humanity would gather for the occasion: the glassy-eyed couple from Des Moines in their matching polyester shirts, the smiling engineer from Tokyo taking movies, the young mother from Luton breast-feeding her infant son on the Altar Stone, the Amesbury bobby standing beneath the outer circle, hands clasped behind his back, the Druid from Leicester in her white ceremonial robes, the longhaired teenager from Dorking who had climbed the great trilithon and was shouting something about Jesus, UFO’s, the sun and the Beatles. The festival has always been one of the great surrounds for getting twisted. The pioneers of hallucinogens had a colorful term for the radical perceptual jolts of such an experience, the fascinating strangeness of it all. They would have called the Stonehenge free festival a mind-blower.
squares

Wynne and Tod had their sleep tube shipped from the rack in Battersea to Stonehenge for the five-day festival. It and a thousand others lay near the old car park across the A360 from the dome which now protected the stones. The tubes looked like giant white Soar capsules scattered in the grass. In between were tension bubbles, gortex tents of varying geometries, hovers and cars and even people sitting in folding chairs beneath gaudy umbrellas. Cage stayed at an inn at Amesbury and watched the festival on telelink.

On solstice eve he was able to coax Tod and Wynne into town with the promise of a free dinner. He proposed his little experiment over dessert.

“I don’t know, man.” Tod looked doubtful. “Tomorrow is the last day, the big one. I don’t know if I ought to be eating experimental drugs.”

Cage had expected that Tod might balk; he was counting on Wynne. “Oh, Tod,” she said, “you’ll be the only one there that won’t be twisted. Why not get into the spirit of the thing?” Her eyes seemed very bright. “Look, how many hours have you shot already? Forty, fifty? They only want a half hour. And even if you miss anything, you can always synthesize it.”

“I know that,” he said irritably. “It’s just that I’m tired. Can hardly think anymore.” He sipped his claret. “Maybe, okay? Just maybe. But start over again. Tell me from the beginning.”

Cage began by claiming that he had been impressed by Burn London; he said he wanted to get to know Tod better, understand his art. Cage spoke of the inspiration he had had while watching the festival on telelink. They would all take Share and go to the solstice celebrations together, relying on Stonehenge, the crowd, and each other for cues to shape their experience. Cage spoke of the aesthetics of randomness as an answer to the problem of selection. He said they might be on the verge of a historic discovery; Share might well be a new way for non-artists to participate in the very act of artistic creation.

Cage did not mention that he had laced Tod’s dose of Share with an anticholinergic which would smash his psychological defenses flat. When Tod was completely vulnerable to suggestion—stripped of the capacity to lie—Cage wold begin interrogating. He would force Tod to tell the truth; force Wynne to see how this shallow boy was using her to further his career. At that moment Wynne, too, would see the ugliness that Cage had seen all along beneath the handsome face. When Tod revealed just how little he cared for her, their affair would be over.

“Come on, Tod,” said Wynne. “We haven’t done chemicals together in a long time. I’m tired of getting twisted alone. And when Tony recommends something like this, you know it has to be a killer flash.”

“You’re sure I’ll be able to function while we’re on this stuff?” Tod’s resistance was wearing down. “I don’t want to waste the day shooting blades of grass.”

“I’ll bring something to neutralize it. If you have problems you can poke yourself straight anytime you want. Don’t worry, Tod. Look, the action of Share should actually help you be more visually oriented. You yourself have said that language gets in the way of art. Share strips away the superstructure of preconceptions. You won’t know what you’re seeing; you’ll just see it. The eyes of a child, Tod. Think of it.”

For a moment Cage wondered if he had overdone it. Wynne’s attention turned; she seemed more interested in what he was saying than in Tod’s reaction to it. He could feel her appraising stare but did not acknowledge it. The waiter came with the check and as Cage signed for it, he dangled the real bait before Tod.

“If you’re afraid to try it, Tod, just say so. It is something new, after all. No one would blame you for backing out.”

“Very good, sir.” A true Englishman, the waiter pretended not to hear as Cage handed him the check. “Thank you, sir.”

“Still,” Cage continued, “I believe in Share and I believe in you. So much so that when you’re done I’d like to show your video to Western Amusement. They haven’t decided yet how to market Share. If this video is as good as I think it can be, the problem will be solved. I’ll make them buy it. You’ll be the spokesman—hell, the father—of a new collaborative art form.”

He knew he had Tod then. This was what the kid had wanted all along. Cage had seen right away that Tod had only seduced Wynne as a career move. All right then, let Tod have his introduction to an entertainment multinational—and on his own terms. Let him believe that he had manipulated Cage. It did not matter as long as Cage got Wynne back.

“What are you doing, Tony?” Wynne said. Beneath her skin tint, she had gone pale. She must havesuspected the stakes Cage was playing for.

“What am I doing?” Cage stood, laughing. “I’m not really sure. That’s what makes it interesting, isn’t it?”

“Okay, man.” Tod stood, too. “I’ll try it.”

“Tony.” Wynne stared up at them.
squares

“What’s that?” said Wynne, pointing at Stonehenge. Bolts of lightning forked through the darkness, illuminating the crowd which stood outside the dome.

“It’s only the son et lumiere,” said Cage. “The holo techs from the Department of Environment put it on to soak a few extra quid from the tourists.” They kept walking up the A360 from where the Amesbury shuttle had dropped them. “Watch what comes next.”

Seconds later two laser rainbows shimmered between the stones. “Stonehenge’s greatest hits,” said Tod with contempt. “Both Constable and Turner did major paintings here. Turner’s was full of his usual bombast, lightning bolts and dead shepherds and howling dogs. Constable tried to jack up his boring watercolor with a double rainbow.”

Cage bit his lip and said nothing. He did not really need a lecture on Stonehenge, especially not from Tod. After all, he owned one of Constable’s Stonehenge sketches.

Tod flipped down the visor of his VidStar helmet; he looked like a mantis with lens eyes. Cage could hear tiny motors buzzing as the twin cameras focused. “Is anyone else starting to feel it?” said Wynne.

“I’ve been doing a lot of research on this place, you know,” Tod continued. “It’s amazing, the people who’ve been here.”

“Yes,” Cage said. “It’s an oozy kind of coolness spreading across the back of my skull—like mud.” Theyhad eaten the capsules of Share in the darkness on the ride over. “What time is it?”

“It’s 4:18.” Tod slipped a fresh disk into the drive clipped to his belt. “Sunrise at 5:07.”

Cage looked to the northeast; the sky had already started to lighten. The stars were like glass mites scuttling away into the grayness.

“They come in waves,” said Wynne. “Hallucinations.”

“Yes,” Cage said. The backs of his eyes seemed to tingle. He knew there was something wrong but he could not think what it was.

They pushed past the inevitable Drug Temperance League picket line; luckily, none of them recognized Cage. At last they reached a barbed-wire corridor leading through the crowd to the entrance of the dome. Down the corridor marched a troop of ghosts. They were dressed in white robes; some wore glasses. They carried copper globes and oak branches and banners with images of snakes and pentacles. They were male and female, and they seemed old. They were murmuring a chant that sounded like wind blowing through fallen leaves. Dry old ghosts, crinkly and intent, turned inward as if they were working out chess problems in their heads.

“The Druids,” said Tod. The words broke the trance and a shiver danced across Cage’s shoulder. He glanced at Wynne and could tell instantly that she had felt the same. A smile of recognition lit her face in the predawn gloom.

“Are you all right?” said Tod.

Wynne laughed. “No.”

Tod frowned and linked his arm through hers. “Let’s go. We have to walk around the dome if we want to see the sun rise over the Heel Stone.”

They began to thread their way through the crowd to the southwest side of the dome. The space between the shells was empty now and Cage could see that the procession of Druids had surrounded the outer sarsen circle. All turned to the northeast to face the Heel Stone and the approaching sunrise.

“This is it,” said Tod. “We’re right on the axis.”

The fat woman standing next to Cage was glowing. Except for knee-high studded leather leggings, she was naked. Her skin gave off a soft green light: her nipples and all of her body hair were bright orange. When she moved the rolls of fat gleamed like moonlit waves. At first he thought she was another hallucination. Something wrong.

“Do you see her too?” Wynne whispered.

“She’s a glowworm.” Tod made no effort to keep his voice down, and the green woman stared at them.

Wynne nodded as if she had understood. Cage cupped his hand to her ear. “What’s a glowworm?”

“She’s had a luminescent body tint,” came the whispered reply.

Tod laughed as he pointed his lenses at her. “Do you know how carcinogenic that stuff is? Eighty percent mortality after five years.”

She waddled over to him. “It’s my body, Flash. Ain’t it?” Cage was surprised when she slipped a hand around Tod’s waist. “Would that be a video you’re making, Flash? Me in it?”

“Sure,” he said. “Everyone gets to be famous for ten minutes. You know the camera loves you, glowworm. That’s why you got tinted.”

She giggled. “You with someone, Flash?”

“Not now, glowworm. The sun is coming.”

Amateur photographers and professional cameramen began to jostle for position around them. Tod, using his elbows with cunning, would not be moved. The sun’s bright lip appeared over the trees to the northeast. Inside the dome Druids raised horns and blew a tribute to the new day. Outside there was inarticulate shouts and polite applause. A man with a long beard rolled on the ground, barking.

“But there’s no alignment,” some fool was complaining. “The sun’s in the wrong place.”

The sun had cleared the trees and crawled across the brick-colored horizon. Cage shut his eyes and still he could see it: blood red, flashing blue, veins pulsing across its surface.

“Sun’s not wrong,” said a man with a camera where his head should have been. “Stonehenge doesn’t really line up. Never did. It’s a myth, man.”

Although he did not immediately recognize the man, Cage knew he hated that mocking voice. When he opened his eyes again the sun had already climbed several of its diameters into the sky. After a few moments it passed over the Heel Stone at the opposite end of Stonehenge. And seemed to hang there, propped in the sky by a single untrimmed sarsen, five meters tall. His view was framed by the uprights and lintels of the outer circle. It was as if he were standing on the backbone of the world. He was spellbound: men in skins had built a structure that could capture a star. The crowd was silent, or perhaps Cage had ceased to perceive anything but his vision of sunfire and stone. Then the moment passed. The sun continued to climb.

“Looks like a doorway,” said the glowworm. “Into another world.” She seemed pale in the light of dawn.

Doorway. The word filled his mind. Doorway raised upon doorway. Someone said, “I make it about four degrees off.” Cage saw people crouching to help the barking man.

“Tony?” A strange and beautiful woman had taken his hand. Her voice echoed and distorted: a baby’s inexact chatter, the joyful cry of a child. He blinked at her in the soft light. Blue-skinned, hair in spikes, she was dressed in silver: the setting for a sapphire. Her face, a jewel. Precious. Cage was falling in love.

“Who are you?” He could not remember.

“They come in waves,” she said. He did not understand.

“He’s so far out he’s breathing space,” said the camera head with the mocking voice.

“Who are you?” Cage held up her hand, clasped in his.

“It’s me, Tony.” The beautiful woman was laughing. Cage wanted to laugh too. “Wynne.”

Wynne. He said the word over and over to himself, shuddering with pleasure at each repetition. Wynne. His Wynne.

“And I’m Tod, remember?” The camera head looked disgusted. “Christ, am I glad I palmed that stuff. Look at you two. She can’t stop laughing and you’re catatonic. How was I supposed to work? Do you realize how twisted you are?”

Tod. Cage battered through yet another wave of hallucinations, trying to remember. A plan … force Tod …make Wynne see … Cage had known it all along. But it was no good if Tod were straight. “You didn’t take…?”

“Hell no!” Tod turned. Cage felt the lens eyes probing him, recording, judging. “I’m not as gullible as you think, man. I decided to fake it, see how the stuff affected you first. If it looked like fun I knew I could always catch up.”

There was a tiny red light flashing in the middle of Tod’s helmet. “Turn it off, you bastard,” Cage said. “Not me into your damn … your god damn…”

“No?” Cage could see a smile beneath the visor. “You’re a public figure, man. We all own a piece of you.”

“Tod,” said Wynne. “Don’t goad him.”

The red light went out. He flipped the visor up and held out his hand to her. She let go of Cage and went to him. “Let’s take a walk, Wynne. I want to talk to you.”

As he watched them walk away together Cage felt as though he were turning to stone. He had lost her. The crowd swirled around them and they were gone.

“Aren’t you Tony Cage?”

He stared without comprehension at a middle-aged woman wearing a mood dress. It changed from blue to silvery-green as she called to her husband. “Marv, come quick.” A paunchy man in isothermals responded to her summons. “You are Tony Cage, aren’t you?”

Cage could not speak. The man shook his nerveless hand. “Sure, we’ve seen you on telelink. Lots of times. We’re from the States. New Hampshire. We’ve tried all your drugs.”

“But Soar’s still our favorite. I’m Sylvie. We’re retired.” The dress lightened from lime to apple green. Cage could not look her in the face.

“I’m Marv. Say, you look pretty twisted. What are you on, anyway? Something new?”

Heads were turning. “Sorry.” His tongue was stone. “Not feeling well. Have to…” By then he was stumbling away from his manic fans. Luckily they did not follow.

He did not remember how long he wandered through the crowd or how he felt or what exactly he was looking for. A terrible suspicion nagged at him … maybe something was wrong with the dose? Eventually the Druids finished their service and the dome was opened to the public. He drifted on a floodtide of humanity and at last washed up on the Slaughter Stone.

The Slaughter Stone was a slab of lichen-covered sarsen about thirty meters away from the outer circle: a good place to sit and watch, away from the hurly-burly around the standing stones. The surface of the stone was pitted and rough. It once was thought that these natural bowls were used to catch sacrifical blood—both human and animal. Another myth, since the stone originally stood upright. Now they were two fallen things, Cage and the stone, their foundations undermined, purposes lost. They existed in roughly the same state of consciousness. Cage thought sandstone thoughts; his understanding was that of rock.

The sun climbed. Cage was hot. The combination of body heat and solar gain had overloaded the dome’s air conditioning. He did nothing. The waves of hallucinations seemed to have receded. People had climbed the outer circle and walked along the lintels. One woman started to strip. The crowd clapped and urged her on. “Vestal virgin, vestal virgin,” they cried. A little boy nearby watched avidly as he squeezed cider from a disposable juice bulb. Cage was thirsty; he did nothing. The boy dropped the bulb on the ground when he had finished and wandered off. A bobby stepped out from beneath the circle to watch as the stripper removed her panties. The crowd roared and she gave them an extra treat. She was an amputee; she unstrapped her prosthetic forearm and waved it over her head. The world was going mad and trying to take Cage with it. He loaded a neuroleptic into his pressure syringe and poked it into his forearm.

“Tony.”

There was no Tony. There was only stone.

“Hey, man.” A stranger shook him. “It’s me. Tod. There’s something wrong with Wynne! We need to know what you took.

“In waves.” Cage started to laugh. “They come in waves.” Now he knew. Hallucinations. But not with Share. He was laughing so hard he fell backwards onto the stone. “Belotti!” Poor Bobby had finally struck—after all these years. The drug was pure but the dose … Too high. Hallucinogen. Dangerous, he had said. Unpredictable. That unpredictable old … “Bastard!” Cage was gasping for air.

“He needs oxygen. Quick.”

“Look at his eyes!”

When the last wave hit him, Cage held on to the stone. The crowd disappeared. The dome vanished. The car park, the A360, all signs of civilization—gone. Then the stones awoke and began to dance. Those that had fallen righted themselves. A road erupted from the grass. The Slaughter Stone bucked and threw him as it stood. A twin appeared beside it: a gate. He wanted to pass through, walk down the road, see Stonehenge whole. But the magic held him back. In an overly explained world, only the subtlest and most powerful magic of all had survived. The magic that works exclusively in the mind. A curse. A dead and illiterate race had placed a curse upon the imagination of the world. In its rude magnificence Stonehenge challenged all to understand its meaning, yet its secret was forever locked behind impenetrable walls of time.

“Lay him down here.”

“Tony!”

“He can’t hear you.”

Suddenly they were all around him, all of those who had stood where Cage now stood. The politicians and writers and painters and historians and scientists and the tourists—yes, even the tourists who, in search of an hour’s diversion, had found instead a timeless mystery. All of those who had accepted Stonehenge’s challenge, and fallen under the curse. They had striven with words and images to find the secret, yet all they had seen was themselves. The sun grew very bright then, and the sides of the stones turned silver. Cage could see all the ghosts reflected in the bright stones. He could see himself.

“Tony, can you hear me? Wynne’s having some kind of fit. You have to tell us.”

Cage saw himself in the Slaughter Stone. What did it matter? He had already lost her. His image seemed to shimmer. He looked like a ghost; the thought of death did not displease him. To be as a stone.

“Wake up, man. You have to save her. She’s your daughter, damn you!”

“No.” At that moment Cage’s reflection in the stone shifted and he saw his mirror image. Wynne. In pain. He realized that she had been in pain for a long time, had hidden it behind a veneer of chemicals and feigned toughness. He should have known. Trapped within the magical logic of the hallucination, now he could actuallyfeel her pain and was racked by the certain knowledge that he was its source. It was no longer the drug, it was Stonehenge itself that forced him to suffer with her, Stonehenge that created a magic landscape where the veil of words was parted and mind could touch mind directly. Or so it seemed to Cage. A Sound tore through the vision: a scream. “No!” Stones fell, disappeared, but Cage could not escape the pain. All the lies Cage had told himself fell away. In a moment of terrible grace, he realized what he had done. To his daughter.

Tod had lost his helmet, probably lying on the turf somewhere, shooting closeups of blades of grass. He seemd very pale beneath his blue skin tint. Cage blinked, trying to remember what it was that he had asked. There were electrodes taped to Cage’s head and wrist. A medic was checking readouts.

“What did you give her?” said the medic.

Cage’s hands trembled as he fumbled the pressure syringe from his pocket. “This … a poke … neuroleptic. She needs it now. Now!”

The medic seemed very young; he looked doubtful. Cage sat up, tore the electrode from his temple. “Do you know who I am?” The world was spinning. “Do it!”

The medic looked briefly at Tod, then took the syringe and ran back toward the standing stones. Tod hesitated, staring at Cage.

“What did you say to her?” Cage tried to stand up.

He put his arm around Cage’s shoulders to steady him. “Are you all right?”

“Did you say it to her? That she was my daughter?”

“That’s what she thinks. We were arguing about it.”

“She was my lover. You know that, I guess. She came to me one night. Three years ago. We were both twisted. I couldn’t … I couldn’t send her away.”

Tod looked straight ahead. “She said that. She said it was her fault. Then the fit hit her.”

“No.” Cage could still see himself; he would never be able to stop seeing himself again. “I was lonely so I made sure that she was lonely too. And called it love.” The word almost shocked him. “Where is she? Take me to her.” They started to walk. “Do you love her, Tod?”

“I don’t know, man.” He considered for some time. “Feels something like.”

She was unconscious but the fit had passed and the medic said her signs were good. Cage went with Tod to the hospital. They waited all day; they talked about everything but what was most on their minds. Cage realized that he had made a mistake about Tod. So many mistakes. When Wynne at last regained consciousness, Tod went to see her. Alone.

“I’m not here,” Cage said. “Tell her I’ve gone away.”

“I can’t do that.”

“Tell her!”

They only gave Tod ten minutes. Cage kept worrying that Tod would call him in.

“Is she all right?”

“Seems to be. She asked about you; I told her you went back to your room to sleep it off. I told her you’d be in tomorrow. They’re going to keep her overnight.”

“I’m leaving, Tod.” Cage offered his hand. “You won’t be seeing me again.”

“What? You can’t do that to her, man. She saw something this morning, something that makes her feelguilty as hell. If you just disappear she’s going to feel worse. Do you understand? You owe it to her to stay.”

Cage let his hand fall to his side. “You want me to be some kind of a hero, Tod. Problem is, I’m a coward—always have been. I saw something today too, and I’ll spend the rest of my life trying to forget it. She’ll … you’llboth be better off without me.”

Tod grabbed him by the shoulders. “You’re damn well going to see her tomorrow. Listen to me, man! If you love her at all…”

“I love her.” Cage shook free. “Like I love myself.”

That night he caught the shuttle from Heathrow to Shannon. He knew Tod was right; it was cruel and selfish to run away. Tod was entitled to think what he wanted. He would never know how much it hurt Cage to give Wynne up this way … If Cage was escaping, it was into pain. He hoped Wynne would understand. Eventually. His beautiful Wynne. It took a few days to put his affairs in order. He assigned a fortune in Western Amusement stock to her. He made a tape for her, said goodbye.
squares

There is a mist clinging to the land. The slaty grayness of Galway Bay reminds Cage of sarsen. The cyogenic box awaits, set for a hundred years. He does not know whether this is enough to save her. Or himself. He knows he will probably never see her again. But for a time, at least, he will be at peace. He will sleep the inscrutable sleep of stones.