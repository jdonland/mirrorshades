\chapter{Preface}

This book showcases writers who have come to prominence within this decade. Their allegiance to Eighties culture has marked them as a group---as a new movement in science fiction.

This movement was quickly recognized and given many labels: Radical Hard SF, the Outlaw Technologists, the Eighties Wave, the Neuromantics, the Mirrorshades Group.

But of all the labels pasted on and peeled throughout the early Eighties, one has stuck: cyberpunk.

Scarcely any writer is happy about labels---especially one with the peculiar ring of ``cyberpunk.'' Literary tags carry an odd kind of double obnoxiousness: those with a label feel pigeonholed; those without feel neglected. And, somehow, group labels never quite fit the individual, giving rise to an abiding itchiness. It follows, then, that the ``typical cyberpunk writer'' does not exist; this person is only a Platonic fiction. For the rest of us, our label is an uneasy bed of Procrustes, where fiendish critics wait to lop and stretch us to fit.

Yet it's possible to make broad statements about cyberpunk and to establish its identifying traits. I'll be doing this too in a moment, for the temptation is far too strong to resist. Critics, myself included, persist in label-mongering, despite all warnings; we must, because it's a valid source of insight---as well as great fun.

Within this book, I hope to present a full overview of the cyberpunk movement, including its early rumblings and the current state of the art. Mirrorshades should give readers new to Movement writing a broad introduction to cyberpunk's tenets themes, and topics. To my mind, these are showcase stories. strong, characteristic examples of each writer's work to date. I've avoided stories widely anthologized elsewhere, so even hardened devotees should find new visions here.

Cyberpunk is a product of the Eighties milieu---in some sense, as I hope to show later, a definitive product. But its roots are deeply sunk in the sixty-year tradition of modem popular SF.

The cyberpunks as a group are steeped in the lore and tradition of the SF field. Their precursors are legion. Individual cyberpunk writers differ in their literary debts; but some older writers, ancestral cyberpunks perhaps, show a clear and striking influence.

From the New Wave: the streetwise edginess of Harlan Ellison. The visionary shimmer of Samuel Delany. The free-wheeling zaniness of Norman Spinrad and the rock esthetic of Michael Moorcock; the intellectual daring of Brian Aldiss; and, always, J. G. Ballard.

From the harder tradition: the cosmic outlook of Olaf Stapledon; the science/politics of H. G. Wells; the steely extrapolation of Larry Niven, Poul Anderson, and Robert Heinlein.

And the cyberpunks treasure a special fondness for SF's native visionaries: the bubbling inventiveness of Philip Jose Farmer; the brio of John Varley, the reality games of Philip K. Dick; the soaring, skipping beatnik tech of Alfred Bester. With a special admiration for a writer whose integration of technology and literature stands unsurpassed: Thomas Pynchon.

Throughout the Sixties and Seventies, the impact of SF's last designated ``movement,'' the New Wave, brought a new concern for literary craftsmanship to SF. Many of the cyberpunks write a quite accomplished and graceful prose; they are in love with style, and are (some say) fashion-conscious to a fault. But, like the punks of '77, they prize their garage-band esthetic. They love to grapple with the raw core of SF: its ideas. This links them strongly to the classic SF tradition. Some critics opine that cyberpunk is disentangling SF from mainstream influence, much as punk stripped rock and roll of the symphonic elegances of Seventies ``progressive rock.'' (And others---hard-line SF traditionalists with a firm distrust of ``artiness''---loudly disagree.) Like punk music, cyberpunk is in some sense a return to roots. The cyberpunks are perhaps the first SF generation to grow up not only within the literary tradition of science fiction but in a truly science-fictional world. For them, the techniques of classical ``hard SF''---extrapolation, technological literacy---are not just literary tools but an aid to daily life. They are a means of understanding, and highly valued.

In pop culture, practice comes first; theory follows limping in its tracks. Before the era of labels, cyberpunk was simply ``the Movement''---a loose generational nexus of ambitious young writers, who swapped letters, manuscripts, ideas, glowing praise, and blistering criticism. These writers---Gibson, Rucker Shiner, Shirley, Sterling---found a friendly unity in their common outlook, common themes, even in certain oddly common symbols, which seemed to crop up in their work with a life of their own. Mirrorshades, for instance.

Mirrored sunglasses have been a Movement totem since the early days of '82. The reasons for this are not hard to grasp. By hiding the eyes, mirrorshades prevent the forces of normalcy from realizing that one is crazed and possibly dangerous. They are the symbol of the sunstaring visionary, the biker, the rocker, the policeman, and similar outlaws. Mirrorshades---preferably in chrome and matte black, the Movement's totem colors---appeared in story after story, as a kind of literary badge.

These proto-cyberpunks were briefly dubbed the Mirrorshades Group. Thus, this anthology's title, a well-deserved homage to a Movement icon. But other young writers, of equal talent and ambition, were soon producing work that linked them unmistakably to the new SF. They were independent explorers, whose work reflected something inherent in the decade, in the spirit of the times. Something loose in the 1980s.

Thus, ``cyberpunk''---a label none of them chose. But the term now seems a fait accompli, and there is a certain justice in it. The term captures something crucial to the work of these writers, something crucial to the decade as a whole: a new kind of integration. The overlapping of worlds that were formerly separate: the realm of high tech, and the modern pop underground.

This integration has become our decade's crucial source of cultural energy. The work of the cyberpunks is paralleled throughout the Eighties pop culture: in rock video; in the hacker underground; in the jarring street tech of hip-hop and scratch music; in the synthesizer rock of London and Tokyo. This phenomenon, this dynamic, has a global range; cyberpunk is its literary incarnation.

In another era this combination might have seemed far-fetched and artificial. Traditionally there has been a yawning cultural gulf between the sciences and the humanities: a gulf between literary culture, the formal world of art and politics. and the culture of science, the world of engineering and industry.
But the gap is crumbling in unexpected fashion. Technical culture has gotten out of hand. The advances of the sciences are so deeply radical, so disturbing, upsetting, and revolutionary, that they can no longer be contained. They are surging into culture at large; they are invasive; they are everywhere. The traditional power structure, the traditional institutions, have lost control of the pace of change.

And suddenly a new alliance is becoming evident: an integration of technology and the Eighties counterculture. An un-holy alliance of the technical world and the world of organized dissent---the underground world of pop culture, visionary fluidity, and street-level anarchy.

The counterculture of the 1960s was rural, romanticized, anti-science, anti-tech. But there was always a lurking contradiction at its heart, symbolized by the electric guitar. Rock technology was the thin edge of the wedge. As the years have passed, rock tech has grown ever more accomplished, expanding into high-tech recording, satellite video, and computer graphics. Slowly it is turning rebel pop culture inside out, until the artists at pop's cutting edge are now, quite often, cutting-edge technicians in the bargain. They are special effects wizards, mixmasters, tape-effects techs, graphics hackers, emerging through new media to dazzle society with head-trip extravaganzas like FX cinema and the global Live Aid benefit. The contradiction has become an integration.

And now that technology has reached a fever pitch, its influence has slipped control and reached street level. As Alvin Toffler pointed out in The Third Wave---a bible to many cyberpunks---the technical revolution reshaping our society is based not in hierarchy but in decentralization, not in rigidity but in fluidity.

The hacker and the rocker are this decade's pop-culture idols, and cyberpunk is very much a pop phenomenon: spontaneous, energetic, close to its roots. Cyberpunk comes from the realm where the computer hacker and the rocker overlap, a cultural Petri dish where writhing gene lines splice. Some find the results bizarre, even monstrous; for others this integration is a powerful source of hope.

Science fiction---at least according to its official dogma---has always been about the impact of technology. But times have changed since the comfortable era of Hugo Gernsback, when Science was safely enshrined---and confined---in an ivory tower. The careless technophilia of those days belongs to a vanished, sluggish era, when authority still had a comfortable margin of control.

For the cyberpunks, by stark contrast, technology is visceral. It is not the bottled genie of remote Big Science boffins; it is pervasive, utterly intimate. Not outside us, but next to us. Under our skin; often, inside our minds.

Technology itself has changed. Not for us the giant steam-snorting wonders of the past: the Hoover Dam, the Empire State Building, the nuclear power plant. Eighties tech sticks to the skin, responds to the touch: the personal computer, the Sony Walkman, the portable telephone, the soft contact lens.

Certain central themes spring up repeatedly in cyberpunk. The theme of body invasion: prosthetic limbs, implanted circuitry, cosmetic surgery, genetic alteration. The even more powerful theme of mind invasion: brain-computer interfaces, artificial intelligence, neurochemistry---techniques radically redefining - the nature of humanity, the nature of the self.

As Norman Spinrad pointed out in his essay on cyberpunk, many drugs, like rock and roll, are definitive high-tech products. No counterculture Earth Mother gave us lysergic acid - it came from a Sandoz lab, and when it escaped it ran through society like wildfire. It is not for nothing that Timothy Leary proclaimed personal computers ``the LSD of the 1980s''---these are both technologies of frighteningly radical potential. And, as such, they are constant points of reference for cyberpunk.

The cyberpunks, being hybrids themselves, are fascinated by interzones: the areas where, in the words of William Gibson, ``the street finds its own uses for things.'' Roiling, irrepressible street graffiti from that classic industrial artifact, the spray can. e subversive potential of the home printer and the photocopier. Scratch music, whose ghetto innovators turn the phonograph itself into an instrument, producing an archetypal Eighties music where funk meets the Burroughs cut-up method. ``It's all in the mix''---this is true of much Eighties art and is as applicable to cyberpunk as it is to punk mix-and-match retro fashion and multitrack digital recording.

The Eighties are an era of reassessment, of integration, of hybridized influences, of old notions shaken loose and reinterpreted with a new sophistication, a broader perspective. The cyberpunks aim for a wide-ranging, global point of view.

William Gibson's Neuromancer, surely the quintessential cyberpunk novel, is set in Tokyo, Istanbul, Paris. Lewis Shiner's Frontera features scenes in Russia and Mexico---as well as the surface of Mars. John Shirley's Eclipse describes Western Europe in turmoil. Greg Bear's Blood Music is global, even cosmic in scope.

The tools of global integration---the satellite media net, the multinational corporation---fascinate the cyberpunks and figure constantly in their work. Cyberpunk has little patience with borders. Tokyo's Hayakawa's SF Magazine was the first publication ever to produce an ``all-cyberpunk'' issue, in November 1986. Britain's innovative SF magazine Interzone has also been a hotbed of cyberpunk activity, publishing Shirley, Gibson, and Sterling as well as a series of groundbreaking editorials, inter-views, and manifestos. Global awareness is more than an article of faith with cyberpunks; it is a deliberate pursuit.

Cyberpunk work is marked by its visionary intensity. Its writers prize the bizarre, the surreal, the formerly unthinkable. They are willing---eager, even---to take an idea and unflinchingly push it past the limits. Like J. G. Ballard---an idolized role model to many cyberpunks---they often use an unblinking, almost clinical objectivity. It is a coldly objective analysis, a technique borrowed from science, then put to literary use for classically punk shock value.

With this intensity of vision comes strong imaginative concentration. Cyberpunk is widely known for its telling use of detail, its carefully constructed intricacy, its willingness to carry extrapolation into the fabric of daily life. It favors ``crammed`` loose: rapid, dizzying bursts of novel information, sensory overIoad that submerges the reader in the literary equivalent of the hard-rock ``wall of sound.``

Cyberpunk is a natural extension of elements already present in science fiction, elements sometimes buried but always seething with potential. Cyberpunk has risen from within the SF genre; it is not an invasion but a modern reform. Because of this, its effect within the genre has been rapid and powerful.

Its future is an open question. Like the artists of punk and New Wave, the cyberpunk writers, as they develop, may soon be galloping in a dozen directions at once.

It seems unlikely that any label will hold them for long. Science fiction today is in a rare state of ferment. The rest of the decade may well see a general plague of movements, led by an increasingly volatile and numerous Eighties generation. The eleven authors here are only a part of this broad wave of writers, and the group as a whole already shows signs of remarkable militancy and fractiousness. Fired by a new sense of SF's potential, writers are debating, rethinking, teaching old dogmas new tricks. Meanwhile, cyberpunk's ripples continue to spread, exciting some, challenging others---and outraging a few, whose pained remonstrances are not yet fully heard.
The future remains unwritten, though not from lack of trying.

And this is a final oddity of our generation in SF---that, for us, the literature of the future has a long and honored past. As writers, we owe a debt to those before us, those SF writers whose conviction, commitment, and talent enthralled us and, in all truth, changed our lives. Such debts are never repaid, only acknowledged and---so we hope - passed on as a legacy to those who follow in turn.

Other acknowledgments are due. The Movement owes much to the patient work of today's editors. A brief look at this book's copyright page shows the central role of Ellen Datlow at Omni, a shades-packing sister in the vanguard of the ideologically correct, whose help in this anthology has been invaluable. Gardner Dozois was among the first to bring critical attention to the nascent Movement. Along with Shawna McCarthy, he has made Isaac Asimov's Science Fiction Magazine a center of energy and controversy in the field. Edward Ferman's Fantasy and Science Fiction is always a source of high standards. Interzone, the most radical periodical in science fiction today, has already been mentioned; its editorial cadre deserves a second thanks. And a special thanks to Yoshio Kobayashi, our Tokyo liaison, translator of Schismatrix and Blood Music, for favors too numerous to mention.

Now, on with the show.

--- Bruce Sterling
