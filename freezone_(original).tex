\chapter{Extra: Freezone (Original Version)}
\chapterauthor{John Shirley}

\firstletter{F}reezone floated in the Atlantic Ocean, a city afloat in the wash of international cultural confluence.

The city was anchored about a hundred miles north of Sidi Ifni, a drowsy city on the coast of Morocco in a warm, gentle current, and in a sector of the sea only rarely troubled by large storms. What storms arose here spent their fury on the maze of concrete wave-baffles Freezone Admin had spent years building up around the artificial island.

But the affluent could feel the crumbling of their kingdom. They didn’t feel safe in the States. They needed someplace outside, somewhere controlled. Europe was out now; Central and South America, too risky. The Pacific theater was another war zone.

So that’s where Freezone came in.

The community was now seventeen square miles of urban raft protected with one of the meanest security forces in the world. Freezone dealt in pleasant distractions for the rich in the exclusive section and—in the second-string places around the edge—for technickis from the drill rigs. And the second-string places sheltered a few thousand semi-illicit hangers-on, and a few hundred performers.

Like Rickenharp.
squares

Rick Rickenharp stood against the south wall of the Semiconductor, letting the club’s glare and blare wash over him, and mentally writing a song. The song went something like, “Glaring blare, lightning stare/Nostalgia for the electric chair.”

Then he thought, Fucking drivel.

All the while he was doing his best to look cool but vulnerable, hoping one of the girls flashing through the crowd would remember having seen him in the band the night before, would try to chat him up, play groupie. But they were mostly into wifi dancers.

And no fucking way Rickenharp was going to wire into minimono.

Rickenharp was a rock classicist; he was retro. He wore a black leather motorcycle jacket that was some seventy-some years old, said to have been worn by John Cale when he was still in the Velvet Underground. The seams were beginning to pop for the third time; three studs were missing from the chrome trimming. The elbows and collar edges were worn through the black dye to the brown animal the leather had come from. But the leather was second skin to Rickenharp. He wore nothing under it. His bony, hairless chest showed translucent-bluewhite between the broken zippers. He wore blue jeans that were only ten years old but looked older than the coat; he wore genuine Harley Davidson boots. Earrings clustered up and down his long, slightly too prominent ears, and his rusty brown hair looked like a cannon-shell explosion.

And he wore dark glasses.

And he did all this because it was gratingly unfashionable.

His band hassled him about it. They wanted their lead-git and frontman minimono.

“If we’re gonna go minimono, we oughta just sell the fucking guitars and go wires,” Rickenharp had told them.

And the drummer had been stupid and tactless enough to say, “Well, fuck, man, maybe we should go to wires.”

Rickenharp had said, “Maybe we should get a fucking drum machine, too, you fucking Neanderthal!” and kicked the drum seat over, sending Murch into the cymbals with a fine crashing, so that Rickenharp added, “you should get that good a sound outta those cymbals on stage. Now we know how to do it.”

Murch had started to throw his sticks at him, but then he’d remembered how you had to have them lathed up special because they didn’t make them anymore, so he’d said, “Suck my ass, big shot,” and got up and walked out, not the first time. But that was the first time it meant anything, and only some heavy ambassadorial action on the part of Ponce had kept Murch from leaving the band.

The call from their agent had set the whole thing off. That’s what it really was. Agency was streamlining its clientele. The band was out. The last two albums hadn’t sold, and in fact the engineers claimed that live drums didn’t digitize well onto the miniaturized soundcaps that passed for CDs now. Rickenharp’s holovid and the videos weren’t getting much airplay.

Anyway, Vid-Co was probably going out of business. Another business sucked into the black hole of the depression. “So it ain’t our fault the stuff’s not selling,” Rickenharp said. “We got fans but we can’t get the distribution to reach ’em.”

Mose had said, “Bullshit, we’re out of the Grid, and you know it. All that was carrying us was the nostalgia wave anyway. You can’t get more’n two bits out of a revival, man.”

Julio the bassist had said something in technicki which Rickenharp hadn’t bothered to translate because it was probably stupid and when Rickenharp had ignored him he’d gotten pissed and it was his turn to walk out. Fucking touchy technickis anyway.

And now the band was in abeyance. Their train was stopped between the stations. They had one gig, just one: opening for a wifi act. And Rickenharp didn’t want to do it. But they had a contract and there were a lot of rock nostalgia freaks on Freezone, so maybe that was their audience anyway and he owed it to them. Blow the goddamn wires off the stage.

He looked around the Semiconductor and wished the Retro-Club was still open. There’d been a strong retro presence at the RC, even some rockabillies, and some of the rockabillies actually knew what rockabilly sounded like. The Semiconductor was a minimono scene.

The minimono crowd wore their hair long, fanned out between the shoulders and narrowing to a point at the crown of the head, and straight, absolutely straight, stiff, so from the back each head had a black or gray or red or white teepee-shape. Those, in monochrome, were the only acceptable colors. Flat tones and no streaks. Their clothes were stylistic extensions of their hairstyles. Minimono was a reaction to Flare—and to the chaos of the war, and the war economy, and the amorphous shifting of the Grid. The Flare style was going, dying.

Rickenharp had always been contemptuous of the trendy Flares, but he preferred them to minimono. Flare had energy, anyway.

A flare was expected to wear his hair up, as far over the top of his head as possible, and that promontory was supposed to express. The more colors the better. In that scene, you weren’t an individual unless you had an expressive flare. Screwshapes, hooks, aureola shapes, layered multicolor snarls. Fortunes were made in flare hair-shaping shops, and lost when it began to go out of fashion. But it had lasted longer than most fashions; it had endless variation and the appeal of its energy to sustain it. A lot of people copped out of the necessity of inventing individual expression by adopting a politically standard flare. Shape your hair like the insignia for your favorite downtrodden third world country (back when they were downtrodden, before the new marketing axis). Flares were so much trouble most people took to having flare wigs. And their drugs were styled to fit the fashion. Excitative neurotransmitters; drugs that made you seem to glow. The wealthier flares had nimbus belts, creating artificial auroras. The hipper flares considered this to be tastelessly narcissistic, which was a joke to nonflares, since all flares were floridly vain.

Rickenharp had never colored or shaped his hair, except to encourage its punk spikiness.

But Rickenharp wasn’t a punkrocker. He identified with prepunk, late 1950s, mid-1960s, early 1970s. Rickenharp was a proud anachronism. He was simply a hard-core rocker, as out of place in the Semiconductor as bebop would have been in the 1980s dance clubs.

Rickenharp looked around at the flat-back, flat-gray, monochrome tunics and jumpsuits, the black wristfones, the cookie-cutter sameness of JAS’s; at the uniform tans and ubiquitous FirStep Colony-shaped earrings (only one, always in the left ear). The high-tech-fetishist minimonos were said to aspire toward a place in the Colony the way Rastas had dreamed of a return to Ethiopia. Rickenharp thought it was funny that the Russians had blockaded the Colony. Funny to see the normally dronelike, antiflamboyant minimonos quietly simmering on ampheticool, standing in tense groups, hissing about the Russian blockade of FirStep, in why-doesn’t-someone-do-something outrage.

The stultifying regularity of their canned music banged from the walls and pulsed from the floor. Lean against the wall and you felt a drill-bit vibration of it in your spine.

There were a few hardy, defiant flares here, and flares were Rickenharp’s best hope for getting laid. They tended to respect old rock.

The music ceased; a voice boomed, “Joel NewHope!” and spots hit the stage. The first wifi act had come on. Rickenharp glanced at his watch. It was ten. He was due to open for the headline act at 11:30. Rickenharp pictured the club emptying as he hit the stage. He wasn’t long for this club.

NewHope hit the stage. He was anorexic and surgically sexless: radical minimono. A fact advertised by his nudity: he wore only gray and black spray-on sheathing, his dick in a drag queen’s tuck. How did the guy piss? Rickenharp, wondered. Maybe it was out of that faint crease at his crotch. A dancing mannequin. His sexuality was clipped to the back of his head: a single chrome electrode that activated the pleasure center of the brain during the weekly legally controlled catharsis. But he was so skinny—hey, who knows, maybe he went to a black-market cerebrostim to interface with the pulser. Though minimonos were supposed to be into stringent law and order.

The neural transmitters jacked into NewHope’s arms and legs and torso transmitted to pickups on the stage floor. The long, funereal wails pealing from hidden speakers were triggered by the muscular contractions of his arms and legs and torso. He wasn’t bad, for a minimono, Rickenharp thought. You can make out the melody, the tune shaped by his dancing, and it had a shade more complexity than the M’n’Ms usually had … The M’n’M crowd moved into their geometrical dance configurations, somewhere between disco dancing and square dance, Busby Berkley kaleidoscopings worked out according to formulas you were simply expected to know, if you had the nerve to participate. Try to dance freestyle in their interlocking choreography, and sheer social rejection, on the wings of body language, would hit you like an arctic wind.

Sometimes Rickenharp did an acid dance in the midst of the minimono configuration, just for the hell of it, just to revel in their rejection. But his band had made him stop that. Don’t alienate the audience at our only gig, man. Probably our last fucking gig …

The wiredancer rippled out bagpipelike riffs over the digitalized rhythm section. The walls came alive.

A good rock club—in 1965 or 1975 or 1985 or 1995 or 2012 or 2039 should be narrow, dark, close, claustrophobic. The walls should be either starkly monochrome—all black or mirrored, say—or deliberately garish. Camp, layered with whatever was the contemporary avant-garde or gaudy graffiti.

The Semiconductor showed both sides. It started out butch, its walls glassy black; during the concert it went in gaudy drag as the sound-sensitive walls reacted to the music with color streaking, wavelengthing in oscilloscope patterns, shades of bluewhite for high end, red and purple for bass and percussion, reacting vividly, hypnotically to each note. The minimonos disliked reactive walls. They called it kitschy.

The dance spazzed the stage, and Rickenharp grudgingly watched, trying to be fair to it. Thinking, It’s another kind of rock ’n’ roll, is all. Like a Christian watching a Buddhist ceremony, telling himself, “Oh, well, it’s all manifestations of the One God in the end.” Rickenharp thinking: But real rock is better. Real rock is coming back, he’d tell almost anyone who’d listen. Almost no one would.

A chaotichick came in, and he watched her, feeling less alone. Chaotics were much closer to real rockers. She was a skinhead, with the sides of her head painted. The Gridfriend insignia was tattooed on her right shoulder. She wore a skirt made of at least two hundred rags of synthetic material sewn to her leather belt—a sort of grass skirt of bright rags. The nipples of her bare breasts were pierced with thin screws. The minimonos looked at her in disgust; they were prudish, and calling attention to one’s breasts was decidedly gauche with the M’n’Ms. She smiled sunnily back at them. Her handsome Semitic features were slashed randomly with paint. Her makeup looked like a spinpainting. Her teeth were filed.

Rickenharp swallowed hard, looking at her. Damn. She was his type.

Only … she wore a blue-mesc sniffer. The sniffer’s inverted question mark ran from its hook at her right ear to just under her right nostril. Now and then she tilted her head to it, and sniffed a little blue powder.

Rickenharp had to look away. Silently cursing.

He’d just written a song called “Stay Clean.”

Blue mesc. Or syncoke. Or heroin. Or amphetamorphine. Or XTZ. But mostly he went for blue mesc. And blue mesc was addictive.

Blue mesc, also called boss blue. It offered some of the effects of mescaline and cocaine together, framed in the gelatinous sweetness of methaqualone. Only … stop taking it after a period of steady use and the world drained of meaning for you. There was no actual withdrawal sickness. There was only a deeply resonant depression, a sense of worthlessness that seemed to settle like dust and maggot dung into each individual cell of the user’s body.

Some people called blue mesc “the suicide ticket.” It could make you feel like a coal miner when the mineshaft caved in, only you were buried in yourself.

Rickenharp had squandered the money from his only major microdisc hit on boss blue and synthmorph. He’d just barely made it clean. And lately, at least before the band squabbles, he’d begun feeling like life was worth living again.

Watching the girl with the sniffer walk past, watching her use, Rickenharp felt stricken, lost, as if he’d seen something to remind him of a lost lover. An ex-user’s syndrome. Pain from guilt of having jilted your drug.

And he could imagine the sweet burn of the stuff in his nostrils, the backward-sweet pharmaceutical taste of it in the back of his palate; the rush; the autoerotic feedback loop of blue mesc. Imagining it, he had a shadow of the sensation, a tantalizing ghost of the rush. In memory he could taste it, smell it, feel it … Seeing her use brought back a hundred iridescent memories and with them came an almost irrepressible longing. (While some small voice in the back of his head tried to get his attention, tried to warn him, Hey, remember the shit makes you want to kill yourself when you run out; remember it makes you stupidly overconfident and boorish; remember it eats your internal organs … a small, dwindling voice … ) The girl was looking at him. There was a flicker of invitation in her eyes.

He wavered.

The small voice got louder.

Rickenharp, if you go to her, go with her, you’ll end up using.

He turned away with an anguished internal wrenching. Stumbled through the wash of sounds and lights and monochrome people to the dressing room; to guitar and earphones and the safer sonic world.
squares

Rickenharp was listening to a collector’s item Velvet Underground tape, from 1968. It was capped into his Earmite. The song was “White Light/White Heat.” The guitarists were doing things that would make Baron Frankenstein say, “There are some things man was not meant to know.” He screwed the Earmite a little deeper so that the vibrations would shiver the bone around his ear, give him chills, chills that lapped through him in harmony with the guitar chords. He’d picked a visorclip to go with the music: a muted documentary on expressionist painters. Listening to the Velvets and looking at Edvard Munch. Man!

And then Julio dug a finger into his shoulder.

“Happiness is fleeting,” Rickenharp muttered, as he flipped the visorclip back. Some visors came with camera eye and fieldstim. The fieldstim you wore snugged to the skin, as if it were a sheer corset. The camera picked up an image of the street you were walking down and routed it to the fieldstim, which tickled your back in the pattern of whatever the camera saw. Some part of your mind assembled a rough image of the street out of that. Developed for blind people in the 1980s. Now used by viddy addicts who walked or drove the streets wearing visors, watching TV, reflexively navigating by using the fieldstim, their eyes blocked off by the screen but never quite bumping into anyone. But Rickenharp didn’t use a fieldstim.

So he had to look at Julio with his own eyes. “What do YOU want?”

“N’ten,” Julio said. Julio the technicki bassist. They went on in ten minutes.

Mose, Ponce, Julio, Murch. Rhythm guitar backup vocals. Keyboards. Bass. Drums.

Rickenharp nodded and reached up to flip the visor back in place, but Ponce flicked the switch on the visor’s headset. The visor image shrank like a landscape vanishing down a tunnel behind a train, and Rickenharp felt like his stomach was shrinking inside him at the same rate. He knew what was coming down. “Okay,” he said, turning to look at them. “What?”

They were in the dressing room. The walls were black with graffiti. All rock club dressing rooms will always be black with graffiti; flayed with it, scourged with it. Like the flat declaration THE PARASITES RULE, the cheerful petulance of symbiosis THE SCREAMIN’ GEEZERS GOT FUCKING BORED HERE, the oblique existentialism of THE ALKOLOID BROTHERS LOVE YOU ALL BUT THINK YOU WOULD BE BETTER OFF DEAD, and the enigmatic ones like SYNC 66 CLICKS NOW. It looked like the patterning of badly wrinkled wallpaper. It was in layers; it was a palimpsest. Hallucinatory stylization as if tracing the electron firings of the visual cortex.

The walls, in the few places they were visible under the graffiti, were a gray-painted pressboard. There was just enough room for Rickenharp’s band, sitting around on broken-backed kitchen chairs and one desk chair with three legs. Crowded between the chairs were instruments in their cases. The edges of the cases were false leather peeling away. Half the snaps broken.

Rickenharp looked at the band, looked clockwise one face to the next, taking a poll from their expressions: Mose on his left, a bruised look to his eyes; his hair a triple-Mohawk, the center spine red, the outer two white and blue; a smoky crystal ring on his left index finger that matched—he knew it matched—his smoky crystal amber eyes. Rickenharp and Mose had been close. Each looked at the other a little accusingly. There was a lover’s sulkiness between them, though they’d never been lovers. Mose was hurt because Rickenharp didn’t want to make the transition: Rickenharp was putting his own taste in music before the survival of the band. Rickenharp was hurt because Mose wanted to go minimono wifi act, a betrayal of the spiritual ethos of the band; and because Mose was willing to sacrifice Rickenharp. Replace him with a wire dancer. They both knew it, though it had never been said. Most of what passed between them was semiotically transmitted with the studied indirection of the terminally cool.

Tonight, Mose looked like serious bad news. His head was tilted as if his neck were broken, his eyes lusterless.

Ponce had gone minimono, at least in his look, and they’d had a ferocious fight over that. Ponce was slender—like everyone in the band—and fox-faced, and now he was sprayed battleship gray from head to toe, including hair and skin. In the smoky atmosphere of the clubs he sometimes vanished completely.

He wore silver contact lenses. Flat-out glum, he stared at a ten-slivered funhouse reflection in his mirrored fingernails.

Julio, yeah, he liked to give Rickenharp shit, and he wanted the change-up. Sure, he was loyal to Rickenharp, up to a point. But he was also a conformist. He’d argue for Rickenharp maybe, but he’d go with the consensus. Julio had lush curly black Puerto Rican hair piled prowlike over his head. He had a woman’s profile and a woman’s long-lashed eyes. He had a silver-stud earring, and wore classic retro-rock black leather like Rickenharp. He twisted the skull-ring on his thumb, returning a scowl for its grin, staring at it as if deeply worried that one of its ruby-red glass eyes was about to come out.

Murch was a thick slug of a guy with a glass crew cut. He was a mediocre drummer, but he was a drummer, with a trap set and everything, a species of musician almost extinct. “Murch’s rare as a dodo,” Rickenharp said once, “and that’s not all he’s got in common with a dodo.” Murch wore horn-rimmed dark glasses, and he was holding a bottle of Jack Daniels on his knee. The Jack Daniels was a part of his outfit. It went with his cowboy boots, or so he thought.

Murch was looking at Rickenharp in open contempt. He didn’t have the brains to dissemble.

“Fuck you, Murch,” Rickenharp said.

“Whuh? I didn’t say nothing.”

“You don’t have to. I can smell your thoughts. Enough to gag a faggot maggot.” Rickenharp stood and looked at the others. “I know what’s on your mind. Give me this: one last good gig. After that you can have it how you want.”

Tension lifted its wings and flew away.

Another bird settled over the room. Rickenharp saw it in his mind’s eye: a thunderbird. Half made of an Indian teepee painting of a thunderbird, and half of chrome T-Bird car parts. When it spread its wings the pinfeathers glistened like polished bumpers. There were two headlights on its chest, and when the band picked up their instruments to go out to the stage, the headlights switched on.

Rickenharp carried his Stratocaster in its black case. The case was bandaged with duct tape and peeling with faded stickers. But the Strat was spotless. It was transparent. Its lines curved hot like a sports car.

They walked down a white plastibrick corridor toward the stage. The corridor narrowed after the first turn, so they had to walk sideways, holding the instruments out in front of them. Space was precious on Freezone.

The stagehand saw Murch go out first, and he signaled the DJ, who cut the canned music and announced the band through the PA. Old-fashioned, like Rickenharp requested: “Please welcome … Rickenharp.”

There was no answering roar from the crowd. There were a few catcalls and a smattering of applause.

Good, you bitch, fight me, Rickenharp thought, waiting for the band to take up their positions. He’d go on stage last, after they’d set up the spot for him. Always.

Rickenharp squinted from the wings to see past the glare of lights into the dark snakepit of the audience. Only about half minimono now. That was good, that gave him a chance to put this one over.

The band took its place, pressed their automatic tuners, fiddled with dials.

Rickenharp was pleasantly surprised to see that the stage was lit with soft red floods, which is what he’d requested. Maybe the lighting director was one of his fans. Maybe the band wouldn’t fuck this one up. Maybe everything would fall into place. Maybe the lock on the cage door would tumble into the right combination, the cage door would open, the T-Bird would fly.

He could hear some of the audience whispering about Murch. Most of them had never seen a live drummer before, except for salsa. Rickenharp caught a scrap of technicki: “Whuzziemackzut?” What’s he making with that, meaning: What are those things he’s adjusting? The drums.

Rickenharp took the Strat out of its case and strapped it on. He adjusted the strap, pressed the tuner. When he walked onto the stage, the amp’s reception field would trigger, transmit the Strat’s signals to the stack of Marshalls behind the drummer. A shame, in a way, about miniaturization of electronics: the amps were small, though just as loud as twentieth century amps and speakers. But they looked less imposing. The audience was muttering about the Marshalls, too. Most of them hadn’t seen old-fashioned amps. “What’s those for?” Murch looked at Rickenharp. Rickenharp nodded.

Murch thudded 4/4, alone for a moment. Then the bass took it up, laid down a sonic strata that was kind of off-center strutting. And the keyboards laid down sheets of infinity.

Now he could walk on stage. It was like there’d been an abyss between Rickenharp and the stage, and the bass and drum and keyboards working together made a bridge to cross the abyss. He walked over the bridge and into the warmth of the floods. He could feel the heat of the lights on his skin. It was like stepping from an air-conditioned room into the tropics. The music suffered deliciously in a tropical lushness. The pure white spotlight caught and held him, focusing on his guitar, as per his directions, and he thought, Good, the lighting guy really is with me.

He felt as if he could feel what the guitar felt. The guitar ached to be touched.
squares

Without consciously knowing it, Rickenharp was moving to the music. Not too much. Not in the pushy, look-at-me way that some performers had. The way they had of trying to force enthusiasm from the audience, every move looking artificial.

No, Rickenharp was a natural. The music flowed through him physically, unimpeded by anxieties or ego knots. His ego was there: it was the fuel for his personal Olympics torch. But it was also as immaculate as a pontiff’s robes.

The band sensed it: Rickenharp was in rare form tonight. Maybe it was because he was freed. The tensions were gone because he knew this was the end of the line: the band had received its death sentence: Now, Rickenharp was as unafraid as a true suicide. He had the courage of despair.

The band sensed it and let it happen. The chemistry was there, this time, when Ponce and Mose came into the verse section. Mose with a sinuous riffing picked low, almost on the chrome-plate that clamped the strings; Ponce with a magnificently redundant theme washed through the brass mode of the synthesizer. The whole band felt the chemistry like a pleasing electric shock, the pleasurable shock of individual egos becoming a group ego.

The audience was listening, but they were also resisting. They didn’t want to like it. Still, the place was crowded—because of the club’s rep, not because of Rickenharp—and all those packed-in bodies make a kind of sensitive atmospheric exo-skeleton, and he knew that made them vulnerable. He knew what to touch.

Feeling the Good Thing begin to happen, Rickenharp looked confident but not quite arrogant—he was too arrogant to show arrogance.

The audience looked at Rickenharp as a man will look at a smug adversary just before a hand-to-hand fight and wonder, “Why’s he so smug, what does he know?”

He knew about timing. And he knew there were feelings even the most aloof among them couldn’t control, once those feelings were released: and he knew how to release them.

Rickenharp hit a chord. He let it shimmer through the room and he looked out at them. He made eye contact.

He liked seeing the defiant stares, because that was going to make his victory more complete.

Because he knew. He’d played five gigs with the band in the last two weeks, and for all five gigs the atmosphere had been strained, the electricity hadn’t been there; like a Jacob’s ladder where the two poles aren’t properly lined up for the sparks to jump.

And like sexual energy, it had built up in them, dammed behind their private resentments; and now it was pouring through the dam, and the band shook with the release of it as Rickenharp thundered into his progression and began to sing …

Strumming over the vocals, he sang,

And for me, yeah for me

PAIN IS EVERYTHING!

Pain is all there is

Babe take some of mine

or lick some of his

PAIN IS EVERYTHING!

Pain is all there is

Babe take some of mine …

Singing it insolently, half shouting, half warbling at the end of each note, with that fuck-you tone, performing that magic act: shouting a melody. He could see doors opening in their faces, even the minimonos, even the neutrals, all the flares, the rebs, the chaotics, the preps, the retros. Forgetting their subcultural classifications in the unification of the music. He was basted in sweat under the lights, he was squeezing sounds with his fingers and it was as if he could feel the sounds taking shape in his hands the way a sculptor feels clay under his fingers, and it was like there was no gap between his hearing the sound in his head and its coming out of the speakers. His brain, his body, his fingers had closed the gap, was one supercooled circuit breaker fused shut.

Some part of him was looking through the crowd for the chaotichick he’d spotted earlier. He was faintly disappointed when he didn’t see her. He told himself, You ought to be happy, you had a narrow escape, she would’ve got you back into boss blue.

But when he saw her press to the front and nod at him ever so slightly in that smug insider’s way, he was simply glad, and he wondered what his subconscious was planning for him … All those thoughts were flickers. Most of the time his conscious mind was completely focused on the sound, and the business of acting out the sound for the audience. He was playing out of sorrow, the sorrow of loss. His family was going to die, and he played tunes that touched the chord of loss, in everyone …

And the band was supernaturally tight. The gestalt was there, uniting them, and he thought: The band feels good, but it’s not going to help when the gig’s over.

It was like a divorced couple having a good time in bed but knowing that wouldn’t make the marriage right again; the good time was a function of having given up.

But in the meantime there were fireworks.

By the last tune in the set the electricity was so thick in the club that—as Mose had said once, with a rocker’s melodrama—“If you could cut it, it would bleed.” The dope and smashweed and tobacco smoke moiling the air seemed to conspire with the stage lights to create an atmosphere of magical apartness. With each song-keyed shift in the light, red to blue to white to sulfurous yellow, a corresponding emotional wavelength rippled through the room. The energy built, and Rickenharp discharged it, his Strat the lightning rod.

And then the set ended.

Rickenharp bashed out the last five notes alone, nailing a climax onto the air. Then he walked offstage, hardly hearing the roar from the crowd. He found himself half running down the white, grimy plastibrick corridor, and then he was in the dressing room and didn’t remember coming there. The graffiti seemed to writhe on the walls as if he’d taken a psychedelic. Everything felt more real than usual. His ears were ringing like Quasimodo’s belfry.

He heard footsteps and turned, working up what he was going to say to the band. But it was the chaotichick and someone else, and then a third dude coming in after the someone else.

The someone else was a skinny guy with brown hair that was naturally messy, not messy as part of one of the cultural subcurrents. His mouth hung a little ajar, and one of his incisors was decayed black. His nose was windburned and the back of his bony hands were gnarled with veins. The third dude was Japanese; small, brown-eyed, nondescript, his expression was mild, just a shade more friendly than neutral. The skinny Caucasian guy wore an army jacket sans insignia, shiny jeans, and rotting tennis shoes. His hands were nervous, like there was something he was used to holding in them that wasn’t there now. An instrument? Maybe.

The Japanese guy wore a Japanese Action Suit—surprise, surprise—sky-blue and neat as a pin. There was a lump on his hip—something he could reach by putting his right arm across his body and through the open zipper down the front of the suit—and Rickenharp was pretty sure it was a gun.

There was one thing all three of them had in common: they looked half-starved.

Rickenharp shivered—his gloss of sweat cooling on him, but he forced himself to say, “Whusappnin’?” It was wooden in his mouth. He was looking past them, waiting for the band.

“Band’s in the wings,” the chaotichick said. “The bass player said to tell you … well, it was Telm zassouter.”

Rickenharp had to smile at her mock of Julio’s technicki. Tell him, get his ass out here.

Then some of the druggy feeling washed away and he heard the shouts from the audience and he realized they wanted an encore.

“Jeez, an encore,” he said without thinking. “Been so fucking long.”

“ ’Ey mate,” the skinny guy said, pronouncing mate like mite. Brit or Aussie. “I saw you at Stone’enge five years ago when you ’ad yer second ’it.”

Rickenharp winced a little when the guy said your second hit, inadvertently underlining the fact that Rickenharp had had only two, and everyone knew he wasn’t likely to have any more.

“I’m Carmen,” the chaotichick said. “This is Willow and Yukio.” Yukio was standing sideways from the others, and something about the way he did it told Rickenharp he was watching down the corridor without seeming to.

Carmen saw Rickenharp looking at Yukio and said, “Cops are coming down.”

“Why?” Rickenharp asked. “The club’s licensed.”

“Not for you or the club. For us.”

He looked at her and said, “Hey, I don’t need to get busted … ” He picked up his guitar and went into the hall. “I got to do my encore before they lose interest.”

She followed along, into the hall and the echo of the encore stomps, and asked, “Can we hang out in the dressing room for a while?”

“Yeah, but it ain’t sacrosanct. You come back here, the cops can, too.” They were in the wings now. Rickenharp signaled to Murch and the band started playing.

Standing beside him, she said, “These aren’t exactly cops. They probably don’t know these kind of places, they’d look for us in the crowd, not the dressing room.”

“You’re an optimist. I’ll tell the bouncer to stand here, and if he sees anyone else start to come back, he’ll tell ’em it’s empty back here ’cause he just checked. Might work, might not.”

“Thanks.” She went back to the dressing room. He spoke to the bouncer and went on stage. Feeling drained, the guitar heavy on him. But he picked up on the energy level in the room and it carried him through two encores. He left them wanting more—and, sticky with sweat, walked back to his dressing room.

They were still there. Carmen, Yukio, Willow.

“Is there a stage door?” Yukio asked. “Into alley?”

Rickenharp nodded. “Wait in the hall; I’ll come out and show you in a minute.”

Yukio nodded, and they went into the hall. The band came in, filed past Carmen and Yukio and the Brit without much noticing them, assuming they were backstage hangout flotsam, except Murch stared at Carmen’s tits and swaggered a bit, twirling his drumsticks.

The band sat around laughing in the dressing room, slapping palms, lighting several kinds of smokes. They didn’t offer Rickenharp any; they knew he didn’t use it.

Rickenharp was packing his guitar away, when Mose said, “You blew good.”

“You mean he gave you a good head?” Murch said, and Julio snickered.

“Yeah,” Ponce said, “the guy gives a good head, good collarbone, good kidneys—”

“Good kidneys? Rick sucks on your kidneys? I think I’m gonna puke.”

And the usual puerile band banter because they were still high from a good set and putting off what they knew had to come, till Rickenharp said, “What you want to talk about, Mose?”

Mose looked at him, and the others shut up.

“I know there’s something on your mind,” Rickenharp said softly. “Something you haven’t come out with yet.”

Mose said, “Well, it’s like—there’s an agent Ponce knows, and this guy could take us on. He’s a technicki agent and we’d be taking on a technicki circuit, but we’d work our way back from there, that’s a good base. But this guy says we have to get a wire act in.”

“You guys been busy,” Rickenharp said, shutting the guitar case.

Mose shrugged, “Hey, we ain’t been doing it behind your back; we didn’t hear from the guy till yesterday night. We didn’t have a real chance to talk to you till now, so, uh, we have the same personnel but we change costumes, change the band’s name, write new tunes.”

“We’d lose it,” Rickenharp said. Feeling caved-in. “We’d lose the thing we got, doing that shit, because it’s all superimposed.”

“Rickenharp—rock ’n’ roll is not a fucking religion,” Mose said.

“No, it’s not a religion, it’s a way of life. Now, here’s my proposal: we write new songs in the same style as always. We did good tonight. It could be the beginning of a turnaround for us. We stay here, build on the base audience we established tonight.”

It was like throwing coins into the Grand Canyon. You couldn’t even hear them hit bottom.

The band just looked back at him.

“Okay,” Rickenharp said. “Okay. We’ve been through this ten fucking times. Okay. That’s all.” He’d had an exit speech worked out for this moment, but it caught in his throat. He turned to Murch and said, “You think they’re going to keep you on, they tell you that? Bullshit! They’ll be doing it without a drummer, man. You better learn to program computers, fast.” Then he looked at Mose. “Fuck you, Mose.” He said it quietly.

He turned to Julio, who was looking at the far wall as if to decipher some particularly cryptic piece of graffiti. “Julio, you can have my amp, I’ll be traveling light.”

He turned and, carrying his guitar, walked out, leaving silence behind him.

He nodded at Yukio and his friends and they followed him to the stage door. At the door, Carmen said, “Any chance you could help us find a little cover?”

Rickenharp needed company, bad. He nodded and said, “Yeah, if you’ll gimme a hit of that blue boss.”

She said, “Sure.” And they went into the alley.
squares

Rickenharp put on his dark glasses, because of the way the Walk tugged at him.

The Walk wound through the interlinked Freezone outfloats for a half-mile, looping up and back, a hairpin canyon of arcades crusted with neon and glowflake, holos and screens. It was involuted, intensified by layering and a blaze of colored light.

Stoned, very stoned: Rickenharp and Carmen walked together through the sticky-warm night, almost in step. Yukio walked behind, Willow ahead, and Rickenharp felt like part of a jungle patrol formation. And he had another feeling: that they were being followed, or watched. Maybe it was suggestion, from seeing Yukio and Willow glance over their shoulders now and then …

Rickenharp felt a ripple of kinetic force under his feet, an arc of wallow moving in languid whiplash through the flexible streetstuff, telling him that the breakers were up today, the baffles around the artificial island feeling the strain.

The arcades ran three levels above the narrow street; each level had its own sidewalk balcony; people stood at the railing to look down at the segmented snake of street traffic. The stack of arcades funneled a rich wash of scents to Rickenharp: the french-fry toastiness of the fast food; the sweet harshness of smashweed smoke, gyno-smoke, tobacco smoke—the cloy of perfumes; the mixed odors of fish-ka-bob stands, urine, rancid beer, popcorn, sea air; and the faint ozone smell of the small, eerily quiet electric cars jockeying on the street. His first time here, Rickenharp had thought the place smelled wrong for a red light cluster. “It’s wimpy,” he’d said. Then he’d realized he was missing the bass-bottom of carbon monoxides. There were no combustion cars on Freezone. Some parts of America still permitted pollutive, resource-greedy gasoline cars, and Rickenharp, being a retro, had preferred those places.

The sounds splashed over Rickenharp in a warm wave of cultural fecundity; pop tunes from thudders and wrist-boxes swelled in volume as they passed, the guys exuding the music insignificant in comparison to the noise they carried, the skanky tripping of protosalsa or the calculatedly redundant pulse of minimono.

Rickenharp and Carmen walked beneath a fiberglass arch—so covered with graffiti its original commemorative meaning was lost—and ambled down the milky walkway under the second-story arcade boardwalk. The multinational crowd thickened as they approached the heart of the Walk. The soft lights glowing upward from beneath the polystyrene walkway gave the crowd a 1940s-horror-movie look; even through the dark glasses the place tugged at Rickenharp with a thousand subliminal come-hithers.

Rickenharp was still riding the blue mesc surf, but the wave was beginning to break; he could feel it crumbling under him. He looked at Carmen. She glanced back at him, and they understood one another. She looked around, then nodded toward the darkened doorway of a defunct movie theater, a trash-cluttered recess twenty feet off the street. They went into the doorway; Yukio and Willow stood with their backs to the door, blocking the view from the street, so that Rickenharp and Carmen could each do a double hit of blue mesc. There was a kind of little-kid pleasure in stepping into seclusion to do drugs, a rush of outlaw in-crowd romance to it. On the second sniff the graffiti on the pad-locked, fiberglass doors seemed to writhe with significance. “I’m running low,” Carmen said, checking her mesc bottle.

“Running low on drugs? Whoever heard of that happening?” Rickenharp said and they both burst in peals of laughter. His mind was racing now, and he felt himself click into the boss blue verbal mode. “You see that graffiti? You’re gonna die young because the ITE took the second half of your life. You know what that is? I didn’t know what ITE was till yesterday, I used to see those things and wonder and then somebody said—”

“Immortality something or other,” she said, licking blue mesc off her sniffer.

“Immortality Treatment Elite. Supposedly some people keeping an immortality treatment to themselves because the government doesn’t want the public to live too long and overpopulate the place. Another bullshit conspiracy theory.”

“You don’t believe in conspiracies?”

“I don’t know—some. Nothing that far-fetched. But—I think people are being manipulated all the time. Even here … this place tugs at you, you know. Like—”

Willow said, “Right, we’ll ’ave our sociology class later children, you gotter? Where’s this place with the bloke can get us off the fooking island?”

“Yeah, okay, come on,” Rickenharp said, leading them back into the flow of the crowd—but seamlessly picking up his blue mesc rap. “I mean, this place is a Times Square, right? You ever read the old novels about that place? That was the archetype. Or some places in Bangkok. I mean, these places are carefully arranged. Maybe subconsciously. But arranged as carefully as Japanese florals, only with the inverse esthetic. Sure, every whining, self-righteous tightassed evangelist who ever preached the diabolic seductiveness of places like this was right—in a way—was fully justified ’cause, yeah, the places titillate and they seduce and they vampirize people. Yeah, they’re Venus’s-flytraps. Architectural Svengalis. Yes to all the clichés about the bad part of town. All the reverend preachers—Reverend who, Reverend—what’s his name?—Rick Crandall … ”

She looked sharply at him. He wondered why but the mesc swept him on.

“All the preachers are right, but the reason they’re right is why they’re wrong, too. Everything here is trying to sell you something. Lots of lights and whirligig suction to seduce you into throwing your energy into it—in the form of money. People mostly come here to buy or to be titillated up to the verge of buying. The tension between wanting to buy and the resistance to buying can give you a charge. That’s what I get into: I let it tickle my glands, but I hold back from paying into it. You know? Just constant titillation but no orgasm, because you waste your money or you get a social disease or mugged or sold bad drugs or something … I mean, anything sold here is pointless bullshit. But it’s harder for me to resist tonight … ” Because I’m stoned. “Makes you susceptible. Receptive to subliminals worked into the design of the signs, that gaudy kinetics, those fucking on/off bulbs—makes you flash on the old computer-thinking models, binomial thinking, on-off, on-off, blink blink—all those neon tubes, pulling you like the hypnotist’s spiral pendant in the old movies … And the kinds of colors they use, the energy of the signs, the rate of pulse, the rate of on/offing in the bulbs, all of it’s engineered according to principles of psychology the people who make them don’t even know they’re using, colors that hint about, you know, glandular discharges and tingly chemical flows to the pleasure center … like obscenities you pay for in the painted mouth of a whore … like video games … I mean—”

“I know what you mean,” she said, in desperation buying a waxpaper cup of beer. “You must be thirsty after that monologue. Here.” She shoved the foaming cup under his nose.

“Talking too much. Sorry.” He drank off half the beer in three gulps, took a breath, finished it, and it was paradise for a moment. A wave of quietude soothed him—and then evaporated mesc burned through again. Yeah, he was wired.

“I don’t mind listening to you talk,” she said, “except you might say too much, and I’m not sure if we’re being scanned.”

Rickenharp nodded sheepishly, and they walked on. He crushed the cup in his hand, began methodically to shred it as they went.

Rickenharp luxuriated in the colors of the place, colors that mixed and washed over the crowd, making the stream of hats and heads into a living swatch of iridescent gingham; shining the cars into multicolored lumps of mobile ice.

You take the word lurid, Rickenharp thought, and you put it raw in a vat filled with the juice of the wordappeal. You leave it and let the acids of appeal leach the colors out of lurid, so that you get a kind of gasoline rainbow on the surface of the vat. You extract the petro-rainbow on the surface of the vat with cheesecloth and strain it into a glass tube, dilute heavily with oil of cartoon innocence and extract of pure subjectivity. Now run a current through the glass tube and all the other tubes of the neon signs interlacing Freezone’s Walk.

The Walk, stretching ahead of them, was itself almost a tube of colored lights, converging in a kaleidoscope; the concave fronts of the buildings to either side were flashing with a dozen varieties of signs. The sensual flow of neon data in primary colors was broken at cunningly irregular intervals by stark trademark signs: Synthlife Systems and Microsoft-Apple and Nike and Coca-Cola and Warner Amex and NASA Chemco and Brazilian Exports Intl and Exxon Electrics and Nessio. In all of that only one hint of the war: two unlit signs, Fabrizzio and Allinne—an Italian and a French company, killed by the Russian blockades. The signs were unlit, dead.

They passed a TV-shirt shop; tourists walked out with their shirts flashing video imagery, fiberoptics woven into the shirtfront playing the moving sequence of your choice.

Sidewalk hawkers of every race sold beta candy spiked with endorphins; sold shellfish from Freezone’s own beds, tempura’d and skewered; sold holocube pornography key rings; sold instapix of you and your wife, oh that’s your boyfriend … Despite the nearness of Africa, black Africans were few here: Freezone Admin considered them a security risk and few on the contiguous coast could afford the trip. The tourists were mostly Japanese, Canadian, Brazilians—riding the crest of the Brazilian boom—South Koreans, Chinese, Arabs, Israelis, and a smattering of Americans; damned few Americans anymore, with the depression. Screens scanned them, one of them caught Rickenharp with a facial recognition program and on it a sexy animated Asian woman cooed, “Rick Rickenharp—try Wilcox Subsensors and walk in a glow of excitement … ”

As they got deeper into the Walk the atmosphere became even more hot-house. It was a multicolored steam bath. The air was sultry, the various smokes of the place warping the neon glow, filtering and smearing the colors of signs and TV shirts and DayGlo jewelry. High up, between the not-quite-fitted jigsaw parts of signs and lights, were blue-black slices of night sky. At street level the jumble was given shape and borders by the doors opening on either side: by people using the doors to check out malls and stimsmoke parlors and memento shops and cubey theaters and, especially, tingler galleries. Dealers drifted up like reef fish, nibbling and moving on, pausing to offer, “DH, gotcher good Dee Ech”: Direct Hookup, illegal cerebral pleasure center stimulation. And drugs: synth-cocaine and smokeable herbs; stims, and downs. About half of the dealers were burn artists, selling baking soda or pseudostims. The dealers tended to hang on to Rickenharp and Carmen because they looked like users, and Carmen was wearing a sniffer. Blue mesc and sniffers were illegal, but so were lots of things the Freezone cops ignored. You could wear a sniffer, carry the stuff, but the understanding was, you don’t use it openly, you step into someplace discreet.

And whores of both sexes cruised the street, flagrantly soliciting. Freezone Admin was supposed to regulate all prostitution, but black-market pros were tolerated as long as somebody paid off the beat security and as long as they didn’t get too numerous.

The crowd streaming past was a perpetually unfolding revelation of human variety. It unfolded again and a specialty pimp appeared, pushing a man and woman ahead of him; they had to hobble because they were straitjacket-packaged in black-rubber bondage gear. Their faces were ciphers in blank black-rubber masks; aluminum racks held their mouths wide-open, intended to be inviting, but to Rickenharp whispered to Carmen, “Victims of a mad orthodontist!” and she laughed.

Studded down the streets were Freezone security guards in bullet-proofed uniforms that made Rickenharp think of baseball umpires, faces caged in helmets. Their guns were locked by combination into their holsters; they were trained to open the four-digit combination in one second.

Mostly they stood around, gossiped on their helmet radios. Now two of them hassled a sidewalk three-card-monte artist—a withered little black guy who couldn’t afford the baksheesh—pushing him back and forth between them, bantering one another through helmet amplifiers, their voices booming over the discothud from the speakers on the download shops: “WHAT THE FUCK YOU DOING ON MY BEAT SCUMBAG. HEY BILL YOU KNOW WHAT THIS GUY’S DOING ON MY BEAT.”

“FUCK NO I DUNNO WHAT’S HE DOING ON YOUR BEAT.”

“HE’S MAKlNG ME SICK WITH THIS RIP-OFF MONTE BULLSHIT IS WHAT HE’S DOING.”

One of them hit the guy too hard with the waldo-enhanced arm of his riot suit and the monte dealer spun to the ground like a top running out of momentum, out cold.

“LOITERING ON THE ZONE’S WALKS, YOU SEE THAT BILL.”

“I SEE AND IT MAKES ME SICK JIM.”

The bulls dragged the little guy by the ankle to a lozenge-shaped kiosk in the street and pushed him into a man-capsule. They sealed the capsule, scribbled out a report, pasted it onto the capsule’s hard plastic hull. Then they shoved the man-capsule into the kiosk’s chute. The capsule was sucked by mail-tube principle to Freezone Lockup.

“Looks like they’re using some kind of garbage disposal to get rid of people here,” Carmen said when they were past the cops.

Rickenharp looked at her. “You weren’t nervous walking by the cops. So it’s not them we’re avoiding, huh?”

“Nope.”

“You wanna tell me who it is we’re supposed to be avoiding?”

“Uh-uh, I do not.”

“How do you know these out-of-town cops you’re worried about haven’t gone to the locals and recruited some help?”

“Yukio says they won’t, they don’t want anybody to scan what they’re doing here because the Freezone admin don’t like ’em.”

Rickenharp guessed: the who they were avoiding was the Second Alliance. Freezone’s chairman was Jewish. The Second Alliance could meet in Freezone—the idea was, the place was open to anyone for meetings, or recreation; anyone, even people the Freezone boss would like to see gassed—but the SA couldn’t operate here, except covertly.

The fucking SA bulls! Shit! … The blue mesc worked with his paranoia. Adrenaline spurted, making his heart bang. He began to feel claustrophobic in the crowd; began to see patterns in the movement around him, patterns charged with meaning superimposed by his own fear-galvanized mind. Patterns that taunted him with, The SA’s close behind. He felt a stomach-churning combination of horror and elation.

All night he’d worked hard at suppressing thoughts of the band. And of his failure to make the band work.He’d lost the band. And it was almost impossible to make anyone understand why that was, to him, like a man losing his wife and children. And there was the career. All those years of pushing for that band, struggling to program a place for it in the Grid. Shot to hell now, his identity along with it. He knew, somehow, that it would be futile to try to put together another band. The Grid just didn’t want him; and he didn’t want the fucking Grid. And the elation was this: that ugly pit of displacement inside him closed up, was just gone, when he thought about the SA bulls. The bulls threatened his life, and the threat caught him up in something that made it possible to forget about the band. He’d found a way out.

But the horror was there, too. If he got caught up in this … if the SA bulls got hold of him …

Fuck it. What else did he have?

He grinned at Carmen, and she looked blankly back at him, wondering what the grin meant.

So now what? he asked himself. Get to the OmeGaity. Find Frankie. Frankie was the doorway.

But it was taking so long to get there. Thinking. The drug’s fucking with your sense of duration. Heightened perception makes it seem to take longer.

The crowd seemed to get thicker, the air hotter, the music louder, the lights brighter. It was getting to Rickenharp. He began to lose the ability to make the distinction between things in his mind and things around him. He began to see himself as an enzyme molecule floating in some macrocosmic bloodstream—the sort of things that always OD’d him when he did an energizing drug in a sensory-overflow environment.

What am I?

Sizzling orange-neon arrows on the marquee overhead seemed to crawl off the marquee, slither down the wall, down into the sidewalk, snaking to twine around his ankles, to try to tug him into a tingler emporium. He stopped and stared. The emporium’s display holos writhed with fleshy intertwinings; breasts and buttocks jutted out at him, and he responded against his will, like all the clichés, getting hard in his pants: visual stimuli; monkey see, monkey respond. He thought: Bell rings and dog salivates.

He looked over his shoulder. Who was that guy with the sunglasses back there? Why was he wearing sunglasses at night? Maybe he’s SA— Noooo, man. I’m wearing sunglasses at night. Means nothing.

He tried to shrug off the paranoia, but somehow it was twined into the undercurrent of sexual excitement. Every time he saw a whore or a pornographic video sign, the paranoia hooked into him as a kind of scorpion stinger on the tail of his adolescent surge of arousal. And he could feel his nerve ends begin to extrude from his skin. After having been clean so long, his-blue mesc tolerance was low.

Who am I? Am I the crowd?

He saw Carmen look at something in the street, then whisper urgently to Yukio.

“What’s the matter?” Rickenharp asked.

She whispered, “You see that silver thing? Kind of a silvery fluttering? There—over the cab … Just look, I don’t wanna point.”

He looked into the street. A cab was pulling up at the curb. Its electric motor whined as it nosed through a heap of refuse. Its windows were dialed to mercuric opacity. Above and a little behind it a chrome bird hovered, its wings a hummingbird blur. It was about thrush-sized, and it had a camera-lens instead of a head. “I see it. Hard to say whose it is.”

“I think it’s run from inside that cab. That’s like them. They’ll send it after us from there. Come on.” She ducked into a tingler gallery; Willow and Yukio and Rickenharp followed her. They had to buy a swipe card to get in. A bald, jowly old dude it the counter took the cards, swiped them without looking, his eyes locked on a wrist-TV screen. On his wrist a miniature newscaster was saying in a small tinny voice, “ … attempted assassination of SA director Crandall today … ” Something mumbled, distorted. “ … Crandall is in serious condition and heavily guarded at Freezone Medicenter … ”

The turnstile spun for them and they went into the gallery. Rickenharp heard Willow mutter to Yukio, “The bastard’s still alive.”

Rickenharp put two and two together.

The tingler gallery was predominantly fleshtone, every available vertical surface taken up by emulsified nude humanity. As you passed from one photo or holo to the next, you saw the people in them were inverted or splayed or toyed with, turned in a thousand variations on coupling, as if a child had been playing with unclothed dolls and left them scattered. A sodden red light hummed in each booth: the light snagged you, a wavelength calculated to produce sexual curiosity. In each “privacy booth” was a screen and a tingler. An oxygen mask that dropped from a ceiling trap pumped out a combination of amyl nitrite and pheromones. The tingler looked like a twentieth-century vacuum cleaner hose with an oversized salt-shaker top on one end: You watched the pictures, listened to the sounds, and ran the tingler over your erogenous zones; the tingler stimulated the appropriate nerve ends with a subcutaneously penetrative electric field, very precisely attenuated. You could pick out the guys in the health-club showers who’d used a tingler too long: use it more than the “recommended thirty-five-minute limit” and it made your skin look sunburned. One time Rickenharp’s drummer had asked him if he had any lotion: “I got ‘tingler dick,’ man.”

“To phrase it in the classic manner,” Yukio said abruptly, “is there another way out of here?”

Rickenharp nodded. “Yeah … Uh—somewhere.”

Willow was staring at a teaser blurb under a still-image of two men, a woman and a goat. He took a step closer, squinting at the goat.

“You looking for a family resemblance, Willow?” Rickenharp said.

“Shut your ’ole, ya retro greaser.”

The booth sensed his nearness: the images on the sample placard began to move, bending, licking, penetrating, reshaping themselves with a weirdly formalized awkwardness; the booth’s light increased its red glow, puffed out a tease of pheromone and amyl nitrite, trying to seduce him.

“Well, where is the other door?” Carmen hissed.

“Huh?” Rickenharp looked at her. “Oh! I’m sorry, I’m so—uh I’m not sure.” He glanced over his shoulder, lowered his voice. “The bird didn’t follow us in.”

Yukio murmured, “The electric fields on the tinglers confuse the bird’s guidance system. But we must keep a step ahead.”

Rickenharp looked around—but he was still stoned: the maze of black booths and fleshtones seemed to twist back on itself, to turn ponderously, as if going down some cubistic drain …

“I will find the other door,” Yukio said. Rickenharp followed him gratefully. He wanted out.

They hurried through the narrow hall between tingler booths. The customers moved pensively—or strolled with excessive nonchalance—from one booth to another, reading the blurbs, scanning the imagery, sorting through fetishistic indexings for their personal libido codes, not looking at one another except peripherally, carefully avoiding the margins of personal-space.

Chuffing, sighing music played from somewhere; the red lights were like the glow of blood in a hand held over a bright light. But the place was rigorously Calvinistic in its obstacle course of tacit regulations. And here and there, at the turns in the hot, narrow passageways between rows of booths, bored security guards rocked on their heels and told the browsers, No loitering, please, you can purchase more time at the front desk.

Rickenharp flashed that the place wanted to drain his sexuality, as if the vacuum-cleaner hoses in the booths were going to vacuum his orgone energy, leave him chilled as a gelding.

Get the fuck out of here.

Then he saw EXIT, and they rushed for it, through it.

They were in an alley. They looked up, around, half expecting to see the metal bird. No bird. Only the gray intersection of styroconcrete planes, stunningly monochrome after the hungry chromatics of the tingler gallery.

They walked out to the end of the alley, stood for a moment watching the crowd. It was like standing on the bank of a torrent. Then they stepped into it, Rickenharp, blue mesc’d, fantasizing that he was getting wet with the liquefied flesh of the rush of humanity as he steered by sheer instinct to his original objective: the OmeGaity.

They pushed through the peeling black chessboard doors into the dark mustiness of the OmeGaity’s entrance hall, and Rickenharp gave Carmen his coat to hide her bare breasts. “Men only, in here,” he said, “but if you don’t shove your femaleness into their line of sight, they might let us slide.”

Carmen pulled the jacket on, zipped it up—very carefully—and Rickenharp gave her his dark glasses.

Rickenharp banged on the window of the screening kiosk beside the locked door that led into the cruising rooms. Beyond the glass, someone looked up from a fat-screen TV. “Hey, Carter,” Rickenharp said.

“Hey.” Carter grinned at him. Carter was, by his own admission, “a trendy faggot.” He was flexicoated battleship gray with white trim, a minimono style. But the real M’n’Ms would have spurned him for wearing a luminous earring—it blinked through a series of words in tiny green letters—Fuck … you … if … you … don’t … like … it … Fuck … you … if—and they’d have considered that unforgivably “Griddy.” And anyway Carter’s wide, froggish face didn’t fit the svelte minimono look. He looked at Carmen. “No girls, Harpie.”

“Drag queen,” Rickenharp said. He slipped a folded twenty newbux note through the slot in the window. “Okay?”

“Okay, but she takes her chances in there,” Carter said, shrugging. He tucked the twenty in his charcoal bikini briefs.

“Sure.”

“You hear about Geary?”

“Nope.”

“Snuffed hisself with China White ’cause he got green pissed.”

“Oh, shit.” Rickenharp’s skin crawled. His paranoia flared up again, and to soothe it he said, “Well, I’m not gonna be licking anybody’s anything. I’m looking for Frankie.”

“That asshole. He’s there, holding court or something. But you still got to pay admission, honey.”

“Sure,” Rickenharp said.

He took another twenty newbux out of his pocket, but Carmen put a hand on his arm and said, “We’ll cover this one.” She slapped a twenty down.

Carter took it, chuckling. “Man, that queen got some real nice larynx work.” Knowing damn well she was a girl. “Hey, Rick, you still playing at the—”

“I blew the gig off,” Rickenharp cut in, trying to head off the pain. The boss blue had peaked and left him feeling like he was made out of cardboard inside, like any pressure might make him buckle. His muscles twitched now and then, fretful as restive children scuffing feet. He was crashing. He needed another hit. When you were up, he thought, things showed you their frontsides, their upsides; when you peaked, things showed you their hideous insides. When you were down, things showed you their backsides, their downsides. File it away for lyrics.

Carter pressed the buzzer that unlocked the door. It razzed them as they walked through.

Inside it was dim, hot, humid.

“I think your blue was cut with coke or meth or something,” Rickenharp told Carmen as they walked past the dented lockers. “Cause I’m crashing harder than I should be.”

“Yeah, probably … What’d he mean ‘he got green pissed’?”

“Positive test for AIDS-three. The HIV that kills you in three weeks. You drop this testing pill in your urine and if the urine turns green you got AIDS. There’s no cure for the new HIV yet, won’t be in three weeks, so the guy … ” He shrugged.

“What the ’ell is this place?” Willow asked.

In a low voice Rickenharp told him, “It’s a kind of bathless gay baths, man. Cruising places for ’mos. But about a lotta the people are straights who ran out of bux at the casinos, use it for a cheap place to sleep, you know?”

“Yeah? And ’ow come you know all about it, ’ey?”

Rickenharp smirked. “You saying I’m gay? The horror, the horror.”

Someone in a darkened alcove to one side laughed at that.

Willow was arguing with Yukio in an undertone. “Oi don’t like it, that’s all, fucking faggots got a million fucking diseases. Some side o’ beef with a tan going to wank on me leg.”

“We just walk through, we don’t touch,” Yukio said. “Rickenharp knows what to do.”

Rickenharp thought, Hope so.

Maybe Frankie could get them safely off Freezone, maybe not.

The walls were black pressboard. It was a maze like a tingler gallery but in the negative. There was a more ordinary red light; there was the peculiar scent that lots of skin on skin generates and the accretion of various smokes, aftershaves, cheap soap, and an ingrained stink of sweat and semen gone rancid. The walls stopped at ten feet up and the shadows gathered the ceiling into themselves, far overhead. It was a converted warehouse space, with a strange vibe of stratification: claustrophobia layered under agoraphobia. They passed mossy dark cruising warrens. Faces blurred by anonymity turned to monitor them as they passed, expressions cool as video cameras.

They strolled through the game room with its stained pool tables and stammering holo-games, its prized-open vending machines. Peeling from the walls between the machines were posters of men—caricatures with oversized genitals and muscles that seemed themselves a kind of sexual organ, faces like California surfers. Carmen bit her finger to keep from laughing at them, marveling at the idiosyncratic narcissism of the place.

They passed through a cruising room designed to look like a barn. Two men ministered to one another on a wooden bench inside a “horse stall” with wet fleshy noises. Willow and Yukio looked away. Carmen stared at the gay sex in fascination. Rickenharp walked past without reacting, led the way through other midnight nests of pawing men; past men sleeping on benches and couches, sleepily slapping unwanted hands away.

And found Frankie in the TV lounge.

The TV lounge was bright, well-lit, the walls cheerful yellow. The OmeGaity was cheap—there were no holo cubes. There were motel-standard living-room lamps on end tables; a couch; a regular color screen showing a rock video channel; and a bank of monitors on the wall. It was like emerging from the underworld. Frankie was sitting on the couch, waiting for customers.

Frankie dealt on a porta-terminal he’d plugged into a Grid-socket. The buyer gave him an account number or credit card; Frankie checked the account, transferred the funds into his own (registered as consultancy fees), and handed over the packets.

The walls of the lounge were inset with video monitors; one showed the orgy room, another a porn vid, another ran a Grid network satellite channel. On that one a newscaster was yammering about the attempted assassination, this time in technicki, and Rickenharp hoped Frankie wouldn’t notice it and make the connection. Frankie the Mirror was into taking profit from whatever came along, and the SA paid for information.

Frankie sat on the torn blue vinyl couch, hunched over the pocket-sized terminal on the coffee table. Frankie’s customer was a disco ’mo with a blue sharkfin flare, steroid muscles, and a white karate robe; the guy was standing to one side, staring at the little black canvas bag of blue packets on the coffee table as Frankie completed the transaction.

Frankie was black. His bald scalp had been painted with reflective chrome; his head was a mirror, reflecting the TV screens in fish-eye miniature. He wore a pinstriped three-piece gray suit. A real one, but rumpled and stained like he’d slept in it, maybe fucked in it. He was smoking a Nat Sherman cigarette, down to the gold filter. His synthcoke eyes were demonically red. He flashed a yellow grin at Rickenharp. He looked at Willow, Yukio, and Carmen, made a mocking scowl. “Fucking narcs—get more fancy with their setups every day. Now they got four agents in here, one of ’em looks like my man Rickenharp, other three took like refugees and a computer designer. But that Jap hasn’t got a camera. Gives him away.”

“What’s this ’ere about—” Willow began.

Rickenharp made a dismissive gesture that said, He isn’t serious, dumbshit. “I got two purchases to make,” he announced and looked at Frankie’s buyer. The buyer took his packet and melted back into the warrens.

“First off,” Rickenharp said, taking his card from his wallet, “I need some blue blow, three grams.”

“You got it, homeboy.” Frankie ran a lightpen over the card, then punched a request for data on that account. The terminal asked for the private code number. Frankie handed the terminal to Rickenharp, who punched in his code, then erased it from visual. Then he punched to transfer funds to Frankie’s account. Frankie took the terminal and double-checked the transfer. The terminal showed Rickenharp’s adjusted balance and Frankie’s gain.

“That’s gonna eat up half your account, Harpie,” Frankie said.

“I got some prospects.”

“I heard you and Mose parted company.”

“How’d you get that so fast?”

“Ponce was here buying.”

“Yeah, well—now I’ve dumped the dead weight, my prospects are even better.” But as he said it he felt dead weight in his gut.

“ ’S your bux, man.” Frankie reached into the canvas carry-on, took out three pre-weighed bags of blue powder. He looked faintly amused. Rickenharp didn’t like the look. It seemed to say, I knew you’d come back, you sorry little wimp.

“Fuck off, Frankie,” Rickenharp said, taking the packets.

“What’s this sudden squall of discontent, my child?”

“None of your business, you smug bastard.”

Frankie’s smugness tripled. He glanced speculatively at Carmen and Yukio and Willow. “There’s something more, right?”

“Yeah. We got a problem. My friends here—they’re getting off the raft. They need to slip out the back way so Tom and Huck don’t see ’em.”

“Mmm. What kind of net’s out for them?”

“It’s a private outfit. They’ll be watching the copter port, everything legit … ”

“We had another way off,” Carmen said suddenly. “But it was blown—”

Yukio silenced her with a look. She shrugged.

“Verr-rry mysterious,” Frankie said. “But there are safety limits to curiosity. Okay. Three grand gets you three berths on my next boat out. My boss’s sending a team to pick up a shipment. I can probably get ’em on there. That’s going east, though. You know? Not west or south or north. One direction and one only.”

“That’s what we need,” Yukio said, nodding, smiling. Like he was talking to a travel agent. “East. Someplace Mediterranean.”

“Malta,” Frankie said. “Island of Malta. Best I can do.” Yukio nodded. Willow shrugged. Carmen assented by her silence.

Rickenharp was sampling the goods. In the nose, to the brain, and right to work. Frankie watched him placidly. Frankie was a connoisseur of the changes drugs made in people. He watched the change of expression on Rickenharp’s face. He watched Rickenharp’s visible shift into ego drive.

“We’re gonna need four berths, Frankie,” Rickenharp said.

Frankie raised an eyebrow. “You better decide after that shit wears off.”

“I decided before I took it,” Rickenharp said, not sure if it was true.

Carmen was staring at him. He took her by the arm and said, “Talk to you a minute?” He led her out of the lounge, into the dark hallway. The skin of her arm was electrically sweet under his fingers. He wanted more. But he dropped his hand from her and said, “Can you get the bux?”

She nodded. “I got a fake card, dips into—well, it’ll get it for us. I mean, for me and Yukio and Willow. I’d have to get authorization to bring you. And I can’t do that.”

“Know what? I won’t help you get out otherwise.”

“You don’t know—”

“Yeah, I do. I’m ready to go. I just go back and get my guitar.”

“The guitar’ll be a burden where we’re going. We’re going into occupied territory, to get where we want to be. You’d have to leave the guitar.”

He almost wavered at that. “I’ll check it into a locker. Pick it up someday. Thing is—if they watched us with that bird, they saw me with you. They’ll assume I’m part of it. Look, I know what you’re doing. The SA’s looking for you. Right? So that means you’re—”

“Okay, hold it, shit; keep your voice down. Look—I can see where maybe they marked you, so you got to get off the raft, too. Okay, you go with us to Malta. But then you—”

“I got to stay with you. The SA’s everywhere. They marked me.”

She took a deep breath and let it out in a soft whistle through her teeth. She stared at the floor. “You can’t do it.” She looked at him. “You’re not the type. You’re a fucking artist.”

He laughed. “You say that like it’s the lowest insult you can come up with. Look—I can do it. I’m going to do it. The band is dead. I need to … ” He shrugged helplessly. Then he reached up and took her sunglasses off, looked at her shadowed eyes. “And when I get you alone I’m going to batter your cervix into jelly.”

She punched him hard in the shoulder. It hurt. But she was smiling. “You think that kind of talk turns me on? Well, it does. But it’s not going to get you into my pants. And as for going with us—What you think this is? You’ve seen too many movies.”

“The SA’s marked me, remember? What else can I do?”

“That’s not a good enough reason to … to become part of this thing. You got to really believe in it, because it’s hard. This is not a celebrity game show.”

“Jesus. Give me a break. I know what I’m doing.”

That was bullshit. He was trashed. He was blown. My computer’s experiencing a power surge. Motherboard fried. Hell, then burn out the rest.

He was living a fantasy. But he wasn’t going to admit it. He repeated, “I know what I’m doing.”

She snorted. She stared at him. “Okay,” she said.

And after that everything was different.